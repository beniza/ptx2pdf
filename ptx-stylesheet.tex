%%%%%%%%%%%%%%%%%%%%%%%%%%%%%%%%%%%%%%%%%%%%%%%%%%%%%%%%%%%%%%%%%%%%%%%
% Part of the ptx2pdf macro package for formatting USFM text
% copyright (c) 2007 by SIL International
% written by Jonathan Kew
%
% Permission is hereby granted, free of charge, to any person obtaining  
% a copy of this software and associated documentation files (the  
% "Software"), to deal in the Software without restriction, including  
% without limitation the rights to use, copy, modify, merge, publish,  
% distribute, sublicense, and/or sell copies of the Software, and to  
% permit persons to whom the Software is furnished to do so, subject to  
% the following conditions:
%
% The above copyright notice and this permission notice shall be  
% included in all copies or substantial portions of the Software.
%
% THE SOFTWARE IS PROVIDED "AS IS", WITHOUT WARRANTY OF ANY KIND,  
% EXPRESS OR IMPLIED, INCLUDING BUT NOT LIMITED TO THE WARRANTIES OF  
% MERCHANTABILITY, FITNESS FOR A PARTICULAR PURPOSE AND  
% NONINFRINGEMENT. IN NO EVENT SHALL SIL INTERNATIONAL BE LIABLE FOR  
% ANY CLAIM, DAMAGES OR OTHER LIABILITY, WHETHER IN AN ACTION OF  
% CONTRACT, TORT OR OTHERWISE, ARISING FROM, OUT OF OR IN CONNECTION  
% WITH THE SOFTWARE OR THE USE OR OTHER DEALINGS IN THE SOFTWARE.
%
% Except as contained in this notice, the name of SIL International  
% shall not be used in advertising or otherwise to promote the sale,  
% use or other dealings in this Software without prior written  
% authorization from SIL International.
%%%%%%%%%%%%%%%%%%%%%%%%%%%%%%%%%%%%%%%%%%%%%%%%%%%%%%%%%%%%%%%%%%%%%%%

%% Macros to read Paratext stylesheets and define TeX control sequences
%% for all markers present in the file

%
% Macros to process each marker used in a Paratext stylesheet,
% either ignoring it or saving the parameter as appropriate
%

% Remember the name of the marker currently being defined
\def \Marker         #1\relax{\def\m@rker{#1}} % store name of marker currently being defined

% Ignore stylesheet markers that do not affect formatting 
\def \Name           #1\relax{}
\def \Description    #1\relax{}
\def \OccursUnder    #1\relax{}
\let \Occursunder\OccursUnder
\def \Rank           #1\relax{}
\def \Underline      {}
\let \underline\Underline
\def \NotRepeatable  {}
\let \Notrepeatable\NotRepeatable
\def \Color          #1\relax{}
\let \color\Color

% For each property that affects formatting, define a macro to store the value
% of this field, these will be accessed via getp@ram.
% Constructs a macro name \<marker>:<parameter> to hold the value
\def \Endmarker      #1\relax{\defp@ram{endmarker}{#1}} % endmarker is not currently checked
\def \TextType       #1\relax{\lowercase{\defp@ram{type}{#1}}}
\let \Texttype\TextType
\def \TextProperties #1\relax{\lowercase{\defp@ram{properties}{#1}}}
\let \Textproperties\TextProperties
\def \FontSize       #1\relax{\defp@ram{fontsize}{#1}}
\let \Fontsize\FontSize
\def \FontName       #1\relax{\defp@ram{fontname}{#1}}
\let \Fontname\FontName
\def \FirstLineIndent#1\relax{\defp@ram{firstindent}{#1}}
\let \Firstlineindent\FirstLineIndent
\def \LeftMargin     #1\relax{\defp@ram{leftmargin}{#1}}
\let \Leftmargin\LeftMargin
\def \RightMargin    #1\relax{\defp@ram{rightmargin}{#1}}
\let \Rightmargin\RightMargin
\def \Italic         {\defp@ram{italic}{true}}
\def \Bold           {\defp@ram{bold}{true}}
\def \Superscript    {\defp@ram{superscript}{true}}
\def \Regular        {\defp@ram{italic}{false}\defp@ram{bold}{false}\defp@ram{superscript}{false}}
\let \superscript\Superscript
\def \SpaceBefore    #1\relax{\defp@ram{spacebefore}{#1}}
\let \Spacebefore\SpaceBefore
\def \SpaceAfter     #1\relax{\defp@ram{spaceafter}{#1}}
\let \Spaceafter\SpaceAfter
\def \Justification  #1\relax{\lowercase{\defp@ram{justification}{#1}}}
\def \CallerStyle    #1\relax{\defp@ram{callerstyle}{#1}} % ptx2pdf extension
\def \TEStyleName    #1\relax{} % Translation Editor extension

% Make a TeX macro based on the StyleType
\def \StyleType      #1\relax{\lowercase{\def\styl@type{#1}}\m@kestyle}
\let \Styletype=\StyleType

%
% \defp@ram: store the value of a parameter from the .sty file
%   #1 -> name of parameter to store
%   #2 -> value
% Constructs a macro name \<marker>:<parameter> to hold the value
% csname is required here in order to build a macro name containing a colon and possibly a number.
%
\def\defp@ram#1#2{\x@\def\csname\m@rker:#1\endcsname{#2}}

%
% \getp@aram: fetch the value of a style parameter into \p@ram
%   #1 -> name of parameter to fetch
%   #2 -> name of marker
% Sets temporary macro \p@ram to the value
%
\def\getp@ram#1#2{\x@\let\x@\p@ram\csname#2:#1\endcsname}

%
%
%
\def\m@kestyle{{\uccode`\|=`\\\uppercase{\message{|\m@rker}}}
 \ifx\styl@type\P@ra \x@\defp@rstyle\x@{\m@rker}
 \else\ifx\styl@type\Ch@r \x@\defch@rstyle\x@{\m@rker}
 \else\ifx\styl@type\N@te \x@\defn@testyle\x@{\m@rker}
 \else \message{unknown style type \styl@type}
 \fi\fi\fi}
\def\P@ra{paragraph}
\def\Ch@r{character}
\def\N@te{note}

%
% \def*style: define a USFM marker as a paragraph, character or note style marker
% which will expand to \p@rstyle, \ch@rstyle or \n@testyle, with the marker name
% as its parameter
%
% \csname...\endcsname is used because the marker may contain numbers as well as letters
%
\def\defp@rstyle#1{\x@\def\csname#1\endcsname{\p@rstyle{#1}}}
\def\defch@rstyle#1{\x@\def\csname#1\endcsname{\ch@rstyle{#1}}}
\def\defn@testyle#1{\x@\def\csname#1\endcsname{\n@testyle{#1}}\m@ken@tecl@ss{#1}}

% Allow "hooking" custom TeX code at the beginning or ending of specific styles.
% e.g. \sethook{start}{s1}{-} % put a hyphen at the start of every section heading
\def\sethook#1#2#3{\x@\def\csname #1-#2\endcsname{#3}}

% All types of marker use this to check for the "nonpublishable" property
\def\t@stpublishability#1{\n@npublishablefalse % check if "nonpublishable" occurred in the marker's \Properties
 \getp@ram{properties}{#1}%
 \x@\t@stnonpub\p@ram nonpublishable!}
\def\t@stnonpub #1nonpublishable#2!{%
 \def\t@st{#2}\ifx\t@st\empty\else\n@npublishabletrue\fi}
\newif\ifn@npublishable

% called after \c and \v have been processed in stylesheet at beginning of typesetting file
% Patch in special handling for \c and \v.
% ~ is set to lowercase to a space in \ch@rstyle
\lowercase{
 \def\@ddcvhooks{
  \let\@V=\v% remember \v defn for (A) below
  % Define a macro to read an actual space delimited verse number
  \def\@v@ ##1 {\gdef\v@rse{##1}%
   \x@\spl@tverses\v@rse--\relax % split verse in a verse bridge (e.g. 3-4) into \v@rsefrom, \v@rseto
   % if not cancelling first verse, output verse number
   \ifc@ncelfirstverse\else
     \@V~\printv@rse\@V*% (A) output verse number using saved style. ~ will lowercase to a space
     \kern\AfterVerseSpaceFactor\FontSizeUnit \fi
   \egroup
   \m@rkverse % generate a milestone for running headers
   % run the hooks for picture insertion, paragraph adjustments, etc.
   % these hooks were added via \addtoversehooks
   \the\v@rsehooks
   % remember current chapter:verse, not currently used. Could potentially be used
   % to insert chapter:verse in footnotes.
   \gdef\reference{\ch@pter:\v@rse}%
   \nobreak\hskip1sp % let \x detect that there's "space" (or maybe drop-chapter) here
  }
  % define handling for usfm \v marker
  \def\v{\leavevmode % ensure we are in horizontal mode to build paragraph
   % set c@ncelfirstversetrue if we are immediately following a drop cap chapter number
   % and the user has requested omission of the first verse number in a chapter
   \c@ncelfirstversefalse
   \ifnum\spacefactor=\n@wchaptersf \kern0sp % override \hangversenumber here
    \ifOmitVerseNumberOne \c@ncelfirstversetrue \fi \fi
   % -1sp indicates that a hanging verse number has been requested here.
   % When this happens we make ll@p use llap to right justify the verse number at the current position.
   % Otherwise make ll@p do nothing.
   \ifdim\lastkern=-1sp \let\ll@p=\llap
   \else \let\ll@p=\relax \fi % hanging verse number?
   \ll@p\bgroup\m@kedigitsother\@v@} %
  % remember chapter macro for later
  \let\@C=\c
  % Define a macro to process space delimited chapter number
  % ## is necessary because this is a nested macro definition
  \def\@c@ ##1 {\ifsk@pping \egroup \fi % if we were skipping, stop it
   \gdef\ch@pter{##1}% remember chapter number
   \gdef\v@rse{}% clear verse number
   \m@kedigitsletters
   % If omitting chapter number, skip this. 
   % We would have to set \OmitChapterNumbertrue on a book by book basis
   % in the configuration file in order to omit chapter numbers for single chapter books.
   \ifOmitChapterNumber\else
    % if chapter label not present, set \ch@ptertrue to cause chapter number to be printed
    % at start of first text paragraph.
    \ifx\ch@plabel\empty \global\ch@ptertrue 
    \else % if \cl present, output chapter label here
      \p@rstyle{cl}\ch@plabel\ \ch@pter
      \pdfb@@kmark \pdfch@ptermark
    \fi
   \fi}
  %
  \def\c{\m@kedigitsother\@c@}
  % set default value for drop cap chapter number size if not already set
  \ifdim\dropnumbersize=0pt
    \getp@ram{fontsize}{p}\dimen255=0\p@ram\FontSizeUnit \multiply\dimen255 by 2
    \ifdim\dimen255=0pt \dimen255=24\FontSizeUnit \fi % just in case \FontSize was missing
    \getp@ram{fontsize}{c}\dropnumbersize=0\p@ram\FontSizeUnit \relax
	% if font size specified for chapter number stylesheet is less than 2*paragraph font size ...
    \ifdim\dropnumbersize<\dimen255
      \s@tfont{p}%workaround for Lotus Elam Bold issue - ensure the \p font is loaded first
      \s@tfont{c}\setbox0=\hbox{0123456789}%set a box at the size specified by the \c marker
	  %clear font<c> so that it will get rebuilt on next setfont for \c (after we have figured out real size)
      \x@\let\csname font<c>\endcsname=\relax 
	  % Calculate drop cap height = base line height + (lower case) x height
      \s@tfont{p}\dimen0=\fontdimen5\font % get x height (appendix F) 5 is x-height of font
      \advance\dimen0 by \baselineskip
	  % The following calculations are scaled by 256 in order to get better accuracy
	  % even though calculations are integer values.
	  % divide desired height by current height
      \multiply\dimen0 by 256
      \divide\dimen0 by \ht0
	  % multiply by original size from \c marker in stylesheet
      \multiply\dimen0 by \dropnumbersize
      \dropnumbersize=\dimen0
      % divide by the \FontSizeUnit to get value for fontsize param of marker \c
      \dimen0=\FontSizeUnit \divide\dimen0 by 256
      \divide\dropnumbersize by \dimen0
    \fi
  \fi
  % set font size for \c
  \edef\dr@psize{\expandafter\strip@pt\the\dropnumbersize}%
  \def\m@rker{c}\defp@ram{fontsize}{\dr@psize}
 }
}
{\catcode`P=12 \catcode`T=12 \lowercase{\gdef\strip@pt#1PT{#1}}}
\def\AfterVerseSpaceFactor{2}
\newif\ifOmitChapterNumber
\newif\ifc@ncelfirstverse
\def\printv@rse{\v@rsefrom\ifx\v@rsefrom\v@rseto\else\endash\v@rseto\fi}
\def\@ne{1}

% put \hangversenumber into the <start> hook for a style such as \q1
% in order to 'hang' verse numbers into the paragraph indent of the style
\def\hangversenumber{\kern-1sp\relax}

% size to use for drop-cap style numbers; automatically calculated if not set
\newdimen\dropnumbersize

% other modules can use \addtoversehooks to insert macros that will be executed at each verse
\def\addtoversehooks#1{\x@\global\x@\v@rsehooks\x@{\the\v@rsehooks #1}}
\newtoks\v@rsehooks

\newif\ifch@pter \def\ch@plabel{}

% macros to switch digits between "letter" and "other" category; must be "letter" when reading USFM data,
% but "other" when we want to read numeric values
\def\m@kedigitsletters{\catcode`0=\el@ven \catcode`1=\el@ven \catcode`2=\el@ven \catcode`3=\el@ven
 \catcode`4=\el@ven \catcode`5=\el@ven \catcode`6=\el@ven \catcode`7=\el@ven
 \catcode`8=\el@ven \catcode`9=\el@ven \relax}
\def\m@kedigitsother{\catcode`0=\tw@lve \catcode`1=\tw@lve \catcode`2=\tw@lve \catcode`3=\tw@lve
 \catcode`4=\tw@lve \catcode`5=\tw@lve \catcode`6=\tw@lve \catcode`7=\tw@lve
 \catcode`8=\tw@lve \catcode`9=\tw@lve \relax}
\def\el@ven{11}
\def\tw@lve{12}

\let\b@ok\relax
\def\ch@pter{}
\def\v@rse{}
\def\m@rkverse{\mark{\b@ok:\ch@pter:\v@rse}}

\begingroup
\obeylines%
\gdef\@ddspecialhooks{% there are a few special USFM markers that we give "magic" properties
 \let\@H=\h%
 \def\h{\bgroup\obeylines\@h}% \h gets stored as the book name (for references)
 \def\@h ##1^^M{\gdef\b@ok{##1}\egroup}% store book id in \b@ok
 %
 \let\@CL=\cl%
 \def\do@CL{\global\ch@pterfalse\@CL}%
 \def\cl{\ifch@pter \let\n@xt=\do@CL \else \let\n@xt=\st@recl \fi \n@xt}%
 \def\st@recl{\bgroup\obeylines\@cl}% \cl gets stored as chapter label and changes \c format
 \def\@cl ##1^^M{\gdef\ch@plabel{##1}\egroup}%
 %
 \let\@ID=\id%
 \def\id{\bgroup\obeylines\unc@tcodespecials \m@kedigitsother
   \catcode32=10 \@id}% \id gets stored and printed in the margin with cropmarks
 \def\@id ##1^^M{\gdef\c@rrID{##1}\uppercase{\@@id##1ZZZ\end}\egroup}%
 %
 \let\fig=\ptx@fig% \fig has its own special macro to parse the fields
 %
 \let\nb=\ptx@nb% \nb is a special code that suppresses a paragraph break across \c
}%
\endgroup
\def\unc@tcodespecials{\def\do##1{\catcode`##1=12 }\dospecials}
\def\@@id#1#2#3#4\end{\gdef\id@@@{#1#2#3}} % get first 3 chars of the \id
\def\ptx@fig #1\fig*{\d@figure{#1}}

%
% \stylesheet{...} reads a Paratext stylesheet line by line
%
\newif\ifc@ntinue
\newread\styl@sheet
\def\stylesheet#1{
 \let\save@regular=\regular \let\regular=\Regular
 \let\save@bold=\bold \let\bold=\Bold
 \let\save@italic=\italic \let\italic=\Italic
 \openin\styl@sheet="#1" \c@ntinuetrue
 \ifeof\styl@sheet\errmessage{Paratext stylesheet "#1" not found}\else
   \message{Reading Paratext stylesheet "#1"...}%
   \endlinechar=-1
   \catcode`\#=5 % paratext comment char
   \loop
    \read\styl@sheet to \th@line
    \th@line \relax
    \ifeof\styl@sheet \c@ntinuefalse \fi
    \ifc@ntinue\repeat
   \endlinechar=13
   \closein\styl@sheet
   \catcode`\#=6 % restore default TeX catcode
 \fi
 \let\regular=\save@regular
 \let\bold=\save@bold
 \let\italic=\save@italic
}

\def\@netimesetup{
 % stuff to execute after loading all stylesheets but before processing the first USFM file
 \s@tupsizes
 \@ddcvhooks
 \@ddspecialhooks
 \the\@nithooks
 \let\@netimesetup=\relax}

% Read the files associated with this book: usfm, adjustment list, picture list
\def\ptxfile#1{
 \@netimesetup
 \gdef\ch@plabel{}
 \initp@rastyles
 \initn@testyles
 \openadjlist "\the\AdjListPath#1.adj"
 \openpiclist "\the\PicListPath#1.piclist"
 %\catcode`\$=12
 \catcode`\^=12 % make these printable
 \catcode`\&=12 \catcode`\~=12
 \catcode`\#=12 %\catcode`\%=12
 \catcode`\{=12 \catcode`\}=12
 \catcode`\_=11 % treat as letter for TE custom styles
 \catcode`\/=\active
 \catcode13=10
 \def\d@something{\afterassignment\@ctivate \count255=}\the\do@ctive
 \m@kedigitsletters %so that marker names can include digits
 % test if file is available, inform user if it's missing
 \openin\t@stread="\the\PtxFilePath#1"
   \ifeof\t@stread \def\n@xt{\MSG{USFM file "\the\PtxFilePath#1" not found -- ignored}}
   \else \closein\t@stread \def\n@xt{\input "\the\PtxFilePath#1"\relax}\fi
 \n@xt
 \ifhmode \unskip\end@llcharstyles \par \fi
 \ifsk@pping\egroup\fi % end skipping, if file ended with a non-printable marker
 \endt@ble % end table if file ended during table processing
 \m@kedigitsother
 \def\d@something{\afterassignment\de@ctivate \count255=}\the\do@ctive
 \catcode13=5
 \closepiclist
 \closeadjlist
 \catcode`\/=12
 \catcode`\#=6 \catcode`\%=14 % restore TeX meanings for those we might use
 \catcode`\{=1 \catcode`\}=2
 \singlecolumn
 \pagebreak}
\newread\t@stread
\newtoks\AdjListPath % path to look for adjustment files
\newtoks\PicListPath % path to look for picture list files
\newtoks\PtxFilePath % path to look for main ptx input files

% mechanism to define active chars within the ptx files
\catcode`\~=\active
\def\DefineActiveChar#1#2{\count255=`#1
  \do@ctive=\x@\x@\x@{\x@\the\x@\do@ctive
    \x@\d@something\the\count255 }
  \lccode`\~=\count255
  \def\t@mp{\noexpand#2} \lowercase{\edef~{\t@mp}}}
\newtoks\do@ctive

\def\@ctivate{\x@\edef\csname save-cat-\the\count255\endcsname{\the\catcode\count255}%
  \catcode\the\count255=\active\relax}
\def\de@ctivate{\catcode\the\count255=\csname save-cat-\the\count255\endcsname\relax}

\def\openbracket{\char`\[\ignorespaces}
\def\openparen{\char`\(\ignorespaces}

\DefineActiveChar{[}{\openbracket} % make these skip following spaces
\DefineActiveChar{(}{\openparen}   % (useful at end-of-line before new verse)

\catcode`\/=\active
\def/{\futurelet\n@xt\sl@sh}
\def\sl@sh{\ifx\n@xt/\let\n@xt\sl@shbreak\else\let\n@xt\sl@shprint\fi\n@xt}
\def\sl@shbreak/{\unskip\penalty-250{} \ignorespaces}
\def\sl@shprint{\char`\/}
\catcode`\/=12

% Create registers to hold configuration information, set defaults
\newdimen\FontSizeUnit		\FontSizeUnit=1bp % units for font sizes in Paratext stylesheet
\newdimen\IndentUnit		\IndentUnit=1in % units for indents
\newdimen\PaperWidth		\PaperWidth=210mm
\newdimen\PaperHeight		\PaperHeight=297mm
\newdimen\MarginUnit		\MarginUnit=1in % units for margins
\newdimen\BindingGutter		\BindingGutter=5mm
\def\LineSpacingFactor{1.0} % scale factor used for line spacing
\def\VerticalSpaceFactor{1.0} % scaling factor applied to space before/after styles
\def\TopMarginFactor{1.0}
\def\BottomMarginFactor{\TopMarginFactor}
\def\SideMarginFactor{1.0}
\def\ColumnGutterFactor{15} % relative to FontSizeUnit
\def\HeaderPosition{0.5} % pos of baseline from top of page, relative to MarginUnit
\def\FooterPosition{0.5} % pos of baseline from bottom of page, relative to MarginUnit

\newcount\TitleColumns \TitleColumns=1
\newcount\IntroColumns \IntroColumns=1
\newcount\BodyColumns  \BodyColumns=1

\newif\ifBindingGutter % \BindingGutter won't be used unless this is set to true
\newif\ifDoubleSided \DoubleSidedtrue % binding gutter on alternate sides; use odd/even headers

% calculate various dimensions based on the factors, units, etc that have been defined
\def\s@tupsizes{
 \le@dingunit=\LineSpacingFactor\FontSizeUnit
 \verticalsp@ceunit=\VerticalSpaceFactor\le@dingunit
 \baselineskip=14\le@dingunit
 \lineskiplimit=-7\le@dingunit
 \lineskip=0pt
 \topskip=\baselineskip
 \gutter=\ColumnGutterFactor\FontSizeUnit
 \dimen0=\PaperWidth \dimen2=\SideMarginFactor\MarginUnit
 \advance\dimen0 by -2\dimen2
 \ifBindingGutter \advance\dimen0 by -\BindingGutter \fi
 \textwidth=\dimen0
 \colwidth=0.5\textwidth \advance\colwidth by -0.5\gutter
 \hsize=\textwidth
 \topm@rgin=\TopMarginFactor\MarginUnit
 \advance\topm@rgin by -\topskip \advance\topm@rgin by 12\FontSizeUnit
 \bottomm@rgin=\BottomMarginFactor\MarginUnit
 \dimen0=\PaperHeight
 \advance\dimen0 by -\topm@rgin
 \advance\dimen0 by -\bottomm@rgin
 \textheight=\dimen0
 \vsize=\textheight
 \pdfpagewidth=\PaperWidth
 \pdfpageheight=\PaperHeight
 \resetvsize
}
\newdimen\le@dingunit
\newdimen\verticalsp@ceunit
\newdimen\topm@rgin
\newdimen\bottomm@rgin
\newdimen\textwidth
\newdimen\textheight
\newdimen\colwidth
\newdimen\gutter \gutter=20pt

\endinput
