%% Macros to read Paratext stylesheets and define TeX control sequences
%% for all markers present in the file

%
% Macros to process each marker used in a Paratext stylesheet,
% either ignoring it or saving the parameter as appropriate
%
\def \Marker         #1\relax{\def\m@rker{#1}} % store name of marker currently being defined
\def \Endmarker      #1\relax{\defp@ram{endmarker}{#1}}
\def \Name           #1\relax{}
\def \Description    #1\relax{}
\def \OccursUnder    #1\relax{}
\let \Occursunder\OccursUnder
\def \Rank           #1\relax{}
\def \TextType       #1\relax{\lowercase{\defp@ram{type}{#1}}}
\let \Texttype\TextType
\def \TextProperties #1\relax{\lowercase{\defp@ram{properties}{#1}}}
\let \Textproperties\TextProperties
\def \StyleType      #1\relax{\lowercase{\def\styl@type{#1}}\m@kestyle}
\let \Styletype=\StyleType
\def \FontSize       #1\relax{\defp@ram{fontsize}{#1}}
\let \Fontsize\FontSize
\def \FontName       #1\relax{\defp@ram{fontname}{#1}}
\let \Fontname\FontName
\def \FirstLineIndent#1\relax{\defp@ram{firstindent}{#1}}
\let \Firstlineindent\FirstLineIndent
\def \LeftMargin     #1\relax{\defp@ram{leftmargin}{#1}}
\let \Leftmargin\LeftMargin
\def \RightMargin    #1\relax{\defp@ram{rightmargin}{#1}}
\let \Rightmargin\RightMargin
\def \Italic         {\defp@ram{italic}{true}}
\def \Bold           {\defp@ram{bold}{true}}
\def \Superscript    {\defp@ram{superscript}{true}}
\let \superscript\Superscript
\def \Underline      {}
\let \underline\Underline
\def \NotRepeatable  {}
\let \Notrepeatable\NotRepeatable
\def \SpaceBefore    #1\relax{\defp@ram{spacebefore}{#1}}
\let \Spacebefore\SpaceBefore
\def \SpaceAfter     #1\relax{\defp@ram{spaceafter}{#1}}
\let \Spaceafter\SpaceAfter
\def \Color          #1\relax{}
\let \color\Color
\def \Justification  #1\relax{\lowercase{\defp@ram{justification}{#1}}}
\def \CallerStyle    #1\relax{\defp@ram{callerstyle}{#1}}

%
% \defp@ram: store the value of a parameter from the .sty file
%   #1 -> name of parameter to store
%   #2 -> value
% Constructs a macro name \<marker>:<parameter> to hold the value
%
\def\defp@ram#1#2{\x@\def\csname\m@rker:#1\endcsname{#2}}

%
% \getp@aram: fetch the value of a style parameter into \p@ram
%   #1 -> name of parameter to fetch
%   #2 -> name of marker
% Sets temporary macro \p@ram to the value
%
\def\getp@ram#1#2{\x@\let\x@\p@ram\csname#2:#1\endcsname}

%
%
%
\def\m@kestyle{{\uccode`\|=`\\\uppercase{\message{|\m@rker}}}
 \ifx\styl@type\P@ra \x@\defp@rstyle\x@{\m@rker}
 \else\ifx\styl@type\Ch@r \x@\defch@rstyle\x@{\m@rker}
 \else\ifx\styl@type\N@te \x@\defn@testyle\x@{\m@rker}
 \else \message{unknown style type \styl@type}
 \fi\fi\fi}
\def\P@ra{paragraph}
\def\Ch@r{character}
\def\N@te{note}

%
% \def*style: define a USFM marker as a paragraph, character or note style marker
% which will expand to \p@rstyle, \ch@rstyle or \n@testyle, with the marker name
% as its parameter
%
% \csname...\endcsname is used because the marker may contain numbers as well as letters
%
\def\defp@rstyle#1{\x@\def\csname#1\endcsname{\p@rstyle{#1}}}
\def\defch@rstyle#1{\x@\def\csname#1\endcsname{\ch@rstyle{#1}}}
\def\defn@testyle#1{\x@\def\csname#1\endcsname{\n@testyle{#1}}\m@ken@tecl@ss{#1}}

\def\sethook#1#2#3{\x@\def\csname #1-#2\endcsname{#3}}

\lowercase{
 \def\@ddcvhooks{
  \let\@V=\v
  \def\@v@ ##1 {\gdef\v@rse{##1}%
   \x@\spl@tverses\v@rse--\relax
   \global\c@ncelfirstversefalse
   \ifOmitVerseNumberOne \ifnum\spacefactor=998 \global\c@ncelfirstversetrue \fi \fi
   \ifc@ncelfirstverse\else
     \@V~\printv@rse\@V*%
     \kern\AfterVerseSpaceFactor\FontSizeUnit \fi
   \egroup
   \m@rkverse
   \the\v@rsehooks
   \gdef\reference{\ch@pter:\v@rse}%
   \nobreak\hskip1sp % let \x detect that there's "space" (or maybe drop-chapter) here
  }
  \def\v{\leavevmode
   \ifdim\lastkern=-1sp \let\ll@p=\llap
   \else \let\ll@p=\relax \fi % hanging verse number?
   \ll@p\bgroup\m@kedigitsother\@v@}
  \let\@C=\c
  \def\@c@ ##1 {\ifsk@pping \egroup \fi
   \gdef\ch@pter{##1}\gdef\v@rse{}\m@kedigitsletters
   \ifOmitChapterNumber\else
    \ifx\ch@plabel\empty \global\ch@ptertrue 
    \else \p@rstyle{cl}\ch@plabel\ \ch@pter\fi
   \fi}
  \def\c{\m@kedigitsother\@c@}
  \ifnum\dropnumbersize=0
    \getp@ram{fontsize}{p}\count255=\p@ram \multiply\count255 by 2
    \getp@ram{fontsize}{c}\dropnumbersize=\p@ram
    \ifnum\dropnumbersize<\count255
      \s@tfont{c}\setbox0=\hbox{0123456789}
      \x@\let\csname font<c>\endcsname=\relax
      \s@tfont{p}\dimen0=\fontdimen5\font
      \advance\dimen0 by \baselineskip
      \multiply\dimen0 by 100 \divide\dimen0 by \ht0
      \multiply\dimen0 by \dropnumbersize \divide\dimen0 by 100
      \dropnumbersize=\dimen0
    \fi
  \fi
  \def\m@rker{c}\defp@ram{fontsize}{\the\dropnumbersize}
 }
}
\def\AfterVerseSpaceFactor{3}
\newif\ifOmitChapterNumber
\def\printv@rse{\v@rsefrom\ifx\v@rsefrom\v@rseto\else\endash\v@rseto\fi}
\def\@ne{1}

% put \hangversenumber into the <start> hook for a style such as \q1
% in order to 'hang' verse numbers into the paragraph indent of the style
\def\hangversenumber{\kern-1sp\relax}

% size to use for drop-cap style numbers; automatically calculated if not set
\newcount\dropnumbersize

% other modules can use \addtoversehooks to insert macros that will be executed at each verse
\def\addtoversehooks#1{\x@\v@rsehooks\x@{\the\v@rsehooks #1}}
\newtoks\v@rsehooks

\newif\ifch@pter \def\ch@plabel{}

% macros to switch digits between "letter" and "other" category; must be "letter" when reading USFM data,
% but "other" when we want to read numeric values
\def\m@kedigitsletters{\catcode`0=\el@ven \catcode`1=\el@ven \catcode`2=\el@ven \catcode`3=\el@ven
 \catcode`4=\el@ven \catcode`5=\el@ven \catcode`6=\el@ven \catcode`7=\el@ven
 \catcode`8=\el@ven \catcode`9=\el@ven \relax}
\def\m@kedigitsother{\catcode`0=\tw@lve \catcode`1=\tw@lve \catcode`2=\tw@lve \catcode`3=\tw@lve
 \catcode`4=\tw@lve \catcode`5=\tw@lve \catcode`6=\tw@lve \catcode`7=\tw@lve
 \catcode`8=\tw@lve \catcode`9=\tw@lve \relax}
\def\el@ven{11}
\def\tw@lve{12}

\def\b@ok{BOOK}
\def\ch@pter{}
\def\v@rse{}
\def\m@rkverse{\mark{\b@ok:\ch@pter:\v@rse}}

\begingroup
\obeylines%
\gdef\@ddspecialhooks{% there are a few special USFM markers that we give "magic" properties
 \let\@H=\h%
 \def\h{\bgroup\obeylines\@h}% \h gets stored as the book name (for references)
 \def\@h ##1^^M{\gdef\b@ok{##1}\egroup}%
 \let\@CL=\cl%
 \def\do@CL{\global\ch@pterfalse\@CL}%
 \def\cl{\ifch@pter \let\n@xt=\do@CL \else \let\n@xt=\st@recl \fi \n@xt}%
 \def\st@recl{\bgroup\obeylines\@cl}% \cl gets stored as chapter label and changes \c format
 \def\@cl ##1^^M{\gdef\ch@plabel{##1}\egroup}%
 \let\@ID=\id%
 \def\id{\bgroup\obeylines\unc@tcodespecials \catcode32=10 \@id}% \id gets stored and printed in the margin with cropmarks
 \def\@id ##1^^M{\gdef\c@rrID{##1}\uppercase{\@@id##1ZZZ\end}\egroup}%
 \let\fig=\ptx@fig% \fig has its own special macro to parse the fields
 \let\nb=\ptx@nb% \nb is a special code that suppresses a paragraph break across \c
}%
\endgroup
\def\unc@tcodespecials{\def\do##1{\catcode`##1=12 }\dospecials}
\def\@@id#1#2#3#4\end{\gdef\id@@@{#1#2#3}} % get first 3 chars of the \id
\def\ptx@fig #1\fig*{\d@figure{#1}}

%
% \stylesheet{...} reads a Paratext stylesheet line by line
%
\newif\ifc@ntinue
\newread\styl@sheet
\def\stylesheet#1{
 \let\save@bold=\bold \let\bold=\Bold
 \let\save@italic=\italic \let\italic=\Italic
 \openin\styl@sheet="#1" \c@ntinuetrue
 \ifeof\styl@sheet\message{Paratext stylesheet "#1" not found!}\else
   \message{Reading Paratext stylesheet "#1"...}%
   \endlinechar=-1
   \catcode`\#=5 % paratext comment char
   \loop
    \read\styl@sheet to \th@line
    \th@line \relax
    \ifeof\styl@sheet \c@ntinuefalse \fi
    \ifc@ntinue\repeat
   \endlinechar=13
   \closein\styl@sheet
   \catcode`\#=6 % restore default TeX catcode
 \fi
 \let\bold=\save@bold
 \let\italic=\save@italic
}

\def\@netimesetup{
 % stuff to execute after loading all stylesheets but before processing the first USFM file
 \s@tupsizes
 \@ddcvhooks
 \@ddspecialhooks
 \let\@netimesetup=\relax}

\def\ptxfile#1{
 \@netimesetup
 \gdef\ch@plabel{}
 \initp@rastyles
 \initn@testyles
 \openadjlist "\the\AdjListPath#1.adj"
 \openpiclist "\the\PicListPath#1.piclist"
 %\catcode`\$=12
 \catcode`\^=12 \catcode`\_=12 % make these printable
 \catcode`\&=12 \catcode`\~=12
 \catcode`\#=12 %\catcode`\%=12
 \catcode`\{=12 \catcode`\}=12
 \catcode`\/=\active
 \catcode13=10
 \m@kedigitsletters
 \input "#1" \ifsk@pping\egroup\fi
 \m@kedigitsother
 \catcode13=5
 \closepiclist
 \closeadjlist
 \catcode`\/=12
 \catcode`\#=6 \catcode`\%=14 % restore TeX meanings for those we might use
 \catcode`\{=1 \catcode`\}=2
 \singlecolumn
 \pagebreak}
\newtoks\AdjListPath % path to look for adjustment files
\newtoks\PicListPath % path to look for picture list files

\catcode`\/=\active
\def/{\futurelet\n@xt\sl@sh}
\def\sl@sh{\ifx\n@xt/\let\n@xt\sl@shbreak\else\let\n@xt\sl@shprint\fi\n@xt}
\def\sl@shbreak/{\unskip\penalty-250{} \ignorespaces}
\def\sl@shprint{\char`\/}
\catcode`\/=12

\newdimen\FontSizeUnit		\FontSizeUnit=1bp
\newdimen\IndentUnit		\IndentUnit=1in
\newdimen\PaperWidth		\PaperWidth=210mm
\newdimen\PaperHeight		\PaperHeight=297mm
\newdimen\MarginUnit		\MarginUnit=1in
\newdimen\BindingGutter		\BindingGutter=5mm
\def\LineSpacingFactor{1.0}
\def\VerticalSpaceFactor{1.0}
\def\TopMarginFactor{1.0}
\let\BottomMarginFactor=\TopMarginFactor % use \def to set separate value if desired
\def\SideMarginFactor{1.0}
\def\ColumnGutterFactor{15}

\newcount\TitleColumns \TitleColumns=1
\newcount\IntroColumns \IntroColumns=1
\newcount\BodyColumns  \BodyColumns=1

\newif\ifBindingGutter % \BindingGutter won't be used unless this is set to true
\newif\ifDoubleSided \DoubleSidedtrue

% calculate various dimensions based on the factors, units, etc that have been defined
\def\s@tupsizes{
 \le@dingunit=\LineSpacingFactor\FontSizeUnit
 \verticalsp@ceunit=\VerticalSpaceFactor\le@dingunit
 \baselineskip=14\le@dingunit
 \lineskiplimit=-7\le@dingunit
 \lineskip=0pt
 \topskip=12\le@dingunit
 \gutter=\ColumnGutterFactor\FontSizeUnit
 \dimen0=\PaperWidth \dimen2=\SideMarginFactor\MarginUnit
 \advance\dimen0 by -2\dimen2
 \ifBindingGutter \advance\dimen0 by -\BindingGutter \fi
 \textwidth=\dimen0
 \colwidth=0.5\textwidth \advance\colwidth by -0.5\gutter
 \hsize=\textwidth
 \dimen0=\PaperHeight
 \advance\dimen0 by -\TopMarginFactor\MarginUnit
 \advance\dimen0 by -\BottomMarginFactor\MarginUnit
 \textheight=\dimen0
 \vsize=\textheight
 \dimen0=\SideMarginFactor\MarginUnit \advance\dimen0 by -1in
 \hoffset=\dimen0
 \dimen0=\TopMarginFactor\MarginUnit \advance\dimen0 by -1in
 \advance\dimen0 by 0.5\baselineskip % nudge down a little for header
 \voffset=\dimen0
 \pdfpagewidth=\PaperWidth
 \pdfpageheight=\PaperHeight
 \resetvsize
}
\newdimen\le@dingunit
\newdimen\verticalsp@ceunit
\newdimen\textwidth
\newdimen\textheight
\newdimen\colwidth
\newdimen\gutter \gutter=20pt

\endinput
