\input paratext2
\tracing{m}
\tracing{A}
%\tracing{s}
%\tracing{F}
\tracing{j}
\tracinggroups=1
%\input "ptx-extended.tex"
\stylesheet{ptx2pdf.sty}
\stylesheet{usfm_sb.sty}
\stylesheet{default-custom.sty}
\stylesheet{test.sty}
%\PaperHeight=210mm
%\PaperWidth=296.9mm

\catcode`\@=11
\gdef\b@okShort{}
\def\doLines{\doGridLines}
\x@\def\csname complex-w\endcsname{\rubyb{w}{lemma}}
%\def\dummy{%
   %\box0
   %{\s@tfont{w:lemma}%
   %\get@ttribute{lemma}\attr@b}
%}%
%\def\GraphPaperLineMajor{0.3pt}
%\def\GraphPaperLineMinor{0.1pt}
%\def\GraphPaperYoffset{-0.5mm}
%\def\GraphPaperColMajor{0.7 1.0 0.7} % Colour (R G B)
%\def\GraphPaperColMinor{1.0 0.8 0.8} % Colour (R G B)
\def\GOTOLinkBorderstyle{/S/U/D [2 2]/W 1} % Underline
\def\URLLinkBorderstyle{/S/U/W 1} % Underline
\def\GOTOLinkBorderCol{.9 .5 .5}% pale red
\def\URLLinkBorderCol{.5 .5 .5}% Gry

\tracingassigns=1
\tracingoutput=1
\showboxdepth=19
\showboxbreadth=19
\pageno=-1
\ptxfile{test.usfm}
\bye
