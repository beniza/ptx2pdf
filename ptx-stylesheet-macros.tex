%% Macros to read Paratext stylesheets and define TeX control sequences
%% for all markers present in the file

\catcode`\@=11

\let\x@=\expandafter
\TeXXeTstate=1

\def\MSG{\immediate\write16 }

\def\TRACE#1{} % default - consume messages
%\let\TRACE=\MSG % use this to echo them

\ifx\regular\undefined   \def\regular{"Times New Roman"}      \fi
\ifx\bold\undefined      \def\bold{"Times New Roman/B"}       \fi
\ifx\italic\undefined    \def\italic{"Times New Roman/I"}     \fi
\ifx\bolditalic\undefined\def\bolditalic{"Times New Roman/BI"}\fi

\def \Marker         #1\relax{\def\m@rker{#1}}
\def \Endmarker      #1\relax{\defp@ram{endmarker}{#1}}
\def \Name           #1\relax{}
\def \Description    #1\relax{}
\def \OccursUnder    #1\relax{}
\let \Occursunder\OccursUnder
\def \Rank           #1\relax{}
\def \TextType       #1\relax{\lowercase{\defp@ram{type}{#1}}}
\let \Texttype\TextType
\def \TextProperties #1\relax{\lowercase{\defp@ram{properties}{#1}}}
\let \Textproperties\TextProperties
\def \StyleType      #1\relax{\lowercase{\def\styl@type{#1}}\m@kestyle}
\let \Styletype=\StyleType
\def \FontSize       #1\relax{\defp@ram{fontsize}{#1}}
\let \Fontsize\FontSize
\def \FontName       #1\relax{\defp@ram{fontname}{#1}}
\let \Fontname\FontName
\def \FirstLineIndent#1\relax{\defp@ram{firstindent}{#1}}
\let \Firstlineindent\FirstLineIndent
\def \LeftMargin     #1\relax{\defp@ram{leftmargin}{#1}}
\let \Leftmargin\LeftMargin
\def \RightMargin    #1\relax{\defp@ram{rightmargin}{#1}}
\let \Rightmargin\RightMargin
\def \Italic         {\defp@ram{italic}{true}}
\let \italic\Italic
\def \Bold           {\defp@ram{bold}{true}}
\let \bold\Bold
\def \Superscript    {\defp@ram{superscript}{true}}
\let \superscript\Superscript
\def \Underline      {}
\let \underline\Underline
\def \NotRepeatable  {}
\let \Notrepeatable\NotRepeatable
\def \SpaceBefore    #1\relax{\defp@ram{spacebefore}{#1}}
\let \Spacebefore\SpaceBefore
\def \SpaceAfter     #1\relax{\defp@ram{spaceafter}{#1}}
\let \Spaceafter\SpaceAfter
\def \Color          #1\relax{}
\let \color\Color
\def \Justification  #1\relax{\lowercase{\defp@ram{justification}{#1}}}
\def \CallerStyle    #1\relax{\defp@ram{callerstyle}{#1}}

\def\defp@ram#1#2{\x@\def\csname\m@rker:#1\endcsname{#2}}
\def\getp@ram#1#2{\edef\p@ram{\csname#2:#1\endcsname}}

\def\defp@rstyle#1{\x@\def\csname#1\endcsname{\p@rstyle{#1}}}
\def\defch@rstyle#1{\x@\def\csname#1\endcsname{\ch@rstyle{#1}}}
\def\defn@testyle#1{\x@\def\csname#1\endcsname{\n@testyle{#1}}\m@ken@tecl@ss{#1}}

\def\m@ken@tecl@ss#1{%
  \x@\n@tecl@sses\x@{\the\n@tecl@sses \\{#1}}%
  \x@\n@winsert\csname note-#1\endcsname
  \s@tn@tep@rams{#1}%
}
\let\n@winsert=\newinsert
\newtoks\n@tecl@sses

\def\s@tn@tep@rams#1{%
  \x@\count\csname note-#1\endcsname=1000
  \x@\skip\csname note-#1\endcsname=\AboveNoteSpace
  \x@\dimen\csname note-#1\endcsname=\maxdimen
}
\newdimen\AboveNoteSpace \AboveNoteSpace=\medskipamount

\def\m@kestyle{{\uccode`\|=`\\\uppercase{\message{|\m@rker}}}
 \ifx\styl@type\P@ra \x@\defp@rstyle\x@{\m@rker}
 \else\ifx\styl@type\Ch@r \x@\defch@rstyle\x@{\m@rker}
 \else\ifx\styl@type\N@te \x@\defn@testyle\x@{\m@rker}
 \else \message{unknown style type \styl@type}
 \fi\fi\fi}
\def\P@ra{paragraph}
\def\Ch@r{character}
\def\N@te{note}

\newif\if@ntromarker
\newif\if@ntro \newif\ift@tle \newif\ifb@dy
\newif\ifd@ublecols \d@ublecolsfalse
\def\t@stintro#1#2\relax{\if#1i\global\@ntromarkertrue\else\global\@ntromarkerfalse\fi}
\def\st@rtintro{\TRACE{st@rtintro}\if@ntro\else
  \ifnum\IntroColumns=\c@rrentcols\else
    \ifnum\IntroColumns=2 \doublecolumns\else\singlecolumn\fi
  \fi
  \global\@ntrotrue\global\t@tlefalse\global\b@dyfalse\fi}
\def\st@rttitle{\TRACE{st@rttitle}\ift@tle\else
  \ifnum\TitleColumns=\c@rrentcols\else
    \ifnum\TitleColumns=2 \doublecolumns\else\singlecolumn\fi
  \fi
  \global\t@tletrue\global\@ntrofalse\global\b@dyfalse\fi}
\def\st@rtbody{\ifb@dy\else\TRACE{st@rtbody}%
  \ifnum\BodyColumns=\c@rrentcols
    \penalty-200\vskip\baselineskip
  \else
    \ifnum\BodyColumns=2 \doublecolumns\else\singlecolumn\fi
  \fi
  \global\b@dytrue\global\@ntrofalse\global\t@tlefalse\fi}
\newcount\c@rrentcols \global\c@rrentcols=1

\newif\ifhe@dings
\newbox\he@dingbox
\def\endhe@dings{\ifhe@dings\TRACE{endhe@dings}%
  \makecutouts \endgraf
  \egroup \cutoutcarryover
  \gridb@x\he@dingbox
  \global\he@dingsfalse
 \fi
}

\def\gridb@x#1{%
% \setbox0=\hbox{\kern-#1pt\vrule\kern#1pt\kern-.4pt\box#1\kern-.4pt\vrule}%
%\MSG{before gridbox: ht0=\the\ht0; dp0=\the\dp0}
 \setbox0=\box#1%
%{\showboxbreadth=100 \showboxdepth=10 \showbox0 }%
 \dimen4=\dp0
 \dimen2=\ht0 \advance\dimen2 by \dp0
 \dimen0=0\baselineskip
 \loop \ifdim\dimen0<\dimen2
   \advance\dimen0 by \baselineskip
%   \line{\kern-5pt\special{color push rgb 1 0 0}.\hfil\special{color pop}}\nobreak \repeat
   \line{}\nobreak \repeat
% \line{\kern-5pt\special{color push rgb 0 1 0}.\hfil\special{color pop}}\nobreak
 \line{}\nobreak
 \setbox0=\vbox to 0pt{\kern-\ht0\unvbox0}
 \dp0=\dimen4
%\MSG{ after gridbox: dimen0=\the\dimen0; dp0=\the\dp0}
 \unvbox0 \nobreak
}

\def\par{\ifhmode \TRACE{par}%
 \makecutouts
 \csname end-\m@rker\endcsname
 \endgraf \cutoutcarryover \fi}

\newif\ifhe@dingstyle
\def\p@rstyle#1{\TRACE{p@rstyle:#1}%
 \ifsk@pping \egroup \fi
 \t@stpublishability{#1}%
 \ifn@npublishable
  \setbox\j@nkbox=\vbox\bgroup \sk@ppingtrue
 \else
  \ifhmode\unskip\fi
  \end@llcharstyles
  \par
  \getp@ram{spaceafter}{\m@rker}%
  \x@\ifx\p@ram\relax \else
    \ifhe@dings\nobreak\fi \vskip\p@ram \VerticalSpaceUnit
  \fi
  \resetp@rstyle
  \csname after-\m@rker\endcsname
  \gdef\m@rker{#1}
  \csname before-\m@rker\endcsname
  \x@\t@stintro\m@rker\relax
  \global\he@dingstylefalse
  \getp@ram{type}{\m@rker}\ifx\p@ram\t@tle \mark{\t@tle}\fi
  \ifx\p@ram\t@tle
    \ift@tle\else\ifhe@dings\endhe@dings\fi\fi
    \st@rttitle
    \global\he@dingstyletrue 
  \else\ifx\p@ram\s@ction
    \ift@tle\ifhe@dings\endhe@dings\fi\fi
    \if@ntromarker\st@rtintro\else\st@rtbody\fi
    \global\he@dingstyletrue
  \else\ifx\p@ram\oth@r
    \ift@tle\ifhe@dings\endhe@dings\fi\fi
    \if@ntromarker\st@rtintro\else\st@rtbody\fi
    \if@ntromarker\else\global\he@dingstyletrue\fi
  \else
    \if@ntromarker\st@rtintro\else\st@rtbody\fi
  \fi\fi\fi
  \ifhe@dingstyle\TRACE{headingstyle true}%
    \ifhe@dings\else
      \setbox\he@dingbox=\vbox\bgroup\global\he@dingstrue
        \linepenalty=1000
    \fi
    \nobreak
  \else\TRACE{headingstyle false}%
    \ifhe@dings\endhe@dings\fi
  \fi
  \getp@ram{spacebefore}{\m@rker}%
  \x@\ifx\p@ram\relax \else
    \ifhe@dings\nobreak\fi \vskip\p@ram \VerticalSpaceUnit
  \fi
  \everypar={%
   \ifRTL\beginR\fi
   \getp@ram{justification}{\m@rker}%
   \ifx\p@ram\c@nter
    \leftskip=0pt plus \hsize \rightskip=\leftskip \parfillskip=0pt
   \else\ifx\p@ram\l@ft
    \rightskip=0pt plus \textwidth
   \else\ifx\p@ram\r@ght
    \leftskip=0pt plus \textwidth \parfillskip=0pt
   \fi\fi\fi
   \getp@ram{leftmargin}{\m@rker}%
   \x@\ifx\p@ram\relax \else \advance \ifRTL\rightskip\else\leftskip\fi\p@ram \IndentUnit \fi
   \getp@ram{rightmargin}{\m@rker}%
   \x@\ifx\p@ram\relax \else \advance \ifRTL\leftskip\else\rightskip\fi\p@ram \IndentUnit \fi
   \s@tfont{\m@rker}%
   \allowp@rindenttrue
   \c@ncelfirstversefalse
   \ifch@pter
    \getp@ram{type}{\m@rker}%
    \ifx\p@ram\v@rsetext
     \m@kechapterbox
     \kern-\wd\ch@pterbox\box\ch@pterbox
     \global\ch@pterfalse
     \ifIndentAtChapter\else \allowp@rindentfalse \fi
     \c@ncelfirstversetrue
     \ifx\XeTeXversion\undefined \else
      \special{pdf:dest (\b@ok.\ch@pter) [@thispage /Fit]}%
      \special{pdf:outline 0 << /Title (\b@ok\space\ch@pter)
               /A << /S /GoTo /D (\b@ok.\ch@pter) >> >>}%
     \fi
    \fi
   \fi
   \ifallowp@rindent
    \getp@ram{firstindent}{\m@rker}%
    \x@\ifx\p@ram\relax \else \kern\p@ram \IndentUnit \fi
   \fi
   \global\p@ranesting=1
   \csname start-\m@rker\endcsname
   \ifc@ncelfirstverse \spacefactor=0\n@wchaptersf \fi
  }%
 \fi}
\newif\ifRTL
\newif\ifallowp@rindent
\newif\ifc@ncelfirstverse
\newif\ifIndentAtChapter
\newif\ifOmitVerseNumberOne \OmitVerseNumberOnefalse
\def\n@wchaptersf{998}

\newif\ifsk@pping \newif\ifn@npublishable \newbox\j@nkbox
\def\t@stpublishability#1{\n@npublishablefalse
 \getp@ram{properties}{#1}%
 \x@\t@stnonpub\p@ram nonpublishable!}
\def\t@stnonpub #1nonpublishable#2!{%
 \def\t@st{#2}\ifx\t@st\empty\else\n@npublishabletrue\fi}

\def\m@kechapterbox{%
 \setbox\ch@pterbox=\hbox{\s@tfont{c}\ch@pter}%
 \setbox\ch@pterbox=\hbox{\lower\baselineskip\box\ch@pterbox
   \box\ch@pternote\kern.2em}%
 \ifRTL\let\c@tcmd=\rightcutout\else\let\c@tcmd=\leftcutout\fi
 \x@\let\x@\d@lay\csname delay-\id@@@.\ch@pter\endcsname
 \x@\c@tcmd\x@{\the\wd\ch@pterbox}{0\d@lay}{2}%
 \dp\ch@pterbox=0pt
 \setbox\ch@pterbox=\hbox{\ifRTL\kern\rightskip\fi
   \box\ch@pterbox
   \ifRTL\else\kern\leftskip\fi}%
}
\newbox\ch@pterbox
\newbox\ch@pternote
\def\DelayChapter#1#2#3{\uppercase{\def\ucb@@k{#1}}%
  \x@\def\csname delay-\ucb@@k.#2\endcsname{#3}}

\def\v@rsetext{versetext}
\def\c@nter{center}
\def\l@ft{left}
\def\r@ght{right}
\def\t@tle{title}
\def\s@ction{section}
\def\oth@r{other}
\def\resetp@rstyle{%
 \leftskip=0pt \rightskip=\leftskip
 \parfillskip=0pt plus 1fil
 \parindent=0pt }

\def\ch@rstyle#1{\TRACE{ch@rstyle:#1}%
 \def\newch@rstyle{#1}%
 \catcode32=12\relax
 \futurelet\n@xt\doch@rstyle}
\def\doch@rstyle{\catcode32=10\relax
 \if\n@xt*\let\n@xt\endch@rstyle\else\let\n@xt\startch@rstyle\fi
 \n@xt}

\catcode`\~=12 \lccode`\~=32 % we'll use \lowercase{~} when we need a cat-12 space
\lowercase{
 \def\startch@rstyle~{\TRACE{startch@rstyle}%
  \csname before-\newch@rstyle\endcsname
  \bgroup
   \let\thisch@rstyle=\newch@rstyle
   \s@tfont{\thisch@rstyle}%
   \ifnum\n@tenesting>0 \global\advance\n@tenesting by 1
   \else \global\advance\p@ranesting by 1 \fi
   \csname start-\thisch@rstyle\endcsname}
 \def\endch@rstyle*{\TRACE{endch@rstyle}%
   \csname end-\thisch@rstyle\endcsname
   \x@\global\x@\let\x@\n@xt
    \csname after-thisch@rstyle\endcsname
   \ifnum\n@tenesting>0 \global\advance\n@tenesting by -1
   \else \global\advance\p@ranesting by -1 \fi
  \egroup
  \n@xt}
}
\newcount\n@tenesting \newcount\p@ranesting
\def\sethook#1#2#3{\x@\def\csname #1-#2\endcsname{#3}}

\def\end@llcharstyles{%
 \ifnum\n@tenesting>0 \doendch@rstyles\n@tenesting
  \else \doendch@rstyles\p@ranesting \fi}
\def\doendch@rstyles#1{\@LOOP \ifnum#1>1 \endch@rstyle*\@REPEAT}
\def\@LOOP #1\@REPEAT{\gdef\@BODY{#1}\@ITERATE}
\def\@ITERATE{\@BODY \global\let\@NEXT\@ITERATE
 \else \global\let\@NEXT\relax \fi \@NEXT}

\def\n@testyle#1{\TRACE{n@testyle:#1}%
 \def\newn@testyle{#1}%
 \catcode32=12\relax
 \futurelet\n@xt\don@testyle}
\def\don@testyle{\catcode32=10\relax 
 \if\n@xt*\let\n@xt\endn@testyle\else\let\n@xt\startn@testyle\fi
 \n@xt}

\lowercase{
 \def\startn@testyle~#1 {\TRACE{startn@testyle}%
  \def\t@st{#1}%
  \ifx\t@st\pl@s
    \inc@utonum{\newn@testyle}%
    \x@\gen@utonum\x@{\newn@testyle}%
  \else\ifx\t@st\min@s \def\them@rk{}%
  \else \def\them@rk{#1}%
  \fi\fi
  \begingroup\resetp@rstyle
  \m@kenote{\newn@testyle}{\everypar={}\cancelcutouts
   \getp@ram{callerstyle}{\newn@testyle}%
   \x@\ifx\p@ram\relax\def\c@llerstyle{v}\else\edef\c@llerstyle{\p@ram}\fi
   \hbox{\h@ndlesuperscript{\c@llerstyle}{\s@tfont{\c@llerstyle}\them@rk}}}\bgroup
  \global\n@tenesting=1\relax
  \csname start-\newn@testyle\endcsname
  \ignorespaces
 }
}

\def\endn@testyle*{\TRACE{endn@testyle}%
 \end@llcharstyles
 \csname end-\newn@testyle\endcsname
 \egroup
 \global\n@tenesting=0 \endgroup}

\def\inc@utonum#1{\count255=0\csname autonum #1\endcsname
 \advance\count255 by 1 \x@\xdef\csname autonum #1\endcsname{\number\count255}}

%% overridden by ptx-callers.tex
\ifx\gen@utonum\undefined
 \def\gen@utonum#1{%
  \count255=0\csname autonum #1\endcsname
  \loop \ifnum\count255>26 \advance\count255 by -26 \repeat
  \advance\count255 by 96 \edef\them@rk{\char\count255}}
\fi

\def\pl@s{+}
\def\min@s{-}
\def\xr@fstyle{x}
\def\fnc@ller#1{\s@tfont{v}#1}

\def\wlog#1{\immediate\write-1{#1}}
\def\s@tfont#1{%
 \x@\ifx\csname font<#1>\endcsname \relax
  \let\typef@ce=\regular
  \getp@ram{fontname}{#1}%
  \x@\ifx\p@ram\relax
	  \getp@ram{superscript}{#1}%
	  \dimen0=\ifx\p@ram\tru@ \SuperscriptFactor\fi\FontSizeUnit
	  \getp@ram{bold}{#1}%
	  \ifx\p@ram\tru@
		\let\typef@ce=\bold
		\getp@ram{italic}{#1}%
		\ifx\p@ram\tru@ \let\typef@ce=\bolditalic \fi
	  \else
		\getp@ram{italic}{#1}%
		\ifx\p@ram\tru@ \let\typef@ce=\italic \fi
	  \fi
  \else
   \edef\typef@ce{"\p@ram"}%
  \fi
  \getp@ram{fontsize}{#1}%
  \x@\global\x@\font
    \csname font<#1>\endcsname=\typef@ce\space at \p@ram \dimen0
 \fi
 \csname font<#1>\endcsname
}
\def\tru@{true}
\def\SuperscriptFactor{0.75}

\def\h@ndlesuperscript#1#2{%
  \getp@ram{superscript}{#1}%
    \ifx\p@ram\tru@
      \setbox0=\hbox{#2}%
      \raise.75ex\box0
    \else\hbox{#2}\fi
}

\lowercase{
 \def\@ddcvhooks{
  \let\@V=\v
  \def\@v@ ##1 {\gdef\v@rse{##1}%
   \x@\spl@tverses\v@rse--\relax
   \c@ncelfirstversefalse
   \ifOmitVerseNumberOne \ifnum\spacefactor=998 \c@ncelfirstversetrue \fi \fi
   \ifc@ncelfirstverse\else \h@ndlesuperscript{v}{\@V~\printv@rse\@V*}\kern.2em \fi
   \egroup
   \m@rkverse
   \the\v@rsehooks
   \gdef\reference{\ch@pter:\v@rse}}
  \def\v{\csname before-v\endcsname
   \leavevmode
   \ifdim\lastkern=-1sp \let\ll@p=\llap
   \else \let\ll@p=\relax \fi % hanging verse number?
   \ll@p\bgroup\m@kedigitsother\@v@}
  \let\@C=\c
  \def\@c@ ##1 {\ifsk@pping \egroup \fi
   \gdef\ch@pter{##1}\gdef\v@rse{}\m@kedigitsletters
   \ifOmitChapterNumber\else
    \ifx\ch@plabel\empty \global\ch@ptertrue 
    \else \p@rstyle{cl}\ch@plabel\ \ch@pter\fi
   \fi}
  \def\c{\m@kedigitsother\@c@}
  \ifnum\dropnumbersize=0
    \getp@ram{fontsize}{p}\count255=\p@ram \multiply\count255 by 2
    \getp@ram{fontsize}{c}\dropnumbersize=\p@ram
    \ifnum\dropnumbersize<\count255
      \s@tfont{c}\setbox0=\hbox{0123456789}
      \x@\let\csname font<c>\endcsname=\relax
      \s@tfont{p}\dimen0=\fontdimen5\font
      \advance\dimen0 by \baselineskip
      \multiply\dimen0 by 100 \divide\dimen0 by \ht0
      \multiply\dimen0 by \dropnumbersize \divide\dimen0 by 100
      \dropnumbersize=\dimen0
    \fi
  \fi
  \def\m@rker{c}\defp@ram{fontsize}{\the\dropnumbersize}
 }
}
\newif\ifOmitChapterNumber
\def\printv@rse{\v@rsefrom\ifx\v@rsefrom\v@rseto\else\endash\v@rseto\fi}
\def\hangversenumber{\kern-1sp\relax}
\def\@ne{1}

\newcount\dropnumbersize

\newtoks\v@rsehooks
\def\addtoversehooks#1{\x@\v@rsehooks\x@{\the\v@rsehooks #1}}

\newif\ifch@pter \def\ch@plabel{}
\def\m@kedigitsletters{\catcode`0=\el@ven \catcode`1=\el@ven \catcode`2=\el@ven \catcode`3=\el@ven
 \catcode`4=\el@ven \catcode`5=\el@ven \catcode`6=\el@ven \catcode`7=\el@ven
 \catcode`8=\el@ven \catcode`9=\el@ven \relax}
\def\m@kedigitsother{\catcode`0=\tw@lve \catcode`1=\tw@lve \catcode`2=\tw@lve \catcode`3=\tw@lve
 \catcode`4=\tw@lve \catcode`5=\tw@lve \catcode`6=\tw@lve \catcode`7=\tw@lve
 \catcode`8=\tw@lve \catcode`9=\tw@lve \relax}
\def\el@ven{11}
\def\tw@lve{12}

\def\b@ok{BOOK}
\def\ch@pter{}
\def\v@rse{}
\def\m@rkverse{\mark{\b@ok:\ch@pter:\v@rse}}

\begingroup
\obeylines%
\gdef\@ddspecialhooks{%
 \let\@H=\h%
 \def\h{\bgroup\obeylines\@h}%
 \def\@h ##1^^M{\gdef\b@ok{##1}\egroup}%
 \let\@CL=\cl%
 \def\cl{\bgroup\obeylines\@cl}%
 \def\@cl ##1^^M{\gdef\ch@plabel{##1}\egroup}%
 \let\@ID=\id%
 \def\id{\bgroup\obeylines\@id}%
 \def\@id ##1^^M{\gdef\c@rrID{##1}\uppercase{\@@id##1ZZZ\end}\egroup}%
 \let\fig=\@fig%
}%
\endgroup
\def\@@id#1#2#3#4\end{\gdef\id@@@{#1#2#3}}
\def\@fig #1\fig*{\d@figure{#1}}

\def\f@rstref#1{\edef\t@st{#1}\x@\extr@ctfirst\t@st\relax
 \ifx\@book\empty\else\@book\ \@chapter\ifVerseRefs\cvs@p\@verse\fi\fi}
\def\l@stref#1{\edef\t@st{#1}\x@\extr@ctlast\t@st\relax
 \ifx\@@book\empty\else\@@book\ \@@chapter\ifVerseRefs\cvs@p\@@verse\fi\fi}
\def\r@ngerefs#1#2{%
 \edef\t@st{#1}\x@\extr@ctfirst\t@st\relax
 \edef\t@st{#2}\x@\extr@ctlast\t@st\relax
 \ifx\@book\empty\else
 \@book\ 
 \ifVerseRefs
  \ifx\@chapter\empty\else\@chapter\cvs@p\fi\@verse
  \ifx\@book\@@book
   \ifx\@chapter\@@chapter
    \ifx\@verse\@@verse\else\ranges@p\@@verse\fi
   \else\ranges@p\@@chapter\cvs@p\@@verse\fi
  \else\ranges@p\@@book\ \@@chapter\cvs@p\@@verse\fi
 \else
  \ifx\@chapter\empty\else
   \@chapter
   \ifx\@chapter\@@chapter\else
    \setbox0=\hbox{\tracinglostchars=0
     \global\c@untA=0\@chapter \global\c@untB=0\@@chapter}%
    \advance\c@untA by 1
    \ifnum\c@untA=\c@untB \pairs@p \else \ranges@p \fi
    \@@chapter
   \fi
  \fi
 \fi\fi}
\newcount\c@untA \newcount\c@untB
\newif\ifVerseRefs
\def\ranges@p{\kern.1em\endash\kern.1em}
%\def\pairs@p{,\kern.2em}
\let\pairs@p\ranges@p
\def\cvs@p{\hbox{:}}
\def\endash{\char"2013\relax}

\def\extr@ctfirst#1:#2:#3\relax{%
 \def\@book{#1}\def\@chapter{#2}\def\t@st{#3}%
 \x@\spl@tverses\t@st--\relax
 \edef\@verse{\v@rsefrom}}
\def\extr@ctlast#1:#2:#3\relax{%
 \def\@@book{#1}\def\@@chapter{#2}\def\t@st{#3}%
 \x@\spl@tverses\t@st--\relax
 \edef\@@verse{\v@rseto}}
\def\spl@tverses#1-#2-#3\relax{%
 \edef\v@rsefrom{#1}\edef\v@rseto{#2}%
 \ifx\v@rseto\empty\let\v@rseto=\v@rsefrom\fi}

\newif\ifc@ntinue \c@ntinuetrue
\newread\styl@sheet
\def\stylesheet#1{
 \openin\styl@sheet=#1
 \ifeof\styl@sheet\message{Paratext stylesheet #1 not found!}\else
   \message{Reading Paratext stylesheet #1...}%
   \endlinechar=-1
   \catcode`\#=5 % paratext comment char
   \loop
    \read\styl@sheet to \th@line
    \th@line \relax
    \ifeof\styl@sheet \c@ntinuefalse \fi
    \ifc@ntinue\repeat
   \endlinechar=13
   \s@tupsizes
   \@ddcvhooks
   \@ddspecialhooks
   \catcode`\#=6 % restore default TeX catcode
 \fi}

\def\ptxfile#1{
 \openadjlist "#1.adj"
 \openpiclist "#1.piclist"
 %\catcode`\$=12
 \catcode`\^=12 \catcode`\_=12 % make these printable
 \catcode`\&=12 \catcode`\~=12
 \catcode`\#=12 %\catcode`\%=12
 \catcode`\{=12 \catcode`\}=12
 \catcode`\/=\active
 \catcode13=10
 \m@kedigitsletters
 \input "#1" \ifsk@pping\egroup\fi
 \catcode13=5
 \closepiclist
 \closeadjlist
 \catcode`\/=12
 \catcode`\#=6 \catcode`\%=14 % restore TeX meanings for those we might use
 \catcode`\{=1 \catcode`\}=2
 \singlecolumn
 \pagebreak}

\catcode`\/=\active
\def/{\futurelet\n@xt\sl@sh}
\def\sl@sh{\ifx\n@xt/\let\n@xt\sl@shbreak\else\let\n@xt\sl@shprint\fi\n@xt}
\def\sl@shbreak/{\penalty-250\ }
\def\sl@shprint{\char`\/}
\catcode`\/=12

\newtoks\oddhead
\newtoks\evenhead
\newtoks\titlehead
\headline={{%
 \pl@ceborder
 \s@tfont{h}%
 \edef\t@st{\p@gefirstmark}%
 \global\rhr@letrue
 \ifx\t@st\empty \hfil \global\rhr@lefalse \else
  \ifx\t@st\t@tle \the\titlehead \global\rhr@lefalse \else
  \ifodd\pageno \the\oddhead \else \the\evenhead
  \fi\fi\fi}}
\oddhead={\rlap{\RHoddleft}\hfil\RHoddcenter\hfil\llap{\RHoddright}}
\evenhead={\rlap{\RHevenleft}\hfil\RHevencenter\hfil\llap{\RHevenright}}
\titlehead={\rlap{\RHtitleleft}\hfil\RHtitlecenter\hfil\llap{\RHtitleright}}

\newtoks\oddfoot
\newtoks\evenfoot
\newtoks\titlefoot
\footline={{%
 \s@tfont{h}%
 \edef\t@st{\p@gefirstmark}%
 \ifx\t@st\empty \hfil \else
  \ifx\t@st\t@tle \the\titlefoot \else
  \ifodd\pageno \the\oddfoot \else \the\evenfoot
  \fi\fi\fi}}
\oddfoot={\rlap{\RFoddleft}\hfil\RFoddcenter\hfil\llap{\RFoddright}}
\evenfoot={\rlap{\RFevenleft}\hfil\RFevencenter\hfil\llap{\RFevenright}}
\titlefoot={\rlap{\RHtitleleft}\hfil\RFtitlecenter\hfil\llap{\RHtitleright}}

\def\firstref{\ifRTL\beginR\fi\f@rstref{\p@gefirstmark}\ifRTL\endR\fi}
\def\lastref{\ifRTL\beginR\fi\l@stref{\p@gebotmark}\ifRTL\endR\fi}
\def\rangeref{\ifRTL\beginR\fi\r@ngerefs{\p@gefirstmark}{\p@gebotmark}\ifRTL\endR\fi}
\let\pagenumber=\folio

\def\RHoddleft{\empty}
\def\RHoddcenter{\rangeref}
\def\RHoddright{\pagenumber}

\def\RHevenleft{\pagenumber}
\def\RHevencenter{\rangeref}
\def\RHevenright{\empty}

\def\RHtitleleft{\empty}
\def\RHtitlecenter{\empty}
\def\RHtitleright{\empty}

\def\RFoddleft{\empty}
\def\RFoddcenter{\empty}
\def\RFoddright{\empty}

\def\RFevenleft{\empty}
\def\RFevencenter{\empty}
\def\RFevenright{\empty}

\def\RFtitleleft{\empty}
\def\RFtitlecenter{\pagenumber}
\def\RFtitleright{\empty}

\def\PageBorder{}
\newbox\b@rder
\def\pl@ceborder{\ifx\PageBorder\empty\else
  \ifvoid\b@rder
    \global\setbox\b@rder=\hbox{\XeTeXpdffile \PageBorder \relax}%
    \global\setbox\b@rder=\vbox to 0pt{\vss\hbox to \textwidth{\hss\box\b@rder\hss}\vss}%
  \fi
  \hbox to 0pt{\vbox to 0pt{\kern0.5\textheight
    \kern-0.5\baselineskip\kern12pt\copy\b@rder\vss}\hss}%
 \fi}

\newdimen\FontSizeUnit		\FontSizeUnit=1bp
\newdimen\IndentUnit		\IndentUnit=1in
\newdimen\PaperWidth		\PaperWidth=210mm
\newdimen\PaperHeight		\PaperHeight=297mm
\newdimen\MarginUnit		\MarginUnit=1in
\newdimen\LeadingUnit		\LeadingUnit=1bp
\newdimen\VerticalSpaceUnit	\VerticalSpaceUnit=0.5bp
\newdimen\BindingGutter		\BindingGutter=5mm
\def\LineSpacingFactor{1.0}
\def\VerticalSpaceFactor{1}
\def\TopMarginFactor{1.0}
\def\SideMarginFactor{1.0}
\def\ColumnGutterFactor{10}

\newcount\TitleColumns
\newcount\IntroColumns
\newcount\BodyColumns

\newif\ifGutter
\newif\ifDoubleSided

\def\s@tupsizes{
 \LeadingUnit=\LineSpacingFactor\FontSizeUnit
 \VerticalSpaceUnit=\VerticalSpaceFactor\LeadingUnit
 \baselineskip=14\LeadingUnit
 \topskip=10\LeadingUnit
 \gutter=\ColumnGutterFactor\FontSizeUnit
 \dimen0=\PaperWidth \dimen2=\SideMarginFactor\MarginUnit
 \advance\dimen0 by -2\dimen2
 \ifGutter \advance\dimen0 by -\BindingGutter \fi
 \textwidth=\dimen0
 \colwidth=0.5\textwidth \advance\colwidth by -0.5\gutter
 \hsize=\textwidth
 \dimen0=\PaperHeight
 \advance\dimen0 by -\TopMarginFactor\MarginUnit
 \advance\dimen0 by -\TopMarginFactor\MarginUnit
 \textheight=\dimen0
 \vsize=\textheight
 \dimen0=\SideMarginFactor\MarginUnit \advance\dimen0 by -1in
 \hoffset=\dimen0
 \dimen0=\TopMarginFactor\MarginUnit \advance\dimen0 by -1in
 \advance\dimen0 by 0.5\baselineskip % nudge down a little for header
 \voffset=\dimen0
 \pdfpagewidth=\PaperWidth
 \pdfpageheight=\PaperHeight
 \resetvsize
}
\newdimen\textwidth
\newdimen\textheight
\newdimen\colwidth
\newdimen\gutter \gutter=20pt

\catcode`\@=12

\endinput
