
\input paratext2.tex

%%%%% Stylesheets %%%%%
\stylesheet{{{/ptxpath}/usfm.sty}}
\stylesheet{{{/ptxpath}/{project/id}/custom.sty}}
%\stylesheet{{NestedStyles.sty}}

\stylesheet{{{/ptxpath}/{project/id}/PrintDraft/PrintDraft-mods.sty}}% override project stylesheet settings here
{document/ifomitallverses}\stylesheet{{D:/Temp/NoVerseNums-mods.sty}}

%%%%% Page Setup %%%%%
% Dimensions
\PaperWidth={paper/width}
\PaperHeight={paper/height}
\CropMarks{paper/ifcropmarks}

%% Margins
%\BindingGutter= mm % paper/gutter mm
\MarginUnit={paper/margins}mm
\def\TopMarginFactor{{{paper/topmarginfactor}}}
\def\BottomMarginFactor{{{paper/bottommarginfactor}}}
%\def\SideMarginFactor{{1.0}} % Not yet sure why we would want this anything other than 1.0

%% Columns
\TitleColumns=1
\IntroColumns=1
\BodyColumns={paper/columns}
\ColumnGutterRule{paper/ifverticalrule}
\def\ColumnGutterFactor{{{document/colgutterfactor}}}

%%%%% Text Spacing %%%%%
\def\LineSpacingFactor{{{paragraph/linespacing}}}
\def\VerticalSpaceFactor{{1.0}}

%% Cross-space contextualization
% None=0 This is how XeTeX behaves by default. Most projects will use this setting.
% Some=1 Spaces between words are adjusted, but the rendering of individual words is *NOT* affected by the spaces.
% Full=2 Spaces between words are adjusted, and the rendering of individual words *IS* affected by the spaces.
\XeTeXinterwordspaceshaping = {document/crossspacecntxt}

%%%%% Fonts %%%%%
\FontSizeUnit={paper/fontfactor:.6f}pt   % Q (from MP to MH): Can you explain what's happening here?

%% Faces
%\def\regular{{"Padauk Book/GR:;script=mymr;mapping=burmesedigits"}}
%\def\bold{{"Padauk Book/GR/B:script=mymr;mapping=burmesedigits"}}
%\def\italic{{"Padauk/GR/I:script=mymr;mapping=burmesedigits"}}
%\def\bolditalic{{"Padauk/GR/BI:script=mymr;mapping=burmesedigits"}}

%\def\regular{{"{fontregular/name}{document/digitmapping}"}}
%\def\bold{{"{fontbold/name}:{document/digitmapping}{fontbold/embolden}{fontbold/slant}"}}
%\def\italic{{"{fontitalic/name}:{document/digitmapping}{fontitalic/embolden}{fontitalic/slant}"}}
%\def\bolditalic{{"{fontbolditalic/name}:{document/digitmapping}{fontbolditalic/embolden}{fontbolditalic/slant}"}}

\def\regular{{"{fontregular/name}"}}
\def\bold{{"{fontbold/name}:{fontbold/embolden}{fontbold/slant}"}}
\def\italic{{"{fontitalic/name}:{fontitalic/embolden}{fontitalic/slant}"}}
\def\bolditalic{{"{fontbolditalic/name}:{fontbolditalic/embolden}{fontbolditalic/slant}"}}

%%%%% Layout %%%%%
\RTL{document/ifrtl}   % Use right-to-left layout mode?
% Specify ISO 639-1 language code.
% XeTeX will use ICU line break info (useful for scripts which do not use word spaces).
{document/iflinebreakon}\XeTeXlinebreaklocale "{document/linebreaklocale}"
% Introduce some stretchability (glue) at each potential break
%\XeTeXlinebreakskip=0em plus 0.1em minus 0.01em
\JustifyPars{document/ifjustify}
\IndentAfterHeading{document/supressindent}

%%% Chapters & Verses %%%
\OmitVerseNumberOne{document/ifomitverseone}
\OmitChapterNumber{document/ifomitchapternum}
{document/ifomitallchapters}\def\AfterChapterSpaceFactor{{10}}
{document/ifomitallverses}\def\AfterVerseSpaceFactor{{0}}

%%% Poetry - Hang Verse Numbers
{document/hangpoetry}\sethook{{start}}{{q1}}{{\hangversenumber}}
{document/hangpoetry}\sethook{{start}}{{q2}}{{\hangversenumber}}


%%%%% Notes %%%%%

%% Disable footnote rule (if next line is commented out then the footnote rule will show and vice versa)
{notes/ifomitfootnoterule}\def\footnoterule{{}} 

%\AutoCallerStartChar=97
%\AutoCallerNumChars=26
%\AutoCallers{{f}}{{*,+,¶,§,**,++,¶¶,§§}}
%\AutoCallers{{f}}{{a,b,c,d,e,f,g,h,i,j,k,l,m,n,o,p,q,r,s,t,u,v,w,x,y,z}}
\AutoCallers{{f}}{{{notes/fncallers}}} % to hide callers in body text this needs to be empty
{notes/fnresetcallers}\PageResetCallers{{f}}
{notes/fnomitcaller}\OmitCallerInNote{{f}}
{notes/fnparagraphednotes}\ParagraphedNotes{{f}}
%\NumericCallers{{f}}

%\AutoCallers{{x}}{{a,b,c,d,e,f,g,h,i,j,k,l,m,n,o,p,q,r,s,t,u,v,w,x,y,z}}
\AutoCallers{{x}}{{{notes/xrcallers}}} % to hide callers in body text this needs to be empty
{notes/xrresetcallers}\PageResetCallers{{x}}
{notes/xromitcaller}\OmitCallerInNote{{x}}
{notes/xrparagraphednotes}\ParagraphedNotes{{x}}
%\NumericCallers{{x}}

%%%%% Running Header/Footer %%%%%
\def\HeaderPosition{{{header/headerposition}}}
\def\FooterPosition{{{header/footerposition}}}
{header/ifrhrule}\RHruleposition={header/ruleposition}pt

%% Odd Header
\def\RHoddleft{{{header/oddleft}}}
\def\RHoddcenter{{{header/oddcenter}}}
\def\RHoddright{{{header/oddright}}}

%% Even Header
\def\RHevenleft{{{header/evenleft}}}
\def\RHevencenter{{{header/evencenter}}}
\def\RHevenright{{{header/evenright}}}

\VerseRefs{header/ifverses}                  % Include verses in reference text?
\OmitChapterNumberRH{header/ifomitrhchapnum} % Omit the chapter numbers in the running header (RH)

%% Footer
\def\RFoddcenter{{{footer/ftrcenter}}}
\def\RFevencenter{{{footer/ftrcenter}}}
\def\RFtitlecenter{{{footer/ftrcenter}}}

%%%%% Illustrations %%%%%
\IncludeFigures{document/iffigures}
\PicPath={{{/ptxpath}/{project/id}/figures/}}
%\PicPath={{{/ptxpath}/{project/id}/local/figures/}} % should check if this folder exists (with HiRes figures)
\FigurePlaceholders{document/iffigplaceholders}

% tweak to enable blank lines
\def\b{{\par\vskip\baselineskip}}
% restore page break definition (if present in USFM stylesheet, it has been re-defined and will not work)
\let\pb=\pagebreak

\input "{/ptxpath}/{project/id}/PrintDraft/PrintDraft-mods.tex"% adjust/override setup parameters here

%%%%% Hyphenation %%%%%
\hyphenpenalty=10000

{paper/watermark}\def\PageBorder{{D:/Temp/A4-Draft.pdf}}
%{paper/watermark}\def\PageBorder{{D:/Temp/A5-Draft.pdf}}

%%%%% Load Paratext USFM %%%%%
\XeTeXdefaultencoding "utf-8"
% job files. The next \ptxfile line is replaced, content is not used
\catcode`\%=12
\ptxfile{{PrintDraft/41MATprojid-draft.SFM}}
\catcode`\%=14

\end
