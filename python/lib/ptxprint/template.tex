
\input paratext2.tex

%%%%% Stylesheets %%%%%
\stylesheet{{{/ptxpath}/usfm.sty}}
\stylesheet{{{/ptxpath}/{project/id}/custom.sty}}
%\stylesheet{{NestedStyles.sty}}

\stylesheet{{{/ptxpath}/{project/id}/PrintDraft/PrintDraft-mods.sty}}% override project stylesheet settings here

%%%%% Page Setup %%%%%
% Dimensions
\PaperWidth={paper/width}
\PaperHeight={paper/height}

%% Margins
\MarginUnit={paper/margins}
\def\TopMarginFactor{{1.15}}
\def\BottomMarginFactor{{1.15}}
\def\SideMarginFactor{{1.0}}

%% Columns
\TitleColumns=1
\IntroColumns=1
\BodyColumns={paper/columns}

%%%%% Text Spacing %%%%%
\def\LineSpacingFactor{{{paragraph/linespacing}}}
\def\VerticalSpaceFactor{{1.0}}

%%%%% Fonts %%%%%
\FontSizeUnit={paper/fontfactor:.6f}pt

%% Faces
%\def\regular{{"Padauk Book/GR:;script=mymr;mapping=burmesedigits"}}
%\def\bold{{"Padauk Book/GR/B:script=mymr;mapping=burmesedigits"}}
%\def\italic{{"Padauk/GR/I:script=mymr;mapping=burmesedigits"}}
%\def\bolditalic{{"Padauk/GR/BI:script=mymr;mapping=burmesedigits"}}

\def\regular{{"{font/regular}"}}
\def\bold{{"{font/bold}"}}
\def\italic{{"{font/italic}"}}
\def\bolditalic{{"{font/bolditalic}"}}

%%%%% Layout %%%%%
\RTLfalse% Use right-to-left layout mode?
\XeTeXlinebreaklocale "my"
\JustifyParstrue

\AutoCallers{{f}}{{a,b,c,d,e,f,g,h,i,j,k,l,m,n,o,p,q,r,s,t,u,v,w,x,y,z}}
\AutoCallers{{x}}{{a,b,c,d,e,f,g,h,i,j,k,l,m,n,o,p,q,r,s,t,u,v,w,x,y,z}}


%%%%% Running Header/Footer %%%%%
\def\HeaderPosition{{0.7}}
\def\FooterPosition{{0.7}}

%% Odd Header
\def\RHoddleft{{\empty}}
\def\RHoddcenter{{\rangeref}}
\def\RHoddright{{\pagenumber}}

%% Even Header
\def\RHevenleft{{\empty}}
\def\RHevencenter{{\rangeref}}
\def\RHevenright{{\pagenumber}}

\VerseRefsfalse% include verses in reference text?

%% Footer
\def\RFoddcenter{{2019-09-04: DRAFT}}
\def\RFevencenter{{2019-09-04: DRAFT}}
\def\RFtitlecenter{{2019-09-04: DRAFT}}

%%%%% Illustrations %%%%%
\IncludeFigurestrue
\PicPath={{{/ptxpath}/{project/id}/figures/}}

% tweak to enable blank lines
\def\b{{\par\vskip\baselineskip}}
% restore page break definition (if present in USFM stylesheet, it has been re-defined and will not work)
\let\pb=\pagebreak

\input {/ptxpath}/{project/id}/PrintDraft/PrintDraft-mods.tex% adjust/override setup parameters here

%%%%% Hyphenation %%%%%
\hyphenpenalty=10000

%%%%% Load Paratext USFM %%%%%
\XeTeXdefaultencoding "utf-8"
\catcode`\%=12
\ptxfile{{/home/mhosken/Paratext8Projects/myBCBv0/PrintDraft/41MATmyBCBv0-draft.SFM}}
\catcode`\%=14

\end
