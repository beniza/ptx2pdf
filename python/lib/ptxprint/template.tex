
\input paratext2.tex

%%%%% Stylesheets %%%%%
\stylesheet{{{/ptxpath}/usfm.sty}}
{project/ifusecustomsty}\stylesheet{{{/ptxpath}/{project/id}/custom.sty}}
{project/ifusemodssty}\stylesheet{{{/ptxpath}/{project/id}/PrintDraft/PrintDraft-mods.sty}} % User can override project stylesheet settings here
{project/ifusenested}\stylesheet{{{/ptxpath}/{project/id}/PrintDraft/NestedStyles.sty}}     % PTXprint overrides all others (for versenums, ftnts, xref)

%%%%% Page Setup %%%%%
% Dimensions
\PaperWidth={paper/width}
\PaperHeight={paper/height}
\CropMarks{paper/ifcropmarks}

%% Margins
\MarginUnit={paper/margins}mm
\def\TopMarginFactor{{{paper/topmarginfactor}}}
\def\BottomMarginFactor{{{paper/bottommarginfactor}}}
\def\SideMarginFactor{{{paper/sidemarginfactor}}}
\BindingGutter{paper/ifaddgutter}
\BindingGutter={paper/gutter}mm

%% Columns
\TitleColumns=1
\IntroColumns=1
\BodyColumns={paper/columns}
\ColumnGutterRule{paper/ifverticalrule}
\def\ColumnGutterFactor{{{document/colgutterfactor}}}
% This next set of options can only be active if NOT using PT macros (as they use the new ptxplus macros)
{project/ifnotptmacros}\ColumnGutterRuleSkip={paper/colgutteroffset}pt
{project/ifnotptmacros}{document/marginalverses}\input ptxplus-marginalverses.tex
{project/ifnotptmacros}{document/marginalverses}\columnshift={document/columnshift}pt
{project/ifnotptmacros}\endbooknoeject{project/ifstarthalfpage}

%%%%% Text Spacing %%%%%
\def\LineSpacingFactor{{{paragraph/linespacingfactor}}}
\def\VerticalSpaceFactor{{1.0}}
{paragraph/useglyphmetrics}\XeTeXuseglyphmetrics=0
\baselineskip={paragraph/linespacing}pt {project/ifnotptmacros}{paragraph/varlinespacing} {paragraph/linemax} {paragraph/linemin}

%% Cross-space contextualization
\XeTeXinterwordspaceshaping = {document/spacecntxtlztn}

%%%%% Fonts %%%%%
\FontSizeUnit={paper/fontfactor}pt
%% Faces
\def\regular{{"{fontregular/name}{fontregular/engine}{font/features}{document/script}{document/digitmapping}"}}
\def\bold{{"{fontbold/name}{fontbold/engine}{font/features}{document/script}{document/digitmapping}{fontbold/embolden}{fontbold/slant}"}}
\def\italic{{"{fontitalic/name}{fontitalic/engine}{font/features}{document/script}{document/digitmapping}{fontitalic/embolden}{fontitalic/slant}"}}
\def\bolditalic{{"{fontbolditalic/name}{fontbolditalic/engine}{font/features}{document/script}{document/digitmapping}{fontbolditalic/embolden}{fontbolditalic/slant}"}}

%\extrafont  %% This will be replaced by code for the fallback fonts to be used for special/missing characters

%%%%% Layout %%%%%
{document/ifhidehboxerrors}\overfullrule=0pt
\RTL{document/ifrtl}   % Use right-to-left layout mode?
{document/toptobottom}\XeTeXupwardsmode=1     % Top-to-Bottom and Left-to-Right layout
{document/toptobottom}\rotatetrue
% Specify ISO 639-1 language code.
% XeTeX will use ICU line break info (useful for scripts which do not use word spaces).
% {document/script}

{document/iflinebreakon}\XeTeXlinebreaklocale "{document/linebreaklocale}"
% Introduce some stretchability (glue) at each potential line break
%\XeTeXlinebreakskip=0em plus 0.1em minus 0.01em
\JustifyPars{paragraph/ifjustify}
\IndentAfterHeading{document/supressindent}

%%% Chapters & Verses %%%
\OmitVerseNumberOne{document/ifomitverseone}
\OmitChapterNumber{document/ifomitchapternum}
{document/ifomitallchapters}\def\AfterChapterSpaceFactor{{10}}
{document/ifomitallverses}\def\AfterVerseSpaceFactor{{0}}

%%% Poetry - Hang Verse Numbers
{document/hangpoetry}\catcode`\@=11
{document/hangpoetry}\sethook{{start}}{{q1}}{{\let\printv@rse=\hangprintv@rse\hangversenumber}}
{document/hangpoetry}\sethook{{start}}{{q2}}{{\let\printv@rse=\hangprintv@rse\hangversenumber}}
{document/hangpoetry}\sethook{{end}}{{q1}}{{\let\printv@rse=\simpleprintv@rse}}
{document/hangpoetry}\sethook{{end}}{{q2}}{{\let\printv@rse=\simpleprintv@rse}}
{document/hangpoetry}\catcode`\@=12

%%%%% Notes %%%%%

%% Disable footnote rule (when line is commented out, the footnote rule will show)
{notes/iffootnoterule}\def\footnoterule{{}} 

%\AutoCallerStartChar=97
%\AutoCallerNumChars=26
{notes/fnresetcallers}\PageResetCallers{{f}}
{notes/fnomitcaller}\OmitCallerInNote{{f}}
{notes/fnparagraphednotes}\ParagraphedNotes{{f}}
%\NumericCallers{{f}}

% no comments inside this catcode area. Who knows what others people will want.
\catcode`\#=12 \catcode`\%=12 \catcode`\&=12 \catcode`\^=12
\AutoCallers{{f}}{{{notes/fncallers}}}
\AutoCallers{{x}}{{{notes/xrcallers}}}
\catcode`\#=6 \catcode`\%=14 \catcode`\&=4 \catcode`\^=7
%

{notes/xrresetcallers}\PageResetCallers{{x}}
{notes/xromitcaller}\OmitCallerInNote{{x}}
{notes/xrparagraphednotes}\ParagraphedNotes{{x}}
%\NumericCallers{{x}}

%%%%% Running Header/Footer %%%%%
\def\HeaderPosition{{{header/headerposition}}}
\def\FooterPosition{{{header/footerposition}}}
{header/ifrhrule}\RHruleposition={header/ruleposition}pt
\def\RangeSeparator{{\kern.1em\char"2013\kern.1em}} % what to put between first - last of a reference range
\def\ChapterVerseSeparator{{\kern.02em{header/chvseparator}\kern.02em}} % what to put between chapter:verse

%% Odd Header
\def\RHoddleft{{{header/oddleft}}}
\def\RHoddcenter{{{header/oddcenter}}}
\def\RHoddright{{{header/oddright}}}

%% Even Header
\def\RHevenleft{{{header/evenleft}}}
\def\RHevencenter{{{header/evencenter}}}
\def\RHevenright{{{header/evenright}}}

\VerseRefs{header/ifverses}                  % Include verses in reference text?
\OmitChapterNumberRH{header/ifomitrhchapnum} % Omit the chapter numbers in the running header (RH)

%% Footer
\def\RFoddcenter{{{footer/oddcenter}}}
\def\RFevencenter{{{footer/oddcenter}}}
\def\RFtitlecenter{{{footer/oddcenter}}}
{footer/ifftrtitlepagenum}\def\RFtitlecenter{{\pagenumber}}
{footer/ifprintConfigName}\def\RFtitlecenter{{{config/name}}}

%%%%% Illustration and Adjustment Locations and Lists %%%%%
\IncludeFigures{document/ifinclfigs}
\FigurePlaceholders{document/iffigplaceholders}
\PicPath={{{document/directory}/tmpPics/}}
{document/ifusepiclist}\PicListPath={{{project/piclists}}}
\AdjListPath={{{project/adjlists}}}

% tweak to enable blank lines
\def\b{{\par\vskip\baselineskip}}
% restore page break definition (if present in USFM stylesheet, it has been re-defined and will not work)
\let\pb=\pagebreak

%%%%% Diglot settings (if applicable)
%{document/diglotsettings}

%%%%% Hyphenation %%%%%
{paragraph/ifhyphenate}\input "{/ptxpath}/{project/id}/shared/ptxprint/hyphen-{project/id}.tex"
{paragraph/ifnothyphenate}\hyphenpenalty=10000
\exhyphenpenalty=2000

%%%%% all normal settings, etc. before this line %%%%%

%\snippets

% Looks in local Config folder for ptxprint-mods.tex, and drops back to shared/ptxprint .tex if not found 
{project/ifusemodstex}\input "{/modspath}"% adjust/override setup parameters here

%%%%%%%%%%%%%%%%%%%%%%%%%%%%%%%%% all settings need to come before this line %%%%%%%%%%%%%%%%%%%%%%%%%%%%%%%%%%%

%%%%% Watermark PDF file or Grid PDFs etc. %%%%%
{paper/ifwatermark}\def\PageBorder{{"{paper/watermarkpdf}"}}

%%%%% Front Matter PDF file(s) get included here %%%%%
{project/ifinclfrontpdf}{project/frontincludes}

{document/toc}\GenerateTOC[{document/toctitle}]{{{jobname}.toc}}
{document/toc}\BodyColumns=1     % Temporarily set this to Single Column for the ToC page(s)
{document/toc}\ptxfile{{{jobname}.toc}}
{document/toc}\BodyColumns={paper/columns}     % But ensure it goes back to the chosen Single/Double setting for the main body text

%%%%% Load Paratext USFM %%%%%
\XeTeXdefaultencoding "utf-8"
\def\closeadjlist{{\relax}}
\def\closepiclist{{\relax}}
% job files:
\catcode`\%=12
\ptxfile{{PrintDraft/41MATprojid-draft.SFM}}
\catcode`\%=14

%%%%% Back Matter PDF file(s) get included here %%%%%
{project/ifinclbackpdf}{project/backincludes}

\end
