%%%%% Page Setup %%%%%

%\CropMarkstrue

%% Margins
%\BindingGutter=4mm
%\BindingGuttertrue
%\def\TopMarginFactor{1.4}
%\def\BottomMarginFactor{0.8}
%\def\SideMarginFactor{0.8}

%% Columns
%\def\ColumnGutterFactor{15.0}
\ColumnGutterRuletrue

%\RHruleposition=6pt

%%%%% Text Spacing %%%%%

%\FontSizeUnit=1.40pt

%%%%% Fonts %%%%%
%% Faces
% This example shows a method for producing fake bold and italic faces.
% Replace "My Font" and "..." with real values (and remove script and/or mapping parameters as needed)
\def\regular{"Suranna:script=telu;embolden=-2.2"}
\def\bold{"Ponnala:script=telu;embolden=-2.0"}
\def\italic{"Ponnala:script=telu;slant=0.2"}
\def\bolditalic{"Ponnala:script=telu;embolden=-2.0;slant=0.2"}

%%%%% Layout %%%%%

% Specify ISO 639-1 language code.
% XeTeX will use ICU line break info (useful for scripts which do not use word spaces).
%\XeTeXlinebreaklocale ""
% Introduce some stretchability (glue) at each potential break
%\XeTeXlinebreakskip=0em plus 0.1em minus 0.01em

%%% Chapters & Verses %%%

\OmitVerseNumberOnetrue
%\def\AfterChapterSpaceFactor{10}
%\def\AfterVerseSpaceFactor{0}

%%%%% Running Header/Footer %%%%%

%\def\HeaderPosition{0.6}
%\def\FooterPosition{0.6}

% Set the items to print at left/center/right of odd and even pages separately
% Possible contents:
%   \rangeref = Scripture reference of the range of text on the page;
%   \firstref = reference of the first verse on the page
%   \lastref = reference of the last verse on the page
%   \pagenumber = the page number
%   \empty = print nothing in this position

%% Even Header
%\def\RHevenleft{\firstref}
%\def\RHevencenter{\empty}
%\def\RHevenright{\pagenumber}

%% Odd Header
%\def\RHoddleft{\pagenumber}
%\def\RHoddcenter{\empty}
%\def\RHoddright{\lastref} 

%% Footer
\def\RFoddcenter{\pagenumber}
\def\RFevencenter{\pagenumber}
\def\RFtitlecenter{\pagenumber}

%% Other Header Setup
%\RHruleposition=6pt

%%%%% Notes %%%%%

%\AutoCallerStartChar=97
%\AutoCallerNumChars=26
\AutoCallers{f}{*}
\AutoCallers{x}{+}
%\NumericCallers{f}
%\NumericCallers{x}
%\PageResetCallers{f}
%\PageResetCallers{x}
%\OmitCallerInNote{f}
\OmitCallerInNote{x}

%% Disable footnote rule
%\def\footnoterule{}

\ParagraphedNotes{x}
\ParagraphedNotes{f}

%%%%% Illustrations %%%%%

%\PicPath={c:/My Paratext Projects/<ProjectName>/figures}

%%%%% Hyphenation %%%%%

% Allocating and using a new language without any hyphenation rules defined effectively disables hyphenation.
% Hyphenation support is provided through hyphenatedWords.txt.
%\newlanguage\printDraftLanguage
%\language\printDraftLanguage

%%%%% Hooks %%%%%

%\sethook{location}{marker}{insert}

%% before -- prior to the start of the par containing the selected style text (before its definition is applied)
%% after -- after the end of the par containing the selected style text (after its definition is terminated)
%% start -- before the start of the selected style text (but after the style par definition is applied)
%% end -- after the end of the selected style text (but before the style par definition is terminated)


% Added these TeX macro commands to this TeX extensions file to enable
% introduction material to be displayed in a two column table and the
% column be separated by leader dots.

% Enable commands with digits in them to be processed
\catcode`@=11
\def\makedigitsother{\m@kedigitsother}
\def\makedigitsletters{\m@kedigitsletters}
\catcode `@=12

% This is the macro to use in the source text for placing
% the introduction text into a table-like format.
% Usage:
%   \iotableleader{First column text}{Second column text}
%\def\iotableleader#1#2{#1\leaders\hbox to 0.8em{\hss.\hss}\hfill#2\par}%

