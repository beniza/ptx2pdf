% Paratext formatting macros, spanning footnotes version

\TeXXeTstate=1

\input ptx-stylesheet-macros.tex
\input ptx-cropmarks.tex
\input ptx-adj-list.tex
\input ptx-pic-list.tex
\input ptx-cutouts.tex
\input ptx-callers.tex

\catcode`\@=11

\let\x@=\expandafter

% based on plain.tex \footnote, \vfootnote
\def\m@kenote#1#2{\let\@sf\empty % #1=class; #2=caller; text is read later
  \ifhmode\edef\@sf{\spacefactor\the\spacefactor}\/\fi
  \ifch@pter \everypar={}\ch@pterfalse
    \global\setbox\ch@pternote=\hbox{\box\ch@pternote #2}
  \else #2\fi
  \@sf \vm@kenote{#1}{#2}}

\def\vm@kenote#1#2{%
  \let\next\relax
  \x@\insert\csname note-#1\endcsname\bgroup
%%% single-column notes:
  \checkp@ranotes{#1}%
  \hsize=\ifp@ranotes\maxdimen\else\textwidth\fi
  \interlinepenalty\interfootnotelinepenalty
  \splittopskip\ht\strutbox % top baseline for broken footnotes
  \splitmaxdepth\dp\strutbox \floatingpenalty\@MM
  \leftskip\z@skip \rightskip\z@skip \spaceskip\z@skip \xspaceskip\z@skip
  \leavevmode\ifRTL\setbox2=\lastbox\beginR\box2\fi
  \testomitc@ller{#1}\ifomitc@ller\else
    \setbox0=\hbox{#2}#2\ifdim\wd0>0pt\kern.2em\fi
  \fi
  \footstrut\futurelet\next\fo@t}

\def\OmitCallerInNote#1{%
  \expandafter\let\csname omit-in-note #1\endcsname=1}
\def\testomitc@ller#1{\expandafter\ifx\csname omit-in-note #1\endcsname\relax
  \omitc@llerfalse \else \omitc@llertrue \fi}
\newif\ifomitc@ller

\def\footstrut{\s@tfont{\newn@testyle}%
  \s@tbaseline{\newn@testyle}%
  \setbox\f@@tstrut=\hbox to 0pt{\XeTeXuseglyphmetrics=0 \char32 \hss}%
  \dimen0=\ht\f@@tstrut \dimen2=\dp\f@@tstrut
  \dimen4=\dimen0 \advance\dimen4 by \dimen2
  \ifdim\dimen4<\baselineskip
    \dimen6=100\baselineskip \divide\dimen6 by \dimen4
    \multiply\dimen0 by \dimen6 \divide\dimen0 by 100
    \multiply\dimen2 by \dimen6 \divide\dimen2 by 100
  \fi
  \setbox\f@@tstrut=\hbox{}\ht\f@@tstrut=\dimen0 \dp\f@@tstrut=\dimen2
  \copy\f@@tstrut}

\def\@foot{\copy\f@@tstrut \ifp@ranotes \parfillskip=0pt \fi
  \par\egroup}
\newbox\f@@tstrut
\def\n@teglue{2em plus 1em minus .5em\relax}

\def\s@tbaseline#1{%
  \x@\ifx\csname baseline<#1>\endcsname \relax
    \getp@ram{fontsize}{#1}\dimen0=\p@ram\LeadingUnit
    \multiply\dimen0 by 14 \divide\dimen0 by 12
    \x@\xdef\csname baseline<#1>\endcsname{\the\dimen0}\fi
  \baselineskip=\csname baseline<#1>\endcsname}

\output={\onecolout}
\def\onecolout{\edef\p@gefirstmark{\firstmark}\xdef\p@gebotmark{\botmark}%
  \plainoutput
  \xdef\p@gefirstmark{}
  \xdef\p@gebotmark{}}
\def\pagebody{\vbox to\textheight{\boxmaxdepth\maxdepth \pagecontents}}
\def\makeheadline{\vbox to\z@{\vskip-22.5\p@
  \hbox to \textwidth{\vbox to8.5\p@{}\the\headline}\vss
  \ifrhr@le\ifdim\RHruleposition=\maxdimen\else
    \kern-\RHruleposition\kern-0.4pt\hrule\kern\RHruleposition
    \fi\fi}\nointerlineskip}
\newif\ifrhr@le
\def\makefootline{\baselineskip24\p@\lineskiplimit\z@
  \hbox to \textwidth{\the\footline}}
\newdimen\RHruleposition \RHruleposition=\maxdimen

\def\pagecontents{\ifvoid\topins\else\unvbox\topins\fi
  \dimen@=\dp\@cclv \unvbox\@cclv % open up \box255
  \ifvoid\bottomins\else\unvbox\bottomins\fi
  \kern-\dimen@ \vfill
  \f@rstnotetrue
  \let\\=\ins@rtn@tecl@ss \the\n@tecl@sses}

\def\ins@rtn@tecl@ss#1{%
  \checkp@ranotes{#1}\ifp@ranotes\let\n@xt=\parains@rtn@tecl@ss
    \else\let\n@xt=\separateins@rtn@tecl@ss\fi
  \n@xt{#1}}

\def\separateins@rtn@tecl@ss#1{%
  \x@\let\x@\th@cl@ss\csname note-#1\endcsname
  \ifvoid\th@cl@ss\else
    \vskip\AboveNoteSpace
    \iff@rstnote\footnoterule\f@rstnotefalse\fi
    \unvbox\th@cl@ss\fi}

\def\parains@rtn@tecl@ss#1{%
  \x@\let\x@\th@cl@ss\csname note-#1\endcsname
  \ifvoid\th@cl@ss\else
    \vskip\AboveNoteSpace
    \iff@rstnote\footnoterule\f@rstnotefalse\fi
    {\maken@tepara{\th@cl@ss}}%
  \fi}

\newif\iff@rstnote
\def\footnoterule{\kern-.4pt % the \hrule is .4pt high
  \hrule width \textwidth}

\def\ParagraphedNotes#1{\x@\let\csname paranotes-#1\endcsname=1}
\newif\ifp@ranotes
\def\checkp@ranotes#1{\x@\ifx\csname paranotes-#1\endcsname\relax
  \p@ranotesfalse\else\p@ranotestrue\fi}

\def\maken@tepara#1{\hsize=\textwidth
  \let\par=\endgraf\ch@pterfalse
  \everypar={}\noindent
  \unvbox#1 \makehboxofhboxes 
  \setbox0=\hbox{\unhbox0 \removehboxes}
  \noindent\unhbox0\unskip\unpenalty\unskip\unskip\linepenalty100\par} 
\def\makehboxofhboxes{\setbox0=\hbox{} 
  \loop\setbox2=\lastbox \ifhbox2 \setbox0=\hbox{\box2\unhbox0}\repeat} 
\def\removehboxes{\setbox0=\lastbox 
  \ifhbox0{\removehboxes}\unhbox0\internotespace\fi}
\def\internotespace{\hfil\hskip\intern@teskip\penalty-10\hfilneg}
\newskip\intern@teskip \intern@teskip=15pt% plus 5pt minus 5pt

\def\doublecolumns{%
  \ifnum\c@rrentcols=1
    \ifhe@dings\endhe@dings\fi
    \penalty-100\vskip\baselineskip
    \global\output={\savepartialpage}\eject
    \ifdim\ht\partial>0.75\textheight
      \global\output={\onecolout}
      \unvbox\partial\vfill\eject
    \fi
    \global\hsize=\colwidth
    \global\vsize=2\textheight \global\advance\vsize by -2\ht\partial
    \gdef\resetvsize{\global\vsize=2\textheight \global\advance\vsize by \baselineskip} 
    \global\output={\twocols}
    \global\c@rrentcols=2
    \global\count\topins=2000
    \global\count\bottomins=2000
    \let\\=\s@tn@tec@unt \the\n@tecl@sses
    \global\holdinginserts=1
  \fi}
\def\s@tn@tec@unt#1{%
  \x@\let\x@\th@cl@ss\csname note-#1\endcsname
  \checkp@ranotes{#1}%
  \global\count\th@cl@ss=\ifp@ranotes 0 \else 2000 \fi
  \global\skip\th@cl@ss=\AboveNoteSpace }

\def\resetvsize{\global\vsize=\textheight} 

\def\msg#1{\immediate\write16{#1}}
\def\savepartialpage{%
%%%  \msg{SAVEPARTIAL}
  \edef\t@st{\p@gefirstmark}%
  \ifx\t@st\empty\xdef\p@gefirstmark{\firstmark}\fi
  \global\setbox\partial=\vbox{\boxmaxdepth\maxdepth \pagecontents}}
\newbox\partial

\def\twocols{% primary output routine in 2-col mode
%%%  \msg{TWOCOLS @ \ch@pter:\v@rse, txtht=\the\textheight, partial=\the\ht\partial}
  \global\setbox\galley=\copy255
  \global\galleypenalty=\outputpenalty
  \global\trialheight=\textheight \global\advance\trialheight by -\ht\partial
  \global\output={\trial}
  \global\holdinginserts=0
  \unvbox255
  \penalty\ifnum\outputpenalty=10000 0 \else \outputpenalty \fi
  }
\newbox\galley
\newcount\galleypenalty
\newdimen\trialheight

\def\@dvn@tecl@ssdim@#1{%
  \checkp@ranotes{#1}%
  \ifp@ranotes\let\n@xt=\@dvnoteclassdim@para
  \else\let\n@xt=\@dvnoteclassdim@sep\fi
  \n@xt{#1}}

\def\@dvnoteclassdim@para#1{
  \x@\let\x@\th@cl@ss\csname note-#1\endcsname
  \ifvoid\th@cl@ss\else
    \setbox0=\copy\th@cl@ss
    \setbox0=\vbox{\maken@tepara{0}}
    \advance\dimen0 by -\ht0
    \advance\dimen0 by -\dp0
    \advance\dimen0 by -\AboveNoteSpace
  \fi}
\def\@dvnoteclassdim@sep#1{
  \x@\let\x@\th@cl@ss\csname note-#1\endcsname
  \ifvoid\th@cl@ss\else
    \advance\dimen0 by -\ht\th@cl@ss
    \advance\dimen0 by -\dp\th@cl@ss
    \advance\dimen0 by -\AboveNoteSpace
  \fi}
\def\cle@rn@tecl@ss#1{%
  \x@\let\x@\th@cl@ss\csname note-#1\endcsname
  \global\setbox\th@cl@ss=\box\voidb@x}

\def\incr#1#2{\ifvoid#2\else\advance#1 by \skip#2\advance#1 by \ht#2\fi}
\def\decr#1#2{\ifvoid#2\else\advance#1 by-\skip#2\advance#1 by-\ht#2\fi}

\def\trial{%
%%%  \msg{TRIAL with ht=\the\trialheight, vsize=\the\vsize}
  \edef\p@gebotmark{\botmark}
  \edef\t@st{\p@gefirstmark}%
  \ifx\t@st\empty\edef\p@gefirstmark{\firstmark}\fi
  \dimen0=\trialheight
  \let\\=\@dvn@tecl@ssdim@ \the\n@tecl@sses
%%%
  \decr{\dimen0}{\topins}
  \decr{\dimen0}{\bottomins}
%%%
  \splittopskip=\topskip
  \setbox8=\copy255
  \dimen2=\dimen0 \decr{\dimen2}{\topleftins} \decr{\dimen2}{\bottomleftins}
  \dimen4=\dimen0 \decr{\dimen4}{\toprightins} \decr{\dimen4}{\bottomrightins}
  \setbox4=\vsplit255 to \dimen2
  \setbox6=\vsplit255 to \dimen4
  \setbox4=\vbox{\unvbox4}
  \setbox6=\vbox{\unvbox6}
  \ifvoid255 \fitonpagetrue \else \fitonpagefalse \fi
%%%  \msg{first col = \the\ht4, second col = \the\ht6, rem = \the\ht255}
  \iffitonpage
%%%    \msg{SUCCEEDED, shipping page}
% re-split to better balance columns, if possible
% at this point:
%   box4+box6 is the page content
%   dimen0 is ht that definitely works
%   try re-splitting with smaller ht while ht6<ht4
    \dimen2=\ht4 \incr{\dimen2}{\topleftins} \incr{\dimen2}{\bottomleftins}
    \dimen4=\ht6 \incr{\dimen4}{\toprightins} \incr{\dimen4}{\bottomrightins}
    \ifdim\dimen4<\dimen2
      \rebalancetrue
      \dimen8=\dimen0
      \loop%\msg{re-balancing}
        \vfuzz=\maxdimen
        \advance\dimen8 by -\baselineskip
        \setbox0=\copy8
        \dimen2=\dimen8 \decr{\dimen2}{\topleftins} \decr{\dimen2}{\bottomleftins}
        \dimen4=\dimen8 \decr{\dimen4}{\toprightins} \decr{\dimen4}{\bottomrightins}
        \setbox4=\vsplit0 to \dimen2 \setbox4=\vbox{\unvbox4}
        \setbox6=\vsplit0 to \dimen4 \setbox6=\vbox{\unvbox6}
        \ifvoid0\else\rebalancefalse\fi
        \dimen6=\ht6 \incr{\dimen6}{\topleftins} \incr{\dimen6}{\bottomleftins}
        \advance\dimen6 by -\ht4 \decr{\dimen6}{\toprightins} \decr{\dimen6}{\bottomrightins}
        \ifdim\dimen6>.3\baselineskip \rebalancefalse\fi
        \ifdim\dimen8<3\baselineskip \rebalancefalse\fi
        \ifrebalance\repeat
      \advance\dimen8 by \baselineskip
      \dimen2=\dimen8 \decr{\dimen2}{\topleftins} \decr{\dimen2}{\bottomleftins}
      \dimen4=\dimen8 \decr{\dimen4}{\toprightins} \decr{\dimen4}{\bottomrightins}
      \setbox4=\vsplit8 to \dimen2 \setbox4=\vbox{\unvbox4}
      \setbox6=\vsplit8 to \dimen4 \setbox6=\vbox{\unvbox6}
    \fi
    \dimen4=\dp4 \setbox4=\vbox to \dimen0{\unvbox\topleftins\unvbox4\vfil\unvbox\bottomleftins}\dp4=\dimen4
    \dimen6=\dp6 \setbox6=\vbox to \dimen0{\unvbox\toprightins\unvbox6\vfil\unvbox\bottomrightins}\dp6=\dimen6
    \def\pagecontents{%
      \ifvoid\partial\else \unvbox\partial \fi
      \ifvoid\topins\else \unvbox\topins \vskip\skip\topins \fi
      \dimen0=\ifdim\dp6>\dp4 \dp6 \else \dp4 \fi
      %\showthe\dimen0
      \hbox to \textwidth{\ifRTL \box6\hfil\box4 \else \box4\hfil\box6 \fi}
      \ifvoid\bottomins\else \vskip\skip\bottomins \unvbox\bottomins \fi
      \kern-\dimen0 \vfil
      \f@rstnotetrue
      \let\\=\ins@rtn@tecl@ss \the\n@tecl@sses
    }
    \resetvsize
    \plainoutput
    \xdef\p@gefirstmark{}
    \xdef\p@gebotmark{}
    \global\holdinginserts=1
    \global\output={\twocols}
  \else
%%%    \msg{REDUCING VSIZE}
    \global\advance\vsize by -\baselineskip
    \global\setbox\topins=\box\voidb@x
    \global\setbox\bottomins=\box\voidb@x
    \global\setbox\topleftins=\box\voidb@x
    \global\setbox\toprightins=\box\voidb@x
    \global\setbox\bottomleftins=\box\voidb@x
    \global\setbox\bottomrightins=\box\voidb@x
    \let\\=\cle@rn@tecl@ss \the\n@tecl@sses
    \global\setbox255=\box\voidb@x
    \global\holdinginserts=1
    \global\output={\backingup}
    \unvbox\galley \penalty\ifnum\galleypenalty=10000 0 \else \galleypenalty \fi
  \fi}
\newif\iffitonpage
\newif\ifrebalance

\def\backingup{
%%%  \msg{BACKING-UP}
  \global\deadcycles=0
  \global\setbox\galley=\copy255
  \global\galleypenalty=\outputpenalty
  \global\output={\trial}
  \global\holdinginserts=0
  \unvbox255 %\eject
  \penalty\ifnum\outputpenalty=10000 0 \else \outputpenalty \fi
}

\xdef\p@gefirstmark{}

\def\singlecolumn{%
  \ifnum\c@rrentcols>1
    \ifhe@dings\endhe@dings\fi
    \vfill\eject
    \global\holdinginserts=0
    \global\hsize=\textwidth
    \global\vsize=\textheight
    \global\output={\onecolout}
    \global\c@rrentcols=1
    \global\count\topins=1000
    \global\count\bottomins=1000
    \box\partial
    \vskip\baselineskip
  \fi}

\count\topins=1000 \dimen\topins=\maxdimen \skip\topins=0pt
\newinsert\bottomins \count\bottomins=1000 \dimen\bottomins=\maxdimen
\newinsert\topleftins \count\topleftins=1000 \dimen\topleftins=\maxdimen
\newinsert\toprightins \count\toprightins=1000 \dimen\toprightins=\maxdimen
\newinsert\bottomleftins \count\bottomleftins=1000 \dimen\bottomleftins=\maxdimen
\newinsert\bottomrightins\count\bottomrightins=1000 \dimen\bottomrightins=\maxdimen

\def\d@figure#1{%
  \gdef\p@rams{#1|}%
  \global\p@ramnumber=1
  \loop
    \x@\getonep@ram\p@rams\end
    \ifx\p@rams\empty \morep@ramsfalse \else \morep@ramstrue \fi
    \ifmorep@rams\repeat
  \lowercase{\edef\size@ption{\csname param-3\endcsname}}%
  \lowercase{\edef\loc@ption{\csname param-4\endcsname}}%
  \ifx\loc@ption\empty
    \ifx\size@ption\size@SPAN\def\loc@ption{t}\else\def\loc@ption{tl}\fi
  \fi
  \p@cwidth=\ifx\size@ption\size@COL \hsize
  \else\ifx\size@ption\size@SPAN \textwidth
  \else \p@cwidth=\textwidth
    \errmessage{Unknown picture size "\size@ption", expected "col" or "span"}\fi\fi
  \let\p@cins=\relax
  \ifx\loc@ption\loc@T \let\p@cins=\topins
  \else\ifx\loc@ption\loc@B \let\p@cins=\bottomins
  \else
      \ifnum\c@rrentcols>1
        \ifx\loc@ption\loc@TL \let\p@cins=\topleftins
        \else\ifx\loc@ption\loc@TR \let\p@cins=\toprightins
        \else\ifx\loc@ption\loc@BL \let\p@cins=\bottomleftins
        \else\ifx\loc@ption\loc@BR \let\p@cins=\bottomrightins
        \fi\fi\fi\fi
      \else
        \ifx\loc@ption\loc@TL \picw@rning{tl}{t}\let\p@cins=\topins
        \else\ifx\loc@ption\loc@TR \picw@rning{tr}{t}\let\p@cins=\topins
        \else\ifx\loc@ption\loc@BL \picw@rning{bl}{b}\let\p@cins=\bottomins
        \else\ifx\loc@ption\loc@BR \picw@rning{br}{b}\let\p@cins=\bottomins
        \fi\fi\fi\fi
      \fi
  \fi\fi
  \ifx\p@cins\relax
    \errmessage{Unknown picture location "\loc@ption",
      expected one of t,b,tl,tr,bl,br}\let\p@cins=\topins
  \fi
  \setbox0=\vbox{
    \hsize=\p@cwidth
    \line{\XeTeXpicfile "\the\PicPath\csname param-2\endcsname" width \p@cwidth }
    \edef\r@f{\csname param-7\endcsname}
    \edef\c@ption{\csname param-6\endcsname}
    \ifx\c@ption\empty\else
      \everypar={}\let\par\endgraf
      \leftskip=0pt plus 1fil \rightskip=\leftskip \parfillskip=0pt
      \linepenalty=1000
      \noindent\leavevmode
        \s@tfont{fig}\c@ption\unskip
        \ifx\r@f\empty\else\nobreak\ (\r@f\unskip)\fi
      \par
    \fi
    \vskip.5\baselineskip
  }%
  \insert\p@cins{\penalty10000 % nonfloating insertion % was 100
    \splittopskip\z@skip
    \splitmaxdepth\maxdimen \floatingpenalty20000 % was zero
    \gridb@x0}%
}
\newtoks\PicPath
\def\picw@rning#1#2{\msg{converted picture placement "#1" to "#2" in single-column layout}}
\def\size@COL{col}
\def\size@SPAN{span}
\def\loc@T{t}
\def\loc@B{b}
\def\loc@TL{tl}
\def\loc@TR{tr}
\def\loc@BL{bl}
\def\loc@BR{br}
\newcount\p@ramnumber
\newdimen\p@cwidth
\newif\ifmorep@rams
\def\getonep@ram#1|#2\end{\gdef\p@rams{#2}%
  \x@\gdef\csname param-\the\p@ramnumber\endcsname{#1}%
  \global\advance\p@ramnumber by 1 }

\def\pagebreak{\vfill\eject %% FIXME
  \ifnum\c@rrentcols=2
%    \iffirstcol\else\line{}\vfill\eject\fi
  \fi
}
\let\pb=\pagebreak

\def\columnbreak{\vfill\eject}

\tolerance=9000
\hbadness=10000
\emergencystretch=1in
\vbadness=10000
\vfuzz=2pt
\frenchspacing


\catcode`\[=\active \def[{\char`\[\ignorespaces}
\catcode`\(=\active \def({\char`\(\ignorespaces}

\catcode`\@=12

\parskip=0pt
\baselineskip=14pt
\lineskip=0pt

\widowpenalty=10000
\clubpenalty=10000
\brokenpenalty=50

\endinput
