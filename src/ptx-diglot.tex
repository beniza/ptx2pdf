%:strip
% ptx-diglot.tex: Diglot(v2) processing for xetex paratext2.tex
% Copyright (c) 2008-2021 by SIL International 
% written by David Gardner
%
% Permission is hereby granted, free of charge, to any person obtaining
% a copy of this software and associated documentation files (the  
% "Software"), to deal in the Software without restriction, including  
% without limitation the rights to use, copy, modify, merge, publish,  
% distribute, sublicense, and/or sell copies of the Software, and to  
% permit persons to whom the Software is furnished to do so, subject to  
% the following conditions:
%
% The above copyright notice and this permission notice shall be  
% included in all copies or substantial portions of the Software.
%
% THE SOFTWARE IS PROVIDED "AS IS", WITHOUT WARRANTY OF ANY KIND,  
% EXPRESS OR IMPLIED, INCLUDING BUT NOT LIMITED TO THE WARRANTIES OF  
% MERCHANTABILITY, FITNESS FOR A PARTICULAR PURPOSE AND  
% NONINFRINGEMENT. IN NO EVENT SHALL SIL INTERNATIONAL BE LIABLE FOR  
% ANY CLAIM, DAMAGES OR OTHER LIABILITY, WHETHER IN AN ACTION OF  
% CONTRACT, TORT OR OTHERWISE, ARISING FROM, OUT OF OR IN CONNECTION  
% WITH THE SOFTWARE OR THE USE OR OTHER DEALINGS IN THE SOFTWARE.
%
% Except as contained in this notice, the name of SIL International  
% shall not be used in advertising or otherwise to promote the sale,  
% use or other dealings in this Software without prior written  
% authorization from SIL International.
%%%%%%%%%%%%%%%%%%%%%%%%%%%%%%%%%%%%%%%%%%%%%%%%%%%%%%%%%%%%%%%%%%%%%%%

\newcount\diglotDbgJoinboxes%
\diglotDbgJoinboxes=-1% Set to the debug message of a joinboxes to execute showbox on that join

\def\diglotbadbrk{\penalty 10000}

%\def\TRshipout#1{\shipout#1}
\def\TRshipout#1{}
\def\b@xbotmark{}

% THis is a crude hack to make main titles line up nicely. For some reason
% they seem to already be vtops with depth info lost. 
%
\def\dstrut#1{{\setbox0\hbox{#1}\hbox{\vrule height \ht0 depth \dp0 width 0pt}}}
%
%
\newif\ifdiglotInnerOuter% Do pages switch columns based on page number (odd pages Left-Right, even Right-left)?
\newif\ifdiglotSwap% Do we invert the columns?
\diglotInnerOuterfalse
\newif\ifdiglotN@rmal % Which way round for this page?
\newif\ifuseLeftMarks %Do we use marks from the left column
\newif\ifuseRightMarks %Do we use them from the right column
\useLeftMarkstrue
\useRightMarkstrue

\def\firstLmark{} % First mark in left column
\def\botLmark{} % last mark in left column
\def\firstRmark{} % First mark in right column
\def\botRmark{} % last mark in right column
\def\nextp@gefirstmark{}
\def\n@xtfirstLmark{} %First Lmark in the n@xt chunk
\def\n@xtfirstRmark{} %First Rmark in the n@xt chunk
\def\LeftMarkstrue{\useLeftMarkstrue\useRigthMarksfalse}
\def\LeftMarksfalse{\useLeftMarkstrue\useRigthMarkstrue}

%Defined penalties
\def\dgl@tPenColSwap{-10001}%
\def\dgl@tPenInternalRpt{-10002}
\def\dgl@tPenLtrialEnd{-10005}
\def\dgl@tPenRtrialEnd{-10006}
\def\oldRmark{}
\def\oldLmark{}
\def\p@gebotmark{}
\newif\ifVisTrace% Show lines where boxes are joined
\newif\ifVisTraceExtra %Do VisTrace even in places where this breaks the layout.
\VisTracefalse%
\VisTraceExtrafalse%
\newif\ifdiglot %If there is diglot material
\newif\ifp@gestart % temporary hold while things get going.
\def\setsid@{\xdef\sid@{\c@rrdstat}}

%+cdig_define-hooks
\newtoks\diglotLho@ks
\newtoks\diglotRho@ks
\def\addToLeftHooks#1{\x@\global\x@\diglotLho@ks\x@{\the\diglotLho@ks #1}}
\def\addToRightHooks#1{\x@\global\x@\diglotRho@ks\x@{\the\diglotRho@ks #1}}
%-cdig_define-hooks

\def\stylesheetL#1{\gdef\ds@ffix{L}\stylesheet{#1}\gdef\ds@ffix{}}%Need to reset ds@ffix to {}, so that side-specific settings can be applied.
\def\stylesheetR#1{\gdef\ds@ffix{R}\stylesheet{#1}\gdef\ds@ffix{}}

%LRspecific holds a list of things that get redefined on side-switching. Now split into Definitions and Dimensions.
\def\LRspecificDef{AdornVerseNumber,VerticalSpaceFactor,LineSpacingFactor,regular,bold,italic,bolditalic,SpaceStretchFactor,SpaceShrinkFactor}
\def\LRspecificDim{FontSizeUnit,le@dingunit,onel@neunit,verticalsp@ceunit,IndentUnit} 


% Code to cycle through \LRspecific, setting them to their L/R values, modified
% from what the code to count callers does in ptx-callers.tex
%Would be nice to re-use \\, but this might get called defining a font in footnotes, so shouldn't mess with that
%Therefore use \wh@t instead.
\def\@rig{@orig}
% These do the looping
\def\pr@cessSp@cific#1,#2\E{\def\t@st{#1}\ifx\t@st\empty\let\n@xt\l@stSpecific\else\let\n@xt\pr@cessSp@cific\wh@t{#1}\fi\n@xt #2,\E}
\def\l@stSpecific#1\E{}
\def\pr@cessSpecific{\edef\LRs{\LRspecificDef,\LRspecificDim}\x@\pr@cessSp@cific \LRs,\E}
\def\pr@cessSpecificDef{\edef\LRs{\LRspecificDef}\x@\pr@cessSp@cific \LRs,\E}
\def\pr@cessSpecificDim{\edef\LRs{\LRspecificDim}\x@\pr@cessSp@cific \LRs,\E}

% These are the things that might get called by the loop.

% If theres a side-specific version, switch to it. If there's no side-specific verision, 
% switch to the \original version, if that exists.
\def\sp@cificSideDef#1{\ifcsname #1\sfx\endcsname\x@\let\csname#1\x@\endcsname\csname #1\sfx\endcsname
  \trace{S}{SpS: #1\sfx\space selected}\else
  \ifcsname #1\@rig\endcsname\x@\let\csname#1\x@\endcsname\csname #1\@rig\endcsname
    \trace{S}{SpS: #1 returned to original}\else\trace{S}{SpS: #1 unmodified}\fi\fi}

% If theres a side-specific version and its >0sp, switch to it. If there's no side-specific verision,
% switch to the \original version, if that exists.
\def\sp@cificSideDim#1{\x@\ifdim\csname #1\sfx\endcsname>0sp\x@\let\csname#1\x@\endcsname\csname #1\sfx\endcsname
  \trace{S}{SpS: #1\sfx\space selected}\else
  \ifcsname #1\@rig\endcsname\x@\let\csname#1\x@\endcsname\csname #1\@rig\endcsname
    \trace{S}{SpS: #1 returned to original}\else\trace{S}{SpS: #1 unmodified}\fi\fi}

\def\s@veSpecificOrig#1{\ifcsname #1\endcsname \x@\let\csname #1\@rig\x@\endcsname\csname #1\endcsname
  \ch@ckLR{#1}{L}\ch@ckLR{#1}{R}\else\MSG{No global definition for #1}\fi} % If the item exists, save its current value. 
\def\ch@ckLR#1#2{\ifcsname#1#2\endcsname\trace{S}{SpS: #1#2 exists}\else\x@\let\csname #1#2\x@\endcsname\csname #1\@rig\endcsname
  \ifdiglot\MSG{Side-specific #1#2 not defined, global #1 will be used}\fi\fi}% Helper function

% If the item exists, save its current value. 
\def\s@veSpecificSide#1{\ifcsname #1\endcsname \x@\let\csname #1\sfx\x@\endcsname\csname #1\endcsname\fi}
% Output routine for normal things.
\def\sh@wSpecificDef#1{\ifcsname #1\sfx\endcsname\trace{S}{#1\sfx: \x@\meaning\csname #1\sfx\endcsname}\else
  \trace{S}{#1\sfx: undefined}\fi}
% Output routine for  dimensions
\def\sh@wSpecificDim#1{\ifcsname #1\sfx\endcsname\trace{S}{#1\sfx: \the\csname #1\sfx\endcsname}\else
  \trace{S}{#1\sfx: undefined}\fi}
%
% And these are the interface functions. They should all set \sfx  to L, R or \@rig
%
\def\setLRspecific{\edef\sfx{\ifdiglot\c@rrdstat\else \@rig\fi}%
  \let\wh@t\sp@cificSideDef\pr@cessSpecificDef
  \let\wh@t\sp@cificSideDim\pr@cessSpecificDim
}% For use with style definitions, etc.

\def\SSsetLRspecific{\ifx\ds@ffix\empty\def\sfx{\@orig}\else\edef\sfx{\ds@ffix}\fi
  \let\wh@t\sp@cificSideDef\pr@cessSpecific}% for use in stylesheet, no need to check dimensions, just existance

\def\showLRspecific{\edef\sfx{\ifdiglot\c@rrdstat\else \@rig\fi}%
  \let\wh@t\sh@wSpecificDef\pr@cessSpecificDef
  \let\wh@t\sh@wSpecificDim\pr@cessSpecificDim
}% Call the output routines
\def\saveLRspecificSide#1{\xdef\sfx{#1}% Save current values (side defined by #1) 
  \let\wh@t\s@veSpecificSide
  \trace{S}{Redefining #1-specific values to current ones}\pr@cessSpecific
}
%
% Except this one:
%
\def\saveLRspecificOrig{% Save original values of side-specific variables. Force it to single use
  \trace{d}{saveLRspecificOrig \c@rrdstat}%
  \let\wh@t\s@veSpecificOrig\pr@cessSpecific\let\saveLRspecificOrig=\relax}

\maxdeadcycles=75
\newdimen\FontSizeUnitL\FontSizeUnitL=-1sp\newdimen\FontSizeUnitR\FontSizeUnitR=-1sp
\newdimen\le@dingunitL\le@dingunitL=-1sp\newdimen\le@dingunitR\le@dingunitR=-1sp
\newdimen\onel@neunitL\onel@neunitL=-1sp\newdimen\onel@neunitR\onel@neunitR=-1sp
\newdimen\verticalsp@ceunitL\verticalsp@ceunitL=-1sp\newdimen\verticalsp@ceunitR\verticalsp@ceunitR=-1sp
\newdimen\IndentUnitL\IndentUnitL=-1sp\newdimen\IndentUnitR\IndentUnitR=-1sp
\def\SpaceStretchFactorL{}\def\SpaceStretchFactorR{}
\def\SpaceShrinkFactorL{}\def\SpaceShrinkFactorR{}
\newif\ifRTLL \newif\ifRTLR
\font\VisTracefont="Andika":color=3f7f3f at 6pt
\def\doVisTrace#1{%
  \setbox0=\vtop to 0pt{\hrule height 0pt depth 0.5pt width 15pt\hbox{\VisTracefont #1 \the\TRACEcount}\vss}\ht0=0pt\dp0=0pt
}
\def\doVisTraceT#1{%
  \setbox0=\vbox to 0pt{\vss\hbox{\VisTracefont #1 \the\TRACEcount}\hrule height 0pt depth 0.5pt width 15pt}\ht0=0pt\dp0=0pt
}

\diglotfalse%
\newif\ifdiglotSepNotes %If the footnotes from the versions should be split (true) or merged together
\diglotSepNotestrue%
\newif\ifdiglotBalNotes %If a left column footnote steals space from the right column also
\diglotBalNotesfalse%
\newif\iftrialfailed
\global\def\n@xtc@mmand{}%

%\partial % fully set Partial page (both columns)
\newbox\n@xtpartialNrml % next chunk that we'll add to partial assuming all goes well - normal orientatin
\newbox\n@xtpartialRev % next chunk that we'll add to partial assuming all goes well - reversed orientation
\let\n@xtpartial=\n@xtpartialNrml

%\galley - as used in monoglot routines. This is the (holdinginserts=1) text. 
% saved by the first output routine. Unless void, it always ends with the appropriate
% end-of-trial penalty
\newbox\galleyexc@ss %This is the bit of the galley that didn't make it onto the current page
\newbox\partialL  % Partial page (fully processed), on the left side
\newbox\excessL  % Excess left material, aligning with the next right chunk.
\newbox\excessR  % Excess right material, once we know we're on the next page 
\newbox\partialR  % Partial page on the right side
\newbox\trialbox

%\newbox\partialPage \setbox\partialPage=\vbox{}
%\diglotLeft={\hsize=\columnLwidth\global\setbox\partialL=\vbox{\unvbox\partialL\unvbox255}}

\newif\ifrunLtrial % logic test in diglotLeft
\newif\ifintrial %flag to let setbox know...

% Not the same as \c@lcavailht from paratext2.tex
\def\dglt@calc@vailht{%
   \trace{D}{dglt@calc@vailht \c@rrdstat}%
   \global\availht=\textheight %
   \global\advance\availht by \adjustp@ge\relax % Panic measure..
   \global\advance\availht by -\ht\partial %
   \global\advance\availht by -\dp\partial %
   \global\advance\availht by -\ht\n@xtpartial %
   \global\advance\availht by -\dp\n@xtpartial %
   \trace{D}{after part1:\the\availht}%
   \ifdim\baselineDelta<0pt %FIXME, shouldn't this be side-dependent?
     \global\advance\availht by \baselineDelta\relax%
   \else
     \global\advance\availht by -\baselineDelta\relax%
   \fi
   \trace{D}{after part2:\the\availht}%
   \f@rstnotetrue
   \let\\=\reduceavailht \the\n@tecl@sses % reduce it by the space needed for each note class
   \iff@rstnote\else
      \advance\availht by -\AboveNoteSpace
   \fi
   \trace{D}{after notes:\the\availht}%
   \decr{\availht}{\topins}%pictures
   % Reduce by the height of whichever top picture takes the most space.
   \setbox1\copy\topleftins\setbox1\vbox{\unvbox1}%
   \setbox2\copy\toprightins\setbox2\vbox{\unvbox2}%
   \trace{D}{inserts: \the\ht1=\the\ht\topleftins? \the\ht2=\the\ht\toprightins?}%
   \ifdim\ht1>\ht\topleftins
     \trace{D}{Box was shrunk: \the\ht1 > \the\ht\topleftins}%
     \ht\topleftins=\ht1
   \fi
   \ifdim\ht2>\ht\toprightins
     \trace{D}{Box was shrunk: \the\ht2 > \the\ht\toprightins}%
     \ht\toprightins=\ht2
   \fi
   \ifdim\ht\topleftins>\ht\toprightins %
     \decr{\availht}{\topleftins}%
   \else%
     \decr{\availht}{\toprightins}%
   \fi%
   \trace{D}{after top:\the\availht (\the\ht\topleftins,\the\ht\toprightins}%
   \ifdim\ht\bottomleftins>\ht\bottomrightins %
     \decr{\availht}{\bottomleftins}%
   \else%
     \decr{\availht}{\bottomrightins}%
   \fi%
   \decr{\availht}{\bottomins}%
   \decr{\availht}{\verybottomins}%
   \x@\global\x@\csname availht\c@rrdstat\endcsname=\availht
   \trace{D}{final:\the\availht}%
}

\newif\ifnastybox  % Signal that box must be treated specially
% The \lastpenalty before a \lefttext or \righttext issues a fake one,
% or similarly penalty to apply at end of trial text, and scratch space

%%%%%%%%%%%
% All penalties
%
\newcount\savedpenalty  % The penalty immediately preceeding a side-swap.
\def\savedLpenalty{0}% just storage.
\def\savedRpenalty{0}

\newcount\tmppenalty%scratchspace
\newcount\Lboxpenalty% The penalty that goes between Lbox and partialL
\newcount\Rboxpenalty%  ~"~
\newcount\partialLpenalty% The penalty that goes between partialL and excessL (also used while loading excessL during excessL
\newcount\partialRpenalty% ~"~
\newcount\endtri@lpen@lty %the penalty we EXPECT at the end of the trial
\newcount\chunkpenalty %highest penalty encoutered yet for this chunk 
\newcount\lastchunkpenalty %highest penalty previouschunk
\newif\ifLneedsemptying
\newif\ifallNeedEmptying %set by lefttext, etc. to indicate that the current chunk ends with the next side-transition. 

% Initial \output routine for any side of scripture.
\def\diglot@any@primary{%
   \ifnum\the\outputpenalty=\dgl@tPenColSwap
     \setf@lwgdstat{\n@xtdstat}%
   \fi
   \trace{D}{diglot@any@primary: \show@dstat}%
   \tmppenalty=\outputpenalty %
   \ifnum\tmppenalty=1000
     \tmppenalty=0
   \fi
   \p@gestarttrue % or at least, this isn't the right place to place things that go after a paragraph
} 


\newif\ifboxmoved%
%\galleypenalty %the penalty we FOUND at the break between the %galley and \galleyexc@ss
\def\@ddtoexcess{%
  \trace{d}{@ddtoexcess hIns=\the\holdinginserts(==1) adding \the\ht255+\the\dp255\space to galleyexc@ss (\the\ht\galleyexc@ss+\the\dp\galleyexc@ss)}%
  \ifvoid\galleyexc@ss
    \global\setbox\galleyexc@ss\vbox{\TempDim=\dp255\unvbox255\kern-\TempDim\ifnum\outputpenalty=10000\else\ifnum\outputpenalty=\endtri@lpen@lty\else\penalty\outputpenalty\fi\fi}%
    %\showbox\galleyexc@ss
  \else
    %\global\setbox\galleyexc@ss\vbox{\joinboxes{\galleyexc@ss}{255}{19}{\ifnum\outputpenalty=10000 0\else\ifnum\outputpenalty=\endtri@lpen@lty 0\else\outputpenalty\fi\fi}}%
    \global\setbox\galleyexc@ss\vbox{\unvbox255\unvbox\galleyexc@ss\ifnum\outputpenalty=10000\else\ifnum\outputpenalty=\endtri@lpen@lty \else\penalty\outputpenalty\fi\fi}%
  \fi
  \trace{D}{\the\ht\galleyexc@ss+\the\dp\galleyexc@ss)}%
  %\showbox\galleyexc@ss
}

\def\undog@lley{%
  \trace{d}{undog@lley}%
  \dimen9=\dp\galley
  \unvbox\galley\count255=\lastpenalty\unpenalty
  \count254=\lastnodetype
  \ifnum\count254=12\else\kern-\dimen9\fi%\lastnotedype=12 is eTeX test for kern
  \ifVisTraceExtra\hbox to 0pt{\doVisTrace{t \the\ht\galleyexc@ss}\box0\hss}\fi
  \traceNum{D}{removed penalty \the\count255, replacing with \the\galleypenalty. lnt:\the\count254}%
  \ifvoid\galleyexc@ss\else
    \penalty\galleypenalty
    \unvbox\galleyexc@ss
  \fi
  \global\setbox\galleyexc@ss=\box\voidb@x\global\setbox\galley=\box\voidb@x
}

\def\rej@ctgalley{%For calling by trial routines that don't like the galley they've just been given 
  \trace{d}{rej@ctgalley (\c@rrdstat)}%
  \r@storenotes{\g@tdstat}{1}%
  \restore@oldmarks{\c@rrdstat}%
  \ifdim\trialheight<0pt
    \boxmovedtrue
  \else
    \global\output={\diglot@backingup}%
    \global\vsize=\trialheight
    \global\holdinginserts=1
    \undog@lley
    \penalty\endtri@lpen@lty
    \relax
  \fi
}

\def\diglot@backingup{%Similar to but more complex than monoglot backing-up
  % routine. Run with holdinginserts=1
  % Used by either side, it will cause a new page
  % break to be found, and
  % once the end-of-trial penalty is found, 
  % the \trial routine is called (again)
  % assumes galley and galleyexc@ss are empty when first run
  % post-galley material is saved  in  \galleyexc@ss
  \trace{d}{diglot@backingup hIns=\the\holdinginserts(==1) op:\the\outputpenalty,  dc:\the\deadcycles}%
  \ifnum\holdinginserts=0
    \MSG{*** Internal error caught. Incorrect internal state found in backing up. Text of footnotes may have been lost.}%
  \fi
  \ifvoid\galley
    \ifnum\deadcycles>0
      \global\deadcycles=\numexpr \deadcycles - 1\relax% it takes 2 cycles to get here, and can't use advance
    \fi
    \global\galleypenalty=\ifnum \outputpenalty=10000 0\else\outputpenalty\fi
    \global\setbox\galley=\vbox{\unvbox255\penalty\endtri@lpen@lty}%
  \else
    \@ddtoexcess
  \fi
  \ifnum\outputpenalty=\endtri@lpen@lty
    \global\holdinginserts=0
    \global\trialfailedfalse
    \global\output={\diglot@any@trial}%calls \whichtrial on success
    \unvcopy\galley %process
    %\ifnum\pageno>100
      %\ifnum\interactionmode=2 %=\scrollmode
        %\showlists
      %\fi
    %\fi
    %\showbox\galleyexc@ss
  \else 
    \ifnum\outputpenalty<-10000
      \ifnum\outputpenalty>-10010
        \MSG{*** Internal error caught. Incorrect penalty encountered. \the\outputpenalty!=\endtri@lpen@lty}%
      \fi
    \fi
    \vsize=\textheight
  \fi
}

\def\run@a@trial{%
  \x@\global\x@\let\x@\endtri@lpen@lty\csname dgl@tPen\c@rrdstat trialEnd\endcsname
  \traceNum{d}{run@a@trial \show@dstat, \the\availht>\the\baselineskip?, etp:\endtri@lpen@lty}%
  \x@\let\x@\p@rtl\csname partial\c@rrdstat\endcsname
  \x@\let\x@\@xcs\csname excess\c@rrdstat\endcsname
  \trace{D}{partial\c@rrdstat: \the\ht\p@rtl+\the\dp\p@rtl x\the\wd\p@rtl}%
  \trace{D}{excess\c@rrdstat: \the\ht\@xcs+\the\dp\@xcs x\the\wd\@xcs}%
  \ifnum\TRACEcount=\diglotDbgJoinboxes
   \showbox\p@rtl\showbox\@xcs
  \fi
  \ifdim\availht>\baselineskip
    \trace{D}{Preparing for trial}%
    \global\trialheight=\availht
    \global\let\whichtrial=\after@a@trial
    \global\vsize=\trialheight
    \global\setbox\trialbox=\box\voidb@x
    \global\setbox\galleyexc@ss=\box\voidb@x
    \global\trialfailedfalse
    \global\intrialtrue
    \global\deadcycles=0
    \global\holdinginserts=0
    %Make sure partialX (last galley) is dropped or recycled ...
    \iflate@stage@reflow
      \trace{D}{restoring notes, re-flowing partial\c@rrdstat\space and excess\c@rrdstat,}%
      \r@storenotes{\g@tdstat}{1}%
      \ifvoid\p@rtl\else
        \global\setbox\@xcs\vtop{\joinboxes{\p@rtl}{\@xcs}{2}{\galleypenalty}}%
      \fi
    \else
      \s@venotes{\g@tdstat}{1}%
    \fi
    %grab 1st trialheight from excessX
    \ifnum\diglotDbgJoinboxes=\TRACEcount
      \showbox\@xcs
    \fi
    \global\setbox\p@rtl=\vsplit\@xcs to \trialheight
    \csname partial\c@rrdstat penalty\endcsname=0 %FIXME: Isn't there a \splitpenalty??
    \global\galleypenalty=0
    \global\setbox\galley=\vbox{\x@\unvcopy\p@rtl\unpenalty\penalty\endtri@lpen@lty}%
    \trace{D}{Running trial \the\trialheight, \the\ht\galley+\the\dp\galley (\the\ht\@xcs+\the\dp\@xcs left for next time)}%
    \ifdim\ht\galley=0pt\ifdim\dp\galley=0pt
      \trace{D}{How come galley has no size? (check with  scrollmode)}%
      \ifnum\interactionmode=2
        \showbox\galley
      \fi
    \fi\fi
    \global\output{\diglot@any@trial}
%    \showbox\galley
    \unvcopy\galley
    \unpenalty
    \penalty\endtri@lpen@lty
    \relax\relax
  \else
    \trace{D}{No space for trial}%
    \global\trialfailedtrue
    \after@after@a@trial
  \fi
}

\xdef\m@rknumL{1}
\xdef\m@rknumR{2}

\def\after@a@trial{%
  %After this runs: (example with L)
  % Lbox contains *typeset* material
  % partialL contains *galley* for typeset material
  % excessL contains surplus material
  % Lboxpenalty contains penalty for a break after Lbox
  \traceNum{d}{after@a@trial \c@rrdstat, \the\holdinginserts, Op:\the\outputpenalty,
    tb:\the\ht\trialbox+\the\dp\trialbox,
    LBP:\the\Lboxpenalty, PLP:\the\partialLpenalty, SLP:\savedLpenalty,
    RBP:\the\Rboxpenalty, PRP:\the\partialRpenalty, SRP:\savedRpenalty,
    DCyc:\the\deadcycles,
    LB:\the\ht\Lbox+\the\dp\Lbox, av:\the\availht, vs:\the\vsize, pg:\the\pagegoal iftrialfailed\iftrialfailed true\else false\fi}%
  \ifnum\TRACEcount=\diglotDbgJoinboxes
   \showbox\trialbox\showbox\galleyexc@ss\showlists
  \fi
  \trace{H}{after@a@trial: b:\botmark/\botmarks1/\botmarks2, t:\topmark/\topmarks1/\topmarks2, f:\firstmarks1/\firstmarks2, fr:\firstRmark, br:\botRmark ol:\oldLmark}%
  %
  %Testing acceptability of result:
  \tempfalse
  \let\after@this=\after@after@a@trial
  \iftrialfailed\else
    \TempDim=\dimexpr \availht-\ht\trialbox\relax
    \ifdim \dp\trialbox > 0.5\baselineskip % Is the depth of the box significant?
      \advance\TempDim by -\dp\trialbox
    \fi
    \ifvoid\galleyexc@ss %Was that the whole input?
      \x@\ifvoid\csname excess\c@rrdstat\endcsname
        \x@\ifnum\csname saved\c@rrdstat penalty\endcsname>9999
          \advance\TempDim by -2\baselineskip
          \trace{D}{Penalty does not allow this as last item on page, reserving 2 lines: -> \the\TempDim}%
        \fi
        \ifdim \TempDim<-0.2\baselineskip %then Unacceptably over-full
          \trace{D}{Not enough space for the suggested galley}%
          \temptrue %
        \fi
      \fi
    \fi
  \fi
  \iftemp
    \traceNum{D}{Rejecting \the\ht\partialL+\the\dp\partialL,\the\ht\galley+\the\dp\galley, :: \the\ht\galleyexc@ss+\the\dp\galleyexc@ss (\the\trialheight}%
    \ifnum\TRACEcount=\diglotDbgJoinboxes
      \x@\showbox\csname partial\c@rrdstat\endcsname 
      \x@\showbox\csname excess\c@rrdstat\endcsname 
      \showbox\galley
    \fi
    \global\advance\trialheight by -1\baselineskip
    \ifdim\trialheight>\baselineskip
      \let\after@this=\rej@ctgalley
    \else
      \global\trialfailedtrue
      \global\pagefulltrue
      \@trial@failed
    \fi
  \else
    \iftrialfailed
      \@trial@failed
    \else
      \makevtop{\trialbox}%
      \ifvoid\galleyexc@ss
        %\tracingmacros=1\tracingassigns=1
        \TempDim=\availht %Recalc in case there's an issue
        \x@\global\csname \c@rrdstat boxpenalty\endcsname=\csname saved\c@rrdstat penalty\endcsname
      \else
        \x@\global\csname\c@rrdstat boxpenalty\endcsname =\galleypenalty
        \x@\global\x@\setbox\csname excess\c@rrdstat\endcsname\vbox{\joinboxes{\galleyexc@ss}{\csname excess\c@rrdstat\endcsname}{4}{\csname partial\c@rrdstat penalty\endcsname}}%
      \fi
      \global\setbox\galleyexc@ss=\box\voidb@x
      \x@\global\x@\setbox\csname partial\c@rrdstat\endcsname\box\galley% Remember the ol galley
      \x@\global\csname \c@rrdstat ht\endcsname=\dimexpr \ht\trialbox + \dp\trialbox\relax
      \x@\global\x@\setbox\x@\csname \c@rrdstat box\endcsname\box\trialbox
      \x@\global\x@\let\x@\m@rknum\csname m@rknum\c@rrdstat\endcsname
      \update@marks
    \fi
    \trace{D}{End of after@a@trial LB:\the\ht\Lbox+\the\dp\Lbox, RB:\the\ht\Rbox+\the\dp\Rbox}%
    %\showlists
  \fi
  \after@this
}

\def\@trial@failed{%
  \trace{D}{trial failed, (end of?) input moved. refilling excess\c@rrdstat}%(the start may have been set already
  %Undo the vsplit (all mtl into excessX)
  \x@\global\x@\setbox\csname excess\c@rrdstat\endcsname\vbox{\joinboxes{\csname partial\c@rrdstat\endcsname}{\csname excess\c@rrdstat\endcsname}{3}{\csname partial\c@rrdstat penalty\endcsname}}%
  \let\col@do=\restore@oldmarks
  \x@\each@col\diglot@list\E
}


\def\update@marks{% Called on successful galley
\x@\let\x@\m@rknum\csname m@rknum\c@rrdstat\endcsname
\trace{H}{update@marks: \c@rrdstat\space uses marks\m@rknum}%
\edef\t@st{\firstmarks\m@rknum}%
\ifx\t@st\empty%
  \trace{H}{update@marks: First mark was empty}% Presumably others also
\else%
  \x@\let\x@\t@@st\csname oldbot\c@rrdstat mark\endcsname%
  \ifx\t@st\t@@st\else % do not use if \firstmark,botmark are a carry-over,
    \trace{H}{u@m: Setting bot\c@rrdstat mark to (\m@rknum) \botmarks\m@rknum [!=\csname oldbot\c@rrdstat mark\endcsname]}%
    \global\m@rksonpagetrue
    \x@\xdef\csname oldfirst\c@rrdstat mark\endcsname{\csname first\c@rrdstat mark\endcsname}%
    \x@\xdef\csname oldbot\c@rrdstat mark\endcsname{\csname bot\c@rrdstat mark\endcsname}% Remember botmark in case this chunk isn't on the page
    \x@\xdef\csname bot\c@rrdstat mark\endcsname{\botmarks\m@rknum}%
    \x@\xdef\csname n@xtfirst\c@rrdstat mark\endcsname{\firstmarks\m@rknum}%If the chunk moves, then this will be the appropriate start-point
    \x@\xdef\csname n@xtbot\c@rrdstat mark\endcsname{\botmarks\m@rknum}%If the chunk moves, then this will be the appropriate end-point
    \x@\ifx\csname first\c@rrdstat mark\endcsname\empty
      \trace{H}{u@m: Setting first\c@rrdstat mark to \firstmarks\m@rknum}%
      \x@\xdef\csname first\c@rrdstat mark\endcsname{\firstmarks\m@rknum}%If the chunk moves, then this will be the appropriate start-point
      \edef\t@st{\p@gefirstmark}%
      \ifx\t@st\empty%
        \trace{H}{dlt3: Setting (default) pagefirstmark to \firstmarks\m@rknum}%
        \xdef\p@gefirstmark{\firstmarks1}%
      \fi
    \fi
  \fi
\fi
}


\def\diglot@any@trial{%output routine for any column
 % Handles calling diglot@backingup
 %Stashes set material in \trialbox, discards extra,
 %if the galley fits then there are no discards and 
 %it calls whichtrial 
 %as though it hadn't been here.
 %Assumptions: holdinginserts=1 material in \galley and \galleyexc@ss
  \trace{d}{diglot@any@trial \the\ht255+\the\dp255\space hIns=\the\holdinginserts(==0 1st time) op=\the\outputpenalty, etp=\endtri@lpen@lty, dc=\the\deadcycles, xs:\the\ht\galleyexc@ss+\the\dp\galleyexc@ss, th:\the\trialheight}%
  %\tracingifs=1
  \ifnum\holdinginserts=0\else
    \ifvoid\trialbox
      \MSG{*** Internal error caught. Incorrect internal state found in during trial run. Text or footnotes may have been lost.}%
      \ifnum\interactionmode=2
        \showbox255
        \showlists
      \fi
    \fi
  \fi
  \global\holdinginserts=1
  \ifvoid\trialbox
    \dglt@calc@vailht
    \TempDim=\availht
    \setbox255\vbox{\unvbox255}%
    \ifdim\ht255>0pt
      \advance\TempDim by -\ht255
      \advance\TempDim by -\dp255
    \else
      \advance\TempDim by \ht255
      \advance\TempDim by -\dp255
    \fi 
    \global\setbox\trialbox=\box255
    \ifdim\TempDim<-0.3\baselineskip % Coarse test. main trial routine can be pickier if it wants
      \trace{D}{Initial Content does not fit (\the\trialheight, \the\TempDim, \the\availht)}%
      \global\trialfailedtrue
    \else
      \trace{D}{Initial content fits(\the\trialheight, \the\TempDim, \the\availht, \the\ht\trialbox)}%
    \fi
  \else
    \ifnum 1=\ifdim\ht255>0pt 1 \else \ifdim\dp255>0pt 1 \else 0\fi\fi
      \trace{D}{Extra content found before end of trial}%
      \global\trialfailedtrue
      \global\vsize=\textheight
      \global\setbox255\box\voidb@x
    \else
      \trace{D}{Zero-sized box found.}%
      \global\setbox\trialbox=\vbox{\unvbox\trialbox\box 255}%
    \fi
  \fi
  \let\after@this=\relax
  \ifnum\outputpenalty=\endtri@lpen@lty
    \iftrialfailed
      \global\advance\trialheight by -1\baselineskip
      \ifdim\trialheight<0.5\baselineskip
        \trace{D}{Abandoning trial, no space for galley at all}%
        \r@storenotes{\g@tdstat}{1}% As if galley hadn't happened.
        \global\setbox\galleyexc@ss\vbox{\undog@lley}%
        \global\galleypenalty=0 
        \global\setbox255=\box\voidb@x
        \let\after@this=\whichtrial% 
      \else 
        \global\output={\diglot@backingup}%
        \global\setbox\trialbox=\box\voidb@x
        \global\vsize=\trialheight
        \traceNum{D}{Trying again}%
        \r@storenotes{\g@tdstat}{1}%
        %\setbox\galley\vbox{\unvbox\galley\unvbox\galleyexc@ss\penalty\endtri@lpen@lty}%
        \ifvoid\galleyexc@ss\else
          \setbox\galley\vbox{\undog@lley\penalty\endtri@lpen@lty}%
        \fi
        \ifnum\TRACEcount=\diglotDbgJoinboxes
           \showbox\galley
        \fi
        \unvbox\galley\global\setbox\galley=\box\voidb@x
      \fi
    \else
      \trace{D}{All looks good!  \the\ht\trialbox+\the\dp\trialbox (gp:\the\galleypenalty, op: \the\outputpenalty)}%
      %\box\trialbox
      %\global\outputpenalty=\galleypenalty
      \let\after@this=\whichtrial
    \fi
  \fi
  \after@this
  \relax
}

\def\ztmon{\tracingmacros=1}
\def\ztmoff{\tracingmacros=0}
% Change which side we're adding to
\def\pr@sideswitch#1#2{%
  \global\savedpenalty=\lastpenalty % This MUST be first, or the last /thing/ won't be a penalty
  \ifhe@dings\endhe@dings
    \global\savedpenalty=\lastpenalty % Pick up revised after-heading penalty
  \fi%
  \endlastp@rstyle{p}%
  \trace{d}{#1text pL:\the\dp\partialL, xL:\the\dp\excessL, pR:\the\dp\partialR, xR\the\dp\excessR, lp:\the\savedpenalty \ifallNeedEmptying allNeedwmptying\fi}%
  %\showlists
  \ifx\c@rrdstat\empty
    \trace{d}{\c@rrdstat is empty, somehow, making it L for sanity}%
    \gdef\c@rrdstat{L}%
  \fi
  \global\output={\diglotCollect}%Why woudn't it be? Don't know, but sometimes it's not.
  \x@\xdef\csname saved\c@rrdstat penalty\endcsname{\the\savedpenalty}%
  \ifinn@te\errmessage{*** #1text called from inside footnote?!?}\fi
  \@@setside{#2}%
  \endgraf\penalty\dgl@tPenColSwap\relax %Trigger collection of text.
  \let\t@mpc@mmand=\n@xtc@mmand% This bit may process the collected input. 
  \global\def\n@xtc@mmand{}%
  \t@mpc@mmand
} 

\outer\def\lefttext{%
  \pr@sideswitch{left}{L}%
  \global\allNeedEmptyingfalse
  \swap@diglot{L}%sets diglotRtrue, etc
  \global\vsize=\maxdimen
  \global\availht=\vsize
  %Other Left parameters
  \output={\diglotCollect}\relax%
  \global\holdinginserts=1
  \the\diglotLho@ks
}%

\outer\def\righttext{%
  \pr@sideswitch{right}{R}%
  \global\allNeedEmptyingtrue % Text after this needs emptying
  \swap@diglot{R}%sets diglotRtrue, etc
  \global\vsize=\maxdimen
  \global\trialheight=\maxdimen
  \relax\output={\diglotCollect}%
  \global\holdinginserts=1
  \the\diglotRho@ks
}
%
% When there's a header on the left but not the right, but we want the verses
% to line up nicely...
\def\norighttext{%
  \global\allNeedEmptyingtrue% Text before the swap needs emptying
  \pr@sideswitch{noright}{L}%
  \swap@diglot{L}%sets diglotLtrue, etc
  \global\vsize=\maxdimen
  \global\trialheight=\maxdimen
  \global\holdinginserts=1
  \the\diglotLho@ks
}

%Similarly when there's no left text, almost-copying \righttext seems OK ...
\def\nolefttext{%
  \global\allNeedEmptyingtrue% Text before the swap needs emptying
  \pr@sideswitch{noleft}{R}%
  \global\allNeedEmptyingtrue% Text after the swap ALSO needs emptying, in case \righttext not used
  \swap@diglot{R}%sets diglotRtrue, etc
  \global\vsize=\maxdimen
  \global\trialheight=\maxdimen
  \global\holdinginserts=1
  \the\diglotLho@ks
}

\def\pagenumberL{\bgroup\setc@rdstat{L}\s@tfont{h}\pagenumber\egroup}%
\def\pagenumberR{\bgroup\setc@rdstat{R}\s@tfont{h}\pagenumber\egroup}

\newdimen\columnLwidth%
\newdimen\columnRwidth%
\newdimen\availhtL
\newdimen\availhtR
\newdimen\chunkDelta % Difference between the last boxes prepared for the page (-ve=left longer)
\newdimen\p@geChunkDelta \p@geChunkDelta=0pt %as above, but *added* to the page.
\newdimen\baselineDelta % semi-constant: difference between the L and R baselines.  Set in ptx-stylesheet (-ve=left longer) 
\newdimen\thisjointDelta % difference between the L and R boxes (-ve=left longer)
\newcount\cumulativeDelta % Sum of \chunkDelta values, in points
\cumulativeDelta=0
\newdimen\adjustp@ge %emergency stretch to page
\adjustp@ge=0pt
\def\m@rkerL{}
\def\m@rkerR{}
\newif\ifdiglotL%
\newif\ifdigl@tL%What is the NEXT text comming?
\let\diglotRtrue=\diglotLfalse\relax%
\let\digl@tRtrue=\digl@tLfalse\relax%
\let\diglotRfalse=\diglotLtrue\relax%
%Diglotstate.
\def\n@xtdstat{L}%What does the diglotstate become at the next \dgl@tPenColSwap penalty?
\def\f@lwgdstat{L}%What is the diglotstate of the 'recent contributions'? (i.e. next material to read)
\def\c@rrdstat{}%What is the current diglotstate?
\def\g@tdstat{\ifdiglot\if\c@rrdstat L\else\c@rrdstat\fi\fi}% Returns "" or "R" (Or other state values in the future)

\def\show@dstat{\c@rrdstat\f@lwgdstat\n@xtdstat}% for debugging output 
\def\setc@rdstat#1{\xdef\c@rrdstat{#1}%
 \ifx\c@rrdstat\empty
   \diglotfalse % Is this necessary? YES.
   \trace{d}{current diglot state set to empty! End of diglot?}%
 \else 
   %\diglottrue % Is this necessary?
   \if#1L\relax\global\diglotLtrue\else\global\diglotLfalse\fi%Compatability
 \fi
}
\def\setf@lwgdstat#1{\xdef\f@lwgdstat{#1}%
 \if#1L\global\digl@tLtrue\else\global\digl@tLfalse\fi%Compatability
}

\def\@setside#1{\setf@lwgdstat{#1}\setc@rdstat{#1}\xdef\n@xtdstat{#1}}

\def\@@setside#1{\trace{D}{@@setside #1}%
   \xdef\n@xtdstat{#1}%
   }%

\def\check@dup#1{%set temptrue if \test is the same as #1
  \edef\t@st{#1}\ifx\t@st\test\temptrue\fi
}

\def\add@to@digl@tc@ntentlist#1{%
  \trace{d}{add@to@digl@tc@ntentlist #1 (\digl@tc@ntentlist)}%
  \edef\test{#1}%
  \tempfalse
  \let\col@do=\check@dup
  \ifx\digl@tc@ntentlist\empty
  \else
    \x@\each@col\digl@tc@ntentlist\E
  \fi
  \iftemp\else
    \xdef\digl@tc@ntentlist{\digl@tc@ntentlist #1}%
  \fi
  \trace{D}{digl@tc@ntentlist now:(\digl@tc@ntentlist)}%
}

\def\swap@diglot#1{% Because we need to set things appropriately.
 %Swapping fonts isn't actually needed, but do need to set
 %baselineskip and other things
 \trace{d}{swap@diglot #1}%
 \trace{D}{t@tle is \ift@tle set\else  not set\fi}%
 \trace{D}{he@dingstyle is \ifhe@dingstyle set\else  not set\fi}%
 \trace{D}{he@dings is \ifhe@dings set\else  not set\fi}%
  %\trace{D}{nsp@cebefore is \ifnsp@cebefore set\else  not set\fi}%
 %All sorts of things should follow if m@rker is set properly...
 \x@\let\csname m@rker\c@rrdstat\endcsname\m@rker %save the old marker 
 \@setside{#1}%
 \add@to@digl@tc@ntentlist{#1}%
 \x@\let\x@\m@rker\csname m@rker\c@rrdstat\endcsname
 \x@\let\x@\w@dth\csname column#1width\endcsname
 \global\hsize=\w@dth\relax
 \x@\global\x@\let\x@\ch@pter\csname @ch@pter\g@tdstat\endcsname
 \x@\global\x@\let\x@\ch@ptert@xt\csname @ch@ptert@xt\g@tdstat\endcsname
 \x@\global\x@\let\x@\v@rse\csname @v@rse\g@tdstat\endcsname
 \x@\global\x@\let\x@\m@rknum\csname m@rknum\c@rrdstat\endcsname
 \setLRspecific %ensure that all the units are correct.
 \trace{D}{Leadingunit: \the\le@dingunit}%
 \ifdim\le@dingunit>0pt %
   \s@tbaseline{p}%
 \fi%
 \ifdim\baselineskip=0pt %
  \message{baseline set to 0pt EEK}%
  \global\baselineskip=12pt
 \fi%
 %Also need to switch hyphenation patterns
 \expandafter\ifx\csname language#1\endcsname\relax\else\uselanguage{\csname language#1\endcsname}\fi}%


%\output={\diglotLeft}
%\hsize=\columnLwidth
\newif\ifpagefull % Is the page full?
\newif\iflate@stage@reflow % Is the trial code re-running a previous galley or processing the next chunk?

\newdimen\Lht
\newdimen\Rht %
\newdimen\TempDim %
\newbox\Lbox % The part of partialL which fits on the page
\newbox\Rbox % The part of partialR which fits on the page

%when there's no more input, make sure partial gets printed
\def\emitpartial{%
       \traceNum{d}{Emitpartial \the\ht\partial+\the\dp\partial, \the\ht\n@xtpartial+\the\dp\n@xtpartial}%
       \@add@n@xtpartial
       \@writep@ge\@writep@ge%
       \global\pagefullfalse%
}

\def\makevtop#1{%There is a strong posibility that a vbox has a depth that
%is unrecoverable on changing it to a vtop, e.g. if the last item in the
%vbox is a \mark. To rejoin boxes on a page accurately we need to preserve
\global%the depth of the box. We therefore assume the box will be joined and if
%the depth is not recoverable we add a kern to remove the orignial depth.
 \s@tbaseline{p}%
 \trace{D}{makevtop #1, before vtop dp=\the\dp#1, ht=\the\ht#1}%
 \TempDim=\dp#1\relax%
 \ifdim\TempDim>\baselineskip \trace{D}{Looks like this is already a vtop}\TempDim=0pt
 \fi%
 %\bgroup
 \setbox0=\copy#1\setbox1=\vtop{\unvbox 0\setbox2=\lastbox}%Work on a copy so we don't break stuff.
 \trace{D}{pd:\the\dp2, ph=\the\ht2, d:\the\TempDim}%
 \global\setbox#1=\vtop{\unvbox#1\ifdim\dp2=0pt\ifdim\TempDim=0pt\else\kern-\TempDim\fi\fi}}%\egroup}%

\def\joinboxes#1#2#3#4{%Join 2 vtops together and preserve baselineskip%
 \traceNum{d}{joinboxes #1(\the\ht#1+\the\dp#1x\the\wd#1) #2 called from locn #3}%
 \ifnum\TRACEcount=\diglotDbgJoinboxes
   \showbox#1\showbox#2
 \fi\relax
 %How do these boxes join?
 \dimen5=0pt
 \ifdim\ht#1>0pt
   \def\join@a{vbox}%
   \ifdim\ht#2=0pt
     \ifdim\dp#2>0pt
       \dimen5=-\baselineskip
       \def\join@b{vtop}%
     \else
       \def\join@b{void}%
     \fi
    \else
       \def\join@b{vbox}%
    \fi
 \else
   \def\join@a{vtop}%
   \def\join@b{unknown}%
 \fi%
 %\bgroup
 %\ifnum#3=10\showbox#1\fi%
 \ifdim\ht#1=0pt \ifdim\dp#1=0pt \ifdim\wd#1=0pt \setbox0=\box#1\fi\fi\fi%make a 0 size box void
 \ifvoid#1%
   \trace{D}{#1 is void}%
 \else%
   \s@tbaseline{p}%
   \ifdim\ht#1=0pt
     \unvbox#1
     %\showlists
     \dimen4=\lastkern%
    % \unkern
     \setbox0=\lastbox%
     \copy0
     \trace{D}{lk: \the\dimen4, final bit of deconstructed first box: \the\ht0+\the\dp0, 2nd box: \the\ht#2+\the\dp#2}%
     \ifdim\dp0>0pt \ifdim\dimen4=0pt \ifdim\ht0>0pt
         \dimen4=-\dp0
     \fi\fi\fi
     \ifdim\dp0=0pt%
  %     \kern \lastkern
       \ifdim\ht0=0pt \ifdim\dimen4<0pt
          \dimen4=0pt
       \fi\fi
     \else
       \ifdim\dp0>\baselineskip
         \kern -\dp0
         %\else
         %\showbox0
       \fi%
     \fi
   \else
     \dimen4=-\dp#1
     \unvbox#1
     \ifnum\lastpenalty<-10000
       \unpenalty
     \fi
   \fi
   \ifvoid#2\else
     \ifdim\ht#2=0pt 
       \dimen3=\baselineskip
       \advance\dimen3 by -\topskip
       \vskip \dimen3
     \fi
   \fi
   %\trace{D}{Box joint is \join@a-\join@b (\the\dimen5)}%
   %\ifdim\dimen5=0pt\else
     %\kern \dimen5
   %\fi
   \ifVisTrace
  %   \hrule\kern -0.4pt
     \dimen6=1sp\count254=0
     \ifdim\dimen4<10sp \ifdim\dimen4>0sp
        \count254=\dimen4
        \dimen6=\dimen4
        \advance\dimen6 by 1sp
        \dimen4=0pt
        \trace{D}{Indenting label by \the\count254}%
     \fi\fi
     \kern \dimen4
     \hbox to 0pt{\hss\doVisTrace{j}\box0\kern\dimexpr -20pt * \count254\relax}%
     \kern \dimen6
   \fi
 \fi%
 %
 \ifvoid#2\else
   \penalty#4 %put the appropriate penalty back 
   %\setbox0=\copy#2
   %\setbox1=\vsplit0 to \baselineskip
   %\trace{D}{Next box, h:\the\ht1, d:\the\dp1}%
   \unvbox#2 %
 \fi%
 \ifnum\diglotDbgJoinboxes=\TRACEcount
   \showlists
 \fi
}

\def\joinb@xes#1#2{%Called (ONLY) by pagebuilder. Join a vtop/vbox and a vtop/vbox together and preserve baselineskip%
%\showbox#1\showbox#2%
\bgroup
 \setbox0=\box#1
 \setbox1=\box#2
 \dimen0=0pt
 \ifdim\ht0>0pt \ifdim\dp0<0.5\baselineskip
   %box0 is not a vtop, it's a vbox
   \advance\dimen0 by -\baselineskip
 \fi\fi
 \ifdim\ht1>0pt \ifdim\dp1<0.5\baselineskip
   %box1 is not a vtop, it's a vbox
   \ifdim\dimen0<0pt
     \dimen0=0pt
   \fi
   %Theoretically, there ought to be a \baselineskip here. It doesn't seem to be needed.
   %\advance\dimen0 by \baselineskip
 \fi\fi
 \trace{D}{joinb@xes #1 #2, skip \the\dimen0}%
 \box0
 \ifdim\dimen0=0pt
 \else
   \vskip\dimen0
 \fi
 \box1
\egroup}


\def\update@cumulative@delta{\bgroup
  \dimen2=1pt
  \traceNum{dP}{Adding \the\dimen0 \space to cumulativeDelta}%
  \divide\dimen0 by \dimen2
  \count255=\dimen0
  \global\advance\cumulativeDelta by \count255
  \egroup
}

\def\diglot@arrange@cols{%
  \trace{D}{diglot@arrange@cols}%
  \ifVisTrace
    \ifdim\ht\Lbox>0pt
      \doVisTraceT{L}% sets box0, used ~43 lines below
    \else
      \doVisTrace{L}% sets box0, used ~43 lines below
    \fi
  \fi
  %Is there actually anything to set?
  \ifnum 1=
      \ifdim\Lht=0pt
        \ifdim\Rht=0pt 0
        \else 1\fi
      \else 1\fi
    \thisjointDelta=0pt %Do we need extra space?
    \ifdim\dp\Lbox>0.3\baselineskip \ifdim\chunkDelta<0pt
      %left are 'touching' and need separating
      \ifdim\baselineDelta<0pt 
        \trace{D}{Separating chunks \the\baselineDelta}%
        \advance\thisjointDelta by -\baselineDelta\relax
      \fi 
      %FIXME: RHS / small L baseline may need more attention
    \fi\fi
    \global\chunkDelta=\Rht
    \global\advance\chunkDelta by -\Lht\relax
    \trace{D}{new chunkDelta: \the\Rht - \the\Lht =  \the\chunkDelta}%
    %save Lht and Rht (as "nxtchunkXht) and reset Xht
    \let\col@do\save@reset@ht
    \x@\each@col\diglot@list\E
    \bgroup
      \dimen0=\chunkDelta
      %\message{c255: \the\dimen0}%
      \update@cumulative@delta
    \egroup
    \dimen8=0pt
    \ifdim\thisjointDelta>0pt
      \advance\dimen8 by \thisjointDelta\relax
    \fi
    \advance\dimen8 by \ht\Lbox
    \ifdim \dimen8 <\ht\Rbox \dimen8=\ht\Rbox \fi
    \dimen9=\dp\Lbox
    \ifdim \dimen9 <\dp\Rbox \dimen9=\dp\Rbox \fi
    % apply columngutterruleskip?
    \tempfalse
    \ifdim\ht\partial=0pt\ifdim\dp\partial=0pt
      \advance\dimen8 by -\ColumnGutterRuleSkip\relax
      \ifdim\ColumnGutterRuleSkip < 0pt
       \ifColumnGutterRule
          \temptrue
       \fi
      \fi
    \fi\fi
    \trace{d}{Gutter \the\dimen8 height, \the\dimen9 deep}%
    \dimen7=\dimen8%.5\baselineskip
    \global\setbox\n@xtpartialNrml\vtop{%
      \iftemp
        \vskip\ColumnGutterRuleSkip
      \fi
      \hbox to \textwidth{\ifVisTrace\llap{\copy0}\fi%
      \hbox to\columnshift{}%
      \hbox to\columnLwidth{\copy\Lbox\hss}\makediglotgutter{\dimen7}{\dimen8}{\dimen9}%
      \hbox to\columnRwidth{\copy\Rbox\hss}%\rlap{\the\ht\Rbox \the\dp\Rbox}%
    }}%
    \global\setbox\n@xtpartialRev\vtop{%
      \iftemp
        \vskip\ColumnGutterRuleSkip
      \fi
      \hbox to \textwidth{\ifVisTrace\llap{\box0}\fi%
      \hbox to\columnRwidth{\box\Rbox\hss}\makediglotgutter{\dimen7}{\dimen8}{\dimen9}%
      \hbox to\columnLwidth{\box\Lbox\hss}%\rlap{\the\ht\Rbox \the\dp\Rbox}%
    }}%
  \else
    \traceNum{d}{No content to output}%
  \fi
}

\def\pickn@xtpartial{\ifdiglotN@rmal\global\let\n@xtpartial=\n@xtpartialNrml\else\global\let\n@xtpartial=\n@xtpartialRev\fi} % decide which n@xtpartial box to use

\def\voidn@xtpartial{\global\setbox\n@xtpartialRev\box\voidb@x\global\setbox\n@xtpartialNrml\box\voidb@x}% Void the other n@xtpartial box

\def\setdiglotN@rmal{\ifnum \numexpr 1 \ifdiglotInnerOuter \ifodd\pageno\else * -1 \fi\fi \ifdiglotSwap * -1\fi\relax >0 
  \diglotN@rmaltrue \else \diglotN@rmalfalse \fi
}

\def\upd@tep@rtial{%Actually add stuff to the page.
  \trace{D}{upd@tep@rtial \the\outputpenalty}%
  \trace{D}{plp:\the\partialLpenalty, prp:\the\partialRpenalty, cp:\the\chunkpenalty lcp:\the\lastchunkpenalty, lbp:\the\Lboxpenalty, rbp:\the\Rboxpenalty}%
  \ifnum\Lboxpenalty=10000 \ifnum\Rboxpenalty=10000
    \trace{D}{Un vtopping heading}%
    \setbox\Lbox=\vbox{\unvbox\Lbox}%
    \setbox\Rbox=\vbox{\unvbox\Rbox}%
  \fi\fi
  \setdiglotN@rmal % Left=left (normal) or not?
  \pickn@xtpartial
  \showboxdepth=3 %
  \showboxbreadth=1000 %
  %\ifnum\pageno=1%
  %\showbox\Lbox%
  %\showbox\Rbox%
  %\fi%
  \global\def\n@xtc@mmand{}%
  \message{|}%
  \hsize=\textwidth%
  \edef\t@st{\p@gebotmark}%
  \ifx\t@st\empty%
    \ifuseLeftMarks \let\b@xbotmark\botLmark\fi
    \ifuseRightMarks \let\b@xbotmark\botRmark\fi
    \trace{H}{updp1: Setting p@gebotmark to \b@xbotmark, since it's empty }%
    \xdef\p@gebotmark{\b@xbotmark}%
  \fi%
  \trace{d}{Adding old n@xtpartial(\the\ht\n@xtpartial+\the\dp\n@xtpartial) to page}%
  \@add@n@xtpartial
  \let\col@do=\store@marks
  \x@\each@col\diglot@list\E %remember old marks state
  \s@venotes{\g@tdstat}{2}% remember old notestate 
  \traceNum{d}{Writing new n@xtpartial : LB:\the\ht\Lbox+\the\dp\Lbox\space RB:\the\ht\Rbox+\the\dp\Rbox\space p@gechunkDelta:\the\p@geChunkDelta}%\space Tot:\the\cumulativeDelta%
  \let\\=\upd
  \diglot@arrange@cols
  \TempDim=\dimexpr \ht\n@xtpartial + \dp\n@xtpartial\relax
  \dglt@calc@vailht
  \ifdim\availht<\baselineskip %
    \global\pagefulltrue%
  \else%
  \fi%
  \ifdim\TempDim>0.1pt
    \global\deadcycles=0 %Reset deadcycles only if there's actual matter being added.
    \global\trialheight=\availht\relax%
    \global\vsize=\trialheight\relax%
  \fi
  \ifpagefull%
    \@add@n@xtpartial
    \@writep@ge%
      \let\col@do=\reset@n@xt
      \x@\each@col\diglot@list\E% 
    \global\pagefullfalse
    \global\deadcycles=0 %That counts as an action too.
  \fi%
}


\newif\ifLRf@@tnotes%
\newif\ifNoteGutterRule \NoteGutterRulefalse
\newif\ifFigGutterRule \FigGutterRulefalse
\newif\ifJoinGutterRule \FigGutterRulefalse
\def\@writep@ge {%
  \trace{d}{@writep@ge}%
  \setdiglotN@rmal
  \pickn@xtpartial
  \global\setbox\partial=\vbox{\unvbox\partial}%
  \ifdiglotSepNotes%
    \trace{D}{Sep notes}%
    \let\s@veddstat=\c@rrdstat
    \LRf@@tnotesfalse%
    \f@rstnotetrue%
    \setc@rdstat{L}%%
    \lastd@pth=0pt 
    %Arrange Left and Right footnotes
    \global\setbox\Lbox=\vbox{\the\diglotLho@ks\let\\=\ins@rtn@tecl@ss \the\n@tecl@sses}%
    \iff@rstnote%
      \trace{f}{No Left footnotes}%
    \else%
      \LRf@@tnotestrue%
      \f@rstnotetrue%
    \fi%
    \lastd@pth=0pt 
    \setc@rdstat{R}%%
    \global\setbox\Rbox=\vbox{\the\diglotRho@ks\let\\=\ins@rtn@tecl@ss \the\n@tecl@sses}%
    \iff@rstnote%
      \trace{f}{No Right footnotes}%
    \else%
      \LRf@@tnotestrue%
    \fi%
    \setc@rdstat{\s@veddstat}%%
    \ifdiglotBalNotes%
      \trace{f}{Balanced notes}%
      \ifLRf@@tnotes%
        \kern\lastd@pth%
        \trace{f}{Footnotes}%
        \ifdim\ht\Lbox>\ht\Rbox%
          \setbox\Rbox=\vbox to \ht\Lbox{\box\Rbox\vss}%
        \else%
          \setbox\Lbox=\vbox to \ht\Rbox{\box\Lbox\vss}%
        \fi%
        \dimen8=\ht\Lbox
        \ifdim \dimen8 <\ht\Rbox \dimen8=\ht\Rbox \fi
        \dimen9=\dp\Lbox
        \ifdim \dimen9 <\dp\Rbox \dimen9=\dp\Rbox \fi
        \trace{d}{Gutter \the\dimen8 high, \the\dimen9 deep}%
        \dimen7=0pt
        \setbox\Lbox=\hbox to \textwidth{%\vllap{\the\ht\Lbox \the\dp\Lbox}%
          \hbox to\columnshift{}%
          \ifdiglotN@rmal%are the sides swapped?
            \hbox to\columnLwidth{\copy\Lbox\hss}\ifNoteGutterRule\makediglotgutter{\dimen7}{\dimen8}{\dimen9}\else\hskip\gutter\fi%%%
            \hbox to\columnRwidth{\copy\Rbox\hss}%\rlap{\the\ht\Rbox \the\dp\Rbox}%
          \else%
            \hbox to\columnshift{}%
            \hbox to\columnRwidth{\copy\Rbox\hss}\ifNoteGutterRule\makediglotgutter{\dimen7}{\dimen8}{\dimen9}\else\hskip\gutter\fi%%
            \hbox to\columnLwidth{\copy\Lbox\hss}%\rlap{\the\ht\Rbox \the\dp\Rbox}%
          \fi%
        }%
      \else%
        \trace{f}{No footnotes}%
      \fi %LRf@@tnotes
      \setbox\partial=\vbox{\unvbox\partial\box\Lbox}%
    \else %BalNotes
      \trace{f}{unBalanced notes}%
      \ifLRf@@tnotes%
        %curLht and curRht hold the height of the last boxes that joined the
        %page,
        %Need to find out where to put the footnotes, as one of them should
        %be close to its respective text.
        {
        %Implement meshing algorithm:
        \trace{fS}{Saving fn dimensions}%
        \edef\v@lpfx{@fn}\let\col@do=\save@boxdim %Save footnote values
        \x@\each@col\diglot@list\E
        \edef\v@lsfx{ht}%
        \trace{fS}{Finding max \v@lpfx X\v@lsfx}%
        \let\col@do=\max@val\dimen1=0pt
        \x@\each@col\diglot@list\E
        \dimen2=\dimen1 %dimen2=note height 
        \dimen8=\dimen1 %dimen8=note height (needed for rule)
        \edef\v@lpfx{@cur}\dimen1=0pt
        \trace{fS}{Finding max \v@lpfx X\v@lsfx}%
        \x@\each@col\diglot@list\E
        \advance\dimen2 by \dimen1%dimen2 now contains total natural height of text from last true synch-point and notes.
        \edef\v@lpfxb{@fn}\dimen1=\dimen2
        \trace{fS}{Finding min note sep \v@lpfxb Xht}%
        \let\col@do=\min@note@sep% subtracts \v@lpfxb Xht and \v@lpfx Xht from dimen2, returning the smalles value in dimen1
        \x@\each@col\diglot@list\E
        \dimen3=\dimen1
        %rule calcs: is #1 (dimen7) high, #3(dimen9) deep, and fits in a box #2(dimen8 - found earlier)high
        \edef\v@lpfx{@fn}
        \edef\v@lsfx{dp}\let\col@do=\max@val\dimen1=0pt
        \trace{fS}{Finding max \v@lpfx X\v@lsfx}%
        \x@\each@col\diglot@list\E
        \dimen9=\dimen1
        \advance\dimen8 by -\AboveNoteSpace\relax %stop the line a little short
        \dimen7=\dimen8
        %
        \trace{d}{Gutter \the\dimen8 tall, \the\dimen9 deep}%
        \setbox\Lbox\hbox to \textwidth{%\vllap{\the\ht\Lbox \the\dp\Lbox}%
          \hbox to\columnshift{}%
          \ifdiglotN@rmal%are the sides swapped?
            \hbox to\columnLwidth{\copy\Lbox\hss}\ifNoteGutterRule\makediglotgutter{\dimen7}{\dimen8}{\dimen9}\else\hskip\gutter\fi%
            \hbox to\columnshift{}%
            \hbox to\columnRwidth{\copy\Rbox\hss}%\rlap{\the\ht\Rbox \the\dp\Rbox}%
          \else%
            \hbox to\columnRwidth{\copy\Rbox\hss}\ifNoteGutterRule\makediglotgutter{\dimen7}{\dimen8}{\dimen9}\else\hskip\gutter\fi%
            \hbox to\columnshift{}%
            \hbox to\columnLwidth{\copy\Lbox\hss}%\rlap{\the\ht\Rbox \the\dp\Rbox}%
          \fi%
        }%
        \trace{f}{Adjusting footnotebox}%
        \trace{D}{Raising footnotebox by \the\dimen3}%
        \global\setbox\partial=\vbox{\unvbox\partial\kern -\dimen3\vskip 0.5\AboveNoteSpace plus 1fil \box\Lbox}%
        }%
      \else%
        \trace{f}{No footnotes}%
      \fi%
    \fi %BalNotes
  \else %SepNotes
    \trace{f}{Merged notes}%
  \fi%
  \traceNum{d}{Forming page}%
  \trace{D}{PFM:\p@gefirstmark\space PBM:\p@gebotmark}%
  %\showbox\partial
  %\showthe\everyhbox
  %\showthe\leftskip
  %\showthe\rightskip
  \p@gestarttrue
  \def\pagecontents{%
   \setdiglotN@rmal
   \pickn@xtpartial
   \ifvoid\topins\else \unvbox\topins \vskip\skip\topins \fi%
   \ifdim\ht\topleftins>\ht\toprightins%
      \setbox\toprightins=\vbox to \ht\topleftins{\box\toprightins\vss}%
   \else%
      \setbox\topleftins=\vbox to \ht\toprightins{\box\topleftins\vss}%
   \fi%
   \dimen8=\ht\bottomleftins
   \ifdim \dimen8 <\ht\toprightins \dimen8=\ht\toprightins \fi
   \dimen9=\dp\topleftins
   \ifdim \dimen9 <\dp\toprightins \dimen9=\dp\toprightins \fi
   \trace{d}{Gutter \the\dimen8 hight, \the\dimen9 deep}%
   \dimen7=\dimen9
   \advance\dimen7 by \dimen8
   \hbox to \textwidth{%\vllap{\the\ht\Lbox \the\dp\Lbox}%
     \hbox to\columnshift{}%
     \ifdiglotN@rmal%Are the sides swapped?
       \hbox to\columnLwidth{\box\topleftins\hss}\ifFigGutterRule\makediglotgutter{\dimen8}{\dimen8}{\dimen9}\else\hskip\gutter\fi %
       \hbox to\columnRwidth{\box\toprightins\hss}%\rlap{\the\ht\Rbox \the\dp\Rbox}%
     \else%
       \hbox to\columnRwidth{\box\toprightins\hss}\ifFigGutterRule\makediglotgutter{\dimen8}{\dimen8}{\dimen9}\else\hskip\gutter\fi %
       \hbox to\columnLwidth{\box\topleftins\hss}%\rlap{\the\ht\Rbox \the\dp\Rbox}%
     \fi%
   }%
   \lastd@pth=\dp\partial\relax%
   \unvbox\partial%
   \ifdim\ht\bottomleftins>\ht\bottomrightins%
      \setbox\bottomrightins=\vbox to \ht\bottomleftins{\box\bottomrightins\vss}%
   \else%
      \setbox\bottomleftins=\vbox to \ht\bottomrightins{\box\bottomleftins\vss}%
   \fi%
   \dimen8=\ht\bottomleftins
   \ifdim \dimen8 <\ht\bottomrightins \dimen8=\ht\bottomrightins \fi
   \dimen9=\dp\bottomleftins
   \ifdim \dimen9 <\dp\bottomrightins \dimen9=\dp\bottomrightins \fi
   \trace{d}{Gutter \the\dimen8 hight, \the\dimen9 deep}%
   \dimen7=\dimen9
   \advance\dimen7 by \dimen8
   \hbox to \textwidth{%\vllap{\the\ht\Lbox \the\dp\Lbox}%
     \hbox to\columnshift{}%
     \ifdiglotN@rmal%Are the sides swapped?
       \hbox to\columnLwidth{\box\bottomleftins\hss}\ifFigGutterRule\makediglotgutter{\dimen8}{\dimen8}{\dimen9}\else\hskip\gutter\fi %
       \hbox to\columnRwidth{\box\bottomrightins\hss}%\rlap{\the\ht\Rbox \the\dp\Rbox}%
     \else%
       \hbox to\columnRwidth{\box\bottomrightins\hss}\ifFigGutterRule\makediglotgutter{\dimen8}{\dimen8}{\dimen9}\else\hskip\gutter\fi %
       \hbox to\columnLwidth{\box\bottomleftins\hss}%\rlap{\the\ht\Rbox \the\dp\Rbox}%
     \fi%
   }%
   \ifvoid\bottomins\else%\kern-\lastd@pth \dimen0=0pt 
     \vskip\skip\bottomins \unvbox\bottomins \fi%
   \ifdiglotSepNotes%
   \else%
     % RTLness will have been asserted (or not) when the note was defined.
     {%
     \ifdiglot\RTLfalse\fi
     \def\c@rrdstat{L}%
     \traceNum{d}{Inserting Notes}%
     \f@rstnotetrue%
     \m@kenotebox
     \unvbox2 % defined by m@kenotebox
     }%
   \fi%
   \ifvoid\verybottomins\else % \kern-\dimen0
     \lastd@pth=0pt \vskip\skip\verybottomins \hbox{\hbox to \columnshift{}\vbox{\unvbox\verybottomins}}\fi
  }%pagecontents
  \trace{D}{PFM:\p@gefirstmark\space PBM:\p@gebotmark\space NPFM:\nextp@gefirstmark}%
  \global\adjustp@ge=0pt
  \resetvsize%
  \global\availhtR=\textheight%
  \global\availhtL=\textheight%
  \global\availht=\textheight%
  \global\vsize=\availht
  \plainoutput%
  \global\chunkDelta=0pt
  \global\pagefullfalse%
  \p@gestartfalse
  \xdef\p@gefirstmark{\nextp@gefirstmark}%
  \xdef\p@gebotmark{}%
  \xdef\nextp@gefirstmark{}%
  \let\col@do=\reset@marks
  \x@\each@col\diglot@list\E% 
  \setc@rdstat{\f@lwgdstat}%Set the current diglotstatus
  \xdef\firstLmark{}%
  \xdef\botLmark{}%
  \xdef\firstRmark{}%
  \xdef\botRmark{}%
  \global\setbox\Rbox=\box\voidb@x%
  \global\setbox\Lbox=\box\voidb@x%
  \nextshipout
  %\nonstopmode
  %\showthe\output
  %%somejunk
}

\def\diglotCollect{%
  %This gets called to collect material into the appropriate input queue.
  % assumes: \availht is correct, 
  % Fills: \excessX
  % Empties: \box255
  \trace{D}{diglotCollect O:\the\outputpenalty, P:\the\savedpenalty, vs:\the\vsize}%
  \x@\let\x@\destb@x\csname excess\c@rrdstat\endcsname
  \diglot@any@primary
  \ifnum\outputpenalty=\dgl@tPenLtrialEnd\relax 
    \trace{D}{Why are we in diglotCollect after a trial?}%
    \ifintrial\diglot@any@tstored\fi
  \fi
  % tempoarily split and see if there are marks in this text
  \bgroup\setbox0=\copy255 \setbox1=\vsplit0 to \maxdimen\egroup
  \edef\t@mp{\splitbotmark}%
  \ifx\t@mp\empty\else\global\m@rksonpagetrue\trace{H}{Found mark \splitbotmark}\fi
  \makevtop{255}%
  \trace{D}{dp:\the\dp255}%
  \tmppenalty=\outputpenalty %
  \ifnum\outputpenalty=10000 % Nothing wrong with breaking here
    \tmppenalty=0\relax%
  \fi
  \ifvoid\destb@x
    \global\setbox\destb@x=\box255
  \else 
    \global\setbox\destb@x=\vtop{\joinboxes{\destb@x}{255}{1}{\csname partial\c@rrdstat penalty\endcsname}}%
  \fi
  \x@\csname partial\c@rrdstat penalty\endcsname=\tmppenalty
  \ifnum\outputpenalty=\dgl@tPenColSwap\relax %Got to the end of the \lefttext
    \trace{D}{Reached end of diglotCollect. \ifallNeedEmptying Running trials\fi}%
    \tmppenalty=\savedpenalty\relax%
    \ifallNeedEmptying
      \global\let\n@xtc@mmand=\diglot@run@trials
      \global\allNeedEmptyingfalse
    \fi
  \fi
}


% for 'Mark' activity around n@xtpartial and aborted trials.
\def\reset@marks#1{% Called at page output
  \trace{H}{Resetting first#1mark and friends}%
  \x@\xdef\csname first#1mark\endcsname{}%
  \x@\xdef\csname oldbot#1mark\endcsname{\csname bot#1mark\endcsname}%
  \x@\xdef\csname bot#1mark\endcsname{}%
}
\def\reset@n@xt#1{% Called at page output
  \trace{H}{Resetting n@xtfirst#1mark}%
  \x@\xdef\csname n@xtfirst#1mark\endcsname{}%
}
\def\store@marks#1{%Save the current mark state
  \trace{H}{Storing pre@ptl ... marks}%
  \x@\xdef\csname pre@ptlfirst#1mark\endcsname{\csname first#1mark\endcsname}%
  \x@\xdef\csname pre@ptlbot#1mark\endcsname{\csname bot#1mark\endcsname}%
}

\def\sw@p@marks#1#2{%
  \edef\tmpmark{\csname pre@ptl#1#2mark\endcsname}%
  \x@\xdef\csname pre@ptl#1#2mark\endcsname{\csname old#1#2mark\endcsname}%
  \x@\xdef\csname old#1#2mark\endcsname{\tmpmark}%
}

\def\swap@marks#1{%Swap the two saved states 
  \trace{H}{Swapping pre@ptl ... and old marks}%
  \sw@p@marks{first}{#1}%
  \sw@p@marks{bot}{#1}%
}

%\def\restore@botmark#1{%
  %\x@\xdef\csname bot#1mark\endcsname{\csname pre@ptlbot#1mark\endcsname}%
%} 

\def\restore@oldmarks#1{%Restore saved mark state
  \trace{H}{restoring old marks (#1)}%
  \x@\xdef\csname first#1mark\endcsname{\csname oldfirst#1mark\endcsname}%
  \x@\xdef\csname bot#1mark\endcsname{\csname oldbot#1mark\endcsname}%
}
\def\restore@storedmarks#1{%Restore saved mark state
  \trace{H}{restoring stored marks}%
  \x@\xdef\csname first#1mark\endcsname{\csname pre@ptlfirst#1mark\endcsname}%
  \x@\xdef\csname bot#1mark\endcsname{\csname pre@ptlbot#1mark\endcsname}%
}

\def\p@rtialmarks#1{%p@rtial mark.
  \store@marks{#1}%
  \trace{H}{using n@xtmarks: first#1mark->\csname n@xtfirst#1mark\endcsname, bot#1mark->\csname n@xtbot#1mark\endcsname}%
  \x@\let\x@\test\csname first#1mark\endcsname
  \ifx\test\empty 
    \x@\xdef\csname first#1mark\endcsname{\csname n@xtfirst#1mark\endcsname}%
  \fi
  \x@\xdef\csname bot#1mark\endcsname{\csname n@xtbot#1mark\endcsname}%
}

%%%%%%%%%
% \Xht calculations, for 'snuggling' footnotes.
\def\@curLht{0pt}
\let\@curRdp\@curLht
\let\@curRht\@curLht
\let\@curRdp\@curLht

\def\save@reset@ht#1{%Save the value of \Xht into \@nxtchunkXht, and set it to 0
  \x@\xdef\csname @nxtchunk#1ht\endcsname{\x@\the\csname #1ht\endcsname}%
  \x@\global\x@\csname #1ht\endcsname=0pt
}
\def\save@boxdim#1{%Save the height & depth of Xbox in \v@lpfx#1ht
  \x@\xdef\csname \v@lpfx#1ht\endcsname{\x@\the\x@\ht\csname #1box\endcsname}%
  \x@\xdef\csname \v@lpfx#1dp\endcsname{\x@\the\x@\dp\csname #1box\endcsname}%
} 
\def\check@nxtht#1{%If the nxtchunk#1 height =0 for a box? Set iftemp if yes 
  \x@\ifdim\csname @nxtchunk#1ht\endcsname=0pt
    \temptrue
  \fi
}
\def\set@ht#1{%Set the values of curXht, using method based on \iftemp  (anything on the page is current by definition)
  \x@\dimen1\x@=\csname @nxtchunk#1ht\endcsname
  \x@\dimen0\x@=\csname @cur#1ht\endcsname
  \x@\xdef\csname @old#1ht\endcsname{\the\dimen0}%
  \iftemp
    \advance\dimen1 by \dimen0
  \fi
  \x@\xdef\csname @cur#1ht\endcsname{\the\dimen1}%
}

\def\max@val#1{\x@\let\x@\t@st\csname \v@lpfx#1\v@lsfx\endcsname
  \ifdim\dimen1<\t@st
    \dimen1=\t@st
  \fi
}
\def\min@note@sep#1{%
  \x@\let\x@\t@st\csname \v@lpfx#1ht\endcsname
  \x@\let\x@\t@stb\csname \v@lpfxb#1ht\endcsname
  \dimen0=\dimen2
  \message{Notesep: (#1) d2:\the\dimen2 - \t@st -\t@stb}%
  \advance\dimen0 by -\t@st\relax
  \advance\dimen0 by -\t@stb\relax
  \ifdim\dimen1>\dimen0
    \dimen1=\dimen0
  \fi
  \trace{fS}{Notesep for #1 is \the\dimen0, min: \the\dimen1}%
}


\def\each@col#1#2\E{% General purpose looper
  %\tracingmacros=1 \tracingassigns=1
  \trace{De}{each@col #1#2}%
  \edef\t@st{#2}%
  \ifx\t@st\empty %#2 is empty
    \let\n@xt@col=\cstackrelax
  \else
    \let\n@xt@col=\each@col
  \fi
  \col@do{#1}%
  \n@xt@col#2\E
}


\newif\ifnot@empty % Is this chunk empty

\def\@add@n@xtpartial{% 
  \trace{d}{@add@n@xtpartial \the\ht\partial+\the\dp\partial, \the\ht\n@xtpartial+\the\dp\n@xtpartial}%
  \ifvoid\n@xtpartial\else
    \ifvoid\partial
      \global\setbox\partial=\box\n@xtpartial
    \else
      \global\setbox\partial=\vtop{\joinb@xes{\partial}{\n@xtpartial}}%
    \fi
    \voidn@xtpartial
    \let\col@do=\p@rtialmarks
    \x@\each@col\diglot@list\E
    % Column height (depth) calcs:
    \tempfalse
    \trace{fS}{Checking for empty boxes}%
    \let\col@do=\check@nxtht %sets iftemp
    \x@\each@col\diglot@list\E
    \trace{fS}{Saving new Xht values}%
    \let\col@do=\set@ht %uses iftemp, 
    \x@\each@col\diglot@list\E
    %
    %\global\p@geChunkDelta=\chunkDelta%
    \global\lastchunkpenalty=\chunkpenalty
    \global\chunkpenalty=-20000 %Everything ought to be bigger than this.
  \fi
}
\def\postponen@xt@partial{%
  \trace{d}{postponen@xtpartial \the\ht\partial+\the\dp\partial, \the\ht\n@xtpartial+\the\dp\n@xtpartial}%
  \global\late@stage@reflowtrue % Need to re-flow all columns.
  \r@storenotes{\cur@ds}{1}%
}

\def\tri@llist{}%Trials to be run.

\def\n@xttrial#1#2\E{%
  \xdef\tri@llist{#2}%
  \trace{D}{n@xttrial #1 (#2)}%
  \setc@rdstat{#1}
}

\def\ch@ckcontents#1#2\E{%
  %sets various state flags and fills the list of trials 
  %left to run.
  %Assumption: called initially with a list of possibly-containing-stuff
  %columns, and empty \tri@llist.
  \edef\c@rrdstat{#1}%
  \edef\oldavht{\x@\the\csname availht\c@rrdstat\endcsname}%
  \x@\ifvoid\csname \c@rrdstat box\endcsname \else
    \trace{D}{side \c@rrdstat\space has material to put on the page}%
    \global\not@emptytrue
    \dglt@calc@vailht
    \ifdim \availht<-2pt 
      \trace{D}{No space (\the\availht) for material (now, was \oldavht)}%
      \late@stage@reflowtrue
      \xdef\refl@wlist{\refl@wlist\c@rrdstat}%
    \fi
  \fi
  \x@\ifvoid\csname excess\c@rrdstat\endcsname
    \trace{D}{side \c@rrdstat\space is finished for now}%
  \else
    \trace{D}{side \c@rrdstat\space has stuff to typeset}%
    \xdef\tri@llist{\tri@llist#1}% More trials to run 
  \fi
  \edef\n@xtarg{#2}%
  \let\afterch@ckcontents\ch@ckcontents
  \ifx\n@xtarg\empty
    \let\afterch@ckcontents\cstackrelax
  \fi
  \x@\afterch@ckcontents\n@xtarg\E
}

\edef\diglot@list{LR}%
\edef\digl@tc@ntentlist{}%

\def\new@diglot@col#1{%
 \xdef\diglot@list{\diglot@list#1}%
 \errmessage{This code isn't written yet}%
 %CREATE: partial#1 excess#1 partial#1penalty m@rknum#1  column#1width picb@x#1
 %FIXMEs needing to be solved before implementing this: 
 % cle@rn@tecl@ss
 %pdfb@@km@rkR 
 %pdfch@pterm@rkR 
 % UI!
 % 
}

\def\check@and@update{%Checks each column to see if there is stuff to output
  \trace{d}{check@and@update \the\triall@@pcount}%
  %and calls upd@tep@rtial as appropriate
  %responds to \ifboxmoved and \pagefull, checking the last chunkpenalty
  \not@emptyfalse
  \x@\ch@ckcontents\diglot@list\E%Sets not@empty if there's stuff to output
  \ifboxmoved
    \ifnum\lastchunkpenalty>9999
      \postponen@xt@partial
      \let\col@do=\restore@storedmarks
      \x@\each@col\diglot@list\E% 
      \@writep@ge
      \@add@n@xtpartial
      \let\col@do=\reset@n@xt
      \x@\each@col\diglot@list\E% 
    \else
      \@add@n@xtpartial
      \@writep@ge
      \let\col@do=\reset@n@xt
      \x@\each@col\diglot@list\E% 
    \fi
    \global\boxmovedfalse
  \else
    \ifnot@empty 
      \global\late@stage@reflowfalse
    \else
      \ifx\tri@llist\empty\else
        \ifnum\triall@@pcount>10
          \pagefulltrue
        \fi
      \fi
    \fi
    \upd@tep@rtial
  \fi
}

\newcount\triall@@pcount

\def\diglot@run@trials{% for X as L, R, (eventually others)
  %expects: collected input in partialX, excessX empty
  %Process: cycle through chunks  
  \traceNum{d}{diglot@run@trials (\digl@tc@ntentlist)}%
  \let\tri@llist\diglot@list%
  %Before running trials, empty(eventually store?) old galleys
  \empty@partials
  \let\tri@llist\digl@tc@ntentlist% Run trials for things that have contents
  \xdef\digl@tc@ntentlist{}%
  \global\late@stage@reflowfalse
  \def\refl@wlist{}% List of columns needing to reflow.
  \ifx\empty\tri@llist\else
    \triall@@pcount=1
    \diglot@run@nxt@trial
  \fi
}

\def\empty@partials{%Empty (eventually store?) partialX
  \ifx\tri@llist\empty\else 
    \loop
      \x@\n@xttrial\tri@llist\E
      \trace{D}{Blanking partial\c@rrdstat}%
      \x@\setbox\csname partial\c@rrdstat\endcsname=\box\voidb@x
    \unless\ifx\tri@llist\empty\repeat
  \fi
}

\def\diglot@run@nxt@trial{%
  \traceNum{d}{diglot@run@nxt@trial}%
  \x@\n@xttrial\tri@llist\E
  \ifx\c@rrdstat\empty %End of the loop.
  \else
    \ifnum 1=\x@\ifvoid\csname excess\c@rrdstat\endcsname \x@\ifvoid\csname partial\c@rrdstat\endcsname 0\else 1\fi\else 1\fi
      \dglt@calc@vailht
      %\ifnum\triall@@pcount=1
        %\showbox\partialL\showbox\excessL
      %\fi
      \run@a@trial
      \relax
    \fi
  \fi
  \relax\relax
}

\def\after@after@a@trial{%
  \trace{d}{after@after@a@trial iftrialfailed\iftrialfailed true\else false\fi,  \the\triall@@pcount}%
  % Check if the run failed (or didn't even run), and update page if that's 
  % appropriate.
  \iftrialfailed
    \ifnum\triall@@pcount=1
      \global\boxmovedtrue %Only count box as moved if the first chunk doesn't fit. 
    \fi
    \global\pagefulltrue
  \else
    \x@\let\x@\tmppen\csname \c@rrdstat boxpenalty\endcsname
    %\tracingassigns=0
    \ifnum \tmppen>\chunkpenalty
      \chunkpenalty=\tmppen
    \fi
  \fi
  \ifboxmoved % Breaks things?
    %things gone from the triallist should be reflowed,
    % for other things, partialX still contains the previous galley
    \empty@partials% empties tri@llist
  \fi
  %\showlists
  \trace{D}{tri@llist: \tri@llist}%
  \ifx\tri@llist\empty
    \advance\triall@@pcount by 1
    \check@and@update %May trigger a refilling of \tri@llist
    \trace{D}{after check@and@update, tri@llist: \tri@llist, refl@wlist:\refl@wlist}%
    \ifx\tri@llist\empty
      \let\n@xt=\relax
    \else
      \let\n@xt=\diglot@run@nxt@trial
    \fi
  \else
    \let\n@xt=\diglot@run@nxt@trial
  \fi
%  \tracingifs=0
  \n@xt
}   

\def\enddigl@t{%
  \endgraf
  \ifdiglot
    \ifsk@pping \egroup \fi% if we were skipping nonpublishable text, end that mode
    \sk@ppingfalse
    \ifhe@dings\endhe@dings\fi%
    \trace{d}{enddigl@t \the\ht\n@xtpartial+\the\dp\n@xtpartial, \the\ht\partial+\the\dp\partial}%
    \eject
    \global\allNeedEmptyingtrue
    \diglot@run@trials
    \penalty\dgl@tPenColSwap
    \global\allNeedEmptyingtrue
    \diglot@run@trials
    \ifnum\interactionmode=2
      \showbox\n@xtpartialNrml\showbox\n@xtpartialRev
    \fi
    \@add@n@xtpartial
    \trace{d}{After that; \the\ht\partial+\the\dp\partial, \the\ht\n@xtpartial+\the\dp\n@xtpartial}%
    \ifvoid\partial\else
      \@writep@ge
      \let\col@do=\reset@n@xt
      \x@\each@col\diglot@list\E% 
    \fi
    \pr@sideswitch{end}{L}%
  \fi
  %FIXME! If \ifendbooknoeject  happens to be true, then what? 
  %Set the 2 columns, put the footnotes up in the air and carry on??
}

\def\makediglotgutter#1#2#3{\hbox to \gutter{\hss
%rule is #1 hight, #3 deep, and fits in a box #2 high
  \trace{D}{makediglotgutter \the #1,\the #2,\the #3}%
  \dimen4=#1\advance \dimen4 by #3
   \setbox4=\vbox to #2{%
     \hbox to 1pt{%
       \ifColumnGutterRule
        \vrule height \the\dimen4 
       \fi
       \hfil}%\dp5=#3 \box5
      \vss}%
    \dp4=#3
    \box4
  \hss}}

\def\diglotgr@db@x#1{%
 \ifdiglot\vbox{\gr@db@@x{#1}\penalty 10000}\penalty10000\else\gr@db@@x{#1}\fi
}
\def\testmarker{\hbox to 0pt{\vrule height 7pt depth 0pt width 0.5pt \kern-0.5pt}\message{_}}

\addtoendptxhooks{%
       \bgroup\ifdiglot\ifnum\pageno>0 \count255=\cumulativeDelta \divide\count255 by \numexpr 1+\pageno-\ptxst@rtp@ge\relax \message{Column delta for book:\the\cumulativeDelta pt (\the\count255 pt/ pg)\space(-ve means left longer)}\fi\fi\egroup\cumulativeDelta=0\relax
 }%
\addtoendhooks{\ifdiglot\ifvoid\partial\else\@writep@ge\fi\fi}

\endinput
