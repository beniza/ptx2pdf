%%%%%%%%%%%%%%%%%%%%%%%%%%%%%%%%%%%%%%%%%%%%%%%%%%%%%%%%%%%%%%%%%%%%%%%
% Part of the ptx2pdf macro package for formatting USFM text
% copyright (c) 2007 by SIL International
% written by Jonathan Kew
%
% Permission is hereby granted, free of charge, to any person obtaining  
% a copy of this software and associated documentation files (the  
% "Software"), to deal in the Software without restriction, including  
% without limitation the rights to use, copy, modify, merge, publish,  
% distribute, sublicense, and/or sell copies of the Software, and to  
% permit persons to whom the Software is furnished to do so, subject to  
% the following conditions:
%
% The above copyright notice and this permission notice shall be  
% included in all copies or substantial portions of the Software.
%
% THE SOFTWARE IS PROVIDED "AS IS", WITHOUT WARRANTY OF ANY KIND,  
% EXPRESS OR IMPLIED, INCLUDING BUT NOT LIMITED TO THE WARRANTIES OF  
% MERCHANTABILITY, FITNESS FOR A PARTICULAR PURPOSE AND  
% NONINFRINGEMENT. IN NO EVENT SHALL SIL INTERNATIONAL BE LIABLE FOR  
% ANY CLAIM, DAMAGES OR OTHER LIABILITY, WHETHER IN AN ACTION OF  
% CONTRACT, TORT OR OTHERWISE, ARISING FROM, OUT OF OR IN CONNECTION  
% WITH THE SOFTWARE OR THE USE OR OTHER DEALINGS IN THE SOFTWARE.
%
% Except as contained in this notice, the name of SIL International  
% shall not be used in advertising or otherwise to promote the sale,  
% use or other dealings in this Software without prior written  
% authorization from SIL International.
%%%%%%%%%%%%%%%%%%%%%%%%%%%%%%%%%%%%%%%%%%%%%%%%%%%%%%%%%%%%%%%%%%%%%%%

% Macros to deal with Scripture references (book, chapter, verse) and running headers

% These macros work on the reference information in the format "Book:C:V"
% that is embedded in the \mark at each verse number.

% Note that "V" may be a verse range, if the USFM data included bridged verses such as "\v 12-15".
% Therefore, we have to do some extra work to extract individual verse numbers.

\newif\ifOmitChapterNumberRH % make this true to omit the ch # (eg for single-chapter books)

%
% Print the first reference from the mark data
%
\newif\ifOmitBookRef
\def\f@rstref#1{\edef\t@st{#1}%
 \ifx\t@st\empty\else\x@\extr@ctfirst\t@st\relax
  \ifx\@book\empty\else
   \ifOmitBookRef\else\@book\ \fi
   \ifOmitChapterNumberRH\else\@chapter\fi
   \ifVerseRefs
    \ifOmitChapterNumberRH\else\cvs@p\fi
    \@verse
   \fi
  \fi
 \fi}

%
% Print the last reference from the mark data
% (same as above unless there are bridged verses)
%
\def\l@stref#1{\edef\t@st{#1}%
 \ifx\t@st\empty\else\x@\extr@ctlast\t@st\relax
 \ifx\@@book\empty\else\ifx\@@book\t@tle\else\ifOmitBookRef\else\@@book\ \fi\ifOmitChapterNumberRH\else\@@chapter\fi\ifVerseRefs\ifOmitChapterNumberRH\else\cvs@p\fi\@@verse\fi\fi\fi\fi}

%
% Print the range of references from a pair of marks
%
\edef\emptym@rk{::}
\edef\n@thing{}
\def\r@ngerefs#1#2{%
 \edef\t@st{#1}\ifx\t@st\t@tle\x@\extr@ctfirst\emptym@rk\relax\else\ifx\t@st\n@thing\x@\extr@ctfirst\emptym@rk\relax\else\x@\extr@ctfirst\t@st\relax\fi\fi
 \edef\t@st{#2}\ifx\t@st\t@tle\x@\extr@ctlast\emptym@rk\relax\else\ifx\t@st\n@thing\x@\extr@ctlast\emptym@rk\relax\else\x@\extr@ctlast\t@st\relax\fi\fi
 \ifx\@book\empty\else
  \@book\ 
  \ifVerseRefs
   \ifOmitChapterNumberRH\else
    \ifx\@chapter\empty\else\@chapter\cvs@p\fi\@verse
    \ifx\@book\@@book
     \ifx\@chapter\@@chapter
      \ifx\@verse\@@verse\else\ranges@p\@@verse\fi
     \else
      \ranges@p\ifOmitChapterNumberRH\else\@@chapter\cvs@p\@@verse\fi
     \fi
    \else
     \ranges@p\@@book\ \ifOmitChapterNumberRH\else\@@chapter\cvs@p\@@verse\fi
    \fi
   \fi
  \else
   \ifOmitChapterNumberRH\else
    \ifx\@chapter\empty\else
     \@chapter
     \ifx\@chapter\@@chapter\else
      \setbox0=\hbox{\tracinglostchars=0
      \global\c@untA=0\@chapter \global\c@untB=0\@@chapter}%
      \advance\c@untA by 1
      \ifnum\c@untA=\c@untB \pairs@p \else \ranges@p \fi
      \@@chapter
     \fi
    \fi
   \fi
  \fi
 \fi}
\newcount\c@untA \newcount\c@untB

%
% extract the starting reference of a (possible) range
% putting the result into \@book, \@chapter, \@verse
%
\def\extr@ctfirst#1:#2:#3\relax{%
 \def\@book{#1}\def\@chapter{#2}\def\t@st{#3}%
 \x@\spl@tverses\t@st--\relax
 \edef\@verse{\v@rsefrom}}

%
% extract the ending reference of a (possible) range
% putting the result into \@@book, \@@chapter, \@@verse
%
\def\extr@ctlast#1:#2:#3\relax{%
 \def\@@book{#1}\def\@@chapter{#2}\def\t@st{#3}%
 \x@\spl@tverses\t@st--\relax
 \edef\@@verse{\v@rseto}}

%
% split a possible verse range on hyphen, setting \v@rsefrom and \v@rseto
%
\def\spl@tverses#1-#2-#3\relax{%
 \edef\v@rsefrom{#1}\edef\v@rseto{#2}%
 \ifx\v@rseto\empty\let\v@rseto=\v@rsefrom\fi}

\newif\ifVerseRefs % whether to include verse numbers, or only book+chapter

%
% Specify separators to use when constructing references
%
\def\ranges@p{\hbox{\RangeSeparator}} % box this to avoid possible bidi problems
%\def\pairs@p{,\kern.2em}
\let\pairs@p\ranges@p
\def\cvs@p{\hbox{\ChapterVerseSeparator}}
\def\endash{\char"2013\relax}

\def\RangeSeparator{\kern.1em\endash\kern.1em} % what to put between first - last of a range
\def\ChapterVerseSeparator{\kern.02em.\kern.02em} % what to put between chapter:verse

%
% Running headers/footers that may use the references defined above
%


% define \headline for use by the output routine
\headline={\hbox to \textwidth{%
 \s@tfont{h}%
 \ifm@rksonpage\trace{H}{marks on page pfm:\p@gefirstmark}\else\trace{H}{No marks on page. old pfm:\p@gefirstmark}\gdef\p@gefirstmark{}\fi
 \edef\t@st{\p@gefirstmark}% check first mark on page to see if this is a "title" page
 \global\rhr@letrue
 \ifnum\ifx\t@st\empty\ifp@geone 0\else 1\fi\else 0\fi =1
  \trace{h}{no p@gefirstmark. Page 1 is \ifp@geone True\else False\fi} 
    \ifodd\pageno
      \the\noVoddhead
      \trace{h}{using noVoddhead} 
    \else
      \ifDoubleSided \trace{h}{using noVevenhead} \the\noVevenhead \else \trace{h}{using noVoddhead} \the\noVoddhead\fi\fi

  %\global\rhr@lefalse
 \else
  \ifendbooknoeject
    \trace{h}{endbooknoeject, so treating as normal page}
    \ifodd\pageno
      \the\oddhead
    \else
      \ifDoubleSided \the\evenhead \else \the\oddhead \fi
    \fi
  \else
    \trace{h}{Testing for \t@tle\space in \t@st}%
    \ifnum\ifx\t@st\t@tle 1\else\ifp@geone 1\else 0\fi\fi =1
      \the\titlehead \global\rhr@lefalse
    \else\ifx\t@st\n@thing
      \the\titlehead \global\rhr@lefalse
      \else
        \ifodd\pageno
          \the\oddhead
        \else
          \ifDoubleSided \the\evenhead \else \the\oddhead\fi
        \fi
      \fi
    \fi
  \fi
 \fi}}

% default headers are made of three components, placed left, center and right
\newtoks\oddhead
\newtoks\evenhead
\newtoks\titlehead
\newtoks\noVoddhead
\newtoks\noVevenhead
\def\defineheads{
  \ifnum\BodyColumns=2
    \oddhead={\hbox to \columnshift{}\rlap{\RHoddleft}\hfil\hbox to 0.5\columnshift{}\RHoddcenter\hfil\hbox to \columnshift{}\llap{\RHoddright}}
    \evenhead={\hbox to \columnshift{}\rlap{\RHevenleft}\hfil\hbox to 0.5\columnshift{}\RHevencenter\hfil\hbox to \columnshift{}\llap{\RHevenright}}
    \noVoddhead={\hbox to \columnshift{}\rlap{\RHnoVoddleft}\hfil\hbox to 0.5\columnshift{}\RHnoVoddcenter\hfil\hbox to \columnshift{}\llap{\RHnoVoddright}}
    \noVevenhead={\hbox to \columnshift{}\rlap{\RHnoVevenleft}\hfil\hbox to 0.5\columnshift{}\RHnoVevencenter\hfil\hbox to \columnshift{}\llap{\RHnoVevenright}}
    \titlehead={\hbox to \columnshift{}\rlap{\RHtitleleft}\hfil\hbox to 0.5\columnshift{}\RHtitlecenter\hfil\hbox to \columnshift{}\llap{\RHtitleright}}
  \else
    \oddhead={\hbox to \columnshift{}\rlap{\RHoddleft}\hfil\hbox to 0.5\columnshift{}\RHoddcenter\hfil\llap{\RHoddright}}
    \evenhead={\hbox to \columnshift{}\rlap{\RHevenleft}\hfil\hbox to 0.5\columnshift{}\RHevencenter\hfil\llap{\RHevenright}}
    \noVoddhead={\hbox to \columnshift{}\rlap{\RHnoVoddleft}\hfil\hbox to 0.5\columnshift{}\RHnoVoddcenter\hfil\llap{\RHnoVoddright}}
    \noVevenhead={\hbox to \columnshift{}\rlap{\RHnoVevenleft}\hfil\hbox to 0.5\columnshift{}\RHnoVevencenter\hfil\llap{\RHnoVevenright}}
    \titlehead={\hbox to \columnshift{}\rlap{\RHtitleleft}\hfil\hbox to 0.5\columnshift{}\RHtitlecenter\hfil\llap{\RHtitleright}}
  \fi
}

% \footline is similar except it doesn't have to place a border graphic
\footline={{%
 \s@tfont{h}%
 \edef\t@st{\p@gefirstmark}% check first mark on page to see if this is a "title" page
 \ifnum\ifx\t@st\empty\ifp@geone 0\else 1\fi\else 0\fi =1
   \ifodd\pageno \the\noVoddfoot \else 
     \ifDoubleSided \the\noVevenfoot \else \the\noVoddfoot \fi\fi
 \else
   \ifnum\ifx\t@st\t@tle 1\else\ifp@geone 1\else 0\fi\fi =1 \the\titlefoot \else
     \ifodd\pageno \the\oddfoot \else
       \ifDoubleSided \the\evenfoot \else \the\oddfoot \fi
 \fi\fi\fi}}
\newtoks\oddfoot
\newtoks\evenfoot
\newtoks\titlefoot
\newtoks\noVoddfoot
\newtoks\noVevenfoot
\oddfoot={\rlap{\RFoddleft}\hfil\RFoddcenter\hfil\llap{\RFoddright}}
\evenfoot={\rlap{\RFevenleft}\hfil\RFevencenter\hfil\llap{\RFevenright}}
\noVoddfoot={\rlap{\RFnoVoddleft}\hfil\RFnoVoddcenter\hfil\llap{\RFnoVoddright}}
\noVevenfoot={\rlap{\RFnoVevenleft}\hfil\RFnoVevencenter\hfil\llap{\RFnoVevenright}}
\titlefoot={\rlap{\RFtitleleft}\hfil\RFtitlecenter\hfil\llap{\RFtitleright}}


\def\s@treffont{\ifuseLeftMarks\ifuseRightMarks\else\diglotLtrue\s@tfont{h}\fi\else\ifuseRightMarks\diglotLfalse\s@tfont{h}\fi\fi}
\def\headfootL#1{\bgroup\ifRTL\beginR\fi\diglotLtrue\s@tfont{h}#1\ifRTL\endR\fi\egroup}
\def\headfootR#1{\bgroup\ifRTL\beginR\fi\diglotLfalse\s@tfont{h}#1\ifRTL\endR\fi\egroup}
%
% user-level macros for use within the running header
%
\def\firstref{\ifRTL\beginR\fi\bgroup\ifdiglot\s@treffont\fi\f@rstref{\p@gefirstmark}\egroup\ifRTL\endR\fi}
\def\lastref{\ifRTL\beginR\fi\bgroup\ifdiglot\s@treffont\fi\l@stref{\p@gebotmark}\egroup\ifRTL\endR\fi}
\def\rangeref{\ifRTL\beginR\fi\bgroup\ifdiglot\s@treffont\fi\r@ngerefs{\p@gefirstmark}{\p@gebotmark}\egroup\ifRTL\endR\fi}

\def\firstrefL{\headfootL{\f@rstref{\firstLmark}}}
\def\lastrefL{\headfootL{\l@stref{\botLmark}}}
\def\rangerefL{\headfootL{\r@ngerefs{\firstLmark}{\botLmark}}}

\def\firstrefR{\headfootR{\f@rstref{\firstRmark}}}
\def\lastrefR{\headfootR{\l@stref{\botRmark}}}
\def\rangerefR{\headfootR{\r@ngerefs{\firstRmark}{\botRmark}}}

\let\pagenumber=\folio
\def\pagenumberL{\headfootL{\folio}}
\def\pagenumberR{\headfootR{\folio}}

\def\ROMANnumeral#1{\expandafter\uppercase\expandafter{\romannumeral#1}}
\def\usdate{\number\month/\number\day/\number\year}
\def\usdateL{\headfootL{\usdate}}
\def\usdateR{\headfootR{\usdate}}
\def\ukdate{\number\day/\number\month/\number\year}
\def\ukdateL{\headfootL{\ukdate}}
\def\ukdateR{\headfootR{\ukdate}}
\def\isodate{\number\year-\ifnum\month<10 0\fi\number\month-\ifnum\day<10 0\fi\number\day}
\def\isodateL{\headfootL{\isodate}}
\def\isodateR{\headfootR{\isodate}}

\def\hrsminsL{\headfootL{\hrsmins}}
\def\hrsminsR{\headfootR{\hrsmins}}
\def\timestampL{\headfootL{\timestamp}}
\def\timestampR{\headfootR{\timestamp}}



%
% default settings of the running header components
%
\def\RHoddleft{\empty}
\def\RHoddcenter{\rangeref}
\def\RHoddright{\pagenumber}

\def\RHevenleft{\pagenumber}
\def\RHevencenter{\rangeref}
\def\RHevenright{\empty}

\def\RHnoVoddleft{\empty}
\def\RHnoVoddcenter{\empty}
\def\RHnoVoddright{\pagenumber}

\def\RHnoVevenleft{\pagenumber}
\def\RHnoVevencenter{\empty}
\def\RHnoVevenright{\empty}

\def\RHtitleleft{\empty}
\def\RHtitlecenter{\empty}
\def\RHtitleright{\empty}

\def\RFoddleft{\empty}
\def\RFoddcenter{\empty}
\def\RFoddright{\empty}

\def\RFevenleft{\empty}
\def\RFevencenter{\empty}
\def\RFevenright{\empty}

\def\RFnoVoddleft{\empty}
\def\RFnoVoddcenter{\empty}
\def\RFnoVoddright{\empty}

\def\RFnoVevenleft{\empty}
\def\RFnoVevencenter{\empty}
\def\RFnoVevenright{\empty}

\def\RFtitleleft{\empty}
\def\RFtitlecenter{\pagenumber}
\def\RFtitleright{\empty}

\endinput
