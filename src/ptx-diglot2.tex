%:strip
% ptx-diglot.tex: Diglot(v2) processing for xetex paratext2.tex
% Copyright (c) 2008-2021 by SIL International 
% written by David Gardner
%
% Permission is hereby granted, free of charge, to any person obtaining
% a copy of this software and associated documentation files (the  
% "Software"), to deal in the Software without restriction, including  
% without limitation the rights to use, copy, modify, merge, publish,  
% distribute, sublicense, and/or sell copies of the Software, and to  
% permit persons to whom the Software is furnished to do so, subject to  
% the following conditions:
%
% The above copyright notice and this permission notice shall be  
% included in all copies or substantial portions of the Software.
%
% THE SOFTWARE IS PROVIDED "AS IS", WITHOUT WARRANTY OF ANY KIND,  
% EXPRESS OR IMPLIED, INCLUDING BUT NOT LIMITED TO THE WARRANTIES OF  
% MERCHANTABILITY, FITNESS FOR A PARTICULAR PURPOSE AND  
% NONINFRINGEMENT. IN NO EVENT SHALL SIL INTERNATIONAL BE LIABLE FOR  
% ANY CLAIM, DAMAGES OR OTHER LIABILITY, WHETHER IN AN ACTION OF  
% CONTRACT, TORT OR OTHERWISE, ARISING FROM, OUT OF OR IN CONNECTION  
% WITH THE SOFTWARE OR THE USE OR OTHER DEALINGS IN THE SOFTWARE.
%
% Except as contained in this notice, the name of SIL International  
% shall not be used in advertising or otherwise to promote the sale,  
% use or other dealings in this Software without prior written  
% authorization from SIL International.
%%%%%%%%%%%%%%%%%%%%%%%%%%%%%%%%%%%%%%%%%%%%%%%%%%%%%%%%%%%%%%%%%%%%%%%

\newcount\diglotDbgJoinboxes%
\diglotDbgJoinboxes=-1% Set to the debug message of a joinboxes to execute showbox on that join

\def\diglotbadbrk{\penalty 10000}

%\def\TRshipout#1{\shipout#1}
\def\TRshipout#1{}
\def\b@xbotmark{}

% THis is a crude hack to make main titles line up nicely. For some reason
% they seem to already be vtops with depth info lost. 
%
\def\dstrut#1{{\setbox0\hbox{#1}\hbox{\vrule height \ht0 depth \dp0 width 0pt}}}
%
%
\newif\ifdiglotInnerOuter% Do pages switch columns based on page number (odd pages Left-Right, even Right-left)?
\newif\ifdiglotSwap% Do we invert the columns?
\diglotInnerOuterfalse
\newif\ifdiglotN@rmal % Which way round for this page?
\newif\ifuseLeftMarks %Do we use marks from the left column
\newif\ifuseRightMarks %Do we use them from the right column
\useLeftMarkstrue
\useRightMarkstrue

\def\firstLmark{} % First mark in left column
\def\botLmark{} % last mark in left column
\def\firstRmark{} % First mark in right column
\def\botRmark{} % last mark in right column
\def\nextp@gefirstmark{}
\def\LeftMarkstrue{\useLeftMarkstrue\useRigthMarksfalse}
\def\LeftMarksfalse{\useLeftMarkstrue\useRigthMarkstrue}

%Defined penalties
\def\dgl@tPenColSwap{-10001}%
\def\dgl@tPenInternalRpt{-10002}
\def\dgl@tPenLtrialEnd{-10005}
\def\dgl@tPenRtrialEnd{-10006}
\def\oldRmark{}
\def\oldLmark{}
\def\p@gebotmark{}

\newif\ifVisTrace% Show lines where boxes are joined
\newif\ifVisTraceExtra %Do VisTrace even in places where this breaks the layout.
\VisTracefalse%
\VisTraceExtrafalse%
%\def\doTRACEdiglot#1{\global\advance\diglotTRcount by 1 \MSG{\the\diglotTRcount: #1}}%
%\def\noTRACEdiglot#1{\relax}%
%\let\TRACEdiglot=\noTRACEdiglot%
\newif\ifdiglot %If there is diglot material
\newif\ifp@gestart % temporary hold while things get going.
\def\setsid@{\ifdiglotL\xdef\sid@{L}\else\xdef\sid@{R}\fi}

%+cdig_define-hooks
\newtoks\leftho@ks
\newtoks\rightho@ks
\def\addToLeftHooks#1{\x@\global\x@\leftho@ks\x@{\the\leftho@ks #1}}
\def\addToRightHooks#1{\x@\global\x@\rightho@ks\x@{\the\rightho@ks #1}}
%-cdig_define-hooks

\def\stylesheetL#1{\gdef\ds@ffix{L}\stylesheet{#1}\gdef\ds@ffix{}}%Need to reset ds@ffix to {}, so that side-specific settings can be applied.
\def\stylesheetR#1{\gdef\ds@ffix{R}\stylesheet{#1}\gdef\ds@ffix{}}

%LRspecific holds a list of things that get redefined on side-switching. Now split into Definitions and Dimensions.
\def\LRspecificDef{AdornVerseNumber,VerticalSpaceFactor,LineSpacingFactor,regular,bold,italic,bolditalic,SpaceStretchFactor,SpaceShrinkFactor}
\def\LRspecificDim{FontSizeUnit,le@dingunit,onel@neunit,verticalsp@ceunit,IndentUnit} 


% Code to cycle through \LRspecific, setting them to their L/R values, modified
% from what the code to count callers does in ptx-callers.tex
%Would be nice to re-use \\, but this might get called defining a font in footnotes, so shouldn't mess with that
%Therefore use \wh@t instead.
\def\@rig{@orig}
% These do the looping
\def\pr@cessSp@cific#1,#2\E{\def\t@st{#1}\ifx\t@st\empty\let\n@xt\l@stSpecific\else\let\n@xt\pr@cessSp@cific\wh@t{#1}\fi\n@xt #2,\E}
\def\l@stSpecific#1\E{}
\def\pr@cessSpecific{\edef\LRs{\LRspecificDef,\LRspecificDim}\x@\pr@cessSp@cific \LRs,\E}
\def\pr@cessSpecificDef{\edef\LRs{\LRspecificDef}\x@\pr@cessSp@cific \LRs,\E}
\def\pr@cessSpecificDim{\edef\LRs{\LRspecificDim}\x@\pr@cessSp@cific \LRs,\E}

% These are the things that might get called by the loop.

% If theres a side-specific version, switch to it. If there's no side-specific verision, 
% switch to the \original version, if that exists.
\def\sp@cificSideDef#1{\ifcsname #1\sfx\endcsname\x@\let\csname#1\x@\endcsname\csname #1\sfx\endcsname
  \trace{S}{SpS: #1\sfx\space selected}\else
  \ifcsname #1\@rig\endcsname\x@\let\csname#1\x@\endcsname\csname #1\@rig\endcsname
    \trace{S}{SpS: #1 returned to original}\else\trace{S}{SpS: #1 unmodified}\fi\fi}

% If theres a side-specific version and its >0sp, switch to it. If there's no side-specific verision,
% switch to the \original version, if that exists.
\def\sp@cificSideDim#1{\x@\ifdim\csname #1\sfx\endcsname>0sp\x@\let\csname#1\x@\endcsname\csname #1\sfx\endcsname
  \trace{S}{SpS: #1\sfx\space selected}\else
  \ifcsname #1\@rig\endcsname\x@\let\csname#1\x@\endcsname\csname #1\@rig\endcsname
    \trace{S}{SpS: #1 returned to original}\else\trace{S}{SpS: #1 unmodified}\fi\fi}

\def\s@veSpecificOrig#1{\ifcsname #1\endcsname \x@\let\csname #1\@rig\x@\endcsname\csname #1\endcsname
  \ch@ckLR{#1}{L}\ch@ckLR{#1}{R}\else\MSG{No global definition for #1}\fi} % If the item exists, save its current value. 
\def\ch@ckLR#1#2{\ifcsname#1#2\endcsname\trace{S}{SpS: #1#2 exists}\else\x@\let\csname #1#2\x@\endcsname\csname #1\@rig\endcsname
  \ifdiglot\MSG{Side-specific #1#2 not defined, global #1 will be used}\fi\fi}% Helper function

% If the item exists, save its current value. 
\def\s@veSpecificSide#1{\ifcsname #1\endcsname \x@\let\csname #1\sfx\x@\endcsname\csname #1\endcsname\fi}
% Output routine for normal things.
\def\sh@wSpecificDef#1{\ifcsname #1\sfx\endcsname\trace{S}{#1\sfx: \csname #1\sfx\endcsname}\else
  \trace{S}{#1\sfx: undefined}\fi}
% Output routine for  dimensions
\def\sh@wSpecificDim#1{\ifcsname #1\sfx\endcsname\trace{S}{#1\sfx: \the\csname #1\sfx\endcsname}\else
  \trace{S}{#1\sfx: undefined}\fi}
%
% And these are the interface functions. They should all set \sfx  to L, R or \@rig
%
\def\setLRspecific{\edef\sfx{\ifdiglot\ifdiglotL L\else R\fi\else \@rig\fi}%
  \let\wh@t\sp@cificSideDef\pr@cessSpecificDef
  \let\wh@t\sp@cificSideDim\pr@cessSpecificDim
}% For use with style definitions, etc.

\def\SSsetLRspecific{\ifx\ds@ffix\empty\def\sfx{@orig}\else\edef\sfx{\ds@ffix}\fi
  \let\wh@t\sp@cificSideDef\pr@cessSpecific}% for use in stylesheet, no need to check dimensions, just existance

\def\showLRspecific{\edef\sfx{\ifdiglot\ifdiglotL L\else R\fi\else \@rig\fi}%
  \let\wh@t\sh@wSpecificDef\pr@cessSpecificDef
  \let\wh@t\sh@wSpecificDim\pr@cessSpecificDim
}% Call the output routines
\def\saveLRspecificSide#1{\edef\sfx{#1}% Save current values (side defined by #1) 
  \let\wh@t\s@veSpecificSide
  \trace{S}{Redefining #1-specific values to current ones}\pr@cessSpecific
}
%
% Except this one:
%
\def\saveLRspecificOrig{% Save original values of side-specific variables. Force it to single use
  \let\wh@t\s@veSpecificOrig\pr@cessSpecific\let\saveLRspecificOrig=\relax}

\maxdeadcycles=75
\newdimen\FontSizeUnitL\FontSizeUnitL=-1sp\newdimen\FontSizeUnitR\FontSizeUnitR=-1sp
\newdimen\le@dingunitL\le@dingunitL=-1sp\newdimen\le@dingunitR\le@dingunitR=-1sp
\newdimen\onel@neunitL\onel@neunitL=-1sp\newdimen\onel@neunitR\onel@neunitR=-1sp
\newdimen\verticalsp@ceunitL\verticalsp@ceunitL=-1sp\newdimen\verticalsp@ceunitR\verticalsp@ceunitR=-1sp
\newdimen\IndentUnitL\IndentUnitL=-1sp\newdimen\IndentUnitR\IndentUnitR=-1sp
\def\SpaceStretchFactorL{}\def\SpaceStretchFactorR{}
\def\SpaceShrinkFactorL{}\def\SpaceShrinkFactorR{}
\newif\ifRTLL \newif\ifRTLR
\font\VisTracefont="Andika":color=3f7f3f at 6pt
\def\doVisTrace#1{%
  \setbox0=\vtop to 0pt{\hrule height 0pt depth 0.5pt width 15pt\hbox{\VisTracefont #1 \the\TRACEcount}\vss}\ht0=0pt\dp0=0pt
}
\def\doVisTraceT#1{%
  \setbox0=\vbox to 0pt{\vss\hbox{\VisTracefont #1 \the\TRACEcount}\hrule height 0pt depth 0.5pt width 15pt}\ht0=0pt\dp0=0pt
}

\diglotfalse%
\newif\ifdiglotSepNotes %If the footnotes from the versions should be split (true) or merged together
\diglotSepNotestrue%
\newif\ifdiglotBalNotes %If a left column footnote steals space from the right column also
\diglotBalNotesfalse%
\newif\iftrialfailed
\global\def\n@xtc@mmand{}%

%\partial % fully set Partial page (both columns)
\newbox\n@xtpartialNrml % next chunk that we'll add to partial assuming all goes well - normal orientatin
\newbox\n@xtpartialRev % next chunk that we'll add to partial assuming all goes well - reversed orientation

%\galley - as used in monoglot routines. This is the (holdinginserts=1) text. 
% saved by the first output routine. Unless void, it always ends with the appropriate
% end-of-trial penalty
\newbox\galleyexc@ss %This is the bit of the galley that didn't make it onto the current page
\newbox\partialL  % Partial page, on the left side
\newbox\excessL  % Excess left material, aligning with the next right chunk. (added to partialL on shipout if partialL empty)
\newbox\excessR  % Excess right material, once we know we're on the next page 
\newbox\partialR  % Partial page on the right side
\newbox\trialbox

%\newbox\partialPage \setbox\partialPage=\vbox{}
%\diglotLeft={\hsize=\columnLwidth\global\setbox\partialL=\vbox{\unvbox\partialL\unvbox255}}

\newif\ifrunLtrial % logic test in diglotLeft
\newif\ifintrial %flag to let setbox know...

% Not the same as \c@lcavailht from paratext2.tex
\def\dglt@calc@vailht{%
   \trace{D}{dglt@calc@vailht \c@rrdiglotstate}%
   \global\availht=\textheight %
   \global\advance\availht by \adjustp@ge % Panic measure..
   \global\advance\availht by -\ht\partial %
   \global\advance\availht by -\dp\partial %
   \global\advance\availht by -\ht\n@xtpartial %
   \global\advance\availht by -\dp\n@xtpartial %
   \trace{D}{after part1:\the\availht}%
   \ifdim\baselineDelta<0pt %FIXME, shouldn't this be side-dependent?
     \global\advance\availht by \baselineDelta%
   \else
     \global\advance\availht by -\baselineDelta%
   \fi
   \trace{D}{after part2:\the\availht}%
   \f@rstnotetrue
   \let\\=\reduceavailht \the\n@tecl@sses % reduce it by the space needed for each note class
   \trace{D}{after notes:\the\availht}%
   \decr{\availht}{\topins}%pictures
   % Reduce by the height of whichever top picture takes the most space.
   \setbox1\copy\topleftins\setbox1\vbox{\unvbox1}%
   \setbox2\copy\toprightins\setbox2\vbox{\unvbox2}%
   \trace{D}{inserts: \the\ht1=\the\ht\topleftins? \the\ht2=\the\ht\toprightins?}%
   \ifdim\ht1>\ht\topleftins
     \trace{D}{Box was shrunk: \the\ht1 > \the\ht\topleftins}%
     \ht\topleftins=\ht1
   \fi
   \ifdim\ht2>\ht\toprightins
     \trace{D}{Box was shrunk: \the\ht2 > \the\ht\toprightins}%
     \ht\toprightins=\ht2
   \fi
   \ifdim\ht\topleftins>\ht\toprightins %
     \decr{\availht}{\topleftins}%
   \else%
     \decr{\availht}{\toprightins}%
   \fi%
   \trace{D}{after top:\the\availht (\the\ht\topleftins,\the\ht\toprightins}%
   \ifdim\ht\bottomleftins>\ht\bottomrightins %
     \decr{\availht}{\bottomleftins}%
   \else%
     \decr{\availht}{\bottomrightins}%
   \fi%
   \decr{\availht}{\bottomins}%
   \decr{\availht}{\verybottomins}%
   \x@\global\x@\csname availht\c@rrdiglotstate\endcsname=\availht
   \trace{D}{final:\the\availht}%
}

\newif\ifnastybox  % Signal that box must be treated specially
% The \lastpenalty before a \lefttext or \righttext issues a fake one,
% or similarly penalty to apply at end of trial text, and scratch space
\newcount\savedpenalty
\newcount\tmppenalty%scratchspace
\newcount\Lchunkpenalty% The penalty that should be applied at the end of the L text (either partialL or excessL), according to the first output routine. 
\newcount\Lboxpenalty% The penalty that goes between LeftBox and partialL
\newcount\partialLpenalty% The penalty that goes between partialL and excessL
\newcount\Rchunkpenalty%
\newcount\Rboxpenalty% The penalty that goes between RightBox and partialR
\newcount\partialRpenalty% The penalty that goes between partialR and excessR

\newcount\lastsavedLpenalty% The penalty for page breaks at the \n@xtpartial / new boundary
\newcount\lastsavedRpenalty%
\newif\ifLneedsemptying

% Initial \output routine for left scripture
\def\diglot@any@primary{%
   \ifnum\the\outputpenalty=\dgl@tPenColSwap
     \setf@lwgdiglotstate{\n@xtdiglotstate}%
   \fi
   \trace{D}{diglot@any@primary: \show@diglotstate}%
   \tmppenalty=\outputpenalty %
   \ifnum\tmppenalty=1000
     \tmppenalty=0
   \fi
   \p@gestarttrue % or at least, this isn't the right place to place things that go after a paragraph
 } 


\newif\ifboxmoved%
\newcount\endtri@lpen@lty %the penalty we EXPECT at the end of the trial
%\galleypenalty %the penalty we FOUND at the break between the %galley and \galleyexc@ss
\def\@ddtoexcess{%
  \trace{d}{@ddtoexcess hIns=\the\holdinginserts(==1) adding \the\ht255+\the\dp255\space to galleyexc@ss (\the\ht\galleyexc@ss+\the\dp\galleyexc@ss)}%
  \ifvoid\galleyexc@ss
    \global\setbox\galleyexc@ss\vbox{\TempDim=\dp255\unvbox255\kern-\TempDim\ifnum\outputpenalty=10000\else\ifnum\outputpenalty=\endtri@lpen@lty\else\penalty\outputpenalty\fi\fi}%
    %\showbox\galleyexc@ss
  \else
    %\global\setbox\galleyexc@ss\vbox{\joinboxes{\galleyexc@ss}{255}{19}{\ifnum\outputpenalty=10000 0\else\ifnum\outputpenalty=\endtri@lpen@lty 0\else\outputpenalty\fi\fi}}%
    \global\setbox\galleyexc@ss\vbox{\unvbox\galleyexc@ss\ifnum\outputpenalty=10000\else\ifnum\outputpenalty=\endtri@lpen@lty \else\penalty\outputpenalty\fi\fi\unvbox255}%
  \fi
}

\def\undog@lley{%
  \dimen9=\dp\galley
  \unvbox\galley\count255=\lastpenalty\unpenalty
  \count254=\lastnodetype
  \ifnum\count254=12\else\kern-\dimen9\fi%\lastnotedype=12 is eTeX test for kern
  \ifVisTraceExtra\hbox to 0pt{\doVisTrace{t \the\ht\galleyexc@ss}\box0\hss}\fi
  \traceNum{D}{removed penalty \the\count255, replacing with \the\galleypenalty. lnt:\the\count254}%
  \ifvoid\galleyexc@ss\else
    \penalty\galleypenalty
    \unvbox\galleyexc@ss
  \fi
}

\def\rej@ctgalley{%For calling by trial routines that don't like the galley they've just been given 
  \trace{d}{rej@ctgalley}%
  \r@storenotes{\ifdiglotL\else R\fi}%
  \global\output={\diglot@backingup}%
  \global\vsize=\trialheight
  \global\holdinginserts=1
  \undog@lley
  \penalty\endtri@lpen@lty
  \relax
}

\def\diglot@backingup{%Similar to but more complex than monoglot backing-up
  % routine. Run with holdinginserts=1
  % Used by either side, it will cause a new page
  % break to be found, and
  % once the end-of-trial penalty is found, 
  % the \trial routine is called (again)
  % assumes galley and galleyexc@ss are empty when first run
  % post-galley material is saved  in  \galleyexc@ss
  \trace{d}{diglot@backingup hIns=\the\holdinginserts(==1) op:\the\outputpenalty,  dc:\the\deadcycles}%
  \ifnum\holdinginserts=0
    \MSG{*** Internal error caught. Incorrect internal state found in backing up. Text of footnotes may have been lost.}%
  \fi
  \ifvoid\galley
    \ifnum\deadcycles>0
      \global\deadcycles=\numexpr \deadcycles - 1\relax% it takes 2 cycles to get here, and can't use advance
    \fi
    \global\galleypenalty=\ifnum \outputpenalty=10000 0\else\outputpenalty\fi
    \global\setbox\galley=\vbox{\unvbox255\penalty\endtri@lpen@lty}%
  \else
    \@ddtoexcess
  \fi
  \ifnum\outputpenalty=\endtri@lpen@lty
    \global\holdinginserts=0
    \global\trialfailedfalse
    \global\output={\diglot@any@trial}%calls \whichtrial on success
    \unvcopy\galley %process
    %\ifnum\pageno>100
      %\ifnum\interactionmode=2 %=\scrollmode
        %\showlists
      %\fi
    %\fi
    %\showbox\galleyexc@ss
  \else 
    \ifnum\outputpenalty<-10000
      \ifnum\outputpenalty>-10010
        \MSG{*** Internal error caught. Incorrect penalty encountered. \the\outputpenalty!=\the\endtri@lpen@lty}%
      \fi
    \fi
    \vsize=\textheight
  \fi
}
  
\def\diglot@any@trial{%output routine for any column
 % Handles calling diglot@backingup
 %Stashes set material in \trialbox, discards extra,
 %if the galley fits then there are no discards and 
 %it calls whichtrial 
 %as though it hadn't been here.
 %Assumptions: holdinginserts=1 material in \galley and \galleyexc@ss
  \trace{d}{diglot@any@trial \the\ht255+\the\dp255\space hIns=\the\holdinginserts(==0 1st time) op=\the\outputpenalty, etp=\the\endtri@lpen@lty, dc=\the\deadcycles}%
  \ifnum\holdinginserts=0\else
    \ifvoid\trialbox
      \MSG{*** Internal error caught. Incorrect internal state found in during trial run. Text or footnotes may have been lost.}%
    \fi
  \fi
  \global\holdinginserts=1
  \ifvoid\trialbox
    \dglt@calc@vailht
    \TempDim=\availht
    \ifdim\ht255>0pt
      \advance\TempDim by -\ht255
      \advance\TempDim by -\dp255
    \else
      \advance\TempDim by \ht255
      \advance\TempDim by -\dp255
    \fi 
    \global\setbox\trialbox=\box255
    \ifdim\TempDim<-0.8\baselineskip % Coarse test. main trial routine can be pickier if it wants
      \trace{D}{Initial Content does not fit (\the\trialheight, \the\TempDim, \the\availht)}%
      \global\trialfailedtrue
    \else
      \trace{D}{Initial content fits(\the\trialheight, \the\TempDim, \the\availht, \the\ht\trialbox)}%
    \fi
  \else
    \ifnum 1=\ifdim\ht255>0pt 1 \else \ifdim\dp255>0pt 1 \else 0\fi\fi
      \trace{D}{Extra content found before end of trial}%
      \global\trialfailedtrue
      \global\vsize=\textheight
      \global\setbox255\box\voidb@x
    \else
      \trace{D}{Zero-sized box found.}%
      \global\setbox\trialbox=\vbox{\unvbox\trialbox\box 255}%
    \fi
  \fi
  \let\after@this=\relax
  \ifnum\outputpenalty=\endtri@lpen@lty
    \iftrialfailed
      \global\advance\trialheight by -1\baselineskip
      \ifdim\trialheight<0.5\baselineskip
        \trace{D}{Abandoning trial, no space for galley at all}%
        \r@storenotes{\ifdiglotL\else R\fi}% As if galley hadn't happened.
        \global\setbox\galleyexc@ss\vbox{\undog@lley}%
        \global\galleypenalty=0 
        \global\setbox255=\box\voidb@x
        \let\after@this=\whichtrial% 
      \else 
        \global\output={\diglot@backingup}%
        \global\setbox\trialbox=\box\voidb@x
        \global\vsize=\trialheight
        \traceNum{D}{Trying again}%
        \r@storenotes{\ifdiglotL\else R\fi}%
        %\setbox\galley\vbox{\unvbox\galley\unvbox\galleyexc@ss\penalty\endtri@lpen@lty}%
        \ifvoid\galleyexc@ss\else
          \setbox\galley\vbox{\undog@lley\penalty\endtri@lpen@lty}%
        \fi
        \ifnum\TRACEcount=\diglotDbgJoinboxes
           \showbox\galley
        \fi
        \unvbox\galley
      \fi
    \else
      \trace{D}{All looks good!  \the\ht\trialbox+\the\dp\trialbox (gp:\the\galleypenalty, op: \the\outputpenalty)}%
      \setbox255=\box\trialbox
      %\global\outputpenalty=\galleypenalty
      \let\after@this=\whichtrial
    \fi
  \fi
  \relax
  \after@this
}


\def\doDiglotLeftTrial{%
  %empties LeftBox and partialL back onto stack, with holdinginserts=0
  \trace{D}{doDiglotLeftTrial LB:\the\ht\LeftBox+\the\dp\LeftBox, pL:(\the\ht\partialL+\the\dp\partialL) \show@diglotstate, diglotL\ifdiglotL true\else false\fi, dig@tL\ifdigl@tL true\else false\fi, leftfull\ifleftfull true\else false\fi}%
  %Set correct current state
  \global\diglotLtrue%
  \global\hsize=\columnLwidth\relax%
  \dglt@calc@vailht%
  \trace{D}{av: \the\availht, tr: \the\trialheight}%
  % don't know why it might not be, but...
  \ifdim\availht<\trialheight
    \ifdim\availht<0pt
      \bgroup
        \dimen0=\trialheight
        \advance\dimen0 by -\availht
        \message{*** SOMETHING BAD, pg. \the\pageno. A picture or footnote (probable range: \p@gefirstmark-\p@gebotmark) has just taken \the\dimen0 of space. (There used to be \the\trialheight, now there's \the\availht).  There's no space left for text.}%
        \global\leftfulltrue
      \egroup
     \fi
    \global\trialheight=\availht
  \fi
  \global\let\whichtrial=\diglotLeftTrial
  \global\endtri@lpen@lty=\dgl@tPenLtrialEnd
  \global\trialfailedfalse
  \global\holdinginserts=0%
  \global\intrialtrue
  \global\deadcycles=0
  \global\savedpenalty=\partialLpenalty %
%  \ifnum\savedpenalty>9999
%    \global\advance\trialheight by -2\baselineskip %hopefully prevents bad breaks 
%  \fi
  \ifleftfull\else
    \global\vsize=\trialheight\relax%
    \dimen3=\ht\LeftBox
    \advance\dimen3 by \dp\LeftBox
    \advance\dimen3 by \ht\partialL
    \advance\dimen3 by \dp\partialL
    \advance\dimen3 by 3\baselineskip
    \global\output={\diglot@any@trial}%
    %\penalty\lastsavedLpenalty
    \ifdim\dimen3>\trialheight %Is this likely to be the last chunk added to the page? 
      \ifm@rksonpage\else\mark{}\marks1{}\fi% FIXME? \marks2 too?
    \fi
    \global\setbox\galley=\vbox{\joinboxes{\LeftBox}{\partialL}{10}{\Lboxpenalty}\endgraf\penalty\dgl@tPenLtrialEnd}%
    \ifvoid\galleyexc@ss\else
      \MSG{*** Internal error caught. Galleyexc@ss was not empty. Text has probably been lost near page \the\pageno. (\the\galleyexc@sspenalty, \the\ht\galleyexc@ss+\the\dp\galleyexc@ss)}%
      \global\setbox\galleyexc@ss=\box\voidb@x
    \fi
    \s@venotes{}%
    \unvcopy\galley
    \relax\relax%
  \fi
}

}

% Change which side we're adding to
\def\pr@sideswitch#1#2{%
  \global\savedpenalty=\lastpenalty % This MUST be first, or the last /thing/ won't be a penalty
  \trace{d}{#1text pL:\the\dp\partialL, xL:\the\dp\excessL, pR:\the\dp\partialR, lp:\the\savedpenalty}%
  \ifhe@dings\endhe@dings\fi%
  \ifsk@pping \egroup \sk@ppingfalse\fi% if we were skipping nonpublishable text, end that mode
  \ifinn@te\errmessage{*** #1text called from inside footnote?!?}\fi
  \@@setside{#2}%
  \n@xtc@mmand%
  \global\def\n@xtc@mmand{}%
  \endgraf\penalty\dgl@tPenColSwap\relax
} 
\outer\def\lefttext{%
  \pr@sideswitch{left}{L}%
  \swapfonts{L}%sets diglotRtrue, etc
  \global\vsize=\maxdimen
  \global\availht=\vsize
  %Other Left parameters
  \output={\diglotCollect}\relax%
  \global\holdinginserts=1
  \the\leftho@ks
}%

\outer\def\righttext{%
  \allNeedEmptyingtrue
  \pr@sideswitch{right}{R}%
  \swapfonts{R}%sets diglotRtrue, etc
  \global\vsize=\maxdimen
  \global\trialheight=\maxdimen
  \relax\output={\diglotCollect}%
  \global\holdinginserts=1
  \the\rightho@ks
}
%
% When there's a header on the left but not the right, but we want the verses
% to line up nicely...
\def\norighttext{%
  \allNeedEmptyingtrue
  \pr@sideswitch{noright}{L}%
  \swapfonts{L}%sets diglotRtrue, etc
  \global\vsize=\maxdimen
  \global\trialheight=\maxdimen
  \global\holdinginserts=1
  \the\leftho@ks
}

%Similarly when there's no left text, almost-copying \righttext seems OK ...
\def\nolefttext{%
  \allNeedEmptyingtrue
  \pr@sideswitch{noleft}{R}%
  \swapfonts{R}%sets diglotRtrue, etc
  \global\vsize=\maxdimen
  \global\trialheight=\maxdimen
  \global\holdinginserts=1
  \the\leftho@ks
}

\def\pagenumberL{\bgroup\setc@rdiglotstate{L}\s@tfont{h}\pagenumber\egroup}%
\def\pagenumberR{\bgroup\setc@rdiglotstate{R}\s@tfont{h}\pagenumber\egroup}

\newdimen\columnLwidth%
\newdimen\columnRwidth%
\newdimen\availhtL
\newdimen\availhtR
\newdimen\chunkDelta % Difference between the last boxes set on the page (-ve=left longer)
\newdimen\baselineDelta % semi-constant: difference between the L and R baselines.  Set in ptx-stylesheet (-ve=left longer) 
\newdimen\thisjointDelta % difference between the L and R boxes (-ve=left longer)
\newcount\cumulativeDelta % Sum of \chunkDelta values, in points
\cumulativeDelta=0
\newdimen\adjustp@ge %emergency stretch to page
\adjustp@ge=0pt
\def\m@rkerL{}
\def\m@rkerR{}
\newif\ifdiglotL%
\newif\ifdigl@tL%What is the NEXT text comming?
\let\diglotRtrue=\diglotLfalse\relax%
\let\digl@tRtrue=\digl@tLfalse\relax%
\let\diglotRfalse=\diglotLtrue\relax%
%Diglotstate.
\def\n@xtdiglotstate{L}%What does the diglotstate become at the next \dgl@tPenColSwap penalty?
\def\f@lwgdiglotstate{L}%What is the diglotstate of the 'recent contributions'? (i.e. next material to read)
\def\c@rrdiglotstate{L}%What is the current diglotstate?
\def\g@tdiglotstate{\if\c@rrdiglotstate L\else\c@rrdiglotstate\fi}% Returns "" or "R" (Or other state values in the future)

\def\show@diglotstate{\c@rrdiglotstate\f@lwgdiglotstate\n@xtdiglotstate}% for debugging output 
\def\setc@rdiglotstate#1{\gdef\c@rrdiglotstate{#1}%
 \if#1L\global\diglotLtrue\else\global\diglotLfalse\fi%Compatability
}
\def\setf@lwgdiglotstate#1{\gdef\f@lwgdiglotstate{#1}%
 \if#1L\global\digl@tLtrue\else\global\digl@tLfalse\fi%Compatability
}

\def\@setside#1{\setf@lwgdiglotstate{#1}\setc@rdiglotstate{#1}\xdef\n@xtdiglotstate{#1}}

\def\@@setside#1{\trace{D}{@@setside #1}%
   \xdef\n@xtdiglotstate{#1}%
   }%

\def\swapfonts#1{% Because it might be desirable to set fonts appropriately.
 %Swapping fonts isn't actually needed, but do need to set
 %baselineskip and other things
 \trace{D}{swapfonts #1}%
 \trace{D}{t@tle is \ift@tle set\else  not set\fi}%
 \trace{D}{he@dingstyle is \ifhe@dingstyle set\else  not set\fi}%
 \trace{D}{he@dings is \ifhe@dings set\else  not set\fi}%
  %\trace{D}{nsp@cebefore is \ifnsp@cebefore set\else  not set\fi}%
 \@setside{#1}%
 \x@\let\x@\w@dth\csname column#1width\endcsname
 \global\hsize=\w@dth\relax
 %All sorts of things should follow if m@rker is set properly...
 \expandafter\ifx\csname m@rker\endcsname\relax\else%
  \ifdiglotL\let\m@rkerR=\m@rker\let\m@rker=\m@rkerL\relax\else%
  \let\m@rkerL=\m@rker\let\m@rker=\m@rkerR\relax\fi%
  %\message{m@rker now \m@rker}%
 \fi%
 \ifdiglotL
   \global\let\ch@pter=\@ch@pter
   \global\let\ch@ptert@xt=\@ch@ptert@xt
   \global\let\v@rse=\@v@rse
 \else
   \global\let\ch@pter=\@ch@pterR
   \global\let\ch@ptert@xt=\@ch@ptert@xtR
   \global\let\v@rse=\@v@rseR
 \fi
 \setLRspecific %ensure that all the units are correct.
 \trace{D}{Leadingunit: \the\le@dingunit}%
 \ifdim\le@dingunit>0pt %
   \s@tbaseline{p}%
 \fi%
 \ifdim\baselineskip=0pt %
  \message{baseline set to 0pt EEK}%
  \global\baselineskip=12pt
 \fi%
 %Also need to switch hyphenation patterns
 \expandafter\ifx\csname language#1\endcsname\relax\else\uselanguage{\csname language#1\endcsname}\fi}%


%\output={\diglotLeft}
%\hsize=\columnLwidth
\newif\ifleftfull % Is the left column full?
\newif\ifpagefull % Is the page full?
\newdimen\Lht
\newdimen\Rht %
\newdimen\TempDim %
\newbox\LeftBox % The part of partialL which fits on the page
\newbox\RightBox % The part of partialR which fits on the page

%when there's no more input, make sure partial gets printed
\def\emitpartial{%
       \traceNum{d}{Emitpartial \the\ht\partial+\the\dp\partial, \the\ht\n@xtpartial+\the\dp\n@xtpartial}%
       \ifvoid\n@xtpartial\else
         \global\setbox\partial=\vtop{\joinb@xes{\partial}{\n@xtpartial}}%
         \voidn@xtpartial
       \fi
       \@writep@ge\@writep@ge%
       \global\leftfullfalse%
       \global\pagefullfalse%
}

}
\def\makevtop#1{%There is a strong posibility that a vbox has a depth that
%is unrecoverable on changing it to a vtop, e.g. if the last item in the
%vbox is a \mark. To rejoin boxes on a page accurately we need to preserve
%the depth of the box. We therefore assume the box will be joined and if
%the depth is not recoverable we add a kern to remove the orignial depth.
 \s@tbaseline{p}%
 \trace{D}{makevtop #1, before vtop dp=\the\dp#1, ht=\the\ht#1}%
 %\ifnum\diglotTRcount<61\showbox#1\fi
 \TempDim=\dp#1\relax%
 \ifdim\TempDim>\baselineskip \trace{D}{Looks like this is already a vtop}\TempDim=0pt
 \fi%
 %\bgroup
 \setbox0=\copy#1\setbox1=\vtop{\unvbox 0\setbox2=\lastbox}%Work on a copy so we don't break stuff.
 \trace{D}{pd:\the\dp2, ph=\the\ht2, d:\the\TempDim}%
 \global\setbox#1=\vtop{\unvbox#1\ifdim\dp2=0pt\ifdim\TempDim=0pt\else\kern-\TempDim\fi\fi}}%\egroup}%

\def\joinboxes#1#2#3#4{%Join 2 vtops together and preserve baselineskip%
 \traceNum{d}{joinboxes #1(\the\ht#1+\the\dp#1x\the\wd#1) #2 called from locn #3}%
 \ifnum\TRACEcount=\diglotDbgJoinboxes
   \showbox#1\showbox#2
 \fi\relax
 %How do these boxes join?
 \dimen5=0pt
 \ifdim\ht#1>0pt
   \def\join@a{vbox}%
   \ifdim\ht#2=0pt
     \ifdim\dp#2>0pt
       \dimen5=-\baselineskip
       \def\join@b{vtop}%
     \else
       \def\join@b{void}%
     \fi
    \else
       \def\join@b{vbox}%
    \fi
 \else
   \def\join@a{vtop}%
   \def\join@b{unknown}%
 \fi%
 %\bgroup
 %\ifnum#3=10\showbox#1\fi%
 \ifdim\ht#1=0pt \ifdim\dp#1=0pt \ifdim\wd#1=0pt \setbox0=\box#1\fi\fi\fi%make a 0 size box void
 \ifvoid#1%
   \trace{D}{#1 is void}%
 \else%
   \s@tbaseline{p}%
   \ifdim\ht#1=0pt
     \unvbox#1
     %\showlists
     \dimen4=\lastkern%
    % \unkern
     \setbox0=\lastbox%
     \copy0
     \trace{D}{lk: \the\dimen4, final bit of deconstructed first box: \the\ht0+\the\dp0, 2nd box: \the\ht#2+\the\dp#2}%
     \ifdim\dp0>0pt \ifdim\dimen4=0pt \ifdim\ht0>0pt
         \dimen4=-\dp0
     \fi\fi\fi
     \ifdim\dp0=0pt%
  %     \kern \lastkern
       \ifdim\ht0=0pt \ifdim\dimen4<0pt
          \dimen4=0pt
       \fi\fi
     \else
       \ifdim\dp0>\baselineskip
         \kern -\dp0
         %\else
         %\showbox0
       \fi%
     \fi
   \else
     \dimen4=-\dp#1
     \unvbox#1
   \fi
   \ifvoid#2\else
     \ifdim\ht#2=0pt 
       \dimen3=\baselineskip
       \advance\dimen3 by -\topskip
       \vskip \dimen3
     \fi
   \fi
   %\trace{D}{Box joint is \join@a-\join@b (\the\dimen5)}%
   %\ifdim\dimen5=0pt\else
     %\kern \dimen5
   %\fi
   \ifVisTrace
  %   \hrule\kern -0.4pt
     \dimen6=1sp\count254=0
     \ifdim\dimen4<10sp \ifdim\dimen4>0sp
        \count254=\dimen4
        \dimen6=\dimen4
        \advance\dimen6 by 1sp
        \dimen4=0pt
        \trace{D}{Indenting label by \the\count254}%
     \fi\fi
     \kern \dimen4
     \hbox to 0pt{\hss\doVisTrace{j}\box0\kern\dimexpr -20pt * \count254\relax}%
     \kern \dimen6
   \fi
 \fi%
 %
 \ifvoid#2\else
   \penalty#4 %put the appropriate penalty back 
   %\setbox0=\copy#2
   %\setbox1=\vsplit0 to \baselineskip
   %\trace{D}{Next box, h:\the\ht1, d:\the\dp1}%
   \unvbox#2 %
 \fi%
 \ifnum\diglotDbgJoinboxes=\TRACEcount
   \showlists
 \fi
}

\def\joinb@xes#1#2{%Called (ONLY) by pagebuilder. Join a vtop/vbox and a vtop/vbox together and preserve baselineskip%
%\showbox#1\showbox#2%
\bgroup
 \setbox0=\box#1
 \setbox1=\box#2
 \dimen0=0pt
 \ifdim\ht0>0pt \ifdim\dp0<0.5\baselineskip
   %box0 is not a vtop, it's a vbox
   \advance\dimen0 by -\baselineskip
 \fi\fi
 \ifdim\ht1>0pt \ifdim\dp1<0.5\baselineskip
   %box1 is not a vtop, it's a vbox
   \ifdim\dimen0<0pt
     \dimen0=0pt
   \fi
   %Theoretically, there ought to be a \baselineskip here. It doesn't seem to be needed.
   %\advance\dimen0 by \baselineskip
 \fi\fi
 \trace{D}{joinb@xes #1 #2, skip \the\dimen0}%
 \box0
 \ifdim\dimen0=0pt
 \else
   \vskip\dimen0
 \fi
 \box1
\egroup}



\def\diglot@arrange@cols{%
  \ifVisTrace
    \ifdim\ht\LeftBox>0pt
      \doVisTraceT{L}% sets box0, used ~43 lines below
    \else
      \doVisTrace{L}% sets box0, used ~43 lines below
    \fi
  \fi
  %Is there actually anything to set?
  \ifnum 1=
      \ifdim\Lht=0pt
        \ifdim\Rht=0pt 0
        \else 1\fi
      \else 1\fi
    \thisjointDelta=0pt %Do we need extra space?
    \ifdim\dp\LeftBox>0.3\baselineskip \ifdim\chunkDelta<0pt
      %left are 'touching' and need separating
      \ifdim\baselineDelta<0pt 
        \trace{D}{Separating chunks \the\baselineDelta}%
        \advance\thisjointDelta by -\baselineDelta
      \fi 
      %FIXME: RHS / small L baseline may need more attention
    \fi\fi
    \chunkDelta=\Rht
    \global\advance\chunkDelta by -\Lht
    \bgroup
      \dimen0=\chunkDelta
      %\message{c255: \the\dimen0}%
      \dimen2=1pt
      \divide\dimen0 by \dimen2
      \count255=\dimen0
      %\message{adding \the\count255 to cumulativeDelta}%
      \global\advance\cumulativeDelta by \count255
    \egroup
    \dimen8=0pt
    \ifdim\thisjointDelta>0pt
      \advance\dimen8 by \thisjointDelta
    \fi
    \advance\dimen8 by \ht\LeftBox
    \ifdim \dimen8 <\ht\RightBox \dimen8=\ht\RightBox \fi
    \dimen9=\dp\LeftBox
    \ifdim \dimen9 <\dp\RightBox \dimen9=\dp\RightBox \fi
    % apply columngutterruleskip?
    \tempfalse
    \ifdim\ht\partial=0pt\ifdim\dp\partial=0pt
      \advance\dimen8 by -\ColumnGutterRuleSkip
      \ifdim\ColumnGutterRuleSkip < 0pt
       \ifColumnGutterRule
          \temptrue
       \fi
      \fi
    \fi\fi
    \trace{d}{Gutter \the\dimen8 height, \the\dimen9 deep}%
    \dimen7=\dimen8%.5\baselineskip
    \global\setbox\n@xtpartialNrml\vtop{%
      \iftemp
        \vskip\ColumnGutterRuleSkip
      \fi
      \hbox to \textwidth{\ifVisTrace\llap{\copy0}\fi%
      \hbox to\columnshift{}%
      \hbox to\columnLwidth{\copy\LeftBox\hss}\makediglotgutter{\dimen7}{\dimen8}{\dimen9}%
      \hbox to\columnRwidth{\copy\RightBox\hss}%\rlap{\the\ht\RightBox \the\dp\RightBox}%
    }}%
    \global\setbox\n@xtpartialRev\vtop{%
      \iftemp
        \vskip\ColumnGutterRuleSkip
      \fi
      \hbox to \textwidth{\ifVisTrace\llap{\box0}\fi%
      \hbox to\columnRwidth{\box\RightBox\hss}\makediglotgutter{\dimen7}{\dimen8}{\dimen9}%
      \hbox to\columnLwidth{\box\LeftBox\hss}%\rlap{\the\ht\RightBox \the\dp\RightBox}%
    }}%
  \else
    \traceNum{d}{No content to output}%
  \fi
}

\def\pickn@xtpartial{\ifdiglotN@rmal\global\let\n@xtpartial=\n@xtpartialNrml\else\global\let\n@xtpartial=\n@xtpartialRev\fi} % decide which n@xtpartial box to use

\def\voidn@xtpartial{\ifdiglotN@rmal\global\setbox\n@xtpartialRev\box\voidb@x\else\global\setbox\n@xtpartialNrml\box\voidb@x\fi}% Void the other n@xtpartial box

\def\setdiglotN@rmal{\ifnum \numexpr 1 \ifdiglotInnerOuter \ifodd\pageno\else * -1 \fi\fi \ifdiglotSwap * -1\fi\relax >0 
  \diglotN@rmaltrue \else \diglotN@rmalfalse \fi
}

\def\upd@tep@rtial{%Actually add stuff to the page.
  \trace{D}{upd@tep@rtial \the\outputpenalty}%
  \trace{D}{plp:\the\partialLpenalty, prp:\the\partialRpenalty, lcp:\the\Lchunkpenalty, rcp:\the\Rchunkpenalty, lbp:\the\Lboxpenalty, rbp:\the\Rboxpenalty}%
  \ifnum\Lboxpenalty=10000 \ifnum\Rboxpenalty=10000
    \trace{D}{Un vtopping heading}%
    \setbox\LeftBox=\vbox{\unvbox\LeftBox}%
    \setbox\RightBox=\vbox{\unvbox\RightBox}%
  \fi\fi
  \setdiglotN@rmal
  \pickn@xtpartial
  \showboxdepth=3 %
  \showboxbreadth=1000 %
  %\ifnum\pageno=1%
  %\showbox\LeftBox%
  %\showbox\RightBox%
  %\fi%
  \global\def\n@xtc@mmand{}%
  \message{|}%
  \hsize=\textwidth%
  \edef\t@st{\p@gebotmark}%
  \ifx\t@st\empty%
    \trace{H}{updp1: Setting bottom mark to \b@xbotmark, since it's empty }%
    \xdef\p@gebotmark{\b@xbotmark}%
  \fi%
  \traceNum{d}{Adding old n@xtpartial to page(\the\ht\n@xtpartial+\the\dp\n@xtpartial)}%
  \ifvoid\n@xtpartial\else
    \global\setbox\partial=\vtop{\joinb@xes{\partial}{\n@xtpartial}}%
    \voidn@xtpartial
  \fi
  \trace{d}{Writing new n@xtpartial : LB:\the\ht\LeftBox+\the\dp\LeftBox\space RB:\the\ht\RightBox+\the\dp\RightBox\space Last Delta:\the\chunkDelta}%\space Tot:\the\cumulativeDelta%
  \diglot@arrange@cols
  \TempDim=\dimexpr \ht\n@xtpartial + \dp\n@xtpartial\relax
  \ifleftfull% This box fills the left column (with footnotes)
    \pagefulltrue%
  \fi%
  \dglt@calc@vailht
  \ifdim\availht<\baselineskip %
    \pagefulltrue%
  \else%
    \global\trialheight=\availht\relax%
    \global\vsize=\trialheight\relax%
  \fi%
  \ifdim\TempDim>0.1pt
    \global\deadcycles=0 %Reset deadcycles only if there's actual matter being added.
  \fi
  %\ifnum 1=\ifnum\lastsavedLpenalty>9999 1\else\ifnum\lastsavedRpenalty>9999 1\fi\fi
  %  \global\setbox\partial=\vtop{\joinb@xes{\partial}{\n@xtpartial}}%
  %\fi
  \global\lastsavedLpenalty=\Lboxpenalty % Remember how happy we are to break at this point
  \global\lastsavedRpenalty=\Rboxpenalty % Remember how happy we are to break at this point
  \ifpagefull%
    \ifvoid\n@xtpartial\else
      \global\setbox\partial=\vtop{\joinb@xes{\partial}{\n@xtpartial}}%
      \voidn@xtpartial
    \fi
    \@writep@ge%
    \global\deadcycles=0 %That counts as an action too.
  \fi%
  \global\setbox\RightBox=\box\voidb@x %
  \ifvoid\partialL %
    \ifvoid\excessL %
      \trace{D}{partialL and excessL now empty}%
      \global\Lneedsemptyingfalse%
    \fi
  \fi
  \ifpagefull % Page was written
    \global\leftfullfalse%
    \global\pagefullfalse%
    \global\holdinginserts=1 %
    \global\availht=\textheight %
    \global\vsize=\availht\relax%
    %\ifdim\ht\partialL=0pt\ifdim\dp\partialL=0pt\ifdim\wd\partialL=0pt
      %\trace{D}{partialL has no size, making void}%
      %\setbox\partialL\box\voidb@x
    %\fi\fi\fi
    \ifvoid\excessL\else \ifvoid\partialL
        \trace{D}{Forcing early refill of partialL. eL:(\the\ht\excessL+\the\dp\excessL) }%
        \global\earlyLrefilltrue
        \global\Lneedsemptyingtrue
        \global\def\n@xtc@mmand{\upd@tep@rtial}%
    \fi\fi
    \ifearlyLrefill%
      %There is material in excessL which should be added now, not later.
      \trace{D}{Refilling partialL early. pL:(\the\ht\partialL+\the\dp\partialL) eL:(\the\ht\excessL+\the\dp\excessL) }%
      \global\setbox\partialL=\vtop{\joinboxes{\partialL}{\excessL}{12}{\partialLpenalty}}%
      \partialLpenalty=\Lchunkpenalty %
      \global\earlyLrefillfalse%
    \else
      \trace{D}{no early refill of  partialL. pL:(\the\ht\partialL+\the\dp\partialL) eL:(\the\ht\excessL+\the\dp\excessL) }%
    \fi
    \trace{D}{pL:(\the\ht\partialL+\the\dp\partialL) pR:(\the\ht\partialR+\the\dp\partialR) Lneedsemptying\ifLneedsemptying true\else false\fi}%
    \ifvoid\partialL
      \traceNum{D}{partialL is void}% 
      \ifnum\outputpenalty=\dgl@tPenColSwap
        % No partialL mtl, but possibly have right column that should be
        % set.
        \trace{D}{B pL:(\the\ht\partialL+\the\dp\partialL) pR:(\the\ht\partialR+\the\dp\partialR)}%
        \ifvoid\partialR\else
          \traceNum{d}{Repeating setboth}%
          \setboth%
        \fi
      \fi %outputpenalty %
    \else
      \traceNum{D}{partialL not void repeatpermitted\ifrepeatpermitted true\else false\fi}% 
      \ifnum\diglotDbgJoinboxes=\TRACEcount
        \showbox\partialL
      \fi
      %Assumptions: Page is empty, LeftBox is empty, partialL contains extra
      %material for left column, excessL is possibly empty (from setboth or above). 
      \ifrepeatpermitted 
        % Quite possibly have enough material already - repeat 'til not a full page.
        \tempfalse
        \ifLneedsemptying\else
          \ifdim\dp\partialR>0pt %Only test right if we've finished emptying left.
            \temptrue
        \fi\fi
        \iftemp
          %There is something to add to the RHS.. After re-setting partialL, setboth should be called.
          \traceNum{d}{Repeating setboth after trial (both sides have data)}%
          \global\def\n@xtc@mmand{\trace{D}{n@xtc@mmand}\setboth}%
        \else
          \ifrepeatok
            %Right is empty, left has something. Don't call setboth, come straight back to updatepartial
            \traceNum{d}{Repeating upd@tep@rtial after trial}%
            \global\def\n@xtc@mmand{\trace{D}{n@xtc@mmand}\upd@tep@rtial}%
          \else
            % Don't repeat - e.g. when called here from \diglotLeft ??? Or only sometimes?
            \traceNum{d}{Not Repeating upd@tep@rtial}%
            \global\def\n@xtc@mmand{\trace{D}{no n@xtc@mmand}}%
            \global\repeatoktrue %ok to do it later though
          \fi
        \fi
      \fi%
      \ifdim\dp\partialL>\availht%
        %Want the first chunk in partialL, the residue in excessL, so vsplit
        %gets it backwards for us, plus excessL may not be empty.
        %Use LeftBox as a temporary store
        \trace{D}{Spliting partialL (\the\ht\partialL+\the\dp\partialL) to \the\availht}%
        \global\setbox\LeftBox=\vsplit\partialL to \availht
        \edef\t@st{\splitbotmark}%
        \ifx\t@st\empty\else
          \xdef\botLmark{\splitbotmark}%
          \ifuseLeftMarks
            \trace{H}{updp2: Setting bottom mark to \splitbotmark, pagefirstmark to \splitfirstmark}%
            \xdef\p@gefirstmark{\splitfirstmark}%
            \xdef\p@gebotmark{\splitbotmark}%
          \fi
        \fi
        \ifvoid\partialL%
           \Lboxpenalty=\partialLpenalty %
        \else%
           \Lboxpenalty=0 %
           \makevtop{\partialL}%
        \fi%
        \makevtop{\LeftBox}%
        \ifvoid\partialL\else
          \trace{D}{Shuffling excessL, PartialL and LeftBox}%
          \ifvoid\excessL
            \global\setbox\excessL=\box\partialL
          \else 
            \global\setbox\excessL=\vtop{\joinboxes{\partialL}{\excessL}{15}{\partialLpenalty}}%
          \fi
        \fi
        \global\setbox\partialL=\box\LeftBox
      \else%
        \ifx\firstLmark\empty
          \trace{D}{need to split, but only to get marks}%
          \setbox0=\copy\partialL
          \setbox0=\vsplit0 to \maxdimen
          \xdef\firstLmark{\splitfirstmark}%
          \ifuseLeftMarks
            \ifx\p@gefirstmark\empty%
              \trace{H}{updp3: Setting pagefirstmark to \splitfirstmark}%
              \xdef\p@gefirstmark{\splitfirstmark}%
            \fi
          \fi
        \else
          \trace{D}{no need to split}%
        \fi
        \global\outputpenalty=\dgl@tPenInternalRpt\relax %no more itterations after this
        % but we do need 
        \global\setbox\partialL=\vtop{\unvbox\partialL}%
        %\makevtop{\partialL}%
      \fi%
      \trace{D}{rerun left Trial}%
      \global\trialheight=\availht\relax%
      \doDiglotLeftTrial%
    \fi %\void\partiaL
  \fi%
  \trace{D}{end of upd@tep@rtial}%
}

\newif\ifLRf@@tnotes%
\newif\ifNoteGutterRule \NoteGutterRulefalse
\newif\ifFigGutterRule \FigGutterRulefalse
\newif\ifJoinGutterRule \FigGutterRulefalse
\def\@writep@ge {%
  \trace{D}{@writep@ge}%
  \setdiglotN@rmal
  \pickn@xtpartial
  \global\setbox\partial=\vbox{\unvbox\partial}%
  \ifdiglotSepNotes%
    \trace{D}{Sep notes}%
    \let\s@veddiglotstate=\c@rrdiglotstate
    \LRf@@tnotesfalse%
    \f@rstnotetrue%
    \setc@rdiglotstate{L}%%
    \lastd@pth=0pt 
    %Arrange Left and Right footnotes
    \global\setbox\LeftBox=\vbox{\the\leftho@ks\let\\=\ins@rtn@tecl@ss \the\n@tecl@sses}%
    \iff@rstnote%
      \trace{f}{No Left footnotes}%
    \else%
      \LRf@@tnotestrue%
      \f@rstnotetrue%
    \fi%
    \lastd@pth=0pt 
    \setc@rdiglotstate{R}%%
    \global\setbox\RightBox=\vbox{\the\rightho@ks\let\\=\ins@rtn@tecl@ss \the\n@tecl@sses}%
    \iff@rstnote%
      \trace{f}{No Right footnotes}%
    \else%
      \LRf@@tnotestrue%
    \fi%
    \setc@rdiglotstate{\s@veddiglotstate}%%
    \ifdiglotBalNotes%
      \trace{f}{Balanced notes}%
      \ifLRf@@tnotes%
        \kern\lastd@pth%
        \trace{f}{Footnotes}%
        \ifdim\ht\LeftBox>\ht\RightBox%
          \setbox\RightBox=\vbox to \ht\LeftBox{\box\RightBox\vss}%
        \else%
          \setbox\LeftBox=\vbox to \ht\RightBox{\box\LeftBox\vss}%
        \fi%
        \dimen8=\ht\LeftBox
        \ifdim \dimen8 <\ht\RightBox \dimen8=\ht\RightBox \fi
        \dimen9=\dp\LeftBox
        \ifdim \dimen9 <\dp\RightBox \dimen9=\dp\RightBox \fi
        \trace{d}{Gutter \the\dimen8 high, \the\dimen9 deep}%
        \dimen7=0pt
        \setbox\LeftBox=\hbox to \textwidth{%\vllap{\the\ht\LeftBox \the\dp\LeftBox}%
          \hbox to\columnshift{}%
          \ifdiglotN@rmal%are the sides swapped?
            \hbox to\columnLwidth{\copy\LeftBox\hss}\ifNoteGutterRule\makediglotgutter{\dimen7}{\dimen8}{\dimen9}\else\hskip\gutter\fi%%%
            \hbox to\columnRwidth{\copy\RightBox\hss}%\rlap{\the\ht\RightBox \the\dp\RightBox}%
          \else%
            \hbox to\columnshift{}%
            \hbox to\columnRwidth{\copy\RightBox\hss}\ifNoteGutterRule\makediglotgutter{\dimen7}{\dimen8}{\dimen9}\else\hskip\gutter\fi%%
            \hbox to\columnLwidth{\copy\LeftBox\hss}%\rlap{\the\ht\RightBox \the\dp\RightBox}%
          \fi%
        }%
      \else%
        \trace{f}{No footnotes}%
      \fi %LRf@@tnotes
      \setbox\partial=\vbox{\unvbox\partial\box\LeftBox}%
    \else %BalNotes
      \trace{f}{unBalanced notes}%
      \ifLRf@@tnotes%
        %Lht and Rht hold the height of the last boxes that wanted to join the
        %page, but both may be empty 
        %chunckDelta preserves the *actual* last content added to the page.
        %Need to find out where to put the footnotes, as one of them should
        %be close to its respective text.
        {\dimen1=-\chunkDelta% dimen1 is step in last actual piece of text (for historic reasons, not the same way round as chunkDelta)
        \dimen2=\ht\LeftBox\relax%
        \trace{D}{L:\the\Lht R:\the\Rht\space Dl:\the\chunkDelta\space lf:(\the\ht\LeftBox+\the\dp\LeftBox) rf:(\the\ht\RightBox+\the\dp\RightBox)}%
        \advance\dimen2 by -\ht\RightBox % dimen2 is step in footnotes
        %
        \dimen8=\ht\LeftBox
        \ifdim \dimen8 <\ht\RightBox \dimen8=\ht\RightBox \fi
        \dimen9=\dp\LeftBox
        \ifdim \dimen9 <\dp\RightBox \dimen9=\dp\RightBox \fi
        \advance\dimen8 by -\AboveNoteSpace
        %\advance\dimen8 by \baselineskip
        \dimen7=\dimen8
          %\ifdim\availht>0pt%
            %\advance\dimen7 by \the\availht
            %\advance\dimen8 by \the\availht
          %\fi
          %\advance\dimen7 by \baselineskip
          %\advance\dimen8 by \baselineskip
        \trace{d}{Gutter \the\dimen8 tall, \the\dimen9 deep}%
        \setbox\LeftBox\hbox to \textwidth{%\vllap{\the\ht\LeftBox \the\dp\LeftBox}%
          \hbox to\columnshift{}%
          \ifdiglotN@rmal%are the sides swapped?
            \hbox to\columnLwidth{\copy\LeftBox\hss}\ifNoteGutterRule\makediglotgutter{\dimen7}{\dimen8}{\dimen9}\else\hskip\gutter\fi%
            \hbox to\columnshift{}%
            \hbox to\columnRwidth{\copy\RightBox\hss}%\rlap{\the\ht\RightBox \the\dp\RightBox}%
          \else%
            \hbox to\columnRwidth{\copy\RightBox\hss}\ifNoteGutterRule\makediglotgutter{\dimen7}{\dimen8}{\dimen9}\else\hskip\gutter\fi%
            \hbox to\columnshift{}%
            \hbox to\columnLwidth{\copy\LeftBox\hss}%\rlap{\the\ht\RightBox \the\dp\RightBox}%
          \fi%
        }%
        %\advance\dimen2 by -\ht\RightBox % dimen2 is step in footnotes
        %%
        %\setbox\LeftBox\hbox to \textwidth{%\vllap{\the\ht\LeftBox \the\dp\LeftBox}%
          %\hbox to\columnLwidth{\copy\LeftBox\hss}\hskip\gutter\hbox to%
          %\columnRwidth{\copy\RightBox\hss}%\rlap{\the\ht\RightBox \the\dp\RightBox}%
        %}%
        %dimen1 is step in text, dimen 2 is step in notes...
        \ifdim\dimen1<0pt %dimen1 -ve
        \ifdim\dimen2<0pt %dimen2 -ve
              \dimen3=0pt %No adjustment
          \else  %dimen2 +ve
            \ifdim\dimen1 > -\dimen2
              \dimen3=-\dimen1
            \else%
              \dimen3=\dimen2
            \fi%
          \fi  %dimen2 %dimen2 -ve
        \else %dimen1 +ve
          \ifdim\dimen2<0pt %
            \ifdim\dimen1 > -\dimen2
              \dimen3=-\dimen2
            \else%
              \dimen3=\dimen1
            \fi%
          \else %dimen2 +ve
            \dimen3=0pt %No adjustment
          \fi%
        \fi%
        \trace{f}{Adjusting footnotebox}%
        \global\setbox\partial=\vbox{\unvbox\partial\kern -\dimen3\vskip 0.5\AboveNoteSpace plus 1fil \box\LeftBox}%
        }%
      \else%
        \trace{f}{No footnotes}%
      \fi%
    \fi %BalNotes
  \else %SepNotes
    \trace{f}{Merged notes}%
  \fi%
  \traceNum{d}{Forming page}%
  \trace{D}{PFM:\p@gefirstmark\space PBM:\p@gebotmark}%
  %\showbox\partial
  %\showthe\everyhbox
  %\showthe\leftskip
  %\showthe\rightskip
  \p@gestarttrue
  \def\pagecontents{%
   \setdiglotN@rmal
   \pickn@xtpartial
   \ifvoid\topins\else \unvbox\topins \vskip\skip\topins \fi%
   \ifdim\ht\topleftins>\ht\toprightins%
      \setbox\toprightins=\vbox to \ht\topleftins{\box\toprightins\vss}%
   \else%
      \setbox\topleftins=\vbox to \ht\toprightins{\box\topleftins\vss}%
   \fi%
   \dimen8=\ht\bottomleftins
   \ifdim \dimen8 <\ht\toprightins \dimen8=\ht\toprightins \fi
   \dimen9=\dp\topleftins
   \ifdim \dimen9 <\dp\toprightins \dimen9=\dp\toprightins \fi
   \trace{d}{Gutter \the\dimen8 hight, \the\dimen9 deep}%
   \dimen7=\dimen9
   \advance\dimen7 by \dimen8
   \hbox to \textwidth{%\vllap{\the\ht\LeftBox \the\dp\LeftBox}%
     \hbox to\columnshift{}%
     \ifdiglotN@rmal%Are the sides swapped?
       \hbox to\columnLwidth{\box\topleftins\hss}\ifFigGutterRule\makediglotgutter{\dimen8}{\dimen8}{\dimen9}\else\hskip\gutter\fi %
       \hbox to\columnRwidth{\box\toprightins\hss}%\rlap{\the\ht\RightBox \the\dp\RightBox}%
     \else%
       \hbox to\columnRwidth{\box\toprightins\hss}\ifFigGutterRule\makediglotgutter{\dimen8}{\dimen8}{\dimen9}\else\hskip\gutter\fi %
       \hbox to\columnLwidth{\box\topleftins\hss}%\rlap{\the\ht\RightBox \the\dp\RightBox}%
     \fi%
   }%
   \lastd@pth=\dp\partial\relax%
   \unvbox\partial%
   \ifdim\ht\bottomleftins>\ht\bottomrightins%
      \setbox\bottomrightins=\vbox to \ht\bottomleftins{\box\bottomrightins\vss}%
   \else%
      \setbox\bottomleftins=\vbox to \ht\bottomrightins{\box\bottomleftins\vss}%
   \fi%
   \dimen8=\ht\bottomleftins
   \ifdim \dimen8 <\ht\bottomrightins \dimen8=\ht\bottomrightins \fi
   \dimen9=\dp\bottomleftins
   \ifdim \dimen9 <\dp\bottomrightins \dimen9=\dp\bottomrightins \fi
   \trace{d}{Gutter \the\dimen8 hight, \the\dimen9 deep}%
   \dimen7=\dimen9
   \advance\dimen7 by \dimen8
   \hbox to \textwidth{%\vllap{\the\ht\LeftBox \the\dp\LeftBox}%
     \hbox to\columnshift{}%
     \ifdiglotN@rmal%Are the sides swapped?
       \hbox to\columnLwidth{\box\bottomleftins\hss}\ifFigGutterRule\makediglotgutter{\dimen8}{\dimen8}{\dimen9}\else\hskip\gutter\fi %
       \hbox to\columnRwidth{\box\bottomrightins\hss}%\rlap{\the\ht\RightBox \the\dp\RightBox}%
     \else%
       \hbox to\columnRwidth{\box\bottomrightins\hss}\ifFigGutterRule\makediglotgutter{\dimen8}{\dimen8}{\dimen9}\else\hskip\gutter\fi %
       \hbox to\columnLwidth{\box\bottomleftins\hss}%\rlap{\the\ht\RightBox \the\dp\RightBox}%
     \fi%
   }%
   \ifvoid\bottomins\else%\kern-\lastd@pth \dimen0=0pt 
     \vskip\skip\bottomins \unvbox\bottomins \fi%
   \ifdiglotSepNotes%
   \else%
     % RTLness will have been asserted (or not) when the note was defined.
     % If the righttext side is RTL, leaving it true now will make a any LTR text also RTL.
     \ifdiglot\RTLfalse\fi
     \traceNum{d}{Inserting Notes}%
     \f@rstnotetrue%
     \m@kenotebox
     \unvbox2 % defined by m@kenotebox
   \fi%
   \ifvoid\verybottomins\else % \kern-\dimen0
     \lastd@pth=0pt \vskip\skip\verybottomins \hbox{\hbox to \columnshift{}\vbox{\unvbox\verybottomins}}\fi
  }%pagecontents
  \trace{D}{PFM:\p@gefirstmark\space PBM:\p@gebotmark\space NPFM:\nextp@gefirstmark}%
  \global\adjustp@ge=0pt
  \resetvsize%
  \global\availhtR=\textheight%
  \global\availhtL=\textheight%
  \global\availht=\textheight%
  \global\vsize=\availht
  \global\lastsavedRpenalty=0 % No longer relevant
  \global\lastsavedLpenalty=0
  \plainoutput%
  \global\chunkDelta=0pt
  \p@gestartfalse
  %\ifm@rksonpage
    \xdef\p@gefirstmark{\nextp@gefirstmark}%
  %\fi
  \let\firstmark=\empty
  \xdef\nextp@gefirstmark{}%
  \edef\botmark{\p@gebotmark}%
  \let\firstmark\empty
  \setc@rdiglotstate{\f@lwgdiglotstate}%Reset things
  \xdef\p@gebotmark{}%
  \xdef\firstLmark{}%
  \xdef\botLmark{}%
  \xdef\firstRmark{}%
  \xdef\botRmark{}%
  \global\setbox\RightBox=\box\voidb@x%
  \global\setbox\LeftBox=\box\voidb@x%
  \nextshipout
  %\nonstopmode
  %\showthe\output
  %%somejunk
}

\def\diglotCollect{%
  %This gets called to collect material into the appropriate input queue.
  % assumes: \availht is correct, \ifleftfull set false on page output
  % Fills: \partialX
  % Empties: \box255
  \trace{D}{diglotLeft O:\the\outputpenalty, P:\the\savedpenalty, LP:\the\lastsavedLpenalty, vs:\the\vsize,
  \x@\let\x@\destb@x\csname partial\c@rrdiglotstate\endcsname
  \diglot@any@primary
  \ifnum\outputpenalty=\dgl@tPenLtrialEnd\relax %Why are we here after a trial?
    \ifintrial\diglot@any@trial\fi
  \fi
  % tempoarily split and see if there are marks in this text
  \bgroup\setbox0=\copy255 \setbox1=\vsplit0 to \maxdimen\egroup
  \edef\t@mp{\splitbotmark}%
  \ifx\t@mp\empty\else\global\m@rksonpagetrue\trace{H}{Found mark \splitbotmark}\fi
  \makevtop{255}%
  \tmppenalty=\outputpenalty %
  \ifnum\outputpenalty=10000 % Nothing wrong with breaking here
    \tmppenalty=0\relax%
  \fi
  \global\setbox\destb@x=\vtop{\joinboxes{\destb@x}{255}{1}{\csname \c@rrdiglotstate chunkpenalty\endcsname}}%
  \x@\csname \c@rrdiglotstate chunkpenalty\endcsname=\tmppenalty

  \ifnum\outputpenalty=\dgl@tPenColSwap\relax %Got to the end of the \lefttext
    \tmppenalty=\savedpenalty\relax%
    \ifallNeedEmptying
      \diglot@run@trials
    \fi
  \fi
}

\n@xttrial#1#2\end{%
  \edef\tri@list{#2}%
  \setc@rdiglotstate{#1}
}

\newif\ifnot@empty % Is this chunk empty

\def\tri@llist{}%Trials to be run.

\def\ch@ckcontents#1#2\end{%
  %sets various state flags and updates the trials 
  %left to run.
  \edef\c@rrdiglotstate{#1}%
  \let\afterch@ckcontents\ch@ckcontents
  \ifx\c@rrdiglotstate\empty
    \let\afterch@ckcontents\relax
  \else
    \x@\ifvoid\csname partial\c@rrdiglotstate\endcsname
      \not@emptyfalse
    \fi
    \x@\ifvoid\csname excess\c@rrdiglotstate\endcsname
      \edef\tri@llist{\tri@llist#1}% 
    \fi
  \fi
  \afterch@ckcontents
}

\edef\diglot@list{LR}%
\def\chck@and@update{%Checks to see if there is stuff to output
  %and calls upd@tep@rtial as appropriate
  %responds to \ifboxmoved and \pagefull, checking the last chunkpenalty
  \not@emptyfalse
  \x@\ch@ckcontents\diglot@list\end%Sets not@empty if there's stuff to output
  \ifnot@empty
    \ifboxmoved
      \ifnum\lastchunkpenalty>
      \upd@tep@rtial
    \fi
  \fi
}

\newcount\trialloopcount
\newcount\chunkpenalty %highest penalty encoutered yet for this chunk 
\def\diglot@run@trials{%
  \let\tri@llist\diglot@list%
  \trialloopcount=1
  \loop
    \x@\n@xttrial\tri@llist\end
    \ifx\c@rrdiglotstate\empty
      \advance\trialloopcount by 1
      \chck@and@update
    \else
      \x@\ifvoid\csname partial\c@rrdiglotstate\endcsname\else
        \run@a@trial
        \iftrialfailed
          \ifnum\trialloopcount=1
            \boxmovedtrue
          \fi
          \x@\ifnum\csname last\c@rrdiglotstate penalty\endcsname>9999
            
          \fi
        \fi
      \fi
    \fi
  \repeat
}   

\def\makediglotgutter#1#2#3{\hbox to \gutter{\hss
%rule is #1 hight, #3 deep, and fits in a box #2 high
  \trace{D}{makediglotgutter \the #1,\the #2,\the #3}%
  \dimen4=#1\advance \dimen4 by #3
   \setbox4=\vbox to #2{%
     \hbox to 1pt{%
       \ifColumnGutterRule
        \vrule height \the\dimen4 
       \fi
       \hfil}%\dp5=#3 \box5
      \vss}%
    \dp4=#3
    \box4
  \hss}}

\def\diglotgr@db@x#1{%
 \ifdiglot\vbox{\gr@db@@x{#1}\penalty 10000}\penalty10000\else\gr@db@@x{#1}\fi
}
\def\testmarker{\hbox to 0pt{\vrule height 7pt depth 0pt width 0.5pt \kern-0.5pt}\message{_}}

\addtoendhooks{
       \bgroup\ifdiglot\ifnum\pageno>0 \count255=\cumulativeDelta \divide\count255 by \pageno \message{Overall column delta \the\cumulativeDelta pt (\the\count255 pt/ pg)\space(-ve means left longer)}\fi\fi\egroup\cumulativeDelta=0}%

\endinput
