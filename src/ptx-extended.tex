%:skip
%%%%%%%%%%%%%%%%%%%%%%%%%%%%%%%%%%%%%%%%%%%%%%%%%%%%%%%%%%%%%%%%%%%%%%%
% Part of the ptx2pdf macro package for formatting USFM text
% copyright (c) 2007 by SIL International
% written by Jonathan Kew
%
% Permission is hereby granted, free of charge, to any person obtaining  
% a copy of this software and associated documentation files (the  
% "Software"), to deal in the Software without restriction, including  
% without limitation the rights to use, copy, modify, merge, publish,  
% distribute, sublicense, and/or sell copies of the Software, and to  
% permit persons to whom the Software is furnished to do so, subject to  
% the following conditions:
%
% The above copyright notice and this permission notice shall be  
% included in all copies or substantial portions of the Software.
%
% THE SOFTWARE IS PROVIDED "AS IS", WITHOUT WARRANTY OF ANY KIND,  
% EXPRESS OR IMPLIED, INCLUDING BUT NOT LIMITED TO THE WARRANTIES OF  
% MERCHANTABILITY, FITNESS FOR A PARTICULAR PURPOSE AND  
% NONINFRINGEMENT. IN NO EVENT SHALL SIL INTERNATIONAL BE LIABLE FOR  
% ANY CLAIM, DAMAGES OR OTHER LIABILITY, WHETHER IN AN ACTION OF  
% CONTRACT, TORT OR OTHERWISE, ARISING FROM, OUT OF OR IN CONNECTION  
% WITH THE SOFTWARE OR THE USE OR OTHER DEALINGS IN THE SOFTWARE.
%
% Except as contained in this notice, the name of SIL International  
% shall not be used in advertising or otherwise to promote the sale,  
% use or other dealings in this Software without prior written  
% authorization from SIL International.
%%%%%%%%%%%%%%%%%%%%%%%%%%%%%%%%%%%%%%%%%%%%%%%%%%%%%%%%%%%%%%%%%%%%%%%

%+c_ext_intro
% Declare things we need.

\newbox\extb@x % Box in which extended study matter is placed
\newinsert\extended
\newif\ifinextended \inextendedfalse
\newtoks\old@bx %store old definition of \everyhbox
\def\c@tegory{}


\def\setcatp@ram#1#2{\x@\xdef\csname c@tegory-\c@tegory-#1\endcsname{#2}}
\def\getcatp@ram#1{\x@\let\x@\cp@ram\csname c@tegory-\c@tegory-#1\endcsname\ifx\cp@ram\empty\global\let\cp@ram\relax\fi}
\setcatp@ram{hascol}{T}
\setcatp@ram{colour}{0 1 0}
\setcatp@ram{alpha}{0.2}% 1=solid 0=invisible
\setcatp@ram{posn}{b}
\setcatp@ram{wid}{\colwidth}
\setcatp@ram{fgfigspec}{}%Normally no picture.
\def\fb@hpadding{1pt}%Added left and right of feintbox content
\def\fb@vpadding{1pt}%added above and below feintbox content


\addtoendhooks{\ifinextended\errmessage{Reached end of book without finding \\esbe.}\fi}
\def\feintb@x#1#2#3#4#5{%
  % 1 - alpha
  % 2 - colour (r g b)
  % 3 4 5 height, depth, width of colour block
  \bgroup
  \hbox to 0pt{%
    \dimen0=#3\advance\dimen0 by \fb@vpadding
    \dimen1=#4\advance\dimen1 by \fb@vpadding
    \dimen3=\fb@hpadding
    \dimen2=#5\advance\dimen2 by 2\dimen3
    \hskip -\dimen3 % Backup so the expanded box is centred properly
    \special{pdf:put @resources << /ExtGState << /GS01 << /Type /ExtGState /CA #1 /ca #1 /AIS false >>  >> >>}%
     \special{pdf:code q /GS01 gs #2 rg}%
     \vrule height \dimen0 depth \dimen1  width \dimen2 %should be possible to get this working in pdf code, but this works.
     \special{pdf:code Q}%
     \hss
  }%
  \egroup}
\def\feintbox#1#2#3{% Set a (transparent) coloured box as background for box given in #3. #1: alpha, #2: {r g b}  (all numbers in range 0-1.0)
  \setbox0#3\hbox{\feintb@x{#1}{#2}{\ht0}{\dp0}{\wd0}\box0}}
\def\esb{\setbox\extb@x\vbox\bgroup\inextendedtrue\old@bx{\the\everyhbox}\everyhbox{\st@rtesb}}

\def\st@rtesb{%
 \everyhbox{\old@bx}%
 \getcatp@ram{wid}\hsize=\cp@ram
 \getcatp@ram{fgfigspec}\ifx\cp@ram\relax\else\x@\fig \cp@ram\fig*\fi
}

\def\esbe{\ifinextended\egroup\else\message{\\ebse called without \\esb}\fi
  \inextendedfalse
  \getcatp@ram{bgfiglow}%
  \if\cp@ram F\getcatp@ram{bgfigspec}\ifx\cp@ram\relax\else
    %FIXME: modify the esbbox
  \fi\fi
  \getcatp@ram{hascol}%
  \if\cp@ram T
    \getcatp@ram{alpha}\let\pdf@lpha=\cp@ram
    \getcatp@ram{colour}\let\pdfBGc@l=\cp@ram
    \getcatp@ram{pos}\let\esbpos\cp@ram
    \getcatp@ram{bgfigspec}%
    \insert\bottomleftins{\vskip\baselineskip\feintbox{\pdf@lpha}{\pdfBGc@l}{\box\extb@x}}%
  \else
    \insert\bottomleftins{%
    \ifx\cp@ram\relax\else\vbox to 0px{\vbox to\ht\extb@x{\vfil\d@figure{\cp@ram}\vfil}\vss}\fi
    \vskip\baselineskip
    \box\extb@x}%
  \fi
}

% Category file parsing 
\def\Category   #1\relax{%Store name of current category and defaults
  \S@tCat{#1}%
  \setcatp@ram{bgfigscale}{1}%
  \setcatp@ram{fgfigsize}{box}%What is image scaled to?
  \setcatp@ram{fgfigscale}{0.5}%
  \setcatp@ram{fgfigpos}{cl}%Default is cutout left
  \setcatp@ram{bgfigscale}{1}%Full size
  \setcatp@ram{bgfigpos}{pc}%Horizontally centred
  \setcatp@ram{hascol}{F}%No default colour.
}
\def\Position   #1\relax{\setcatp@ram{posn}{#1}}% Where does it go on the page? (like figures, but also B for below notes)
\def\Breakable   #1\relax{\setcatp@ram{break}{\uppercase #1}} %Can the box page-break?
\def\FgImage      #1\relax{\setcatp@ram{fgfigname}{#1}\GenC@tFig{fg}}
\def\FgImagePos   #1\relax{\setcatp@ram{fgfigpos}{#1}\GenC@tFig{fg}}
\def\FgImageScale   #1\relax{\setcatp@ram{fgfigscale}{#1}\GenC@tFig{fg}}
\def\BgImage      #1\relax{\setcatp@ram{bgfigname}{#1}\GenC@tFig{bg}}
\def\BgImageScale   #1\relax{\setcatp@ram{bgfigscale}{#1}\GenC@tFig{bg}}
\def\BgImageLow   #1\relax{\setcatp@ram{bgfiglow}{\uppercase #1}}% What's the right sequence for colour layer and background image

\def\BgColour   #1\relax{\setcatp@ram{hascol}{T}\setcatp@ram{colour}{#1}}%For British/Australian/NZ/etc.
\let\BgColor\BgColour%For Americans
\def\Alpha   #1\relax{\setcatp@ram{alpha}{#1}}% 1=solid 0=invisible

\def\GenC@tFig#1{
 \getcatp@ram{#1figname}\ifx\cp@ram\relax\let\cp@ram\empty\fi\ifx\cp@ram\empty\else
   \let\c@tfigname\cp@ram
   \getcatp@ram{#1figsize}
 \fi
}
\def\EndCategory{\gdef\c@tegory{}\global\let\c@tprefix\empty}

\def\S@tCat#1{\message{Started Category #1}%
  \gdef\c@tegory{#1}%
  \s@tc@tpr@fix%
}
\message{Started Category #1}\def\StyleCategory#1#2{\S@tCat{#1}#2\EndCategory}
\def\categorysheet#1{%Just wrap \stylesheet with category-blanking commands
  \EndCategory\stylesheet{#1}\EndCategory
}

