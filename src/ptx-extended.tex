%:skip
%%%%%%%%%%%%%%%%%%%%%%%%%%%%%%%%%%%%%%%%%%%%%%%%%%%%%%%%%%%%%%%%%%%%%%%
% Part of the ptx2pdf macro package for formatting USFM text
% copyright (c) 2007 by SIL International
% written by Jonathan Kew
%
% Permission is hereby granted, free of charge, to any person obtaining  
% a copy of this software and associated documentation files (the  
% "Software"), to deal in the Software without restriction, including  
% without limitation the rights to use, copy, modify, merge, publish,  
% distribute, sublicense, and/or sell copies of the Software, and to  
% permit persons to whom the Software is furnished to do so, subject to  
% the following conditions:
%
% The above copyright notice and this permission notice shall be  
% included in all copies or substantial portions of the Software.
%
% THE SOFTWARE IS PROVIDED "AS IS", WITHOUT WARRANTY OF ANY KIND,  
% EXPRESS OR IMPLIED, INCLUDING BUT NOT LIMITED TO THE WARRANTIES OF  
% MERCHANTABILITY, FITNESS FOR A PARTICULAR PURPOSE AND  
% NONINFRINGEMENT. IN NO EVENT SHALL SIL INTERNATIONAL BE LIABLE FOR  
% ANY CLAIM, DAMAGES OR OTHER LIABILITY, WHETHER IN AN ACTION OF  
% CONTRACT, TORT OR OTHERWISE, ARISING FROM, OUT OF OR IN CONNECTION  
% WITH THE SOFTWARE OR THE USE OR OTHER DEALINGS IN THE SOFTWARE.
%
% Except as contained in this notice, the name of SIL International  
% shall not be used in advertising or otherwise to promote the sale,  
% use or other dealings in this Software without prior written  
% authorization from SIL International.
%%%%%%%%%%%%%%%%%%%%%%%%%%%%%%%%%%%%%%%%%%%%%%%%%%%%%%%%%%%%%%%%%%%%%%%

%+c_ext_intro
% Declare things we need.

\newbox\extb@x % Box in which extended study matter is placed
\newbox\extchunkb@x % Box in which extended study matter is placed
\newbox\sid@barnotes % Box in which notes are placed in a sidebar
\newtoks\old@bx %store old definition of \everyhbox
\def\c@tegory{}
\gdef\sb@rmarker{esb} % future-proofing. What's the current sidebar marker?

\def\setsbp@ram#1#2{\trace{sc}{Styling \m@rker:#1 #2}\setp@ram{#1}{\m@rker}{#2}}
\def\getsbp@ram#1{\getp@ram{#1}{\sb@rmarker}{\sb@rmarker}\let\cp@ram\p@ram\ifx\cp@ram\empty\global\let\cp@ram\relax\fi\trace{sc}{Got result for (\c@tegory)\sb@rmarker:#1: \cp@ram}}
\def\getsbp@ramNoInh#1{\edef\d@sf{\g@tdstat}%
  \getp@r@m{#1}{\c@tprefix\sb@rmarker\d@sf}\ifx\p@ram\relax\getp@ram{#1}{\c@tprefix\sb@rmarker}{\c@tprefix\sb@rmarker}\fi
  \let\cp@ram\p@ram
  \ifx\cp@ram\empty\global\let\cp@ram\relax\fi
  \ifx\cp@ram\relax
    \trace{sc}{No result for \c@tprefix\sb@rmarker:#1}\else
    \trace{sc}{Got result for \c@tprefix\sb@rmarker:#1: \cp@ram}%
  \fi
}
%\setsbp@ram{hascol}{T}
%\setsbp@ram{bgcolour}{0 1 0}
%\setsbp@ram{alpha}{0.2}% 1=solid 0=invisible
\def\m@rker{esb}
\setsbp@ram{bgfigscale}{1}%
\setsbp@ram{fgfigsize}{box}%What is image scaled to?
\setsbp@ram{fgfigscale}{0.2}%
\setsbp@ram{fgfigpos}{cl}%Default is cutout left
\setsbp@ram{bgfigscale}{1}%Full size
\setsbp@ram{bgfigpos}{pc}%Horizontally centred
\setsbp@ram{hascol}{F}%No default colour.
\setsbp@ram{alpha}{0.2}% 1=solid 0=invisible
\setsbp@ram{borderwidth}{0.5}% in FontSizeUnit
\setsbp@ram{posn}{b}
\setsbp@ram{fgfigspec}{}%Normally no picture.
\setsbp@ram{bgfigspec}{}%Normally no picture.
\setsbp@ram{bgfigalpha}{1.0}%Normally solid colour
\setsbp@ram{bgfigcolour}{0 0 0}%Normally black
%\setsbp@ram{borderhpadding}{0} %FontSizeUnits
%\setsbp@ram{bordervpadding}{0} %FontSizeUnits
\let\m@rker\relax
\def\fb@hpadding{1pt}%Added left and right of feintbox content
\def\fb@vpadding{1pt}%added above and below feintbox content
\newif\ifNoTransparency
\NoTransparencyfalse
\newcount\feintb@xnum
\feintb@xnum=1
\def\remove@transpancy#1 #2 #3|#4\E{\@lphacalc{#4}{#1} \@lphacalc{#4}{#2} \@lphacalc{#4}{#3}}
\def\@lphacalc#1#2{\strip@pt{\dimexpr #2\dimexpr #1 pt\relax + 1pt -\dimexpr #1 pt\relax\relax} }

\addtoendhooks{\ifinextended\errmessage{Reached end of book without finding \\esbe.}\fi}
\def\f@intthing#1#2#3{%
  % 1 - alpha
  % 2 - colour (r g b)
  % 3 - item to set
  \trace{e}{f@intthing #1 #2}%
  \edef\thisalpha{#1}%
  \edef\thisRGB{#2}%
  \ifNoTransparency
    \ifdim 1pt = #1 pt\relax\else
      \edef\thisRGB{\x@\remove@transpancy\thisRGB|\thisalpha\noexpand\E }%
      \edef\thisalpha{1}%
      \MSG{* Transparency is disabled, anything behind transparent item will be hidden. #2 -> \thisRGB}%
    \fi
  \fi
  \ifdim 1pt = \thisalpha pt\relax %Not feint
    \x@\special\x@{pdf:code q \thisRGB\space rg}%
  \else 
    \edef\thisalpha{#1 }% Important space
    \x@\special\x@{pdf:put @resources << /ExtGState << /GS0\the\feintb@xnum\space << /Type /ExtGState /CA \thisalpha /ca \thisalpha /AIS false >>  >> >>}%
    \x@\special\x@{pdf:code q /GS0\the\feintb@xnum\space gs \thisRGB\space rg}%
    \global\advance\feintb@xnum by 1
  \fi
  #3%
  \special{pdf:code Q}%
}%

\def\feintb@x#1#2#3#4#5{%
  % 1 - alpha
  % 2 - colour (r g b)
  % 3 4 5 height, depth, width of colour block
  \trace{e}{feintbox #1 #2 #3, #4, #5}%
  \bgroup
  \hbox to 0pt{%
    \dimen0=#3\advance\dimen0 by \fb@vpadding
    \dimen1=#4\advance\dimen1 by \fb@vpadding
    \dimen3=\fb@hpadding
    \dimen2=#5\advance\dimen2 by 2\dimen3
    \hskip -\dimen3 % Backup so the expanded box is centred properly
    \def\c@de{\vrule height \dimen0 depth \dimen1  width \dimen2}%should be possible to get this working in pdf code, but this works.
    \f@intthing{#1}{#2}{\c@de}%
    \hss
  }%
  \egroup}
\def\fb@htadjust{0pt}
\def\feintbox#1#2#3{% Set a (transparent) coloured box as background for box given in #3. #1: alpha, #2: {r g b}  (all numbers in range 0-1.0)
  \vskip-\fb@vpadding% ensure box expands to border line
  \setbox0#3%
  \trace{e}{pre-feintb@x, ht=\the\ht0 - \fb@htadjust, dp=\the\dp0}%
  \setbox0\hbox{\feintb@x{#1}{#2}{\the\dimexpr \ht0 - \fb@htadjust\relax}{\the\dp0}{\the\wd0}\box0}%
  \trace{e}{post-feintb@x, ht=\the\ht0, dp=\the\dp0}%
  \box0
  %\vskip-\fb@vpadding % preserve grid
  %\ifdim  \fb@vpadding > 0.3\baselineskip \vskip \baselineskip \fi %bigger gap may be needed.
  %\vskip-\fb@vpadding % preserve grid
  }

\def\minip@rs@two#1#2\end{%
   \edef\tempc{#1}\edef\tempd{#2}}%
\def\minip@rsepos#1#2\mid#3\end{%Parse just enough of the FgImagePos and pos parameters to determine the \hsize
   \edef\tempa{#1}\edef\tempb{#2}%
   \x@\minip@rs@two#3\end
   \trace{e}{parsed: \tempa(\tempb), \tempc(\tempd)}%
}

\def\p@rseBorder#1 #2\relax{%
  \let\n@xtB=\p@rseBorder
  \def\t@st{#1}%
  \trace{e}{Parsing border parameter '#1' '#2'}%
  \ifx\t@st\empty
    \let\n@xtB\endP@rseBorder
  \else
    \uppercase{\def\uc@ption{#1}}%
    \x@\let\x@\t@st\csname B@rder\uc@ption\endcsname
    \ifx\t@st\relax
      \message{Could not parse Border #1}%
    \else
      \t@st \relax
    \fi
  \fi
  \def\t@st{#2}%
  \ifx\t@st\empty
    \let\n@xtB\endP@rseBorder
  \fi
  \n@xtB#2 \relax
}

\def\endP@rseBorder#1 \relax{%
  \trace{e}{Finished parsing border}% 
}
\def\sb@leftedge{}\def\sb@rightedge{}%
\edef\sb@leftshift{0pt}\edef\sb@rightshift{0pt}%

\newdimen\sb@rwidth
\def\s@tsb@rwidth{%Set default / specific width of esb box
 %h*,c*,p* take the current page width as their base
 %tX,bX take the column width 
 %t,b,B take the text width
 \trace{e}{setsb@rwidth c:\the\colwidth, t:\the\textwidth, h:\the\hsize, \the\c@rrentcols}%
 \gdef\sb@leftedge{}\gdef\sb@rightedge{}%
 \gdef\sb@rtopedge{}\gdef\sb@rbotedge{}%
 \xdef\sb@leftshift{0pt}\xdef\sb@rightshift{0pt}%
 \getsbp@ram{borderhpadding}\ifx\cp@ram\relax\xdef\border@hpadding{0pt}\else\xdef\border@hpadding{\the\dimexpr \cp@ram\FontSizeUnit\relax}\fi
 \getsbp@ram{bordervpadding}\ifx\cp@ram\relax\xdef\border@vpadding{0pt}\else\xdef\border@vpadding{\the\dimexpr \cp@ram\FontSizeUnit\relax}\fi
 \getsbp@ram{boxhpadding}\ifx\cp@ram\relax\else\edef\fb@hpadding{\the\dimexpr \cp@ram\FontSizeUnit\relax}\fi
 \getsbp@ram{boxvpadding}\ifx\cp@ram\relax\else\edef\fb@vpadding{\the\dimexpr \cp@ram\FontSizeUnit\relax}\fi
 \getsbp@ram{posn}%
  \global\advance\im@gecount by 1 % A side bar is a strange sort of image, but there are similarities
  \ifcsname fig\the\im@gecount p@ge\endcsname
    \x@\let\x@\pgn@\csname fig\the\im@gecount p@ge\endcsname
  \else
    \let\pgn@\relax
  \fi
  \ifx\pgn@\relax
    \whichp@ge=\pageno
    \trace{e}{Sidebar \the\im@gecount may not be mirrored/rotated/aligned properly}%
  \else
    \whichp@ge=\pgn@
    \trace{e}{Sidebar \the\im@gecount was on page \pgn@ last time. [\the\pageno]}%
  \fi
  \ifx\cp@ram\relax
    \gdef\sb@rpos{b}%
    \setsbp@ram{posn}{\sb@rpos}%
  \else
    \global\let\sb@rpos\cp@ram
  \fi
  \getsbp@ram{fgfigpos}\let\fgfigp@s\cp@ram
  \getsbp@ram{fgfigspec}\let\fgfigsp@c\cp@ram
  \getsbp@ramNoInh{w@dth}% Width must NOT be inherited.
  \global\let\fgfig@cutout\false
  \ifx\cp@ram\relax
    \x@\minip@rsepos\fgfigp@s\mid\sb@rpos\end
    \dimen1=\ifnum \c@rrentcols=1 \textwidth \else \colwidth\fi
    \x@\let\csname sb@onlypure-t\endcsname\tr@e
    \x@\let\csname sb@onlypure-b\endcsname\tr@e
    \let\@t@st\tempc
    \ifcsname sb@onlypure-\tempc\endcsname
      \x@\ifx\csname sb@onlypure-\tempc\endcsname\tr@e
        \ifx\relax\tempd\else
          \edef\@t@st{\tempc\tempd}%
        \fi
      \fi
    \fi
    \trace{sc}{Initial width set to \the\dimen1 '\@t@st' (\tempd)}%
    \x@\let\csname sb@width-t\endcsname\textwidth
    \x@\let\csname sb@width-b\endcsname\textwidth
    \x@\let\csname sb@width-B\endcsname\textwidth
    \x@\let\csname sb@width-P\endcsname\textwidth
    \x@\let\csname sb@width-F\endcsname\PaperWidth %Good luck with that!
    \ifcsname sb@width-\@t@st\endcsname
      \x@\dimen1\csname sb@width-\@t@st\endcsname
      \trace{sc}{Revised width set to \the\dimen1}%
    \else
      \temptrue
      \ifnum\c@rrentcols>1\else
        \tempfalse
      \fi
      \iftemp
        \if\tempc t\relax\dimen1=\colwidth\else
          \if\tempc b\relax\dimen1=\colwidth
        \fi\fi
      \fi
    \fi
   % Now apply scale..
    \p@cinswid=\dimen1
    \getsbp@ram{scale}\ifx\cp@ram\relax\else
      \dimen1=\cp@ram\dimen1
    \fi
    \getsbp@ram{\ifodd\pageno borderoddleft\else borderevenleft\fi}\let\b@drleft\cp@ram
    \getsbp@ram{\ifodd\pageno borderoddright\else borderevenright\fi}\let\b@drright\cp@ram
    \getsbp@ram{borderwidth}\let\b@drwidth\cp@ram
    \trace{e}{basic width of \sb@rpos: \the\dimen1}%
    % And shrink further if s has been specified for the fg picture
    \ifx\fgfigsp@c\relax\else
      \p@rsefigscale{fg}%
      \dimen0=\fgfig@dim@wd
      \edef\fgfigsp@c{\fgfigsp@c \fgfig@scale@wd}%
      %\getsbp@ram{fgfigscale}%
      %$\ifx\cp@ram\relax
        %$\dimen0=0.2\dimen1
      %$\else
        %$\dimen0=\cp@ram\dimen1
      %$\fi
      \x@\x@\x@\p@rseLoc\x@\tempa\tempb\end
      \ifx\tempa \pos@Side
        \trace{e}{Forground picture will be \fgfigsp@c}%
        \advance\dimen1 by -\dimen0
        \ifx\tempb\empty\message{No side defined for foreground image in sidebar class '\c@tegory'. Assuming outer.}%
          \let\tempb\@lignOuter
        \fi
        \def\t@pedge{\vfil}%
        \def\b@tedge{\vfil}%
        \ifx\l@cspec@b\@lignLeft
          \ifx\l@cspec@c\@lignTop\def\t@pedge{}\fi
          \ifx\l@cspec@c\@lignBot\def\b@tedge{}\fi
          \xdef\sb@leftedge{\t@pedge\noexpand\sb@ins@rtpic\fgfigsp@c|\the\dimen0 \noexpand\E\noexpand\let\noexpand\sb@leftedge\noexpand\empty\b@tedge}%
          \trace{e}{leftedge defined to sb@ins@rtpic\fgfigsp@c...}%
          \advance\dimen0 by \fb@hpadding
          \xdef\sb@leftshift{\the\dimen0}%
        \else
          \ifx\l@cspec@c\@lignTop\def\t@pedge{}\fi
          \ifx\l@cspec@c\@lignBot\def\b@tedge{}\fi
          \xdef\sb@rightedge{\t@pedge\noexpand\sb@ins@rtpic\fgfigsp@c|\the\dimen0 \noexpand\E\noexpand\let\noexpand\sb@rightedge\noexpand\empty\b@tedge}%
          \advance\dimen0 by \fb@hpadding
          \xdef\sb@rightshift{\the\dimen0}%
        \fi
        \let\fgfigsp@c\empty
      \else% tempa!=s
        \def\l@ftedge{\hfil}%
        \ifx\l@cspec@b\@lignLeft\def\l@ftedge{}\else\ifx\l@cspec@b\@lignRight\def\l@ftedge{\hfill}\fi\fi
        \ifx\tempa\@lignTop
          \xdef\sb@rtopedge{\l@ftedge\noexpand\sb@ins@rtpic\fgfigsp@c|\the\dimen0 \noexpand\E}%
        \fi
        \ifx\tempa\@lignBot
          \xdef\sb@rbotedge{\l@ftedge\noexpand\sb@ins@rtpic\fgfigsp@c|\the\dimen0 \noexpand\E}%
        \fi
        \x@\x@\x@\p@rseLoc\x@\tempa\tempb\end
        \ifx\tempa\loc@Cut
          \let\w@tsit\gr@phic
          \def\w@tsit{graphicInSidebar}%
          \trace{e}{Cutout for esb category '\c@tegory' (\tempa\tempb) will be: \fgfigsp@c}%
          \xdef\sb@rboxname{\w@tsit|\c@tegory}%
          \x@\let\x@\this@pic\csname\sb@rboxname\endcsname
          \ifx\this@pic\relax
            \trace{e}{Creating a new box}% 
            \g@nbox{\sb@rboxname}%
          \fi
          \x@\let\x@\this@pic\csname\sb@rboxname\endcsname
          \global\setbox\this@pic\hbox{\x@\sb@ins@rtpic\fgfigsp@c|\the\dimen0 \E}%
          \trace{e}{Saved box \the\wd\this@pic x\the\ht\this@pic top:\b@drtop}%
          \global\let\fgfig@cutout\tr@e
          \xdef\l@cspec{\l@cspec@b}%
          \ifx\l@cspec@c\empty\gdef\c@tskip{0}\else\xdef\c@tskip{\l@cspec@c}\fi%
          \x@\xdef\x@\c@tskip{\strip@pt{\dimexpr \c@tskip pt + 1 pt\relax}}% floating pt addition
          \x@\c@tskipparse\c@tskip..\E
          \trace{e}{Adjusted c@tskip \c@tskip (\c@tskipfrac)}%
          \gdef\sb@rtopedge{%
            \dimen0=\ht\this@pic \advance\dimen0 by \dp\this@pic %
            \count255=\dimen0
            \getsbp@ram{spacebeside}\ifx\cp@ram\relax\let\sp@cebeside=\DefaultSpaceBeside\else\let\sp@cebeside\cp@ram\fi
            \divide\count255 by \baselineskip
            \advance\count255 by 1
            \trace{e}{Cutout should be \the\count255 lines}%
            \def\thisw@tsitslop{{0}{0}}%
            \d@figureCut\this@pic\unvbox\voidb@x}%
        \fi
      \fi
    \fi
    \advance\dimen1 by -2\dimexpr \fb@hpadding\relax
    \ifx\b@drleft\tr@e \ifdim \border@hpadding>-\b@drwidth\FontSizeUnit 
        \advance\dimen1 by -\b@drwidth\FontSizeUnit
        \advance\dimen1 by -\border@hpadding
    \fi\fi
    \ifx\b@drright\tr@e\ifdim \border@hpadding>-\b@drwidth\FontSizeUnit  
        \advance\dimen1 by -\b@drwidth\FontSizeUnit
        \advance\dimen1 by -\border@hpadding
    \fi\fi
    \if h\tempc\relax\else% Don't store width for class h or class p sidebars, as they may appear full width or single column
      \if p\tempc\relax\else
        \setsbp@ram{w@dth}{\the\dimen1}%
      \fi
    \fi
 \else
   \dimen1=\cp@ram
 \fi
 \trace{e}{Box width is \the\dimen1}%
 \hsize=\dimen1
}
% Sidebar 
\def\@sb{\let\oldc@tegory\c@tegory\let\oldc@prefix\c@tprefix 
  \trace{e}{Starting side bar from \m@rker, \the\lastdepth}%
  \endlastp@rstyle{\ss@Sbar}% Apply spaceafter, etc.
  \xdef\oldlastdepth{\the\lastdepth}\lastdepth=0pt
  \xdef\stylet@pe{\ss@Sbar}%                                                                   %(2)
  \ifhe@dings\endhe@dings\fi% Sidebars should close a heading section.
  % kill cached possParam-\m@rker values  
  \kill@PossParamCache
  \edef\hs@ze{\the\hsize}%
  \s@tc@tpr@fix
  %\end@llpoppedstyles{\ss@Sbar}
  \mcpush{\ss@Sbar}{\sb@rmarker}\trace{e}{ESB}\setbox\extb@x\vbox\bgroup
    \s@tsb@rwidth\inextendedtrue
    \global\sb@rchunkh@ight=\maxdimen
    \global\c@th@@ks{\st@rtsb@r}%
}
\newdimen\sb@rchunkh@ight
\newdimen\sb@rus@dheightl
\newdimen\sb@rus@dheightr
\def\st@rtsb@r{%Gets run by \cat..\cat*, if one is used. 
 \c@th@@ks{}%Once per box is enough!
 \s@tsb@rwidth
 \trace{e}{startESB \the\dimen1}%
 \kill@PossParamCache
 \global\sb@rchunkh@ight=\maxdimen
 \getsbp@ram{break}\ifx\cp@ram\relax\else \if\cp@ram F\else
   \ifx\cp@ram\tr@e
     \global\sb@rchunkh@ight=0.2\textheight
   \else
     \global\sb@rchunkh@ight=\cp@ram\textheight
   \fi
   \floatingpenalty=500\insertpenalties=0
 \fi\fi
 %\egroup\setbox\extb@x\vbox\bgroup
 \sb@rtopedge%\ifhmode\endgraf\fi
}
\let\b@drcol\relax
\newif\ifdoIntskip % Skip should be applied in the text area, not 
\newbox\b@rderbox
\newif\ifhorizb@rder
\newif\ifhasb@rder
\def\prepb@rderbox#1{% Prepare b@rderbox
  \hasb@rderfalse\horizb@rderfalse
  \ifx\b@drleft\tr@e\hasb@rdertrue\trace{eb}{Border: left}\fi
  \ifx\b@drright\tr@e\hasb@rdertrue\trace{eb}{Border: right}\fi
  \ifx\b@drtop\tr@e\hasb@rdertrue\horizb@rdertrue\trace{eb}{Border: Top}\fi
  \ifx\b@drbottom\tr@e\hasb@rdertrue\horizb@rdertrue\trace{eb}{Border: Bottom}\fi
  \ifhasb@rder
    \setbox\b@rderbox\vbox to \ht#1{\hsize=\wd#1\hss\vss}%Placeholder
    \dp\b@rderbox=\dp#1\relax
    %Spacing for the content-box:
    \xdef\b@dr@left@pad{\the\dimexpr \fb@hpadding  \ifx\b@drleft\tr@e +\border@hpadding + \b@drwidth\FontSizeUnit \fi\relax}%
    \xdef\b@dr@right@pad{\the\dimexpr \fb@hpadding  \ifx\b@drright\tr@e +\border@hpadding + \b@drwidth\FontSizeUnit \fi\relax}%
    \xdef\b@dr@top@pad{\the\dimexpr \fb@hpadding  \ifx\b@drtop\tr@e +\border@hpadding + \b@drwidth\FontSizeUnit \fi\relax}%
    \edef\b@drsides{\the\numexpr \ifx\b@drleft\tr@e 1 +\fi \ifx\b@drright\tr@e 1+\fi 0\relax}%
    \edef\b@rderbox@width{\the\dimexpr \wd\b@rderbox +\fb@hpadding*2 + \b@drwidth\FontSizeUnit *\b@drsides+ \border@hpadding * \b@drsides\relax}%% Final width of border artwork
    \edef\b@drsides{\the\numexpr \ifx\b@drtop\tr@e 1 +\fi \ifx\b@drbottom\tr@e 1+\fi 0\relax}%
    \edef\b@rderbox@height{\the\dimexpr \ht\b@rderbox +\fb@vpadding +\dp\b@rderbox + \b@drwidth\FontSizeUnit *\b@drsides + \border@vpadding *\b@drsides\relax}% Final height of border artwork 
    \edef\b@xminadj{\ifdim\border@vpadding< -\b@drwidth\FontSizeUnit \ifx\b@drtop\tr@e\the\dimexpr -\b@drwidth\FontSizeUnit-\border@vpadding\relax\else 0pt\fi\else 0pt\fi}% b@xminadj is for inserts, so that the top of the box is at exactly the page top.
    \let\b@xadj\b@xminadj %b@xadj is for in-line boxes.
    %\let\b@xextadj\b@xminadj %b@xadj is for in-line boxes.
    \edef\b@xextadj{\dimexpr 0pt - \ifdim\border@vpadding< -\b@drwidth\FontSizeUnit \ifx\b@drbottom\tr@e\the\dimexpr \b@drwidth\FontSizeUnit +\border@vpadding\relax\else 0pt\fi\else 0pt\fi\relax}% b@xminadj is for inserts, so that the top of the box is at exactly the page top.
    \edef\b@xextadj{\ifdim\border@vpadding> -\b@drwidth\FontSizeUnit \ifx\b@drtop\tr@e 3pt\else 0pt\fi\else 0pt\fi}%b@xextadj puts some space at the top of a box that will be gridded. This protects it from descenders
    \edef\b@rderbox@depth{\the\dimexpr \dp\b@rderbox  \ifx\b@drbottom\tr@e + \border@vpadding +\b@drwidth\FontSizeUnit \fi \relax}% 
    %\trace{e}{Depth b@drbox=\the\dp\b@rderbox / \the\dp\extchunkb@x / \thsch@nkdp}%
  \else
    \setbox\b@rderbox\box\voidb@x
    \edef\b@xextadj{0pt}%
    \edef\b@xminadj{0pt}%
  \fi
}
\def\mkb@rder{%Put a simple border around a box
  \ifx\b@drcol\relax
    \def\bc@l{}\def\endbc@l{}%
  \else
    \def\bc@l{\special{color push rgb \b@drcol}}\def\endbc@l{\special{color pop}}%
  \fi
  \dimen0=\b@rderbox@height \advance\dimen0 by \dimexpr -\b@drwidth\FontSizeUnit * \b@drsides\relax
  \dimen1=0pt
  \setbox\b@rderbox\vbox to \b@rderbox@height{\baselineskip=0pt\lineskip=0pt\hsize=\b@rderbox@width\bc@l
    \ifx\b@drtop\tr@e\trace{eb}{Drawing Border:top}\hbox{\vrule height\b@drwidth\FontSizeUnit width \b@rderbox@width}\fi%
    \hbox to \b@rderbox@width{%
      \ifx\b@drleft\tr@e\trace{eb}{Drawing Border:left}%
        \vrule width \b@drwidth\FontSizeUnit height \dimen0 depth \dimen1 
      \fi
    \hss
      \ifx\b@drleft\tr@e\trace{eb}{Drawing Border:left}%
        \vrule width \b@drwidth\FontSizeUnit height \dimen0 depth \dimen1
      \fi
    }\vss
    \ifx\b@drbottom\tr@e\trace{eb}{Drawing Border:bottom}\hbox{\vrule height\b@drwidth\FontSizeUnit width \b@rderbox@width}\fi
\endbc@l}%
}

\def\borderstylelist{plain, double}
\let\mkb@rder@plain\mkb@rder
\def\mkb@rder@double#1{%
  \let\Ob@drwidth\b@drwidth
  \let\Os@vedhpadding\fb@hpadding
  \edef\b@drwidth{\strip@pt{\dimexpr \b@drwidth pt /3\relax}}%
  \mkb@rder#1\edef\fb@hpadding{\the\dimexpr\b@drwidth\FontSizeUnit\relax}%
  \setbox#1\vbox{\vskip-1\dimen1\unvbox#1}% This helps, but it's not enough, for some reason.
  \mkb@rder#1\let\b@drwidth\Ob@drwidth
  \let\fb@hpadding\Os@vedhpadding}

\def\p@rsef@gsc@le#1x#2x#3\E{% \v@lpfx  fg or bg, params scale which might be of form Xscale x Yscale
  \edef\fig@h@dim{#1}\edef\fig@v@dim{#2}%
  \trace{e}{p@rsefigscale:#1,#2,#3}%
  \ifdim\ht\extb@x=0pt
    \def\fig@v@dim{}% 
  \fi
  \def\ch@ckfigdim{\tempfalse}%
  \ifx\fig@v@dim\empty
    \x@\def\csname \v@lpfx fig@scale@ht\endcsname{}%
    \def\ch@ckfigdim{\trace{e}{Checking image width \the\dimen3 <\the\ht\extb@x?}\ifdim\dimen4>\ht\extb@x \def\resc@le{ height \the\ht\extb@x}\temptrue\fi}%
  \else
    \dimen0=\fig@v@dim\ht\extb@x
    \x@\edef\csname \v@lpfx fig@scale@ht\endcsname{ height \the\dimen0}%
    \trace{e}{Set \v@lpfx fig@scale@ht}%
  \fi
  \ifx\fig@h@dim\empty
    \x@\def\csname \v@lpfx fig@scale@wd\endcsname{}%
    \x@\edef\csname \v@lpfx fig@dim@wd\endcsname{\the\dimen1}%
    \def\ch@ckfigdim{\trace{e}{Checking height \the\dimen4 <\the\dimen1?}\ifdim\dimen3>\dimen1 \def\resc@le{ width \the\dimen1}\temptrue\fi}%
  \else
    \dimen0=\fig@h@dim\dimen1
    \x@\edef\csname \v@lpfx fig@scale@wd\endcsname{ width \the\dimen0}%
    \x@\edef\csname \v@lpfx fig@dim@wd\endcsname{\the\dimen0}%
    \trace{e}{Set \v@lpfx fig@scale@wd}%
  \fi
}

\def\p@rsefigscale#1{%interpret bacground figure scale
  \ifdim\wd\extb@x>0pt
    \dimen1=\wd\extb@x
  \fi
  \getsbp@ram{#1figscale}%
  \edef\v@lpfx{#1}
  \x@\p@rsef@gsc@le\cp@ram xx\E
  \trace{e}{parsed figure scale: \csname #1fig@scale@wd\endcsname \csname #1fig@scale@ht\endcsname}%
}

\x@\def\csname \sb@rb@x @warning\endcsname#1#2{\msg{converted sidebar placement "#1" to "#2" in single-column layout}}

\def\@sbe{%
  \endlastp@rstyle{esbe}%
  \ifhe@dings\endhe@dings\fi
  \relax
  \getsbp@ram{posn}%
  \edef\l@gstring{{\the\im@gecount}{Sidebar-\c@tegory}{\cp@ram}{\c@rref}{\the\hsize}}% Fully expand figure parameters, but leave page no. to be expanded later.
  \x@\writefigp@gelog\x@{\l@gstring}%
  \global\let\t@st\false
  \ifinextended
   \end@llpoppedstyles{\ss@Sbar*}% May also close groups
   \kill@PossParamCache
   \ifx\fgfigp@s
   \ifhmode\par\fi
   \sb@rbotedge\ifhmode\endgraf\fi
   \global\let\c@t@gory=\c@tegory\egroup 
  \else\global\let\c@t@gory=\empty \trace{e}{\\ebse called without \\esb}%
  \fi
  \trace{e}{ESBE \the\sb@rchunkh@ight}%
  \ifx\fgfig@cutout\tr@e
    \cancelcutouts
  \fi
  \let\c@tegory=\c@t@gory %recover value from box, without making global changes.
  \s@tc@tpr@fix
  \trace{e}{ESBE \c@tprefix(\the\ht\extb@x*\the\wd\extb@x)}%
  %\showbox\extb@x
  \inextendedfalse
  \let\os@vedvpadding=\fb@vpadding%%Preserve old value
  \let\os@vedhpadding=\fb@hpadding%
  \getsbp@ram{borderhpadding}\ifx\cp@ram\relax\xdef\border@hpadding{0pt}\else\edef\border@hpadding{\the\dimexpr \cp@ram\FontSizeUnit\relax}\fi
  \getsbp@ram{bordervpadding}\ifx\cp@ram\relax\xdef\border@vpadding{0pt}\else\edef\border@vpadding{\the\dimexpr \cp@ram\FontSizeUnit\relax}\fi
  \getsbp@ram{boxhpadding}\ifx\cp@ram\relax\else\xdef\fb@hpadding{\the\dimexpr \cp@ram\FontSizeUnit\relax}\fi
  \getsbp@ram{boxvpadding}\ifx\cp@ram\relax\else\xdef\fb@vpadding{\the\dimexpr \cp@ram\FontSizeUnit\relax}\fi
  \getsbp@ram{bgfigspec}\let\bgf@gsp@c\cp@ram
  \getsbp@ram{bgfiglow}\let\bgf@gl@w\cp@ram%
  \getsbp@ram{hascol}\let\h@scol\cp@ram%
  \getsbp@ram{\ifodd\pageno borderoddleft\else borderevenleft\fi}\let\b@drleft\cp@ram
  \getsbp@ram{\ifodd\pageno borderoddright\else borderevenright\fi}\let\b@drright\cp@ram
  \getsbp@ram{bordertop}\let\b@drtop\cp@ram
  \getsbp@ram{borderbottom}\let\b@drbottom\cp@ram 
  \getsbp@ram{borderwidth}\let\b@drwidth\cp@ram
  \getsbp@ram{borderstyle}\let\b@drstyle\cp@ram
  \getsbp@ram{bgfigalpha}\let\bgf@galpha\cp@ram
  \getsbp@ram{bgfigcolour}\let\bgf@gcol\cp@ram
  \getsbp@ram{spacebeside}\ifx\cp@ram\relax\let\sp@cebeside=\DefaultSpaceBeside\else\let\sp@cebeside\cp@ram\fi
  \let\s@vedvpadding=\fb@vpadding%%Preserve old value
  \tempfalse
  \dimen4=-\dimexpr\ifx\b@drtop\tr@e \ifdim  \border@vpadding > -\b@drwidth\FontSizeUnit \border@vpadding + \fb@vpadding + \b@drwidth\FontSizeUnit +  \else \fb@vpadding  +\fi\else \fb@vpadding \fi  + 0pt\relax
  %\dimen4=-\fb@vpadding
  \trace{e}{b@xadj =\ifx\b@drtop\tr@e \ifdim \border@vpadding > -\b@drwidth\FontSizeUnit  \b@drwidth FSU + \border@vpadding +\fi \fi  \fb@vpadding + 0pt = \the\dimen4 |baseline: \the\baselineskip}%
  \loop\ifdim\dimen4> \baselineskip
    \advance\dimen4 by -\baselineskip
  \repeat
  %\loop\ifdim\dimen4< -0.1\baselineskip
    %\advance\dimen4 by \baselineskip
  %\repeat
  \edef\b@xadj{\the\dimen4}%
  \trace{e}{=\b@xadj}%
  \ifx\bgf@gsp@c\relax\else
    \p@rsefigscale{bg}%
    \edef\t@mpfigspec{\bgf@gsp@c \bgfig@scale@ht \bgfig@scale@wd|\bgfig@dim@wd}%
    \setbox\newpicb@x=\hbox{\x@\sb@ins@rtpic\t@mpfigspec\E}%
    \ch@ckfigdim
    \iftemp% It doesn't fit
      \getsbp@ram{bgfigoversize}%
      \ifx\cp@ram\Img@Shrink\edef\t@mpfigspec{\bgf@gsp@c \resc@le|\bgfig@dim@wd}\else
        \ifx\cp@ram\Img@Distort\edef\t@mpfigspec{\bgf@gsp@c \bgfig@scale@ht \bgfig@scale@wd \resc@le|\bgfig@dim@wd}\else
          \ifx\cp@ram\Img@Crop\message{Image crop not supported}\tempfalse\else
            \tempfalse
          \fi
        \fi
      \fi
      \iftemp
        \setbox\newpicb@x=\hbox{\x@\sb@ins@rtpic\t@mpfigspec\E}%
      \fi
    \fi
    \sb@rchunkh@ight=\maxdimen% Background images and page-broken chunks are incompatible 
    \tempfalse
    \if\bgf@gl@w F
      \temptrue
    \fi
    \if\h@scol F
      \temptrue
    \fi
  \fi
  \let\w@tsit\sb@rb@x % Make sure the item is identified properly - used in debugging and .delayed file
  \let\loc@ption\sb@rpos
  \picUsesInstrue \picNarrowfalse
  \ifdim\wd\extb@x=\p@cinswid\else
    \picNarrowtrue
  \fi
  \x@\p@rsePicUse\loc@ption\end  %set loc@ption and discover if there's to be an insert. (sets \picUsesInsfalse if not)
  \doIntskipfalse% Upper baseline-preserving skip should be outside the border, not in the text area.
  \ifpicUsesIns
    \doIntskiptrue
  \fi
  \ifdoIntskip
    \setbox\extb@x\vbox{%\dbghrule
      \kern\b@xadj\unvbox\extb@x}%
  \fi
  \iftemp
    %modify the esbbox to include the picture
    \trace{e}{Including background image below contents: \t@mpfigspec}%
    \setbox\extb@x\vbox to \ht\extb@x{\vbox to 0pt{\vbox to\ht\extb@x{\vss \hbox to \wd\extb@x{\f@intthing{\bgf@galpha}{\bgf@gcol}{\box\newpicb@x}} \vss}\vss}\box\extb@x}%
    \let\bgf@gsp@c\relax %used it, so if it's still there later, we need to use it.
  \fi
  \ifx\h@scol\tr@e
    \getsbp@ram{alpha}\global\let\pdf@lpha=\cp@ram
    \getsbp@ram{bgcolour}\let\pdfBGc@l=\cp@ram
  \fi
  \let\fstb@drtop=\b@drtop
  \let\lstb@drbot=\b@drbottom
  \ifdim\ht\extb@x > 0.9\textheight
    \message{*** WARNING: Sidebar or colophon might not print on page \folio. (\the\ht\extb@x\space high, and  page is \the\textheight).}%
  \fi
  \def\lstch@nkdp{0pt}%
  \loop %Breakable box:
    \ifdim\ht\extb@x>\sb@rchunkh@ight % to enable page splitting, the box should be chunked.
      \dimen1=\sb@rchunkh@ight
      {% 
        \loop
          \setbox2=\copy\extb@x
          \global\setbox\extchunkb@x=\vsplit\extb@x to \dimen1
          \ifnum\badness>999999
            \setbox\extb@x=\box2
            \advance\dimen1 by \baselineskip %Couldnt split, make the chunk bigger.
            \trace{e}{Chunk grown to \the\dimen1}%
        \repeat
      }%
      \setbox\extchunkb@x=\vbox{\unvbox\extchunkb@x}% Reset to natural sizekk
      \let\b@drbottom=\relax
    \else
      \setbox\extchunkb@x=\box\extb@x
      \let\fb@vpadding=\s@vedvpadding
      \let\b@drbottom=\lstb@drbot
    \fi
    % Shift text for side image(s) 
    \ifdim\sb@leftshift>0pt
      \setbox0\vbox to\ht\extchunkb@x{\sb@leftedge}%
      \trace{e}{Adding \sb@leftshift to left of box}%
      \setbox\extchunkb@x\vbox{\hbox{\hbox to \sb@leftshift{\hss\box0\hskip\fb@hpadding}\box\extchunkb@x}}\fi
    \ifdim\sb@rightshift>0pt
      \setbox0\vbox to\ht\extchunkb@x{\sb@rightedge}%
      \trace{e}{Adding \sb@rightshift to right of box}%
      \setbox\extchunkb@x\vbox{%
        \hbox{\box\extchunkb@x\hbox to \sb@rightshift{\hskip\fb@hpadding\box0\hss}}}\fi
    \trace{e}{Chunk of extbox is \the\ht\extchunkb@x+\the\dp\extchunkb@x*\the\wd\extb@x (\the\ht\extb@x*\the\wd\extb@x)}%
    \xdef\thsch@nkdp{\the\dimexpr \dp\extchunkb@x\relax}%
    \ifx\h@scol\tr@e
      %\dimen0=\ht\extchunkb@x
      %\advance\dimen0 by -\lstch@nkdp
      %\dp\extchunkb@x=\lstch@nkdp
      \setbox\extchunkb@x\vbox{%
        \ifx\bgf@gsp@c\relax\else
         \trace{e}{Including background image below everything: \t@mpfigspec}%
          \vbox to 0pt{\vbox to\ht\extchunkb@x{\vfil\f@intthing{\bgf@galpha}{\bgf@gcol}{\box\newpicb@x}\vfil}\vss}%
        \fi
       % \vskip\baselineskip
        \let\fb@htadjust=\lstch@nkdp
        \feintbox{\pdf@lpha}{\pdfBGc@l}{\box\extchunkb@x}}%
      %\setbox\extchunkb@x\vbox{%
        %\vskip \lstch@nkdp
        %\unvbox\extchunkb@x}%
      %\dimen0=\ht\extchunkb@x
      %\advance\dimen0 by -\thsch@nkdp
      %\ht\extchunkb@x=\dimen0
      %\dp\extchunkb@x=\thsch@nkdp
      \xdef\lstch@nkdp{\thsch@nkdp}%
      \trace{e}{Chunk is (with box) \the\ht\extchunkb@x+\the\dp\extchunkb@x}%
    \fi
    \let\b@drtop=\fstb@drtop
    \getsbp@ram{bordercolour}%
    \let\b@drcol\cp@ram
    \prepb@rderbox\extchunkb@x %sets ifhasb@rder
    \ifhasb@rder % any border
      \ifx\b@drstyle\relax
        \message{making box}%
        \mkb@rder
      \else
        \ifcsname mkb@rder@\b@drstyle\endcsname 
          \message{making box for \b@drstyle}%
          \csname mkb@rder@\b@drstyle\endcsname
        \else
          \message{Unknown borderstyle '\b@drstyle'. Known styles: \borderstylelist}%
        \fi 
      \fi
      %\setbox\extchunkb@x\vbox{\vbox to 0pt{\vskip\dimen7\hbox to \b@rderbox@width{\hss\hskip\dimen8\box\extchunkb@x\hskip\dimen9\hss}\vss}\box\b@rderbox}%
      %\showbox\extchunkb@x
      \dimen1=\dimexpr -\b@rderbox@depth\relax
      \relax
      \@LOOP\ifdim\dimen1 <0.1\baselineskip
        \advance\dimen1 by \baselineskip
      \@REPEAT
      %\repeat
      \trace{e}{Box \the\ht\b@rderbox +\the\dp\b@rderbox x \the\wd\b@rderbox (\b@rderbox@height x\b@rderbox@width)  is offset \the\dimen7 (v) and <\b@dr@left@pad|\b@dr@right@pad> (h). Final adjust\the\dimen1}%
      \trace{e}{Box adj (\c@tegory):\the\dimen7, spacing: \b@xminadj / \b@xadj,  \ifdoIntskip 1st \else 2nd\fi}%
      \x@\setbox\extchunkb@x\vbox{\ifdoIntskip\kern\b@xminadj\else\vskip\b@xextadj\fi\vbox to 0pt{\vskip\b@dr@top@pad\hbox to \b@rderbox@width{\hskip\b@dr@left@pad\box\extchunkb@x\hskip\b@dr@right@pad}\vss}\penalty10000\box\b@rderbox\kern\dimen1\ifdoIntskip\else\fi}%
    \else
      \x@\setbox\extchunkb@x\vbox{\ifdoIntskip\kern\b@xminadj\else\fi\box\extchunkb@x}%
    \fi
    \let\fstb@drtop=\relax
    \trace{e}{After border, chunk is  now \the\ht\extchunkb@x+\the\dp\extchunkb@x*\the\wd\extchunkb@x}%
    \ifpicUsesIns
      \x@\let\x@\@xtins\csname ins-\loc@ption\endcsname
      \trace{e}{using ins-\loc@ption}%
      \insert\@xtins{%\vskip\baselineskip
        \dimen0=\dp\extchunkb@x
        \penalty10000
        \pdfsavepos
        \write\p@rlocs{\noexpand\@nontextstart{\the\pdflastxpos}{\the\pdflastypos}}%
        \penalty10000
        \unvbox\extchunkb@x%\vskip-\the\dimen0
        \penalty10000
        \pdfsavepos
        \write\p@rlocs{\noexpand\@nontextstop{\the\pdflastxpos}{\the\pdflastypos}}%
        \ifdim\ht\extb@x>0pt \penalty5000 \fi}%
    \else
      \ifx\picl@c\empty
        \message{Could not interpret position \loc@ption for \c@tegory:\sb@rmarker}%
      \else
        \dimen0=\ht\extchunkb@x \advance\dimen0 by \dp\extchunkb@x \advance\dimen0 by 0.5\baselineskip % Rounding correction
        \divide\dimen0 by \baselineskip \count255=\dimen0
        \advance\count255 by 1
        \trace{g}{Inline \w@tsit takes \the\count255 lines, ls:\l@cspec@b, oldlastsdepth:\oldlastdepth}%
        \ifx\picl@c\loc@Inl
          \ifx\l@cspec@b\@lignLeft\picNarrowtrue\fi
          \ifx\l@cspec@b\@lignRight\picNarrowtrue\fi
	  %\vskip -\lastdepth
          \ifx\h@scol\tr@e
            \ifdim\lastdepth>0.2pt\vskip\baselineskip\fi%
          \fi
          \lastdepth=\oldlastdepth
          \d@figureInl\extchunkb@x
        \fi
        \ifx\picl@c\loc@Par
          \ifx\l@cspec@b\@lignLeft\picNarrowtrue\fi
          \ifx\l@cspec@b\@lignRight\picNarrowtrue\fi
          \lastdepth=0pt
          \d@figurePar\extchunkb@x
        \fi
        \ifx\picl@c\loc@Page
          \d@figurePage\extchunkb@x
        \fi
        \ifx\picl@c\loc@Full
          \d@figureFull\extchunkb@x
        \fi
        \ifx\picl@c\loc@Cut
          \penalty 3
          \d@figureCut\extchunkb@x
          %\ifhmode |+\fi
        \fi
        \ifvoid\extchunkb@x\else
          \message{Could not understand / interpret position \loc@ption for \c@tegory:\sb@rmarker}%
        \fi
      \fi
    \fi
    \ifdim\ht\extb@x>0pt
      \def\fb@vpadding{0pt}%
  \repeat
  \let\c@tegory\oldc@tegory 
  \s@tc@tpr@fix
  \ifvoid\sid@barnotes\else
    \unvbox\sid@barnotes
  \fi
  \let\fb@hpadding\os@vedhpadding
  \let\fb@vpadding\os@vedvpadding
  \kill@PossParamCache
  \global\c@th@@ks{}%
  %\ifhmode +\fi
}

% Category file parsing 
\def\Category   #1\relax{%Store name of current category and defaults
  \S@tCat{#1}%
  \xdef\m@rker{esb}%
}

\def\initc@t{%
  \ifx\c@tprefix\empty
    \edef\t@stname{\m@rker\ds@ffix:is_defined}%
  \else 
    \edef\t@stname{\c@tprefix \m@rker\ds@ffix:is_defined}%
  \fi
  \x@\xdef\csname m@rkerexists-\c@tprefix\m@rker\endcsname{t}%
  \x@\let\x@\t@st\csname \t@stname\endcsname
  \ifx\t@st\relax 
    \trace{e}{First definition of \c@tprefix\m@rker\ds@ffix}%
    \x@\xdef\csname \t@stname\endcsname{Done}%
  \fi
}
\def\Position   #1\relax{\initc@t\setsbp@ram{posn}{#1}}% Where does it go on the page? (like figures, but also B for below notes)
\def\Scale      #1\relax{\initc@t\setsbp@ram{scale}{#1}}% Where does it go on the page? (like figures, but also B for below notes)
\def\Breakable   #1\relax{\initc@t\tempfalse\setbox0\hbox{\ifcat 1#1 \global\temptrue\fi}\iftemp\setsbp@ram{break}{#1}\else\setsbp@ram{break}{\uppercase{#1}}\fi} %Can the box page-break?
\def\FgImage      #1\relax{\initc@t\setsbp@ram{fgfigname}{#1}\GenC@tFig{fg}}
\def\FgImagePos   #1\relax{\initc@t\setsbp@ram{fgfigpos}{#1}\GenC@tFig{fg}}
\def\FgImageScale   #1\relax{\initc@t\setsbp@ram{fgfigscale}{#1}\GenC@tFig{fg}}
\def\BgImage      #1\relax{\initc@t\setsbp@ram{bgfigname}{#1}\GenC@tFig{bg}}
\def\BgImageScale   #1\relax{\initc@t\setsbp@ram{bgfigscale}{#1}\GenC@tFig{bg}}
\def\BgImagePos   #1\relax{\initc@t\setsbp@ram{bgfigpos}{#1}\GenC@tFig{bg}}
\def\BgImageLow   #1\relax{\initc@t\uppercase{\edef\t@st{#1}}\setsbp@ram{bgfiglow}{\t@st}}% What's the right sequence for colour layer and background image
\def\BgImageColour       #1\relax{\initc@t\setsbp@ram{bgfigcolour}{#1}}
\let\BgImageColor=\BgImageColour%for Americans
\def\BgImageAlpha       #1\relax{\initc@t\setsbp@ram{bgfigalpha}{#1}}
\def\BgImageOversize       #1\relax{\initc@t\setsbp@ram{bgfigoversize}{#1}}

\def\BgColour   #1\relax{\initc@t\tempfalse\uppercase{\edef\t@st{#1}}\setbox0\hbox{\if F\t@st \global\temptrue\trace{e}{Cancelling colour}\else\if\tr@e\t@st\global\temptrue\fi\fi}\iftemp\setsbp@ram{hascol}{\t@st}\else\setsbp@ram{hascol}{T}\setsbp@ram{bgcolour}{#1}\fi}%For British/Australian/NZ/etc.
\let\BgColor\BgColour%For Americans
\def\Alpha   #1\relax{\initc@t\setsbp@ram{alpha}{#1}}% 1=solid 0=invisible
\def\BorderWidth        #1\relax{\initc@t\setsbp@ram{borderwidth}{#1}}
\def\BorderColour       #1\relax{\initc@t\setsbp@ram{bordercolour}{#1}}
\def\BorderStyle       #1\relax{\initc@t\setsbp@ram{borderstyle}{#1}}
\def\BoxHPadding #1\relax{\initc@t\setsbp@ram{boxhpadding}{#1}}
\def\BoxVPadding #1\relax{\initc@t\setsbp@ram{boxvpadding}{#1}}
\def\BoxPadding #1\relax{\initc@t\setsbp@ram{boxhpadding}{#1}\setsbp@ram{boxvpadding}{#1}}
\def\BorderHPadding #1\relax{\initc@t\setsbp@ram{borderhpadding}{#1}}
\def\BorderVPadding #1\relax{\initc@t\setsbp@ram{bordervpadding}{#1}}
\def\BorderPadding #1\relax{\initc@t\setsbp@ram{borderhpadding}{#1}\setsbp@ram{bordervpadding}{#1}}
\def\SpaceBeside #1\relax{\initc@t\setsbp@ram{spacebeside}{#1}}
\let\BorderColor=\BorderColour%for Americans
\def\Border	#1\relax{\initc@t\x@\p@rseBorder#1 \relax \relax}

\def\GenC@tFig#1{%
 \getp@r@m{#1figname}{\c@tprefix\m@rker}\ifx\p@ram\relax
   \trace{e}{No figname yet for \c@tprefix\m@rker}%
 \else
   \let\c@tfigname\p@ram
   %\getbp@r@m{#1figscale}%
   \edef\tmp{\c@tfigname|}
   \x@\setsbp@ram{#1figspec}{\tmp}%
   \trace{e}{figspec for \c@tprefix\m@rker:  \tmp}%
 \fi
}
\def\sb@ins@rtpic#1|#2|#3\E{%
  \trace{e}{Figure "#1" "#2" "#3"}%
  \dimen0=#3\relax
  \openin\t@stread="#1" 
   \ifeof\t@stread 
     \m@kepl@ceholder{\dimen0}{0.618 \dimen0}{#1}%
   \else\closein\t@stread
    \let\picfilecomm@nd=\XeTeXpicfile
    \expandafter\ch@ckpdf#1..\endf@lename
    \setbox0\hbox{\picfilecomm@nd "#1" #2}%
    \global\dimen3=\wd0
    \global\dimen4=\ht0
    \trace{e}{Image dims: \the\dimen3 x \the\dimen4}%
    \hbox{\hskip\sb@leftshift\hbox to #3{\hss\box0\hss}\hskip\sb@rightshift}%
   \fi}

\def\EndCategory{\gdef\c@tegory{}\global\let\c@tprefix\empty}

\def\S@tCat#1{\trace{sc}{Started Category #1}%
  \gdef\c@tegory{#1}%
  \s@tc@tpr@fix%
}
\def\showTheCategory{\c@tegory\message{The category is \c@tegory}}

\def\StyleCategory#1#2{\S@tCat{#1}#2\EndCategory}
\def\categorysheet#1{%Just wrap \stylesheet with category-blanking commands
  \subc@lltrue
  \EndCategory\stylesheet{#1}\EndCategory
  \subc@llfalse
}
\def\B@rderTOP{\setsbp@ram{bordertop}{T}}
\def\B@rderBOTTOM{\setsbp@ram{borderbottom}{T}}
\def\B@rderLEFT{\setsbp@ram{borderoddleft}{T}\setsbp@ram{borderevenleft}{T}}
\def\B@rderRIGHT{\setsbp@ram{borderoddright}{T}\setsbp@ram{borderevenright}{T}}
\def\B@rderINNER{\ifBookOpenLeft\setsbp@ram{borderoddright}{T}\setsbp@ram{borderevenleft}{T}\else\setsbp@ram{borderoddleft}{T}\setsbp@ram{borderevenright}{T}\fi}
\def\B@rderOUTER{\ifBookOpenLeft\setsbp@ram{borderoddleft}{T}\setsbp@ram{borderevenright}{T}\else\setsbp@ram{borderoddright}{T}\setsbp@ram{borderevenleft}{T}\fi}
\def\B@rderALL{\setsbp@ram{borderoddleft}{T}\setsbp@ram{borderevenleft}{T}\setsbp@ram{borderoddright}{T}\setsbp@ram{borderevenright}{T}\setsbp@ram{bordertop}{T}\setsbp@ram{borderbottom}{T}}
\def\B@rderNone{\setsbp@ram{borderoddleft}{F}\setsbp@ram{borderevenleft}{F}\setsbp@ram{borderoddright}{F}\setsbp@ram{borderevenright}{F}\setsbp@ram{bordertop}{F}\setsbp@ram{borderbottom}{F}}

\addtoinithooks{\let\@SB=\esb\let\@SBE=\esbe \let\esb=\@sb\let\esbe=\@sbe}
\def\g@nbox#1{%For some reason, this prefers to be in its own macro.
  \x@\newb@x\csname #1\endcsname
}
\x@\def\csname d@code-t\endcsname{\edef\tmp{cat:\tmp}}
%\x@\def\csname d@code-T\endcsname{\edef\tmp{cat:\tmp}}
\x@\def\csname d@code-T\endcsname{\let\tmp\h@phen} % a paragraph-catergory does not altering the stystak

\def\dbghrule{\vbox to 0pt{\vrule width 0.5\hsize height 0.1pt \vss}}
\def\Override{\m@kedigitsother}
\def\EndOverride{\m@kedigitsletters}
