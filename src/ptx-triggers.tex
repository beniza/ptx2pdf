%:strip
% Part of the ptx2pdf macro package for formatting USFM text
% copyright (c) 2020 by SIL International
% written by David Gardner 
%
% Permission is hereby granted, free of charge, to any person obtaining  
% a copy of this software and associated documentation files (the  
% "Software"), to deal in the Software without restriction, including  
% without limitation the rights to use, copy, modify, merge, publish,  
% distribute, sublicense, and/or sell copies of the Software, and to  
% permit persons to whom the Software is furnished to do so, subject to  
% the following conditions:
%
% The above copyright notice and this permission notice shall be  
% included in all copies or substantial portions of the Software.
%
% THE SOFTWARE IS PROVIDED "AS IS", WITHOUT WARRANTY OF ANY KIND,  
% EXPRESS OR IMPLIED, INCLUDING BUT NOT LIMITED TO THE WARRANTIES OF  
% MERCHANTABILITY, FITNESS FOR A PARTICULAR PURPOSE AND  
% NONINFRINGEMENT. IN NO EVENT SHALL SIL INTERNATIONAL BE LIABLE FOR  
% ANY CLAIM, DAMAGES OR OTHER LIABILITY, WHETHER IN AN ACTION OF  
% CONTRACT, TORT OR OTHERWISE, ARISING FROM, OUT OF OR IN CONNECTION  
% WITH THE SOFTWARE OR THE USE OR OTHER DEALINGS IN THE SOFTWARE.
%
% Except as contained in this notice, the name of SIL International  
% shall not be used in advertising or otherwise to promote the sale,  
% use or other dealings in this Software without prior written  
% authorization from SIL International.
%%%%%%%%%%%%%%%%%%%%%%%%%%%%%%%%%%%%%%%%%%%%%%%%%%%%%%%%%%%%%%%%%%%%%%%

% generic chapter.verse / milestone / etc trigger support for the ptx2pdf package
% A trigger is a piece of code that is run at a particular chapter.verse or other event.
% a triggercheck  is code to check if a certain type of trigger e.g. piclist entry is to be run.
% Triggering is based on the actual verse number/range specified in \v, not \vp (printed verse)  
% The setting is set in c@rref (and for diglots, dc@rref). Triggers should be careful
% to not alter the values of c@rref or dc@rref, as these are used by other trigger code.
% The following conventions are applied:
% GEN14.5-preverse  is triggered before the verse number is output. It is possible that this is in vertical mode after a
% paragraph break.
% GEN14.5 is triggered after the verse number has been output. 
%

\newtoks\trigg@rchecks
\trigg@rchecks{}
\def\addtotrigg@rchecks#1{\x@\global\x@\trigg@rchecks\x@{\the\trigg@rchecks #1}}

\def\runtrigg@rscv{%after the verse number
  \p@rnum=1
  \settrig@refcv
  \trace{T}{running triggers \c@rref}%
  \the\trigg@rchecks
}
\def\runtrigg@rsprecv{% before the verse number
  \p@rnum=0
  \settrig@refprecv
  \trace{T}{running triggers \c@rref}%
  \the\trigg@rchecks
}

\def\runtrigg@rspar{%Paragraph trigger
  \advance\p@rnum by 1
  \ifx\v@rse\empty\else
    \settrig@refcv
  \fi
  \let\oc@rref=\c@rref\let\odc@rref=\dc@rref % Preserve old references so that we can restore it later.
  \xdef\c@rref{\c@rref-\the\p@rnum}%
  \ifdiglot
    \xdef\dc@rref{\odc@rref-\the\p@rnum}%
  \fi
  \trace{T}{running triggers \c@rref}%
  \the\trigg@rchecks
  \let\c@rref=\oc@rref\let\dc@rref=\odc@rref
}

\def\runtrigg@rs#1{%Generic something trigger (milestone, or...)
  \def\c@rref{#1}%
  \ifdiglot\def\dc@rref{#1\g@tdstat}%
  \trace{T}{running triggers \c@rref}%
  \the\trigg@rchecks
}
\def\settrign@me#1{\edef\trign@me{trigger-#1}}

\def\addtrigger#1#2{%Add trigger code trigger on reference #1 (to do #2). 
   %Much of the complication is so the triggers concatenate roughly like a toklist
   \settrign@me{#1}\x@\let\x@\tr@gtmp\csname\trign@me\endcsname
   \ifx\tr@gtmp\relax
     \def\tr@gtmp{#2}%
   \else
     \x@\def\x@\tr@gtmp\x@{\tr@gtmp#2}%
   \fi
   \x@\let\csname\trign@me\endcsname\tr@gtmp
}

\def\stdtrigger{%Run any triggers if they exist.
   \settrign@me{\c@rref}\csname\trign@me\endcsname
   \ifdiglot
     \settrign@me{\dc@rref}\csname\trign@me\endcsname
   \fi
}

\addtotrigg@rchecks{\stdtrigger}

\def\settrig@refcv{%
  \xdef\c@rref{\id@@@\ch@pter.\v@rse}%
  \ifdiglot
    \xdef\dc@rref{\id@@@\c@rrdstat\ch@pter.\v@rse}%
  \fi
}

\newcount\p@rnum


\def\settrig@refprecv{%
  \edef\c@rref{\id@@@\ch@pter.\v@rse-preverse}%
  \ifdiglot
    \edef\dc@rref{\id@@@\c@rrdstat\ch@pter.\v@rse-preverse}%
  \fi
}

\addtopreversehooks{\runtrigg@rsprecv} % e.g. piclists
\addtoversehooks{\runtrigg@rscv} % e.g. adjustlist
\addtoeveryparhooks{\runtrigg@rspar} 
\def\c@rref{start}
\def\dc@rref{start\c@rrdstat}
