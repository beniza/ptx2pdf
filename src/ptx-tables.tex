%%%%%%%%%%%%%%%%%%%%%%%%%%%%%%%%%%%%%%%%%%%%%%%%%%%%%%%%%%%%%%%%%%%%%%%
% Part of the ptx2pdf macro package for formatting USFM text
% copyright (c) 2007-2008 by SIL International
% written by Jonathan Kew
%
% Permission is hereby granted, free of charge, to any person obtaining  
% a copy of this software and associated documentation files (the  
% "Software"), to deal in the Software without restriction, including  
% without limitation the rights to use, copy, modify, merge, publish,  
% distribute, sublicense, and/or sell copies of the Software, and to  
% permit persons to whom the Software is furnished to do so, subject to  
% the following conditions:
%
% The above copyright notice and this permission notice shall be  
% included in all copies or substantial portions of the Software.
%
% THE SOFTWARE IS PROVIDED "AS IS", WITHOUT WARRANTY OF ANY KIND,  
% EXPRESS OR IMPLIED, INCLUDING BUT NOT LIMITED TO THE WARRANTIES OF  
% MERCHANTABILITY, FITNESS FOR A PARTICULAR PURPOSE AND  
% NONINFRINGEMENT. IN NO EVENT SHALL SIL INTERNATIONAL BE LIABLE FOR  
% ANY CLAIM, DAMAGES OR OTHER LIABILITY, WHETHER IN AN ACTION OF  
% CONTRACT, TORT OR OTHERWISE, ARISING FROM, OUT OF OR IN CONNECTION  
% WITH THE SOFTWARE OR THE USE OR OTHER DEALINGS IN THE SOFTWARE.
%
% Except as contained in this notice, the name of SIL International  
% shall not be used in advertising or otherwise to promote the sale,  
% use or other dealings in this Software without prior written  
% authorization from SIL International.
%%%%%%%%%%%%%%%%%%%%%%%%%%%%%%%%%%%%%%%%%%%%%%%%%%%%%%%%%%%%%%%%%%%%%%%

% ptx-tables.tex
% support for USFM table markup

\addtoinithooks{\set@ptablemarkers}

\m@kedigitsletters
\def\set@ptablemarkers{% redefine the markers from usfm.sty
  \def\th1{\t@blecell{th1}{l}}
  \def\th2{\t@blecell{th2}{l}}
  \def\th3{\t@blecell{th3}{l}}
  \def\th4{\t@blecell{th4}{l}}
  \def\th5{\t@blecell{th5}{l}}
  \def\th6{\t@blecell{th6}{l}}
  \def\th7{\t@blecell{th7}{l}}
  \def\th8{\t@blecell{th8}{l}}
  \def\th9{\t@blecell{th9}{l}}
  \def\thc1{\t@blecell{thc1}{c}}
  \def\thc2{\t@blecell{thc2}{c}}
  \def\thc3{\t@blecell{thc3}{c}}
  \def\thc4{\t@blecell{thc4}{c}}
  \def\thc5{\t@blecell{thc5}{c}}
  \def\thc6{\t@blecell{thc6}{c}}
  \def\thc7{\t@blecell{thc7}{c}}
  \def\thc8{\t@blecell{thc8}{c}}
  \def\thc9{\t@blecell{thc9}{c}}
  \def\thr1{\t@blecell{thr1}{r}}
  \def\thr2{\t@blecell{thr2}{r}}
  \def\thr3{\t@blecell{thr3}{r}}
  \def\thr4{\t@blecell{thr4}{r}}
  \def\thr5{\t@blecell{thr5}{r}}
  \def\thr6{\t@blecell{thr6}{r}}
  \def\thr7{\t@blecell{thr7}{r}}
  \def\thr8{\t@blecell{thr8}{r}}
  \def\thr9{\t@blecell{thr9}{r}}
  \def\tc1{\t@blecell{tc1}{l}}
  \def\tc2{\t@blecell{tc2}{l}}
  \def\tc3{\t@blecell{tc3}{l}}
  \def\tc4{\t@blecell{tc4}{l}}
  \def\tc5{\t@blecell{tc5}{l}}
  \def\tc6{\t@blecell{tc6}{l}}
  \def\tc7{\t@blecell{tc7}{l}}
  \def\tc8{\t@blecell{tc8}{l}}
  \def\tc9{\t@blecell{tc9}{l}}
  \def\tcc1{\t@blecell{tcc1}{c}}
  \def\tcc2{\t@blecell{tcc2}{c}}
  \def\tcc3{\t@blecell{tcc3}{c}}
  \def\tcc4{\t@blecell{tcc4}{c}}
  \def\tcc5{\t@blecell{tcc5}{c}}
  \def\tcc6{\t@blecell{tcc6}{c}}
  \def\tcc7{\t@blecell{tcc7}{c}}
  \def\tcc8{\t@blecell{tcc8}{c}}
  \def\tcc9{\t@blecell{tcc9}{c}}
  \def\tcr1{\t@blecell{tcr1}{r}}
  \def\tcr2{\t@blecell{tcr2}{r}}
  \def\tcr3{\t@blecell{tcr3}{r}}
  \def\tcr4{\t@blecell{tcr4}{r}}
  \def\tcr5{\t@blecell{tcr5}{r}}
  \def\tcr6{\t@blecell{tcr6}{r}}
  \def\tcr7{\t@blecell{tcr7}{r}}
  \def\tcr8{\t@blecell{tcr8}{r}}
  \def\tcr9{\t@blecell{tcr9}{r}}
  \let\@TR=\tr
  \let\tr=\t@blerow
}
\m@kedigitsother

\def\t@blerow{%
  \startt@ble
  \@ndcell
  \@ndrow
  \advance\tabler@ws by 1
  \tablec@l=0
  \@tablerowtrue \@TR \@tablerowfalse
  \@rowtrue
  \ignorespaces}

\addtoparstylehooks{\if@tablerow\else\endt@ble\fi}
\newif\if@tablerow
\newif\if@row

\def\@ndrow{%
  \if@row
    \loop \ifnum\tablec@l<\maxtablec@l
      \advance\tablec@l by 1
      \x@\let\x@\c@rrentcolbox\csname tablecol-\the\tablec@l\endcsname
      \setbox\c@rrentcolbox=\vbox{\hbox{}\unvbox\c@rrentcolbox}%
      \repeat
    \@rowfalse
  \fi}

\catcode`\~=12 \lccode`\~=32
\lowercase{%
  \def\t@blecell#1#2{%
    \@ndcell
    \st@rtcell
    \x@\xdef\csname align-\the\tabler@ws-\the\tablec@l\endcsname{#2}%
    \trace{t}{Define cell [\the\tabler@ws, \the\tablec@l] = #2 in style #1}%
    \ifnum\tablec@l>\maxtablec@l \global\maxtablec@l=\tablec@l \fi
    \ch@rstyle{#1}~\ignorespaces}%
}

\newtoks\s@vedm@rks
\let\origm@rk\mark
\let\origm@rks\marks
\s@vedm@rks{}
\def\st@rtcell{%
  \advance\tablec@l by 1
  \ifnum\tablec@l>\allocatedtablec@ls \@llocnewc@l \fi
  \setbox\cellb@x=\vbox\bgroup
    \def\mark##1{\x@\global\x@\s@vedm@rks\x@{\the\s@vedm@rks \origm@rk{##1}}}%
    \def\marks##1##2{\x@\global\x@\s@vedm@rks\x@{\the\s@vedm@rks \origm@rks{##1}{##2}}}%
    \aftergroup\dos@vedm@rk%
    \hsize=\maxdimen
    \resetp@rstyle \everypar={}\noindent
    \ifRTL\beginR\fi
    \dimen255=0pt
    \getp@ram{leftmargin}{tr}%
    \ifx\p@ram\relax \else \advance\dimen255 \p@ram\IndentUnit \fi
    \getp@ram{firstindent}{tr}%
    \ifx\p@ram\relax \else \advance\dimen255 \p@ram\IndentUnit \fi
    \kern\dimen255
    \s@tfont{tr}%
    \global\p@ranesting=1
    \@celltrue}
\def\dos@vedm@rk{\the\s@vedm@rks\relax\s@vedm@rks{}}

\def\@ndcell{%
  \if@cell
    \unskip
    \end@llpoppedstyles{tc*}%
    \getp@ram{rightmargin}{tr}%
    \ifx\p@ram\relax\else\kern\p@ram\IndentUnit\fi
    \egroup
    \setbox0=\vbox{\unvbox\cellb@x \global\setbox\cellb@x=\lastbox}%
    \global\setbox\cellb@x=\hbox{\unhbox\cellb@x \unskip\unskip\unpenalty}%
    \x@\let\x@\c@rrentcolbox\csname tablecol-\the\tablec@l\endcsname
    \ifvoid\c@rrentcolbox % adding a column, may need to add empty cells above
      \count255=1
      \loop \ifnum\count255<\tabler@ws \advance\count255 by 1
        \setbox\c@rrentcolbox=\vbox{\hbox{}\unvbox\c@rrentcolbox}%
        \repeat
    \fi
    \trace{t}{Add cell width \the\wd\cellb@x, to row \the\tabler@ws}%
    \setbox\c@rrentcolbox=\vbox{\box\cellb@x\unvbox\c@rrentcolbox}%
      % note that this stacks the cells upwards! they'll be reversed later
  \fi}

\newif\if@cell
\newbox\cellb@x

\def\startt@ble{%
  \ifdoingt@ble \else
    \ifhe@dings\endhe@dings\fi
    \tabler@ws=0
    \tablec@l=0
    \global\maxtablec@l=0
    \doingt@bletrue\inn@tetrue
    \message{starting table}%
  \fi}

\newdimen\t@talwidth
\def\endt@ble{%
  \ifdoingt@ble
    \@ndcell
    \@ndrow
    \t@talwidth=0pt
    \f@reachcol{\advance\t@talwidth by \wd\c@rrentcolbox}%
    \endt@blewrap
    \f@reachcol{\setbox\c@rrentcolbox=\box\voidb@x}%
    \doingt@blefalse\inn@tefalse
  \fi}
\newif\ifdoingt@ble

\def\endt@blewrap{%
  % calculate column widths as:
  % (a) any column whose max is < \hsize / numcols gets its natural width
  % (b) calculate exc@ss width of the remaining columns
  % (c) calculate sp@re width available from the narrow ones
  % (d) distribute this in proportion to the relative exc@ss of each
  \sp@re=0pt \exc@ss=0pt
  \thr@shold=\hsize \divide\thr@shold by \maxtablec@l
  \f@reachcol{%
    \c@rrentcolwidth=-1sp % "unset" flag
    \ifdim\wd\c@rrentcolbox>\thr@shold
      \advance\exc@ss by \wd\c@rrentcolbox \advance\exc@ss by -\thr@shold
    \else
      \c@rrentcolwidth=\wd\c@rrentcolbox
      \advance\sp@re by \thr@shold \advance\sp@re by -\c@rrentcolwidth
    \fi}%
  \f@reachcol{%
    \ifdim\c@rrentcolwidth=-1sp
      \ifdim\exc@ss>\sp@re % we have to wrap at least one column
        \c@rrentcolwidth=\wd\c@rrentcolbox \advance\c@rrentcolwidth by -\thr@shold
        \multiply\c@rrentcolwidth by 20
        \divide\c@rrentcolwidth by \exc@ss
        \multiply\c@rrentcolwidth by \sp@re
        \divide\c@rrentcolwidth by 20
        \advance\c@rrentcolwidth by \thr@shold
      \else % there was enough sp@re, just give columns their desired width
        \c@rrentcolwidth=\wd\c@rrentcolbox
      \fi
    \fi
    %\message{Table col width \the\c@rrentcolwidth}%
  }%
%  \f@reachcol{%
%    \MSG{column \the\@col: width=\the\c@rrentcolwidth\space
%      \ifdim\c@rrentcolwidth<\wd\c@rrentcolbox
%        (wrapped, natural was \the\wd\c@rrentcolbox)\fi}%
%  }%
  \f@reachcol{\wr@pcolumn}%
  \f@reachrow{%
    \ifnum\@row=2 \nobreak \fi
    \ifnum\@row=\tabler@ws \nobreak \fi
    \linec@unt=0
    \f@reachcol{\extractl@stcell \updatelinec@unt}%
    \nth@line=0
    \@@LOOP \ifnum\nth@line<\linec@unt
      \advance\nth@line by 1
      \line{\ifRTL\hfil\lochcaer@f{\getnth@line}\hfil\else
        \hfil\f@reachcol{\getnth@line}\hfil\fi}%
      \ifnum\nth@line<\linec@unt \nobreak \fi
      \@@REPEAT
  }%
}
\newcount\linec@unt \newcount\lct@mp
\newdimen\sp@re \newdimen\exc@ss \newdimen\thr@shold

\def\@LOOP #1\@REPEAT{\gdef\@BODY{#1}\@ITERATE}
\def\@ITERATE{\@BODY \global\let\@NEXT\@ITERATE
 \else \global\let\@NEXT\relax \fi \@NEXT}
\let\@REPEAT\fi
% more copies of plain.tex loop macros (see also ptx-char-style)
\def\@@LOOP #1\@@REPEAT{\gdef\@@BODY{#1}\@@ITERATE}
\def\@@ITERATE{\@@BODY \global\let\@@NEXT\@@ITERATE
 \else \global\let\@@NEXT\relax \fi \@@NEXT}
\let\@@REPEAT\fi

\def\@@@LOOP #1\@@@REPEAT{\gdef\@@@BODY{#1}\@@@ITERATE}
\def\@@@ITERATE{\@@@BODY \global\let\@@@NEXT\@@@ITERATE
 \else \global\let\@@@NEXT\relax \fi \@@@NEXT}
\let\@@@REPEAT\fi

\def\wr@pcolumn{\@row=1\setbox\c@rrentcolbox=\vbox{\unvbox\c@rrentcolbox \wr@pboxes}}
\def\wr@pboxes{% line-wrap a list of \hboxes
  \setbox0=\lastbox 
  \ifhbox0 {\advance\@row by 1 \wr@pboxes}\fi
  \setbox0=\vbox{\hsize=\c@rrentcolwidth
    \everypar={}\resetp@rstyle\noindent
    \ifRTL\beginR\fi
    \getp@ram{leftmargin}{tr}%
    \ifx\p@ram\relax \else \advance \ifRTL\rightskip\else\leftskip\fi\p@ram \IndentUnit\fi
    \kern-\p@ram\IndentUnit
    \getp@ram{rightmargin}{tr}%
    \ifx\p@ram\relax \else \advance \ifRTL\leftskip\else\rightskip\fi\p@ram \IndentUnit\fi
    \x@\let\x@\the@lign\csname align-\the\@row-\the\@col\endcsname
    \ifx\the@lign\cell@r
      \ifRTL\let\n@xt=\rightskip\else\let\n@xt=\leftskip\fi
    \else\ifx\the@lign\cell@c
      \let\n@xt=\rightskip \advance\leftskip by 0pt plus 1fil
    \else
      \ifRTL\let\n@xt=\leftskip\else\let\n@xt=\rightskip\fi
    \fi\fi
    \advance\n@xt by 0pt plus 1fil
    \ifx\the@lign\cell@r
        \nobreak\csname leaders-tcr\the\@col\endcsname\hfil
    \else\ifx\the@lign\cell@c
        \nobreak\csname leaders-tcc\the\@col\endcsname
    \fi\fi
    \trace{t}{cell(\the@lign)[\the\@row, \the\@col]: leaders. leftskip=\the\leftskip, rightskip=\the\rightskip, parfillskip=\the\parfillskip}%
    \unhbox0\unskip \parfillskip=0pt
    \ifx\the@lign\cell@r\else
      \ifx\the@lign\cell@c
        \nobreak\csname leaders-tcc\the\@col\endcsname\null
      \else
        \nobreak\csname leaders-tc\the\@col\endcsname\hfil\null
    \fi\fi
    \par}%
  \wd0=\c@rrentcolwidth
  \box0 }
\def\cell@r{r}
\def\cell@c{c}
\def\extractl@stcell{% delete last cell from box \c@rrentcolbox and return it in \c@rrentcell
  \setbox\c@rrentcolbox=\vbox{\unvbox\c@rrentcolbox \unskip\unpenalty
    \global\setbox\c@rrentcell=\lastbox}}
\def\updatelinec@unt{% count lines in \c@rrentcell, update \linec@unt if more than current value
  \global\lct@mp=0
  \setbox0=\vbox{\unvcopy\c@rrentcell \c@untlines}%
  \ifnum\lct@mp>\linec@unt \global\linec@unt=\lct@mp \fi}
\def\c@untlines{\setbox0=\lastbox
  \ifhbox0 \global\advance\lct@mp by 1 \unskip\unpenalty\c@untlines
  \trace{t}{countd \the\lct@mp\space lines}%
  \fi}
\def\getnth@line{% return content of line nth@line from box \c@rrentcell
  \lct@mp=0
  \setbox0=\vbox{\unvcopy\c@rrentcell \c@untlines}% count lines into \lct@mp
  \ifnum\nth@line>\lct@mp \global\setbox1=\hbox to \c@rrentcolwidth{\hfil}\else
    \setbox0=\vbox{\unvcopy\c@rrentcell
      \@@@LOOP \ifnum\nth@line<\lct@mp
        \setbox0=\lastbox \unskip\unpenalty
        \advance\lct@mp by -1 \@@@REPEAT
      \global\setbox1=\lastbox}%
  \fi
  \box1
}
\newcount\nth@line

\def\f@reachcol#1{%
  \@col=0 \loop \ifnum\@col<\maxtablec@l
    \advance\@col by 1
    \x@\let\x@\c@rrentcolbox\csname tablecol-\the\@col\endcsname
    \x@\let\x@\c@rrentcolwidth\csname colwidth-\the\@col\endcsname
    \x@\let\x@\c@rrentcell\csname cell-\the\@col\endcsname
    #1\repeat}

\def\lochcaer@f#1{%
  \@col=\maxtablec@l \loop \ifnum\@col>0
    \x@\let\x@\c@rrentcolbox\csname tablecol-\the\@col\endcsname
    \x@\let\x@\c@rrentcolwidth\csname colwidth-\the\@col\endcsname
    \x@\let\x@\c@rrentcell\csname cell-\the\@col\endcsname
    #1%
    \advance\@col by -1
    \repeat}

\def\f@reachrow#1{%
  \@row=0 \@LOOP \ifnum\@row<\tabler@ws
    \advance\@row by 1
    #1\@REPEAT}

\newcount\tabler@ws
\newcount\maxtablec@l
\newcount\tablec@l
\newcount\@col \newcount\@row
\newcount\allocatedtablec@ls

\def\@llocnewc@l{%
  \advance\allocatedtablec@ls by 1
  \x@\n@wbox\csname tablecol-\the\allocatedtablec@ls\endcsname
  \x@\n@wbox\csname cell-\the\allocatedtablec@ls\endcsname
  \x@\n@wdimen\csname colwidth-\the\allocatedtablec@ls\endcsname
}
\let\n@wbox=\newbox
\let\n@wdimen=\newdimen

