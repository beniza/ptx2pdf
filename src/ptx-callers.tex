%%%%%%%%%%%%%%%%%%%%%%%%%%%%%%%%%%%%%%%%%%%%%%%%%%%%%%%%%%%%%%%%%%%%%%%
% Part of the ptx2pdf macro package for formatting USFM text
% copyright (c) 2007 by SIL International
% written by Jonathan Kew
%
% Permission is hereby granted, free of charge, to any person obtaining  
% a copy of this software and associated documentation files (the  
% "Software"), to deal in the Software without restriction, including  
% without limitation the rights to use, copy, modify, merge, publish,  
% distribute, sublicense, and/or sell copies of the Software, and to  
% permit persons to whom the Software is furnished to do so, subject to  
% the following conditions:
%
% The above copyright notice and this permission notice shall be  
% included in all copies or substantial portions of the Software.
%
% THE SOFTWARE IS PROVIDED "AS IS", WITHOUT WARRANTY OF ANY KIND,  
% EXPRESS OR IMPLIED, INCLUDING BUT NOT LIMITED TO THE WARRANTIES OF  
% MERCHANTABILITY, FITNESS FOR A PARTICULAR PURPOSE AND  
% NONINFRINGEMENT. IN NO EVENT SHALL SIL INTERNATIONAL BE LIABLE FOR  
% ANY CLAIM, DAMAGES OR OTHER LIABILITY, WHETHER IN AN ACTION OF  
% CONTRACT, TORT OR OTHERWISE, ARISING FROM, OUT OF OR IN CONNECTION  
% WITH THE SOFTWARE OR THE USE OR OTHER DEALINGS IN THE SOFTWARE.
%
% Except as contained in this notice, the name of SIL International  
% shall not be used in advertising or otherwise to promote the sale,  
% use or other dealings in this Software without prior written  
% authorization from SIL International.
%%%%%%%%%%%%%%%%%%%%%%%%%%%%%%%%%%%%%%%%%%%%%%%%%%%%%%%%%%%%%%%%%%%%%%%

% footnote caller support for the ptx2pdf package

%
% Set a list of callers to be used for a given note style
%   #1 -> the note style, e.g., f or x
%   #2 -> comma-separated list of callers (may be multi-character)
%
\def\AutoCallers#1#2{%
  \expandafter\def\csname callers-#1\endcsname{#2}%
      % store the caller list in a macro \callers-#1
  \numc@llers=0
  \expandafter\expandafter\expandafter
    \countc@llers\csname callers-#1\endcsname,\end
      % count number of callers in the list and store in \callercount-#1
  \expandafter\edef\csname callercount-#1\endcsname{\the\numc@llers}}

\def\countc@llers#1,#2\end{\def\t@st{#1}% recursively count items in comma-separated list
  \ifx\t@st\empty \let\n@xt\donec@llers
  \else \advance\numc@llers by 1 \let\n@xt\countc@llers \fi
  \n@xt#2,\end}
\def\donec@llers#1\end{}

%
% Generate an autocaller
%   #1 -> note number (sequential from 1)
%   #2 -> note style, e.g., f or x
%
\def\getcaller#1#2{%
  \c@llernum=#1\relax
  \trace{n}{Caller requested for #2 (#1)}%
  \expandafter\ifx\csname page-caller #2\endcsname\relax \else % if this note class is using per-page callers...
    \write\n@tepages{\string\noteonpage{#2}{#1}{\the\pageno}}% record the page where this note occurred
    \expandafter\let\expandafter\thisc@llerpage\csname notepage-#2-#1\endcsname
    \ifx\thisc@llerpage\relax \else % check where the note occurred last time
      \count255=\c@llernum \c@llernum=1 % reset c@llernum to 1
      \loop \advance\count255 by -1 % then increment it by the number of prev callers on same page
        \expandafter\let\expandafter\prevc@llerpage\csname notepage-#2-\number\count255\endcsname
        \ifx\thisc@llerpage\prevc@llerpage \advance\c@llernum by 1 \repeat
    \fi
  \fi
  \expandafter\ifx\csname callers-#2\endcsname\relax % if caller list is empty
    \expandafter\ifx\csname numeric-callers #2\endcsname\relax % and numeric callers not specified
      \loop\ifnum\c@llernum>\AutoCallerNumChars
        \advance\c@llernum by -\AutoCallerNumChars \repeat
      \advance\c@llernum by -1 \advance\c@llernum by \AutoCallerStartChar
      \char\c@llernum % generate an alphabetic auto-caller
    \else
      \number\c@llernum % generate a numeric caller
    \fi
  \else % caller list is not empty
    \edef\c@llers{\csname callers-#2\endcsname}%
    \ifx\c@llers\empty \else
    \numc@llers=\csname callercount-#2\endcsname
    \loop \ifnum\c@llernum>\numc@llers % reduce \c@llernum modulo \numc@llers
      \advance\c@llernum by -\numc@llers \repeat
    \loop \ifnum\c@llernum>1 % then strip \c@llernum-1 callers from the front of the list
      \advance\c@llernum by -1
      \expandafter\dropfirstc@ller\c@llers,\end \repeat
    \expandafter\getfirstc@ller\c@llers,\end % and extract the one we want
    \fi
  \fi
}

\def\dropfirstc@ller#1,#2\end{\def\c@llers{#2}}
\def\getfirstc@ller#1,#2\end{#1}
\newcount\c@llernum \newcount\numc@llers

\newwrite\n@tepages
\newread\n@tepagetest
\def\@nputn@tepages{
    \openin\n@tepagetest = "\jobname.notepages" % test if notepages file exists from prev run
    \ifeof\n@tepagetest \let\n@xt\relax
    \else \def\n@xt{\input "\jobname.notepages"}
    \fi \closein\n@tepagetest
    \n@xt
}

\def\PageResetCallers#1{% if we want per-page callers, we have to generate and read an auxiliary file
  \ifpagec@llers \else
    \@nputn@tepages
    \pagec@llerstrue
    \addtoendhooks{\finishc@llers}
    \openout\n@tepages="\jobname.notepages"
  \fi
  \expandafter\let\csname page-caller #1\endcsname=1 % record that this note class wants per-page callers
}
\newif\ifpagec@llers
\def\noteonpage#1#2#3{% style, number, page: defines \notepage-<style>-<number> as the page number where this note occurs
  \expandafter\def\csname notepage-#1-#2\endcsname{#3}}
\def\finishc@llers{\immediate\closeout\n@tepages % close the notepages file, then re-read it and check for changes
  \n@teschangedfalse
  \m@kedigitsother
  \let\noteonpage=\checkn@tepage \@nputn@tepages
  \ifn@teschanged \msg{*** Notes may have changed; re-run to update callers}\fi % if so, notify the user
}
\def\checkn@tepage#1#2#3{\def\t@st{#3}% test whether note has moved since the previous run
  \expandafter\ifx\expandafter\t@st\csname notepage-#1-#2\endcsname
    \else \n@teschangedtrue \fi}
\newif\ifn@teschanged

\def\NumericCallers#1{% use numeric rather than alphabetic auto-callers for style #1
  \expandafter\let\csname numeric-callers #1\endcsname=1}

% Character numbers used for default caller sequence
\newcount\AutoCallerStartChar
\newcount\AutoCallerNumChars

\AutoCallerStartChar=97 % start at 'a'
\AutoCallerNumChars=26 % and generate 26 callers before restarting

% To restart numbering class "f" on each page:
% \PageResetCallers{f}

% To make note class "f" use a specific sequence of symbols:
% \AutoCallers{f}{*,†,‡,¶,§,**,††,‡‡,¶¶,§§}

% To make note class "x" omit callers:
% \AutoCallers{x}{}

% To make note class "f" default to numbers rather than chars
% (an explicit list of \AutoCallers will still override this)
% \NumericCallers{f}

%% override the autonumber macro from ptx-note-style.tex:
\def\gen@utonum#1{\count255=0\csname autonum #1\endcsname
  \edef\n@mber{\the\count255}%
  \expandafter\def\expandafter\them@rk
    \expandafter{\expandafter\getcaller\expandafter{\n@mber}{#1}}}

\endinput
