%:strip
% Part of the ptx2pdf macro package for formatting USFM text
% copyright (c) 2020 by SIL International
% written by David Gardner 
%
% Permission is hereby granted, free of charge, to any person obtaining  
% a copy of this software and associated documentation files (the  
% "Software"), to deal in the Software without restriction, including  
% without limitation the rights to use, copy, modify, merge, publish,  
% distribute, sublicense, and/or sell copies of the Software, and to  
% permit persons to whom the Software is furnished to do so, subject to  
% the following conditions:
%
% The above copyright notice and this permission notice shall be  
% included in all copies or substantial portions of the Software.
%
% THE SOFTWARE IS PROVIDED "AS IS", WITHOUT WARRANTY OF ANY KIND,  
% EXPRESS OR IMPLIED, INCLUDING BUT NOT LIMITED TO THE WARRANTIES OF  
% MERCHANTABILITY, FITNESS FOR A PARTICULAR PURPOSE AND  
% NONINFRINGEMENT. IN NO EVENT SHALL SIL INTERNATIONAL BE LIABLE FOR  
% ANY CLAIM, DAMAGES OR OTHER LIABILITY, WHETHER IN AN ACTION OF  
% CONTRACT, TORT OR OTHERWISE, ARISING FROM, OUT OF OR IN CONNECTION  
% WITH THE SOFTWARE OR THE USE OR OTHER DEALINGS IN THE SOFTWARE.
%
% Except as contained in this notice, the name of SIL International  
% shall not be used in advertising or otherwise to promote the sale,  
% use or other dealings in this Software without prior written  
% authorization from SIL International.
%%%%%%%%%%%%%%%%%%%%%%%%%%%%%%%%%%%%%%%%%%%%%%%%%%%%%%%%%%%%%%%%%%%%%%%

\newif\ifin@ttrib



%\tracingall=1

\catcode`\|=\active
\def\use@ttrSlash{%
    \let|=\@ttrSlash %this one is easy (for use with milestones)
}
\def\use@ttrGrab{%
    \let|=\start@ttributegrab %have to build attributes char-by-char, sadly.
}
\catcode`\|=12
\def\kill@ttrib #1\E{%
  \trace{A}{removing (old) attribute '#1'}%
  \x@\global\x@\let\csname attr:#1\endcsname=\undefined
}
\def\unset@ttribs{% Remove any current attibute definitions
  \let\d@=\kill@ttrib
  \x@\cstackdown \attribsus@d,\E
  \xdef\attribsus@d{}%
  \relax\relax
  \trace{A}{attributes list reset}%
}
\def\init@ttribs{%Attribute pre-setupcode 
  %\tracingassigns=1
  \edef\thiswh@tstyle{\ifmst@nestyle\thismil@stone\else\thisch@rstyle\fi}%
  \trace{A}{InitAttribs \thiswh@tstyle (\csname thisch@rstyle\endcsname,\csname thismil@stone\endcsname)}%
  \unset@ttribs% Clear any old attributes.
  \let\attrkey\relax
  %\x@\let\x@\attrkey\csname defaultattrkey-\thisch@rstyle\endcsname
  \x@\let\x@\t@st\csname thiswh@tstyle\endcsname
  \ifx\t@st\relax
    \x@\let\x@\attrkey\csname defaultattrkey-\thiswh@tstyle\endcsname
  \else
%
    \x@\let\x@\attrkey\csname defaultattrkey-\thiswh@tstyle\endcsname
  \fi
  \ifx\attrkey\relax
    \let\thisdefault@ttrkey=\default@ttrkey
  \else
    \let\thisdefault@ttrkey=\attrkey
    \trace{A}{Attributes with no keyname will be \attrkey}%
  \fi
  \edef\attrid{\milestoneOp id}% id / sid / eid, depending.
  \catcode`\|=\active
  \ifmst@nestyle
    \use@ttrSlash
  \else
    \use@ttrGrab
  \fi
  \in@ttribfalse
  \xdef\@ttributes{}%
}


\def\proc@ttribs{%
  \trace{A}{Processing attributes (\ifin@ttrib true\else false\fi)}%
  \ifin@ttrib
    \trace{A}{Attributes specified:\attrib@rgs}%
    \parse@ttribs{\attrib@rgs}%
    \in@ttribfalse
  \fi
  \catcode`\|=12
}


\def\setdefaultattrib#1#2{\trace{A}{Default attrib for #1 is #2}\x@\xdef\csname defaultattrkey-#1\endcsname{#2}}

\x@\def\csname nostore-attrib-eid\endcsname{}
\x@\def\csname nostore-attrib-sid\endcsname{}
\def\extend@ttriblist#1#2{%
  \edef\tmpa{#2}%
  \edef\@ttrlistname{@ttriblist-#1}%
  \x@\let\x@\tmp\csname\@ttrlistname\endcsname
  \ifx\tmp\relax
    \x@\let\x@\tmp\csname defaultattrkey-#1\endcsname
  \fi
  \x@\xdef\csname \@ttrlistname\endcsname{\tmp,#2}}

\edef\p@stattribcmd{}% Parse the attributes, run any code.
\def\store@ttributes#1{\def\newbit{#1}\message{'\newbit'}\edef\@ttributes{\@ttributes\newbit}\futurelet\nxt\isitslash}
\def\startp@stattribcmd#1{\futurelet\nxt\isitcmd}
\def\storep@stattribcmd#1{\def\newbit{#1}\message{"\newbit"}\edef\p@stattribcmd{\p@stattribcmd\newbit}\futurelet\nxt\isitcmd}
\lccode`\~=32
\lowercase{
 \gdef\isitcmd{%
        \let\c@ntinue\storep@stattribcmd
        \ifcat a\nxt\else%\ifcat=\nxt \else
	\let\c@ntinue\end@ttributegrab\fi%\fi
	\c@ntinue}%
}
%\def\foo#1{Thatsit "#1"}

\def\oddc@tcodes{\catcode`"=11 \catcode`==11 
\catcode`\ =11 \catcode`|=0\catcode`\\=12}                                                                                 
\def\normalc@tcodes{\catcode`"=12 \catcode`==12 
\catcode`\ =10\catcode`\\=0\catcode`\|=12}                                                                              
\trace{A}{CATCODES: "\the\catcode`",  =\the\catcode`=,  \the\catcode`\ , *\the\catcode`\*,  \the\catcode`\\,  \the\catcode`\|}%


\xdef\attribsus@d{}% List of used attributes.

\def\start@ttributegrab{\let\c@ntinue\isitslash\oddc@tcodes\futurelet\nxt\c@ntinue}
\def\end@ttributegrab{%
  \xdef\@ttributes{\x@\detokenize\x@{\@ttributes}}% Make it all strings.
  \trace{A}{Now executing \p@stattribcmd}%
  \csname\p@stattribcmd\endcsname}% 

\oddc@tcodes%NO SPACES until |normalc@tcodes is called.
|gdef|isitslash{|let|c@ntinue|store@ttributes|if|nxt\|let|c@ntinue|startp@stattribcmd|normalc@tcodes|fi|c@ntinue}%
|normalc@tcodes%
%

%Parse key="value" or value.
\def\default@ttrkey{} % What unnamed attributes get saved 
\def\thisdefault@ttrkey{} % What do unnamed attributes get saved as?
\def\relaxval{\relax}
\def\milestoneJoinChar{ } % While replacing ' ' with '_' is  nice for styling things like \qt-s |the crowd\*, this then breaks equivalence with ... who="The crowd"
\def\save@ttribute#1="#2"=#3\E{%
  \lowercase{\def\attr@key{#1}}\def\attr@val{#2}%
  \ifx\attr@val\relaxval
    \ifx\thisdefault@ttrkey\empty
      \edef\thisdefault@ttrkey{UnNamed}%
    \fi
    \get@ttribute{\thisdefault@ttrkey}%
    \if\attr@b\relax
      \edef\attr@val{#1}%
    \else
      \ifmst@nestyle
	\edef\attr@val{\attr@b\milestoneJoinChar#1}% Join words with milestoneJoinChar instead of space.
      \else
	\edef\attr@val{\attr@b\space #1}% Join words with a space.
      \fi 
    \fi
    \trace{A}{Got unnamed(\thisdefault@ttrkey) attribute #1 -> \attr@val}%
    \set@ttribute{\thisdefault@ttrkey}{\attr@val}%
  \else
    \trace{A}{Got named attribute \attr@key="\attr@val"}%
    \set@ttribute{\attr@key}{\attr@val}%
  \fi
}

\def\get@ttribute#1{\x@\let\x@\attr@b\csname attr:#1\endcsname \trace{A}{attrib #1 is \attr@b}} %simple access function
\def\set@ttribute#1#2{%
  \x@\xdef\csname attr:#1\endcsname{#2}%
  \xdef\attribsus@d{\ifx\attribsus@d\empty\else\attribsus@d,\fi#1}% Keep a list so we can junk it later.
}

% Parse a space-separated list.
\def\relax@tem#1\E{}

\def\zap@space#1 #2{%borrowed from latex.ltx. Eats all spaces in arg, including mid-space
  #1%
  \ifx#2\empty\else\expandafter\zap@space\fi
  #2}

\def\geton@item #1 #2\E{% Parse space separated list
  \edef\it@m{#1}%
  \edef\it@mtwo{\zap@space #2 \empty}%
  \trace{A}{getone pair (#1) (#2) -> (\it@m) (\it@mtwo)}%
  \ifx\it@m\empty\ifx\it@mtwo\empty\let\nxt@tem=\relax@tem\fi\else
    \D@first{#1}%
  \fi
  \nxt@tem #2 \E
}


% Parse an attribute="value" from the USFM code.
\def\s@ve@ttribute#1{%
   \x@\save@ttribute #1="\relax"=\E
}

\def\parse@ttribs#1{%
  \let\nxt@tem=\geton@item
  \let\D@first=\s@ve@ttribute
  \x@\geton@item #1 \E
}

% Parse list of attributes from style file to read the default attribute
% and the complete list(comma separated).
\def\Attributes#1\relax{% Override empty \Attributes from ptx-stylesheets
  \let\nxt@tem=\geton@item %only need the first one.
  \let\D@first=\interpr@tfirstattrib
  \let\save@ttrib=\setdefaultattrib
  \let\@ttrlistname\relax
  \x@\geton@item #1 \E
  \ifx\@ttrlistname\relax\else 
    \trace{A}{Attribute list for \m@rker\space is \csname\@ttrlistname\endcsname}%
  \fi
}


\def\interpr@tfirstattrib#1{%
  \x@\interpr@t@ttrib #1??\E
  \let\D@first=\interpr@tother@ttrib
  \let\save@ttrib=\extend@ttriblist
}
\def\interpr@tother@ttrib#1{
  \x@\interpr@t@ttrib #1??\E
}
\def\store@nattrib#1-#2|#3\E{%Ignore - in milestone markers
  \save@ttrib{#1}{#3}%
}


\def\interpr@t@ttrib#1?#2?#3\E{% There may be a preceding ? mark. Also setdefaultattrib needs to not have the -s / -e for milestones. 
  \def\tmp{#1}
  \ifx\tmp\empty
     \trace{A}{Attribute '#2' is optional (#1,#2,#3)}%
     \x@\store@nattrib\m@rker-|#2\E
  \else
     \trace{A}{Attribute '#1' is officially required (#1,#2,#3)}%
     \x@\store@nattrib\m@rker-|#1\E
  \fi
}

\def\alignb@x#1#2{\getp@ram{justification}{#1}%
    \ifx\p@ram\c@nter\hfil\else\ifx\p@ram\r@ght\hfil\else\ifx\p@ram\relax\hfil\fi\fi\fi
    \box#2%
    \ifx\p@ram\c@nter\hfil\else\ifx\p@ram\l@ft\hfil\else\ifx\p@ram\relax\hfil\fi\fi\fi}
\def\rubyb#1#2{\setbox1=\hbox{\s@tfont{#2|#1}\get@ttribute{#2}\attr@b}
    \dimen0=\ifdim\wd0>\wd1 \wd0\else\wd1\fi
    \setbox0=\hbox{\vtop{\s@tbaseline{#1}\hbox to \dimen0{\alignb@x{#1}{0}}
        \hbox to \dimen0{\alignb@x{#2|#1}{1}}}}}
\def\rubyt#1#2{\setbox1=\hbox{\s@tfont{#2|#1}\get@ttribute{#2}\attr@b}
    \dimen0=\ifdim\wd0>\wd1 \wd0\else\wd1\fi
    \setbox0=\hbox{\vbox{\s@tbaseline{#2|#1}\hbox to \dimen0{\alignb@x{#2|#1}{1}}
        \hbox to \dimen0{\alignb@x{#1}{0}}}}}

%usfm 3.0 defaults - now loaded from stylesheet.
%\setdefaultattrib{qt}{who}
%\setdefaultattrib{rb}{gloss}
%\setdefaultattrib{w}{lemma}

