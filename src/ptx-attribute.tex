%:strip
% Part of the ptx2pdf macro package for formatting USFM text
% copyright (c) 2020 by SIL International
% written by David Gardner 
%
% Permission is hereby granted, free of charge, to any person obtaining  
% a copy of this software and associated documentation files (the  
% "Software"), to deal in the Software without restriction, including  
% without limitation the rights to use, copy, modify, merge, publish,  
% distribute, sublicense, and/or sell copies of the Software, and to  
% permit persons to whom the Software is furnished to do so, subject to  
% the following conditions:
%
% The above copyright notice and this permission notice shall be  
% included in all copies or substantial portions of the Software.
%
% THE SOFTWARE IS PROVIDED "AS IS", WITHOUT WARRANTY OF ANY KIND,  
% EXPRESS OR IMPLIED, INCLUDING BUT NOT LIMITED TO THE WARRANTIES OF  
% MERCHANTABILITY, FITNESS FOR A PARTICULAR PURPOSE AND  
% NONINFRINGEMENT. IN NO EVENT SHALL SIL INTERNATIONAL BE LIABLE FOR  
% ANY CLAIM, DAMAGES OR OTHER LIABILITY, WHETHER IN AN ACTION OF  
% CONTRACT, TORT OR OTHERWISE, ARISING FROM, OUT OF OR IN CONNECTION  
% WITH THE SOFTWARE OR THE USE OR OTHER DEALINGS IN THE SOFTWARE.
%
% Except as contained in this notice, the name of SIL International  
% shall not be used in advertising or otherwise to promote the sale,  
% use or other dealings in this Software without prior written  
% authorization from SIL International.
%%%%%%%%%%%%%%%%%%%%%%%%%%%%%%%%%%%%%%%%%%%%%%%%%%%%%%%%%%%%%%%%%%%%%%%

\newif\ifin@ttrib



\def\@ttrSlash#1\*{\xdef\attrib@rgs{#1}\message{Attr:\attrib@rgs}\*}
%\tracingall=1

\catcode`\|=\active
\def\use@ttrSlash{%
    \let|=\@ttrSlash %this one is easy
}
\def\use@ttrGrab{%
    \let|=\start@ttributegrab %have to build attributes char-by-char, sadly.
}
\catcode`\|=12
\def\kill@ttrib #1\E{%
  \trace{A}{Destroying attribute #1}%
  \x@\let\csname attr:#1\endcsname=\undefined
}
\def\unset@ttribs{% Destroy any current attibute definitions
  \let\d@=\kill@ttrib
  \x@\cstackdown \attribsus@d,\E
  \xdef\attribsus@d{}%
  \relax\relax
  \trace{A}{attributes list reset}%
}
\def\init@ttribs{%Attribute pre-setupcode 
  \tracingassigns=1
  \trace{A}{InitAttribs}%
  \unset@ttribs% Clear any old attributes.
  \let\attrkey\relax
  %\x@\let\x@\attrkey\csname defaultattrkey-\thisch@rstyle\endcsname
  \x@\let\x@\t@st\csname thismil@stone\endcsname
  \ifx\t@st\relax
    \x@\let\x@\attrkey\csname defaultattrkey-\thisch@rstyle\endcsname
  \else
%
    \x@\let\x@\attrkey\csname defaultattrkey-\thismil@stone\endcsname
  \fi
  \ifx\attrkey\relax
    \let\thisdefault@ttrkey=\default@ttrkey
  \else
    \let\thisdefault@ttrkey=\attrkey
    \trace{A}{Attributes with no keyname will be \attrkey}%
  \fi
  \catcode`\|=\active
  \ifmst@nestyle
  %\iffalse
    \use@ttrSlash
  \else
    \use@ttrGrab
  \fi
  \def\attrst@ck{,}%
  \in@ttribfalse
  \xdef\@ttributes{}%
}


\def\proc@ttribs{%
  \ifin@ttrib
    \trace{A}{Attributes specified:\attrib@rgs}%
    \parse@ttribs{\attrib@rgs}%
    \in@ttribfalse
  \fi
  \catcode`\|=12
}


\def\setdefaultattrib#1#2{\trace{A}{Default attrib for #1 is #2}\x@\def\csname defaultattrkey-#1\endcsname{#2}}

\edef\p@stattribcmd{}% Parse the attributes, run any code.
\def\store@ttributes#1{\def\newbit{#1}\message{'\newbit'}\edef\@ttributes{\@ttributes\newbit}\futurelet\nxt\isitslash}                                                                    
\def\startp@stattribcmd#1{\futurelet\nxt\isitcmd}
\def\storep@stattribcmd#1{\def\newbit{#1}\message{"\newbit"}\edef\p@stattribcmd{\p@stattribcmd\newbit}\futurelet\nxt\isitcmd}                                                                          
\lccode`\~=32
\lowercase{
 \gdef\isitcmd{%
        \let\c@ntinue\storep@stattribcmd
        \ifcat a\nxt\else\ifcat=\nxt
        \else \let\c@ntinue\end@ttributegrab\fi\fi\c@ntinue}
}
%\def\foo#1{Thatsit "#1"}

\def\oddc@tcodes{\catcode`"=11 \catcode`==11 
\catcode`\ =11 \catcode`|=0\catcode`\\=12}                                                                                 
\def\normalc@tcodes{\catcode`"=12 \catcode`==12 
\catcode`\ =10\catcode`\\=0\catcode`\|=12}                                                                              
\message{CATCODES: "\the\catcode`",  =\the\catcode`=,  \the\catcode`\ , *\the\catcode`\*,  \the\catcode`\\,  \the\catcode`\|}%


\xdef\attribsus@d{}% List of used attributes.

\def\start@ttributegrab{\let\c@ntinue\isitslash\oddc@tcodes\futurelet\nxt\c@ntinue}
\def\end@ttributegrab{%
  \xdef\@ttributes{\x@\detokenize\x@{\@ttributes}}% Make it all strings.
  \message{Now executing \p@stattribcmd}%
  \csname\p@stattribcmd\endcsname}


\oddc@tcodes%NO SPACES until |normalc@tcodes is called.
|gdef|isitslash{|let|c@ntinue|store@ttributes|if|nxt\|let|c@ntinue|startp@stattribcmd|normalc@tcodes|fi|c@ntinue}%
|normalc@tcodes%
%

%Parse key="value" or value.
\def\default@ttrkey{} % What unnamed attributes get saved 
\def\thisdefault@ttrkey{} % What do unnamed attributes get saved as?
\def\relaxval{\relax}

\def\save@ttribute#1="#2"=#3\E{
  \lowercase{\def\attr@key{#1}}\def\attr@val{#2}%
  \ifx\attr@val\relaxval
    \trace{A}{Got unnamed(\thisdefault@ttrkey) attribute #1}%
    \ifx\thisdefault@ttrkey\empty
      \x@\edef\csname attr:UnNamed\endcsname{#1}%don't use attrkey here as it's been lowercased
      \edef\attribsus@d{\attribsus@d,UnNamed}% Keep a list so we can junk it later.
    \else
      \x@\edef\csname attr:\thisdefault@ttrkey\endcsname{#1}%
      \edef\attribsus@d{\attribsus@d,\thisdefault@ttrkey}% Keep a list so we can junk it later.
    \fi
  \else
    \trace{A}{Got named attribute \attr@key="\attr@val"}%
    \x@\edef\csname attr:\attr@key\endcsname{\attr@val}%
    \edef\attribsus@d{\attribsus@d,\attr@val}% Keep a list so we can junk it later.
  \fi
}

% Parse a space-separated list.
\def\relax@pair#1\E{}

\def\geton@pair #1 #2\E{% Parse space separated list
  \trace{A}{getone pair "#1" "#2"}%
  \edef\it@m{#1}\ifx\it@m\empty\let\nxt@ttrib=\relax@pair\else
  \x@\save@ttribute #1="\relax"=\E
  \fi
  \nxt@ttrib #2 \E
}
\def\parse@ttribs#1{%
  \let\nxt@ttrib=\geton@pair
  \x@\geton@pair #1 \E
}
