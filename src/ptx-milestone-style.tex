%:strip
% Part of the ptx2pdf macro package for formatting USFM text
% copyright (c) 2007-2020 by SIL International
% written by David Gardner and pior editors of ptx-char-style
%
% Permission is hereby granted, free of charge, to any person obtaining  
% a copy of this software and associated documentation files (the  
% "Software"), to deal in the Software without restriction, including  
% without limitation the rights to use, copy, modify, merge, publish,  
% distribute, sublicense, and/or sell copies of the Software, and to  
% permit persons to whom the Software is furnished to do so, subject to  
% the following conditions:
%
% The above copyright notice and this permission notice shall be  
% included in all copies or substantial portions of the Software.
%
% THE SOFTWARE IS PROVIDED "AS IS", WITHOUT WARRANTY OF ANY KIND,  
% EXPRESS OR IMPLIED, INCLUDING BUT NOT LIMITED TO THE WARRANTIES OF  
% MERCHANTABILITY, FITNESS FOR A PARTICULAR PURPOSE AND  
% NONINFRINGEMENT. IN NO EVENT SHALL SIL INTERNATIONAL BE LIABLE FOR  
% ANY CLAIM, DAMAGES OR OTHER LIABILITY, WHETHER IN AN ACTION OF  
% CONTRACT, TORT OR OTHERWISE, ARISING FROM, OUT OF OR IN CONNECTION  
% WITH THE SOFTWARE OR THE USE OR OTHER DEALINGS IN THE SOFTWARE.
%
% Except as contained in this notice, the name of SIL International  
% shall not be used in advertising or otherwise to promote the sale,  
% use or other dealings in this Software without prior written  
% authorization from SIL International.
%%%%%%%%%%%%%%%%%%%%%%%%%%%%%%%%%%%%%%%%%%%%%%%%%%%%%%%%%%%%%%%%%%%%%%%

% Milestone macros
\def\notlocation#1{\x@\let\csname n@tLocn@#1\endcsname\@ne} %Not all milestones are locations.                                         
\def\ch@ckislocation#1{\keeptriggerfalse\ifcsname n@tLocn@#1\endcsname\keeptriggertrue\fi}

\def\mst@nestyle#1{\trace{m}{mst@nestyle:#1 (\milestoneOp)}%
 \gdef\thismil@stone{\detokenize{#1}}% record the name of the style
 \catcode32=12 % make <space> an "other" character, so it won't be skipped by \futurelet
 \catcode13=12 % ditto for <return>
 %\tracingassigns=1
 %\use@ttrSlash
 \futurelet\n@xt\domst@nestyle % look at following character and call \domst@nestyle
}

\let\p@pe=| % for matching
\catcode`\~=12 \lccode`\~=32 % we'll use \lowercase{~} when we need a category-12 space
\catcode`\_=12 \lccode`\_=13 % and \lowercase{_} for category-12 <return>
\lowercase{%
 \def\domst@nestyle{% here, \n@xt has been \let to the next character after the marker
  \mst@nestyletrue
  \initmil@stone%
  \deactiv@tecustomch@rs
  \catcode32=10 % reset <space> to act like a space again
  \catcode13=10 % and <return> is also a space (we don't want blank line -> \par)
  \ifx\n@xt\h@phen\let\n@xt@\startmst@nestyle@minus\else
    \ifx\n@xt~\let\n@xt@\startmst@nestyle@spc\else
      \ifx\n@xt_\let\n@xt@\startmst@nestyle@nl\else
       \ifx\n@xt\p@pe\let\n@xt@\startmst@nestyle@pipe\else
	\let\n@xt@\startmst@nestyle\fi\fi\fi\fi
  %\tracingassigns=0
  \trace{m}{Start style}%
  \n@xt@
 }
 \def\startmst@nestyle@spc~{\startmst@nestyle}%                                             
 \def\startmst@nestyle@nl_{\startmst@nestyle}%
}
\def\startmst@nestyle@pipe|{\trace{m}{Immediate start of attributes}\in@ttribtrue\startmst@nestyle\@ttrSlash}%
\def\startmst@nestyle@minus-#1{%
  \def\milestoneOp{#1}%
  \startmst@nestyle
}

\def\initmil@stone{%called by startmst@nestyle@minus and startch@rstyle@minus.
  \trace{m}{initmil@stone \milestoneOp}%
  \ifmst@nestyle\else %Was this marker defined as a milestone or as a character style
    \edef\thismil@stone{\newch@rstyle}%
  \fi
  \edef\thismil@stoneKey{}\edef\thismil@stoneVal{}% Set by Attributes
  %\xdef\milestoneOp{#1}%
  \edef\attrid{\milestoneOp id}% id / sid / eid, depending.
}

\def\@ttrSlash{\in@ttribtrue
  \trace{m}{Starting attribute collection}%
  \catcode`\/=12\relax % FIXME Other active chars too?
  \@ttrSl@sh}
% using #1#2 eats initial spaces. This means  #1#2 nicely ignores annoying initial spaces, but it also crashes on a spaces-only arg, so check for that.
\def\@ttrSl@sh #1\*{\edef\tmp{\zap@space #1 \empty}\ifx\tmp\empty\else\x@\@@ttrSlash #1\relax\relax\fi}
\def\@@ttrSlash #1#2\relax{\xdef\attrib@rgs{#1#2}\trace{m}{@ttrSlash: Attr:\attrib@rgs}\*}

\def\*{%This might end a + style (c)  or a normal style character style(C), or a milestone style
  \trace{m}{slash *}%
  \ifmst@nestyle
    \proc@ttribs
    \get@ttribute{\thisdefault@ttrkey}%
    \ifx\attr@b\relax\else
      \let\thismil@stoneKey\thisdefault@ttrkey
      \let\thismil@stoneVal\attr@b
    \fi
  %  \ifmst@nestyle\else
  %    \ifx \ss@ChrP\stylet@pe
  %      \endch@rstylepls*
  %    \else
  %      \endch@rstyle*
  %    \fi
  %  \fi
    \mst@nestylefalse
    \processmil@stone
  \else{*}%
  \fi}% USFM3 'self closing marker' milestone

\let\milestoneOp\empty
\let\thismil@stone\empty
\let\thisattrmil@stone\empty
\newif\ifmst@nestyle \mst@nestylefalse

\def\processmil@stone{\trace{m}{Milestone \thismil@stone (\milestoneOp), \csname thisch@rstyle\endcsname}%
  \ifx\milestoneOp\empty
    \st@ndalonemilestone
  \else 
    \if\milestoneOp s\relax\st@rtmilestone\else
      \if\milestoneOp e\relax\@ndmilestone\else
      \st@ndalonemilestone\fi 
    \fi
  \fi
   %\xdef\thisch@rstyle{\mcpeek}%
   \def\d@##1+##2+##3\E{\d@code{##1}{##2}\edef\thisch@rstyle{\tmp}}\mctop
   \trace{m}{\if e\milestoneOp After-\fi Milestone style is \thisch@rstyle, stystack is: \styst@k }%
   \global\let\milestoneOp\empty
   \s@tfont{\thisch@rstyle}{\styst@k}% set up font attributes
}

\def\startmst@nestyle{%What actually changes between the beginning of a milestone and the end-marker? Attribute values!
   \trace{m}{startmst@nestyle}%
   \op@ninghooks{before}{\thismil@stone}%
   \csname init@ttribs\endcsname
}

\def\dr@pmilest@ne#1+#2+#3\E{% This itterates the stack to kills the TOP matching milestone 
  \edef\MSt@mp{#1}%
  \ifx\MSt@mp\empty
    \let\d@=\cstackrelax %Stop processing (permanent change)
  \else
    \ifx\ss@Mstn\MSt@mp\relax %It's a milestone
      \x@\pars@msid #2;;;\E
      \ifx\MStyp@\MSch@cking\relax
        \ifx\mstone@id\MSch@ckid
          %\csname endit@#1\endcsname
          \trace{m}{Found \MStyp@ (\MSch@ckid)}%
          \let\d@=\empty %just a signal to skip this one.
          \tempfalse % clear error flag if set
        \else
          \trace{m}{Found \MStyp@, but ids do not match "\MSch@ckid"!="\mstone@id"}%
          \temptrue% flag error
        \fi
      \fi
    \fi
    \ifx\d@\empty
      %Two possible actions here: clear all matching milestones or only the top one.
      %standard is vague about nesting/cancellation. Current implementation 
      %keeps going down if there's no sid/eid to match. 
      %\ifx\MSch@ckid\empty
        \let\d@=\dr@pmilest@neB % Keep tracking down, to build \@ut, but no more testing.
      %\else
        %\let\d@=\cstackrelax %Top match only.  permanent change
      %\fi
    \else
      \ifx\@ut\empty \xdef\@ut{#1+#2}\else
	\xdef\@ut{\@ut,#1+#2}%
      \fi
    \fi
  \fi
  \trace{m}{dr@pmilest@ne: \@ut}%
}

\def\dr@pmilest@neB#1+#2+#3\E{% This itterates the stack to kills the TOP matching milestone 
  \edef\MSt@mp{#1}%
  \ifx\MSt@mp\empty
    \let\d@=\cstackrelax %Stop processing 
  \else
    \ifx\@ut\empty \xdef\@ut{#1+#2}\else
      \xdef\@ut{\@ut,#1+#2}%
    \fi
  \fi
  \trace{m}{dr@pmilest@neB: \@ut}%
}

\def\dr@pmilestone#1#2{%Kill a milestone that might be burried in the stack, and might be nested.
  \trace{m}{Dropping milestone #1 from \mcstack}% 
  \edef\MSch@cking{#1}%
  \edef\MSch@ckid{#2}%
  \tempfalse
  \let\@ut=\empty
  \let\d@=\dr@pmilest@ne
  \mcdown
  \xdef\mcstack{\mcstack@mpty}%
  \ifx\@ut\empty
    \s@tstyst@k
  \else
    \rebuild@mcstack{\@ut,}%
    %\xdef\mcstack{\@ut,\mcstack@mpty}%
  \fi
  \trace{m}{Stack now: \mcstack, stystack:\styst@k}%
  \iftemp
    \message{End-milestone of class '\MSch@cking', id '\MSch@ckid' partially matched one or 
      more open milestones, but no match on the id was found, sid and eid must match exactly}%
  \fi
}
%+csty_milestone
% Operations on the current milestone(s)
\def\zapres@rved#1{%
    \edef\smst@mp{\zap@space #1 \empty}%
    \edef\smst@mp{\expandafter\zap@comma \smst@mp,\empty}%
    \edef\smst@mp{\expandafter\zap@plus \smst@mp+\empty}%
} 
% If there is '\s \qt-s\*' in the input, should the qt-s have effect on the paragraph style? If the boolean is true, 
% then it does not, and the paragaph starts differently to '\qt-s\* \s'. Note that this does not trigger any extra paragraph break,
% all it does is ensure that a parargaph waiting to start *is* safely started.
\newif\ifDefineParBeforeMilestone\DefineParBeforeMilestonetrue 
\def\st@rtmilestone{%
  % There may multiple milestones of a given type (can't sensibly have both \qt-s |Jesus\* and 
  % \qt-s |Pilate\* active at the same time (except quote in quote), but in some cases it might make sense to do something similar.)
  \deactiv@tecustomch@rs
  \get@ttribute{\attrid}\ifx\attr@b\relax %if sid is set, sid/eid must match, and may contain letters,numbers,underscore.
    \x@\zapres@rved\x@{\thismil@stoneVal}%
    \xdef\thisattrmil@stone{\thismil@stone;\smst@mp}%
    \def\smst@mp{without ID}%
  \else
    \x@\zapres@rved\x@{\thismil@stoneVal;\attr@b}%
    \xdef\thisattrmil@stone{\thismil@stone;\smst@mp}%
    \def\smst@mp{with \attrid =\attr@b\space}%
  \fi
  \ifDefineParBeforeMilestone\leavevmode\fi
  \mcpush{\ss@Mstn}{\thisattrmil@stone}%
  \trace{m}{Milestone \smst@mp\space stacked, \mcstack}%
  \let\tmp\thisattrmil@stone
  \csname d@code-m\endcsname
  \global\let\thisattrmil@stone\tmp
  \upd@teMsPrefix
  \let\styst@k=\empty
  \s@tstyst@k%
  \@ttrMilestonefalse
  \ifx\thismil@stoneVal\empty\else
    \@ttrMilestonetrue
  \fi
  \kill@PossParamCache
  \trace{m}{Style stack now: \mcstack}%
  \op@ninghooks{start}{\thismil@stone}%
}
\def\@ndmilestone{%
  \trace{m}{@ndmilestone.  stack now: \mcstack}%
  \cl@singhooks{end}{\thismil@stone}%
  \get@ttribute{\attrid}%attrid is id/sid/eid (set in init@trrtibs)
  \ifx\attr@b\relax 
  \dr@pmilestone{\thismil@stone}{}%
  \else
  \dr@pmilestone{\thismil@stone}{\attr@b}%
  \fi
  \trace{m}{Style stack now: \mcstack}%
  \upd@teMsPrefix
  \cl@singhooks{after}{\thismil@stone}\the\afterh@@ks
  \global\let\attr@b\empty\global\let\attrid\empty
  \global\let\thismil@stone\empty
  \global\let\thisattrmil@stone\empty
}
\def\figmil@stone{zfiga}

\def\st@ndalonemilestone{% FIXME:Potentially some kind of hook? Or use this as an image anchor?
  \ifx\attrid\empty
    \let\attr@b\relax
  \else 
    \get@ttribute{\attrid}%
  \fi
  \trace{m}{st@ndalonemilestone(\thismil@stone) \attrid=\attr@b}%
  \t@stpublishability{\thismil@stone}%
  \ifx\p@ram\relax
    \trace{m}{Properties not defined, assuming standalone milestone has no publishable text}%
    \n@npublishabletrue
  \fi
  \ifn@npublishable\else
    \mcpush{\ss@Mstn}{\thismil@stone}%
    \trace{m}{Milestone stacked, \mcstack}%
  \fi
  \op@ninghooks{start}{\thismil@stone}%
  \let\oldc@rref\c@rref\let\olddc@rref\dc@rref
  %\tracingmacros=1
  %\tracingassigns=1
  \ifx\attr@b\relax\else
    \trace{m}{Checking for  ms:\thismil@stone=\attr@b}%
    \edef\@@tmp{ms:\thismil@stone=\x@\detokenize\x@{\attr@b}}%
    \ifcsname \@@tmp\endcsname
      \trace{m}{Running ms:\@@tmp}%
      \csname \@@tmp\endcsname
  \fi\fi
  %\tracingassigns=0
  %\tracingmacros=0
  \ifx\attr@b\relax
    \runtrigg@rs{ms:\thismil@stone}%
  \else
    \runtrigg@rs{\mkp@cref{\id@@@}{\x@\detokenize{\attr@b}}}%
    \runtrigg@rs{ms:\thismil@stone=\x@\detokenize{\attr@b}}%
    \ch@ckislocation{\thismil@stone}% Non-locations can be multiple use, so their value is kept. Sets keeptrigger true if this is not a location. 
    \ifkeeptrigger% Not a location, leave c@rref what it used to be
      \global\let\c@rref\oldc@rref\global\let\dc@rref\olddc@rref
    \else% This milestone is a named location 
      \xdef\c@rref{\id@@@\attr@b}%
      \ifx\v@rse\empty\global\p@rnum=\ifhmode 1\else 0\fi
       \ch@ckadjustments
      \fi
    \fi
  \fi
  \cl@singhooks{end}{\thismil@stone}%\
  \ifn@npublishable
    \n@npublishablefalse %restore to publishable
  \else
    \dr@pmilestone{\thismil@stone}{}%
  \fi
  \cl@singhooks{after}{\thismil@stone}\the\afterh@@ks\relax
  \let\thismil@stone\empty
}%

\def\upd@teMsPrefix{%Build a prefix based on all currently-in-force milestones.
  \trace{m}{upd@teMsPrefix}%
  \xdef\mspr@fix{}\let\d@=\mil@st@necheck\mcup
  \ifx\mspr@fix\empty\else\xdef\mspr@fix{ms:\mspr@fix|}\trace{m}{msPrefix set to \mspr@fix}\fi
}
\xdef\equ@l{=}
\def\pars@msid#1;#2;#3;#4\E{\edef\MStyp@{#1}\edef\k@yv@al{#2}\def\mstone@id{#3}}

\x@\def\csname d@code-m\endcsname{%Modify contents of \tmp to only include bits of the stack value of use for styles.
  \x@\pars@msid \tmp;;;\E
  \edef\tmp{\ifx\k@yv@al\empty\else\k@yv@al|\fi\MStyp@}%
}
\def\mil@st@necheck#1+#2+#3\E{%The milestone has a key value and possibly an ID field. Ignore the id for building the prefix
  \edef\MSt@mp{#1}%
  \ifx\MSt@mp\ss@Mstn
    \x@\pars@msid #2;;;\E
    \trace{m}{parsed #2->\k@yv@al, \mstone@id}%
  \xdef\mspr@fix{\ifx\mspr@fix\empty\else \mspr@fix+\fi\ifx\k@yv@al\equ@l\else\k@yv@al|\fi\MStyp@}\fi}
 
\newtoks\tmpt@ks
\def\exp@ndmspr@fix#1{%for each potential prefix, call #1 with \@lso defined
   \tmpt@ks{#1}%
   \trace{m}{Expanding \mspr@fix}%
   \def\D@IT##1{\edef\@lso{ms:##1|}\trace{m}{adding option \@lso to styles}\the\tmpt@ks}%
   \it@mcount=0
   \x@\@xp@ndmspr@fix \mspr@fix\E
   %\ifnum \it@mcount>1
     %\let\@lso\mspr@fix \the\tmpt@ks
   %\fi
}

%simple + separated list processing
\def\l@stitem #1\E{}
\newcount\it@mcount
\def\e@chitem#1+#2\E{%
  \edef\it@mtmp{#1}\ifx\it@mtmp\empty\let\nxt@item\l@stitem\else
    \advance\it@mcount by 1
    \D@IT{#1}\fi
  \x@\nxt@item #2+\E
}

\def\@xp@ndmspr@fix ms:#1|{%
   \trace{m}{list: #1}%
   \let\nxt@item\e@chitem
   \x@\nxt@item #1+\E
}
%-csty_milestone
\def\attrid{}
\notlocation{zvar}
\def\m@kenumber#1{\m@kedigitsother\x@\m@kenumb@r #1 =}
\def\m@kenumb@r#1={\edef\@@result{\scantokens{#1\noexpand}}\m@kedigitsletters}
\def\defzvar#1#2{\x@\def\csname ms:zvar=#1\endcsname{#2}}
\x@\def\csname start-zrule\endcsname{%
  \endgraf
  \get@ttribute{width}%A
  \ifx\attr@b\relax
    \def\rule@wid{0.5}%
  \else
    \m@kenumber{\attr@b}%
    \let\rule@wid\@@result%
  \fi
  \get@ttribute{thick}%A
  \ifx\attr@b\relax
    \def\rule@thk{0.5pt}%
  \else
    \m@kenumber{\attr@b}%
    \let\rule@thk\@@result%
  \fi
  \get@ttribute{style}%
  \ifx\attr@b\relax
    \def\rule@style{plain}%
  \else
    \ifcsname drawzrule-\attr@b\endcsname
      \let\rule@style\attr@b
    \else
      \def\rule@style{plain}%
      \message{unrecognised rule style '\attr@b' near \c@rref}%
    \fi
  \fi
  \get@ttribute{align}%
  \ifx\attr@b\relax
    \def\rule@pos{c}%
  \else
    \let\rule@pos\attr@b
  \fi
  \csname drawzrule-\rule@style\endcsname
}
\x@\def\csname drawzrule-simple\endcsname{%
  \hbox to \hsize{\ifx\rule@pos\@lignLeft\else\hfil\fi
    \vrule height \rule@thk width \rule@wid\hsize depth 0pt
    \ifx\rule@pos\@lignRight\else\hfil\fi}%
}

\x@\def\csname start-zgap\endcsname{%
  \endlastp@rstyle{zgap}%
  \endgraf
  \get@ttribute{dimension}%A
  \ifx\attr@b\relax
    \let\gap@dim\baselineskip%
  \else
    \m@kenumber{\attr@b}%
    \let\gap@dim\@@result%
  \fi
  \trace{m}{zgap \gap@dim}%
  \vbox{}\penalty10000\vskip\gap@dim\relax%
}

\def\proc@strong#1{%
  \get@ttribute{align}%
  \ifx\attr@b\relax \let\al@gn\@lignLeft\else\let\al@gn\attr@b\fi
  \get@ttribute{strong}%
  \trace{m}{Got strongs \attr@b}%
  \ifx\attr@b\relax \else\edef\tmpz{#1}%
    \x@\dostr@ngs\x@{\x@\tmpz\x@}\x@{\x@\al@gn\x@}\attr@b+++\E
  \fi
}
\def\dostr@ngs#1#2#3#4#5#6#7\E{%
% #1 style, #2 align, #3 #4 #5 #6 4 digits, #7 dummy
  \trace{m}{dostr@ngs #1, #2, #3, #4, #5, #6, #7}%
  \bgroup\s@tfont{#1}{\styst@k}%
    \setbox0=\hbox{\if #4+\relax #3\else\if #6+\relax #3\else #3#4\fi\fi}%
    \setbox1=\hbox{\if #4+\relax\else\if #5+\relax #4\else\if #6+\relax #4#5\else #5#6\fi\fi\fi}%
    \dimen0=\ifdim\wd0>\wd1 \wd0\else\wd1\fi
    \getp@r@m{raise}{#1}\ifx\p@ram\relax\else\dimen1=\p@ram\trace{m}{raising by \the\dimen1, from \p@ram}\raise\dimen1\fi
    \vbox{\s@tbaseline{#1}{\styst@k}\trace{m}{dostr@ngs #1 baselineskip=\the\baselineskip}%
                          \hbox to \dimen0{\if#2r\relax\hfil\fi\unhbox0\if#2l\relax\hfil\fi}%
                          \hbox to \dimen0{\if#2r\relax\hfil\fi\unhbox1\if#2l\relax\hfil\fi}%
    }\egroup
}
