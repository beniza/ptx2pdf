%:strip
% Part of the ptx2pdf macro package for formatting USFM text
% copyright (c) 2007-2020 by SIL International
% written by David Gardner and pior editors of ptx-char-style
%
% Permission is hereby granted, free of charge, to any person obtaining  
% a copy of this software and associated documentation files (the  
% "Software"), to deal in the Software without restriction, including  
% without limitation the rights to use, copy, modify, merge, publish,  
% distribute, sublicense, and/or sell copies of the Software, and to  
% permit persons to whom the Software is furnished to do so, subject to  
% the following conditions:
%
% The above copyright notice and this permission notice shall be  
% included in all copies or substantial portions of the Software.
%
% THE SOFTWARE IS PROVIDED "AS IS", WITHOUT WARRANTY OF ANY KIND,  
% EXPRESS OR IMPLIED, INCLUDING BUT NOT LIMITED TO THE WARRANTIES OF  
% MERCHANTABILITY, FITNESS FOR A PARTICULAR PURPOSE AND  
% NONINFRINGEMENT. IN NO EVENT SHALL SIL INTERNATIONAL BE LIABLE FOR  
% ANY CLAIM, DAMAGES OR OTHER LIABILITY, WHETHER IN AN ACTION OF  
% CONTRACT, TORT OR OTHERWISE, ARISING FROM, OUT OF OR IN CONNECTION  
% WITH THE SOFTWARE OR THE USE OR OTHER DEALINGS IN THE SOFTWARE.
%
% Except as contained in this notice, the name of SIL International  
% shall not be used in advertising or otherwise to promote the sale,  
% use or other dealings in this Software without prior written  
% authorization from SIL International.
%%%%%%%%%%%%%%%%%%%%%%%%%%%%%%%%%%%%%%%%%%%%%%%%%%%%%%%%%%%%%%%%%%%%%%%

% Milestone macros
\def\mst@nestyle#1{\trace{m}{mst@nestyle:#1}%
 \gdef\thismil@stone{\detokenize{#1}}% record the name of the style
 \catcode32=12 % make <space> an "other" character, so it won't be skipped by \futurelet
 \catcode13=12 % ditto for <return>
 \futurelet\n@xt\domst@nestyle % look at following character and call \domst@nestyle
}

\catcode`\~=12 \lccode`\~=32 % we'll use \lowercase{~} when we need a category-12 space
\catcode`\_=12 \lccode`\_=13 % and \lowercase{_} for category-12 <return>
\lowercase{
 \def\domst@nestyle{% here, \n@xt has been \let to the next character after the marker
  \catcode32=10 % reset <space> to act like a space again
  \catcode13=10 % and <return> is also a space (we don't want blank line -> \par)
  \if\n@xt-\let\n@xt@\startmst@nestyle@minus\else
    \if\n@xt~\let\n@xt@\startmst@nestyle@spc\else
      \if\n@xt_\let\n@xt@\startmst@nestyle@nl\else
	\let\n@xt@\startmst@nestyle\fi\fi\fi
  \n@xt@
 }
 \def\startmst@nestyle@spc~{\startmst@nestyle}%                                             
 \def\startmst@nestyle@nl_{\startmst@nestyle}%
}
\def\startmst@nestyle@minus-#1{%
  \mst@nestyletrue
  \initmil@stone{#1}%
  \startmst@nestyle
}

\def\initmil@stone#1{%called by startmst@nestyle@minus and startch@rstyle@minus.
   \trace{m}{initmil@stone #1}%
   \ifmst@nestyle\else %Was this marker defined as a milestone or as a character style
     \edef\thismil@stone{\newch@rstyle}%
   \fi
   \edef\thismil@stoneKey{}\edef\thismil@stoneVal{}% Set by Attributes
   \xdef\milestoneOp{#1}%
}

\def\*{%This might end a + style (c)  or a normal style character style(C), or a milestone style
  \proc@ttribs
  \trace{m}{slash *}%
  \ifmst@nestyle\else
    \if c\stylet@pe\relax
      \endch@rstylepls*
    \else
      \endch@rstyle*
    \fi
  \fi
  \mst@nestylefalse
  \mil@stone}% USFM3 'self closing marker' milestone

\let\milestoneOp\empty
\let\thismil@stone\empty
\newif\ifmst@nestyle \mst@nestylefalse

\def\mil@stone{\trace{m}{Milestone \thismil@stone (\milestoneOp), \csname thisch@rstyle\endcsname}%
  \if\milestoneOp s\relax\st@rtmilestone\else
   \if\milestoneOp e\relax\@ndmilestone\else
    \st@ndalonemilestone\fi\fi
 \let\milestoneOp\empty
 %\xdef\thisch@rstyle{\mcpeek}%
 \def\d@##1+##2\E{\edef\thisch@rstyle{##2}}\mctop
 \s@tfont{\thisch@rstyle}% set up font attributes
}

\def\startmst@nestyle{%What actually changes between the beginning of a milestone and the end-marker? 
   \trace{m}{startmst@nestyle}%
   \init@ttribs
}

\def\dr@pmilest@ne#1+#2\E{% This kills the TOP matching milestone that might be burried in the stack, and might be nexted.
  \edef\MSt@mp{#1}%
  \ifx\MSt@mp\MSch@cking\relax
    %\csname endit@#1\endcsname
    \trace{m}{Found #1}%
    \let\d@=\cstackrelax
  \else
    \ifx\MSt@mp\empty
      \let\d@=\cstackrelax
    \else
      \ifx\@ut\empty \xdef\@ut{#1+#2}\else
      \xdef\@ut{\@ut,#1+#2}\fi%
    \fi
  \fi
  \trace{m}{dr@pmilest@ne: \@ut}%
}
\def\dr@pmilestone#1#2{
  \trace{m}{Dropping milestone #1 from \MScstack}% 
  \edef\MSch@cking{#1}%
  \let\@ut=\empty
  \let\d@=\dr@pmilest@ne
  \MScdown
  \ifx\@ut\empty
    \xdef\MScstack{\MScstack@mpty}%
  \else
    \xdef\MScstack{\@ut,\MScstack@mpty}%
  \fi
  \trace{m}{Stack now: \MScstack}%
}
%+csty_milestone
% Operations on the current milestone(s)
\def\st@rtmilestone{%
  % There may multiple milestones of a given type (can't sensibly have both \qt+s |Jesus\* and 
  % \qt+s |Pilate\* active at the same time, but in some cases it might make sense to do something siilar.)
  \MScpush{\thismil@stone+\thismil@stoneKey=\thismil@stoneVal}%
  \upd@teMsPrefix
}
\def\@ndmilestone{%
  \dr@pmilestone{\thismil@stone}{\thismil@stoneVal}{\thismil@stoneKey}%
  \upd@teMsPrefix
}
\def\st@ndalonemilestone{
  % FIXME:Potentially some kind of hook? Or use this as an image anchor?
}%
\def\upd@teMsPrefix{%Build a prefix based on all currently-in-force milestones.
  \trace{m}{upd@teMsPrefix}%
  \xdef\mspr@fix{}\let\d@=\mil@st@necheck\MScup
  \ifx\mspr@fix\empty\else\xdef\mspr@fix{ms:\mspr@fix|}\trace{m}{msPrefix set to \mspr@fix}\fi
}
\xdef\equ@l{=}
\def\mil@st@necheck#1+#2\E{%
  \edef\k@yv@al{#2}%
  \xdef\mspr@fix{\ifx\mspr@fix\empty\else \mspr@fix+\fi#1\ifx\k@yv@al\equ@l\else#2\fi}}
   

%-csty_milestone


