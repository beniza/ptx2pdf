%%%%%%%%%%%%%%%%%%%%%%%%%%%%%%%%%%%%%%%%%%%%%%%%%%%%%%%%%%%%%%%%%%%%%%%
% Part of the ptx2pdf macro package for formatting USFM text
% copyright (c) 2007 by SIL International
% written by Jonathan Kew
%
% Permission is hereby granted, free of charge, to any person obtaining  
% a copy of this software and associated documentation files (the  
% "Software"), to deal in the Software without restriction, including  
% without limitation the rights to use, copy, modify, merge, publish,  
% distribute, sublicense, and/or sell copies of the Software, and to  
% permit persons to whom the Software is furnished to do so, subject to  
% the following conditions:
%
% The above copyright notice and this permission notice shall be  
% included in all copies or substantial portions of the Software.
%
% THE SOFTWARE IS PROVIDED "AS IS", WITHOUT WARRANTY OF ANY KIND,  
% EXPRESS OR IMPLIED, INCLUDING BUT NOT LIMITED TO THE WARRANTIES OF  
% MERCHANTABILITY, FITNESS FOR A PARTICULAR PURPOSE AND  
% NONINFRINGEMENT. IN NO EVENT SHALL SIL INTERNATIONAL BE LIABLE FOR  
% ANY CLAIM, DAMAGES OR OTHER LIABILITY, WHETHER IN AN ACTION OF  
% CONTRACT, TORT OR OTHERWISE, ARISING FROM, OUT OF OR IN CONNECTION  
% WITH THE SOFTWARE OR THE USE OR OTHER DEALINGS IN THE SOFTWARE.
%
% Except as contained in this notice, the name of SIL International  
% shall not be used in advertising or otherwise to promote the sale,  
% use or other dealings in this Software without prior written  
% authorization from SIL International.
%%%%%%%%%%%%%%%%%%%%%%%%%%%%%%%%%%%%%%%%%%%%%%%%%%%%%%%%%%%%%%%%%%%%%%%

% Paratext formatting macros, spanning footnotes version

\catcode`\@=11
\let\x@=\expandafter

\TeXXeTstate=1 % enable the eTeX bidi extensions, in case we need RTL support

\def\MSG{\immediate\write16 } % shorthand to write a message to the terminal
\def\TRACE#1{}%\let\TRACE=\MSG % default - consume messages
\def\wlog{\immediate\write-1 } % write message to log file only
\newif\iftemp \tempfalse
\newif\ifinn@te\inn@tefalse
\newif\ifm@rksonpage % Try to keep track of marks, so we can kill them in end-sections. 
\m@rksonpagefalse
% \addtoendhooks collects macros to be executed at the end of the job
\def\addtoendhooks#1{\x@\global\x@\@ndhooks\x@{\the\@ndhooks #1}}
\newtoks\@ndhooks
\let\s@ve@nd=\end
\def\end{\par\vfill\supereject \the\@ndhooks \s@ve@nd}

% \addtoinithooks is for stuff we do during one-time-setup before the first PTX file
\def\addtoinithooks#1{\x@\global\x@\@nithooks\x@{\the\@nithooks #1}}
\newtoks\@nithooks

% \addtoeveryparhooks is for the start of every paragraph
\def\addtoeveryparhooks#1{\x@\global\x@\@veryparhooks\x@{\the\@veryparhooks #1}}
\newtoks\@veryparhooks

% \addtoparstylehooks is for stuff to do at each new parstyle marker
\def\addtoparstylehooks#1{\x@\global\x@\p@rstylehooks\x@{\the\p@rstylehooks #1}}
\newtoks\p@rstylehooks

% initialize a \timestamp macro for the cropmarks etc to use
\edef\timestamp{\number\year.% print the date and time of the run
  \ifnum\month<10 0\fi \number\month.%
  \ifnum\day<10 0\fi \number\day\space :: }%
\count255=\time \divide\count255 by 60
\edef\timestamp{\timestamp
  \ifnum\count255<10 0\fi \number\count255:}%
\multiply\count255 by 60 \advance\count255 by -\time
\count255=-\count255
\edef\timestamp{\timestamp
  \ifnum\count255<10 0\fi \number\count255}%

\input ptx-diglot.tex
\input ptx-tracing.tex
\input ptx-para-style.tex
\input ptx-char-style.tex
\input ptx-note-style.tex
\input ptx-stylesheet.tex % must come after the ptx-*-style.tex macros
\input ptx-references.tex
\input ptx-cropmarks.tex
\input ptx-toc.tex
\input ptx-tables.tex
\input ptx-adj-list.tex % must come after ptx-stylesheet.tex
\input ptx-pic-list.tex % must come after ptx-stylesheet.tex
\input ptx-cutouts.tex
\input ptx-callers.tex % must come after ptx-note-style.tex
% Additional modules that are not part of the normal ptx2pdf module
\input ptxplus-character-kerning.tex
%\input ptxplus-marginalverses.tex  % This is where we would like
                                    % load this module but has to
                                    % be loaded after the stylesheet
                                    % information is loaded.
%\input ptxplus-maps.tex            % This still needs development yet

% default font names (override in setup file)
\ifx\regular\undefined   \def\regular{"Times New Roman"}      \fi
\ifx\bold\undefined      \def\bold{"Times New Roman/B"}       \fi
\ifx\italic\undefined    \def\italic{"Times New Roman/I"}     \fi
\ifx\bolditalic\undefined\def\bolditalic{"Times New Roman/BI"}\fi

% footnote macros based on plain.tex \footnote, \vfootnote
\newbox\he@dingnotes
\def\m@kenote#1#2#3#4{\let\@sf\empty % #1=class(e.g. f, x);
  % #2=class of styling
  % #3=caller + all the macros to setup the styling for the caller
  % #4=caller + styling in the footnote if there is a notecallerstyle
  % text is read later
  \ifhmode\edef\@sf{\spacefactor\the\spacefactor}\/\fi%
  % if footnote is on a chapter number ...
  \ifhe@dings #3\edef\@wrap{\global\setbox\he@dingnotes=\vbox\bgroup\unvbox\he@dingnotes}\else\edef\@wrap{}%
    \ifch@pter \everypar={}\ch@pterfalse%
      \global\setbox\ch@pternote=\hbox{\box\ch@pternote #3}%
    \else #3\fi % output caller
  \fi%
  % @sf preserves the space factor (e.g. extra space after a period), restore it now
  \@sf \@wrap\vm@kenote{#1}{#2}{\getp@ram{notecallerstyle}{#2}\ifx\p@ram\relax #3\else #4\fi}}

\def\vm@kenote#1#2#3{%
  \inn@tetrue%
  \let\next\relax%
  \def\n@tetype{#1}%
  \ifdiglot\ifdiglotSepNotes\ifdiglotL\def\n@tetype{#1}\else\def\n@tetype{#1R}\fi\fi\fi%
  \TRACE{vm@kenote \n@tetype}%
  \x@\insert\csname note-\n@tetype\endcsname\bgroup% insert note-f (or note-x)
%%% single-column notes:
  \checkp@ranotes{#1}% check whether this note class is to be paragraphed 
  \hsize=\ifp@ranotes\maxdimen\else\textwidth\advance\hsize by -\columnshift\fi % insert penalties to control breaking of footnotes
  \interlinepenalty\interfootnotelinepenalty % set penalty to break lines
  \floatingpenalty\@MM % make sure note does not float away from caller to another page
  \leftskip\z@skip \rightskip\z@skip \spaceskip\z@skip \xspaceskip\z@skip % reset extra space around paragraph 0
  \s@tbaseline{#1}%
  \ifp@ranotes\else\setbox0=\hbox{\XeTeXuseglyphmetrics=0 \char32}\dimen0=\ht0
      \ifdim\dimen0<\baselineskip
        \dimen1=\baselineskip\advance\dimen1 by -\dimen0
        \vskip\dimen1
		\trace{f}{\reference : Footnote vskip=\the\dimen1 baselineskip=\the\baselineskip strut height=\the\dimen0}
	  \fi\fi
  \leavevmode % begin paragraph
  \ifdiglot\ifdiglotL\the\leftho@ks\else\the\rightho@ks\fi\fi%
  \ifRTL\setbox2=\lastbox\beginR\box2\fi % if RTL text this paragraph needs to be RTL
  % if note omitting caller from note (i.e. all callers are *'s)
  \testomitc@ller{#2}\ifomitc@ller\else
    % save copy of caller in temporary box, if non-empty add a little space
    \setbox0=\hbox{#3}#3\ifdim\wd0>0pt\kern.2em\fi
  \fi
  % currently we do not allow footnotes to break to the next page, so this may not be necessary
%  \splittopskip\ht\f@@tstrut % top baseline for broken footnotes
%  \splitmaxdepth\dp\f@@tstrut
  \mcpush{N+#2}%
  %\getp@ram{fontsize}{#2}\edef\c@rrfontsize{\ifx\p@ram\relax12\else\p@ram\fi}%
  \s@tfont{#2}%
  \futurelet\next\fo@t}% use plain.tex footnote processor

% use \OmitCallerInNote{f} to omit callers from the note at foot of page
% but leave them in the body text (e.g. all the callers are *'s)
\def\OmitCallerInNote#1{%
  \expandafter\let\csname omit-in-note #1\endcsname=1}
%
\def\testomitc@ller#1{\expandafter\ifx\csname omit-in-note #1\endcsname\relax
  \omitc@llerfalse \else \omitc@llertrue \fi}
\newif\ifomitc@ller

% create a "strut" (see TeXbook) of suitable size for the note style
\def\footstrut{\s@tfont{\newn@testyle}%
  \s@tbaseline{\newn@testyle}%
  \setbox\f@@tstrut=\hbox to 0pt{\XeTeXuseglyphmetrics=0 \char32 \hss}%
  \dimen0=\ht\f@@tstrut \dimen2=\dp\f@@tstrut
  \dimen4=\dimen0 \advance\dimen4 by \dimen2
%  \ifdim\dimen4<\baselineskip
    \dimen6=100\baselineskip \divide\dimen6 by \dimen4
    \multiply\dimen0 by \dimen6 \divide\dimen0 by 100
    \multiply\dimen2 by \dimen6 \divide\dimen2 by 100
%  \fi
  \setbox\f@@tstrut=\hbox{}\ht\f@@tstrut=\dimen0 \dp\f@@tstrut=\dimen2
  \copy\f@@tstrut}

\def\@foot{\ifp@ranotes \parfillskip=0pt \else\strut\fi
  \par\egroup\ifRTL\endR\fi\end@llcharstyles
  \def\d@##1+##2\E{\if ##1N\else\MSG{Bad marker ##2\space but expected a closing note marker}\fi}\mctop
  \mcpop}
\newbox\f@@tstrut
\def\n@teglue{2em plus 1em minus .5em\relax} % glue to be used between paragraphed notes

% set baseline appropriately for the given style (may be using much smaller font than body)
% baselineskip = leading
\def\s@tbaseline#1{%
  \ifdiglot\ifdiglotL\def\f@ntstyle{#1}\else\def\f@ntstyle{#1R}\fi\else\def\f@ntstyle{#1}\fi%
  \getp@ram{baseline}{#1}\ifx\p@ram\relax
    \getp@ram{fontsize}{#1}\dimen0=\p@ram\le@dingunit
    \multiply\dimen0 by 14 \divide\dimen0 by 12 % default .75 shift of .85ex (ex=.5 fontsize) against 14/12
    \trace{F}{baseline [\f@ntstyle] = \the\dimen0}%
    \setp@ram{baseline}{\f@ntstyle}{\the\dimen0}\baselineskip=\dimen0\else
  \baselineskip=\p@ram\fi
  \trace{j}{set baselineskip=\the\baselineskip}}

\def\ins@rtn@tecl@ss#1{% insert the given note class, either paragraphed or separately
  \iff@rstnote\setn@tewidth\fi % set appropriate hsize width for diglot or not.
  \checkp@ranotes{#1}\ifp@ranotes\let\n@xt=\parains@rtn@tecl@ss
    \else\let\n@xt=\separateins@rtn@tecl@ss\fi
  \n@xt{#1}}

% insert a note class in which each note is on its own line
\def\separateins@rtn@tecl@ss#1{%
  \def\cl@ss{#1}
  \ifdiglot\ifdiglotSepNotes\ifdiglotL\def\cl@ss{#1}\else\def\cl@ss{#1R}\fi\fi\fi%
  \TRACE{seperateins@rtn@tecl@ss \cl@ss}%
  \x@\let\x@\th@cl@ss\csname note-\cl@ss\endcsname % make \th@cl@ss be a synonym for the current note class
  % if the noteclass has content to output ...
  \ifvoid\th@cl@ss\TRACE{no note}\else%
    \iff@rstnote\kern-\dimen0\vfil\fi % ignore depth of body text; fill space
    \vskip\AboveNoteSpace % output above note class space
    \iff@rstnote\footnoterule\global\f@rstnotefalse\fi % output rule before first note
    \hbox{\hskip\columnshift \vbox{\unvbox\th@cl@ss}}\fi}% output notes

% insert a note class in which each note is in the same paragraph
\def\parains@rtn@tecl@ss#1{%
  \def\cl@ss{#1}%
  \ifdiglot\ifdiglotSepNotes\ifdiglotL\def\cl@ss{#1}\else\def\cl@ss{#1R}\fi\fi\fi%
  \x@\let\x@\th@cl@ss\csname note-\cl@ss\endcsname % make \th@cl@ss be a synonym for the current note class
  % if the noteclass has content to output ...
  \ifvoid\th@cl@ss\else
    \iff@rstnote\kern-\dimen0\vfil % ignore depth of body text; fill space
      \vskip\AboveNoteSpace\else\vskip\InterNoteSpace\fi
    \iff@rstnote{\footnoterule}\global\f@rstnotefalse\fi % output rule before first note
    {\maken@tepara{\th@cl@ss}{#1}}%
  \fi}

\newif\iff@rstnote
\f@rstnotetrue
\def\footnoterule{\kern-0.5\AboveNoteSpace\kern-.4pt % the \hrule is .4pt high
  \setn@tewidth
  \dimen0=\n@tewidth\advance\dimen0 by -\columnshift\toks0=\everypar\everypar={}\parskip=0pt\baselineskip=0pt\parindent=0pt\let\par=\endgraf\parfillskip=0pt%
%  \noindent\hbox{\noindent\kern\columnshift\vbox{\hrule width \dimen0}}\par
  \hbox{\kern\columnshift
    \vrule height 0.4pt width \dimen0}\kern0pt%
    \kern0.5\AboveNoteSpace%This was commented out, without it the line was going through through some callers
\everypar=\toks0}

% determine if a given note class is to be paragraphed
\def\ParagraphedNotes#1{\Par@gr@phedNotes{#1}\ifdiglot\ifdiglotSepNotes\Par@gr@phedNotes{#1R}\fi\fi}
\def\Par@gr@phedNotes#1{\TRACE{Par@gr@phedNotes #1}\x@\let\csname paranotes-#1\endcsname=1}
\newif\ifp@ranotes
\def\checkp@ranotes#1{\x@\ifx\csname paranotes-#1\endcsname\relax
  \p@ranotesfalse\else\p@ranotestrue\fi}

% Notewidth for diglots depends on settings and column. 
\newdimen\n@tewidth
\def\setn@tewidth{%
  \ifdiglot
    \ifdiglotSepNotes
      \ifdiglotL\n@tewidth=\columnLwidth\else\n@tewidth=\columnRwidth\fi
    \else\n@tewidth=\textwidth\fi
  \else\n@tewidth=\textwidth\fi
}
% reformat the contents of a note class insertion into a single paragraph.
% this is usually done for \x. It is sometimes done for \f.
% (based on code from the TeXbook, appendix D)
% #1 is vbox containing notes as individual paragraphs.
\newif\ifNoteTracing \NoteTracingfalse
\newskip\intern@teskip \intern@teskip=15pt plus 3pt minus 5pt
\newskip\noteRag\noteRag=0pt plus 18pt
\def\maken@tepara#1#2{%
  \setn@tewidth
  \hsize=\n@tewidth%\advance\hsize by -\columnshift % width is full page size
  \let\par=\endgraf\ch@pterfalse
  \everypar={}% don't do body text formatting
  \unvbox#1 % open up the vbox of notes to get at the list of individual note boxes
  \makehboxofhboxes % make a single hbox for all notes of this class
  \setbox0=\hbox{\unhbox0 \removehboxes} % add internote space
  \getp@ram{baseline}{#2}\ifx\p@ram\relax\else\baselineskip=\p@ram\fi
% Enable justification according to marker.
  \lineskiplimit=-10pt\leftskip=0pt\rightskip=0pt\parskip=0pt\parfillskip=0pt plus 1fil%\lineskip=10pt
  \getp@ram{justification}{#2}%
  \ifx\p@ram\c@nter
     \leftskip=\noteRag\rightskip=\noteRag
  \else\ifx\p@ram\l@ft
     \rightskip=\noteRag
  \else\ifx\p@ram\r@ght
     \leftskip=\noteRag
  \fi\fi\fi
  \advance\leftskip by \columnshift%
  %\showbox0
  \noindent % starting making new paragraph
  \ifRTL\beginR\fi % respect directionality
  \ifNoteTracing\tracingparagraphs=1\fi
  \unhbox0 % unbox the text so it can be line-wrapped
  \unskip\unpenalty\unskip\unskip % remove internote skip info after last note
  % set penalty which allows breaking between notes unless this would cause
  % an extra line to be created.
  \linepenalty50
  \trace{f}{Note #2 baselineskip=\the\baselineskip}
  \par\leftskip=0pt\ifNoteTracing\tracingparagraphs=0\fi}
%
% make box0 = an hbox contining all the contents of this class
\def\makehboxofhboxes{%
  \setbox0=\hbox{}%
  \loop\setbox2=\lastbox \ifhbox2 \setbox0=\hbox{\box2\unhbox0}\repeat} 
%
% remove inside level of boxing and adding inter note space after each
%     [[a][b][c]] --> [a \internotespace b \internotespace c \internotespace]
\def\removehboxes{%
  \setbox0=\lastbox 
  \ifhbox0{\removehboxes}\unhbox0\internotespace\fi}
%
% skip between notes in paragraph. skip is good place to break.
\def\internotepenalty{-10}
%\def\internotespace{\hfil\hskip\intern@teskip\penalty\internotepenalty\hfilneg}
\def\internotespace{\hskip\intern@teskip\kern 0pt\penalty\internotepenalty}
%\def\internotespace{\hfil\hskip\intern@teskip\penalty-10\hfilneg}

% don't allow stretching between notes.

%%%%%%%%%%%%%% OUTPUT ROUTINES %%%%%%%%%%%%%% 
% default output routine is single-column, somewhat based on Plain TeX output routine (see TeXbook)
\global\output={\onecol}
\global\holdinginserts=1
\def\onecol{%
  \trace{b}{BALANCE pagebuild: cols=1: textheight=\the\textheight}
  \global\setbox\galley=\copy255
  % tempoarily split and see if there are marks in this text
  \bgroup\setbox0=\copy255 \setbox1=\vsplit0 to \maxdimen\egroup
  \edef\t@mp{\splitbotmark}
  \ifx\t@mp\empty\else\global\m@rksonpagetrue\trace{H}{Found mark \splitbotmark}\fi
  \global\galleypenalty=\outputpenalty
  \global\trialheight=\textheight \global\advance\trialheight by -\ht\partial
  \global\output={\onecoltrial}
  \ifm@rksonpage\else\gdef\p@gefirstmark{}\trace{H}{No marks found. Setting empty mark}\fi%No marks on the page, and it didn't fit, so add a blank mark
  \global\holdinginserts=0
  \unvbox255
  \penalty\ifnum\outputpenalty=10000 0 \else \outputpenalty \fi
}
\def\onecoltrial{% single-column version of \twocoltrial (see below)
  \tracingparagraphs=0
  %\tracingall=1\tracingoutput=0\tracingpages=0\tracingparagraphs=0\tracingassigns=0\tracingscantokens=0
  \trace{i}{1c TRIAL with ht=\the\trialheight, vsize=\the\vsize, hIns=\the\holdinginserts}
  \edef\p@gebotmark{\botmark}% remember last \mark for running header
  \edef\t@st{\p@gefirstmark}%
  \ifx\t@st\empty\xdef\p@gefirstmark{\firstmark}\fi % remember first, if not already set
  \availht=\trialheight % amount of space we think is available
  \f@rstnotetrue
  \let\\=\reduceavailht \the\n@tecl@sses % reduce it by the space needed for each note class
  \decr{\availht}{\topins}% and by the space needed for spanning pictures
  \decr{\availht}{\bottomins}
  \ifdim\availht<0pt
    \MSG{Page overfull with inserts. Perhaps a little more text and less pictures would help}
  \fi
  \setbox\s@vedpage=\copy255
  % split the galley to the actual size available
  \setbox\colA=\vsplit255 to \availht
  %\setbox\colA=\vbox{\unvbox\colA}
  % and check if it all fit; if not, we'll have to back up and try again
  \ifvoid255 \fitonpagetrue \else \fitonpagefalse \fi
  \iffitonpage
    \trace{i}{1 SUCCEEDED, shipping page hIns=\the\holdinginserts}
% at this point:
%   \box\colA is the page content
%   \availht is ht that definitely works
    \def\pagecontents{%
%      \msg{upwards mode = \the\XeTeXupwardsmode}
      \dimen1=\textwidth \advance\dimen1 \ExtraRMargin
      \trace{b}{BALANCE pangeoutins: cols=1: text=\the\ht\colA, \the\dp\colA: partial=\the\ht\partial, \the\dp\partial: topins=\the\ht\topins: bottomins=\the\ht\bottomins}
      \ifvoid\partial\else \vbox{\hbox to \dimen1{\hskip\columnshift\vbox{\unvbox\partial}}} \fi
      \ifvoid\topins\else \hbox{\hbox to \columnshift{}\box\topins} \vskip\skip\topins \fi
      \dimen0=\dp\colA
      \hbox to \dimen1{\hbox to \columnshift{}%
        \box\colA\hbox to \ExtraRMargin{}\hfil}
      \ifvoid\bottomins\else \vfil\kern-\dimen0 \dimen0=0pt \vskip\skip\bottomins \hbox{\hbox to \columnshift{}\box\bottomins} \fi % ouput bottom spanning pictures
      \f@rstnotetrue%
      \m@kenotebox
      \trace{b}{BALANCE pageouttxt: notes=\the\ht2 , \the\dp2 \space Leaving \the\ht255, \the\dp255}
      \ifdim\ht2<0pt\trace{b}{Strange notes height}\global\tracingmacros=1\global\tracingcommands=1\global\tracingpages=1\fi
      \unvbox2
    }
    \resetvsize
    \ifdim\ht\colA>\baselineskip\plainoutput\trace{p}{plainoutput from onecoltrial}%
      \xdef\p@gefirstmark{}\xdef\p@gebotmark{}%
    \else\setbox0=\box255\deadcycles=0\fi % dump empty pages (typically at end)
    \global\holdinginserts=1
    \ifrerunsavepartialpaged\trace{o}{onecoltrial: rerunsavepartialpage}\global\output={\savepartialpage}\global\rerunsavepartialpagedfalse
    \else\global\output={\onecol}\fi
  \else % the contents of the "galley" didn't fit into the actual page,
        % so reduce \vsize and try again with an earlier break
    \trace{i}{1c REDUCING VSIZE hIns=\the\holdinginserts}
    \global\advance\vsize by -\baselineskip
    \global\setbox\topins=\box\voidb@x
    \global\setbox\bottomins=\box\voidb@x
    \let\\=\cle@rn@tecl@ss \the\n@tecl@sses
    \global\setbox255=\box\voidb@x
    \global\holdinginserts=1
    \global\let\whichtrial=\onecoltrial
    \global\output={\backingup}
    \unvbox\galley \penalty\ifnum\galleypenalty=10000 0 \else \galleypenalty \fi
  \fi
}

\def\pagebody{\vbox to\textheight{\boxmaxdepth\maxdepth \pagecontents}}
\def\makeheadline{%
  \g@tfontname{h}%
  %\def\f@ntstyle{h\ifdiglot\ifdiglotL L\else R\fi\fi}
  \vbox to 0pt{\kern-\topm@rgin
   \vbox{\kern\HeaderPosition\MarginUnit
    \setbox0=\vbox{\hbox to \textwidth{\the\headline}}
    \ht0=0.7\fontdimen6\csname font<\f@ntstyle>\endcsname \dp0=0pt \box0
    \ifrhr@le\ifdim\RHruleposition=\maxdimen\else
      \kern\RHruleposition \hrule
    \fi\fi}\vss}\nointerlineskip}
\newif\ifrhr@le
\def\makefootline{%
  \vbox to 0pt{\dimen0=\textwidth\advance\dimen0 by -\columnshift
    \kern\bottomm@rgin
    \kern-\FooterPosition\MarginUnit
     \hbox to \textwidth{\the\footline}\vss}}
\newdimen\RHruleposition \RHruleposition=\maxdimen

\def\PageFullFactor{0.9}
\def\doublecolumns{
  \ifnum\c@rrentcols=1
    % make switch from single to double column
    \ifhe@dings\endhe@dings\fi % if headings in process, end headings
    \penalty-100\vskip\baselineskip % ensure blank line between single and double column material
    \global\output={\savepartialpage}\eject % save any partially-full page
	% if single column material already fills 3/4 page, go to next page to start double columns
    \dimen0=\ht\partial
    \ifdim\dimen0>\PageFullFactor\textheight
%%%      \msg{Output full page from 1 to 2 columns}%
      \global\output={\onecol} % but output it immediately if 75% full
      \loop\ifdim\dimen0>\textheight\advance\dimen0-\textheight\repeat
     %\vbox{\hskip-\columnshift\box\partial}
      \unvbox\partial
      \ifdim\dimen0>\PageFullFactor\textheight\vfill\eject
        \else\global\output={\savepartialpage}\eject
      \fi
    \else
      %\hsize=\textwidth
      %\setbox\partial=\vbox{\hskip\columnshift\box\partial}
    \fi
    % reset parameters for 2-column formatting
    \global\hsize=\colwidth\trace{o}{doublecolumn vsize doubles}
    \global\vsize=2\textheight % in 2 col mode you can put twice the height of text
	\global\advance\vsize by -2\ht\partial % subtract height of 1 column material
	\global\advance\vsize by 2\baselineskip % don't get caught short by 1/2 line or so
	% make a macro reset vsize for remaining pages which do not have 1 column material
    \trace{o}{Going to 2 col, textheight \the\textheight - \the\ht\partial}
    \gdef\resetvsize{\trace{o}{resetvsize2}\global\vsize=2\textheight \global\advance\vsize by \baselineskip} 
    \global\output={\twocols}
    \global\c@rrentcols=2
    % top and bottom inserts effectively use twice their height
	% (\count of an insertion class is a scaling factor)
    \count255=2000
    \global\count\topins=\count255
    \global\count\bottomins=\count255
    \let\\=\s@tn@tec@unt \the\n@tecl@sses % reset \count for each note class
    \global\holdinginserts=1 % don't pull out inserts yet, we are still adjusting page
    %\msg{Starting 2col partial: \the\ht\partial, pagetotal: \the\pagetotal, vsize: \the\vsize}
  \fi}

% iterate over note classes, set \count for each class  
\def\s@tn@tec@unt#1{%
  \x@\let\x@\th@cl@ss\csname note-#1\endcsname
  \checkp@ranotes{#1}%
  % notes made into paragraphs (e.g. \x) are counted as 0 height for now,
  % later we will backup making page size smaller until things fit
  % Diglot prefers to get things closer 1st time round...
  \global\count\th@cl@ss=\ifp@ranotes \ifdiglot 500\else 0\fi \else \count255 \fi
  \global\skip\th@cl@ss=\AboveNoteSpace }

\def\resetvsize{\trace{o}{resetvsize}\global\vsize=\textheight} 

\def\msg#1{\immediate\write16{#1}}

\def\savepartialpage{% save a partially-full page when switching to 2-column format,
  \xdef\p@gebotmark{\botmark}
  \global\trialheight=\textheight \global\advance\trialheight by -\ht\partial
  \c@lcavailht
  \setbox\s@vedpage=\copy255
  % split the galley to the actual size available
  \setbox\colA=\vsplit255 to \availht
  \setbox\colA=\vbox{\unvbox\colA}
  % and check if it all fit; if not, we'll have to back up and try again
  \ifvoid255 \fitonpagetrue \else \fitonpagefalse \fi
  \trace{o}{savepartialpage \the\ht\colA rem=\the\ht255}
  \iffitonpage
    \global\setbox\partial=\vbox{
      \ifvoid\partial\else \unvbox\partial \fi
      \ifvoid\topins\else \unvbox\topins \vskip\skip\topins \fi
      \dimen0=\dp\colA
      \dimen1=\textwidth \advance\dimen1 \ExtraRMargin \advance\dimen1 -\columnshift
      \hbox to \dimen1{%\hbox to \columnshift{}%
        \box\colA\hbox to \ExtraRMargin{}\hfil}
      \ifvoid\bottomins\else\kern-\dimen0 \dimen0=0pt \vskip\skip\bottomins \unvbox\bottomins \fi
      \f@rstnotetrue
      \m@kenotebox
      \trace{b}{BALANCE pageouttxt: notes=\the\ht2 , \the\dp2}
      \unvbox2
    }
    \resetvsize%
    \ifdim\ht\colA>\baselineskip\plainoutput
    \xdef\p@gefirstmark{}
    \xdef\p@gebotmark{}
    \else\trace{o}{Page empty \the\ht\colA, \ht\colB, \the\ht\partial}\setbox0=\box255\deadcycles=0\fi % dump empty pages (typically at end)
    \tempfalse%
    \ifdim\ht\partial>\PageFullFactor\textheight \temptrue\fi%
    \ifnum\outputpenalty<-10000 \temptrue\fi%
    \iftemp\def\pagecontents{\box\partial}\plainoutput 
    \xdef\p@gefirstmark{}
    \xdef\p@gebotmark{}
    \fi
    \global\holdinginserts=1
  \else % the contents of the "galley" didn't fit into the actual page,
        % so reduce \vsize and try again with an earlier break
    \trace{i}{2c REDUCING VSIZE hIns=\the\holdinginserts}
    \@mptyinserts
    \global\advance\vsize by -\baselineskip
    \let\\=\cle@rn@tecl@ss \the\n@tecl@sses
    \global\let\whichtrial=\onecoltrial
    \global\output={\backingup}
    \global\rerunsavepartialpagedtrue
    \ifm@rksonpage\else\gdef\p@gefirstmark{}\trace{H}{No marks found. Setting empty mark}\fi%No marks on the page, and it didn't fit, so add a blank mark
    \unvbox\s@vedpage \penalty\ifnum\galleypenalty=10000 0 \else \galleypenalty \fi
  \fi}
  % save 1 column material since we are switching to 2 columns
\newbox\partial

\def\twocols{% primary output routine in 2-col mode
  \trace{i}{TWOCOLS @ \ch@pter:\v@rse, txtht=\the\textheight, partial=\the\ht\partial, hIns=\the\holdinginserts}
  % save copy of current page so we can retry with different heights
  \trace{b}{BALANCE pagebuild: cols=2: textheight=\the\textheight}
  \global\setbox\galley=\copy255
  \bgroup\setbox0=\copy255 \setbox1=\vsplit0 to \maxdimen\egroup
   \edef\t@mp{\splitbotmark}
   \ifx\t@mp\empty\else\global\m@rksonpagetrue\trace{H}{Found mark \splitbotmark}\fi
  \global\galleypenalty=\outputpenalty % save current penalty so we can restore it at (A)
  \global\trialheight=\textheight \global\advance\trialheight by -\ht\partial
  \global\output={\twocoltrial}
  \global\holdinginserts=0 % when doing trial place insertions into boxes
  \unvbox255 % force invoking \twocoltrial
  \penalty\ifnum\outputpenalty=10000 0 \else \outputpenalty \fi % (A) restore output penalty
  }
\newbox\galley
\newcount\galleypenalty
\newdimen\trialheight

% for measuring the space needed for each class of notes;
% this will decrease \availht by the space needed for the given class
\def\reduceavailht#1{%
  \checkp@ranotes{#1}%
  %\TRACE{av: \the\availht =>}
  \ifp@ranotes\let\n@xt=\reduceavailht@para
    \ifdiglot\ifdiglotSepNotes\ifdiglotBalNotes\let\n@xt=\reduceavailht@para@bal\fi\fi\fi
  \else\let\n@xt=\reduceavailht@sep
    \ifdiglot\ifdiglotSepNotes\ifdiglotBalNotes\let\n@xt=\reduceavailht@sep@bal\fi\fi\fi
  \fi
  \iff@rstnote\else\global\noteseentrue\fi\n@xt{#1}}

\def\reduceavailht@para#1{
  \def\cl@sss{#1}%
  \ifdiglot\ifdiglotSepNotes\ifdiglotL\def\cl@sss{#1}\else\def\cl@sss{#1R}\fi\fi\fi%
  \TRACE{class \cl@sss}%
  \x@\let\x@\th@cl@ss\csname note-\cl@sss\endcsname
  \ifvoid\th@cl@ss\else
    \setbox0=\copy\th@cl@ss
    \setbox0=\vbox{\maken@tepara{0}{#1}}
    \advance\availht by -\ht0
    \advance\availht by -\dp0
    \iff@rstnote\advance\availht by -\AboveNoteSpace \f@rstnotefalse
    \else \advance\availht by -\InterNoteSpace\fi
    %\getp@ram{baseline}{#1}\ifx\p@ram\relax\else\advance\availht-\p@ram\fi
  \fi}

\def\reduceavailht@para@bal#1{%
  \TRACE{reduceavailht@para@bal #1}%
  \x@\let\x@\th@cl@ssL\csname note-#1\endcsname%
  \bgroup%leave dimen0 etc unchanged
  \dimen0=0pt
  %FIXME: Should this code adjust & restore diglotL?%
  \ifvoid\th@cl@ssL\else
    \setbox0=\copy\th@cl@ssL
    \setbox0=\vbox{\maken@tepara{0}{#1}}%
    \message{reduceavailht@para@bal for #1 ht0=\the\ht0 dp0=\the\dp0}%
    \dimen0=\ht0
    \advance\dimen0 by \dp0
  \fi
  \x@\let\x@\th@cl@ssR\csname note-#1R\endcsname
  \dimen1=0pt
  \ifvoid\th@cl@ssR\else
    \setbox0=\copy\th@cl@ssR
    \setbox0=\vbox{\maken@tepara{0}{#1}}%
    \message{reduceavailht@para@bal for #1R ht1=\the\ht1 dp1=\the\dp1}%
    \dimen1=\ht0
    \advance\dimen1 by \dp0
  \fi
  \ifnum\dimen1>\dimen0 \dimen0=\dimen1 \fi
  \ifdim\dimen0>0pt
    \advance\availht by -\dimen0
    \iff@rstnote\advance\availht by -\AboveNoteSpace \f@rstnotefalse
    \else \advance\availht by -\InterNoteSpace\fi
    %\getp@ram{baseline}{#1}\advance\availht-\p@ram%
    \message{p@ram=\p@ram}%
  \fi\egroup}

\def\reduceavailht@sep#1{
  \def\cl@sss{#1}%
  \ifdiglot\ifdiglotSepNotes\ifdiglotL\def\cl@sss{#1}\else\def\cl@sss{#1R}\fi\fi\fi
  \TRACE{class \cl@sss}%
  \x@\let\x@\th@cl@ss\csname note-\cl@sss\endcsname
  \ifvoid\th@cl@ss\else
    \f@rstnotefalse
    \advance\availht by -\ht\th@cl@ss
    \advance\availht by -\dp\th@cl@ss
    \advance\availht by -\AboveNoteSpace
%FIXME?: Shouldn't the above be using InterNoteSpace? 
  \fi}

\def\reduceavailht@sep@bal#1{%
  \TRACE{reduceavailht@sep@bal}%
  \x@\let\x@\th@cl@ssR\csname note-#1\endcsname
  \x@\let\x@\th@cl@ssL\csname note-#1R\endcsname
  \dimen0=0pt\dimen1=0pt
  \ifvoid\th@cl@ssL\else
    \dimen0=\ht\th@cl@ssL
    \advance\dimen0 by \dp\th@cl@ssL
  \fi
  \ifvoid\th@cl@ssR\else
    \dimen1=\ht\th@cl@ssR 
    \advance\dimen1 by \dp\th@cl@ssR
  \fi
  \ifdim\dimen0>0pt 
    \advance\availht by -\dimen0
    \advance\availht by -\AboveNoteSpace
  \fi}

\def\cle@rn@tecl@ss#1{%
  \ifdiglot
    \ifdiglotSepNotes
       \cle@rn@t@cl@ss{#1}%
       \cle@rn@t@cl@ss{#1R}%
    \else\cle@rn@t@cl@ss{#1}\fi
  \else\cle@rn@t@cl@ss{#1}\fi
}

\def\cle@rn@t@cl@ss#1{%
  \x@\let\x@\th@cl@ss\csname note-#1\endcsname
  \global\setbox\th@cl@ss=\box\voidb@x}

% increment or decrement a given \dimen by the height of a given \box, unless void
% round to baselineskip assuming 0.5\baselineskip already added
\def\incr#1#2{\ifvoid#2\else\advance#1 by \skip#2\advance#1 by \ht#2\fi}
\def\decr#1#2{\ifvoid#2\else\advance#1 by -\skip#2\advance#1 by -\ht#2\fi}

% allocate some named registers for the \trial routine to use
\newbox\s@vedpage % to save the page contents for re-splitting columns
\newdimen\availht % overall available height
\newdimen\shortavail % shortened version of \availht for re-balancing loop
\newdimen\colhtA \newdimen\colhtB % dimen registers for calculating available ht for each col
\newbox\colA \newbox\colB % box registers to hold contents of the columns
\newcount\loopcount
\newif\ifrem@inder \rem@indertrue
\newif\ifunbalanced \unbalancedfalse% \unbalancedtrue
\newif\ifnoteseen \noteseenfalse% A trigger to help alert user that note might have gone missing

\def\s@tcolhts#1{
    \colhtA=#1 \decr{\colhtA}{\topleftins} \decr{\colhtA}{\bottomleftins}
    \colhtB=#1 \decr{\colhtB}{\toprightins} \decr{\colhtB}{\bottomrightins}
}

\def\g@tcolhts{
    \colhtA=\ht\colA \incr{\colhtA}{\topleftins} \incr{\colhtA}{\bottomleftins}
    \colhtB=\ht\colB \incr{\colhtB}{\toprightins} \incr{\colhtB}{\bottomrightins}
}

\def\spl@tcols#1{
    % even if \vsplit to 0pt, TeX will always pull one line from the input box over
    %\trace{o}{split params maxdepth=\the\splitmaxdepth, topskip=\the\splittopskip, colhtA=\the\colhtA, colhtB=\the\colhtB}
    \splittopskip=\topskip
    \setbox9=\copy#1
    \setbox\colA=\vsplit#1 to \colhtA \ifnum\badness>999999\setbox#1=\box9\setbox\colA\box\voidb@x\else\setbox\colA=\vbox{\unvbox\colA}\fi
    \setbox9=\copy#1
    \setbox\colB=\vsplit#1 to \colhtB \ifnum\badness>999999\setbox#1=\box9\setbox\colB\box\voidb@x\else\setbox\colB=\vbox{\unvbox\colB}\fi
}

\newdimen\availA \newdimen\availB
\def\BalanceThreshold{0}

\def\balanced{
  \setbox0=\copy\s@vedpage
  \s@tcolhts{\availht}\spl@tcols0
  \ifnum\BalanceThreshold>0
    \availA=\colhtA \availB=\colhtB
    \advance\availA by -\BalanceThreshold\baselineskip
    \advance\availB by -\BalanceThreshold\baselineskip
  \else
    \availA=0pt \availB=0pt
  \fi
  % and check if it all fit; if not, we'll have to back up and try again
  \ifvoid0 \fitonpagetrue \else \fitonpagefalse \fi
  \rebalancefalse
  \trace{o}{first \the\availht . col=\the\ht\colA, \the\colhtA . second col=\the\ht\colB, \the\colhtB . rem=\the\ht0}
  \ifdim\ht\colA<.5\baselineskip\ifdim\ht\colB<.5\baselineskip\MSG{I can't break this page!}
    \iffitonpage\s@ve@nd % the split failed and there is nothing left!
    \else\setbox\colA=\box0\setbox\s@vedpage=\box\voidb@x % dump it all in colA and bail
  \fi\fi\fi
  \iffitonpage
    \shortavail=\colhtB \advance\shortavail by -\ht\colB
    \advance\shortavail by -\colhtA \advance\shortavail by \ht\colA
    \ifdim\shortavail>5\baselineskip
      \trace{o}{Rebalance trying from average \the\shortavail of \the\colhtA =\the\ht\colA, \the\colhtB =\the\ht\colB}
      \divide\shortavail 2% \advance\shortavail by -0.5\ht\colA \advance\shortavail by -0.5\ht\colB
      \advance\colhtA by -\shortavail \advance\colhtB by -\shortavail
      \loopcount=0
      \loop\setbox0=\copy\s@vedpage
        \advance\colhtA by \baselineskip
        \advance\colhtB by \baselineskip
        \spl@tcols0
        \ifvoid0\loopcount=10\fi  
        \ifnum\loopcount=10\repeat % so ugly you can't \else\repeat
      \trace{o}{Rebalance starting with \the\colhtA, \the\colhtB}
    \fi
    \g@tcolhts
    \ifunbalanced\shortavail=\availht\else\ifdim\colhtB<\colhtA \rebalancetrue \shortavail=\colhtA\fi\fi
    \advance\colhtA -\baselineskip\ifdim\colhtA<\colhtB \rebalancetrue \shortavail=\colhtB\fi
    \advance\shortavail \baselineskip
    \loopcount=0
    \ifrebalance
      \loop
        \vfuzz=\maxdimen
        \advance\loopcount by 1
        \advance\shortavail by -\baselineskip
        \setbox0=\copy\s@vedpage
        \dimen0=\shortavail\advance\dimen0 by \ifdim\dp\colA>\dp\colB \dp\colA\else\dp\colB\fi
        \s@tcolhts{\dimen0}\spl@tcols0
		% if something left in box0, it didn't fit, quit loop
        \trace{o}{re-balancing \the\ht\colA =\the\colhtA, \the\ht\colB =\the\colhtB, from \the\shortavail, rem \the\ht0}%
        \ifrem@inder\else\ifvoid0\else
          \advance\shortavail by \baselineskip
          \advance\shortavail by \ifdim\dp\colA>\dp\colB \dp\colA\else\dp\colB\fi
          \trace{o}{rebalance found extra back to \the\shortavail}\rebalancefalse
          \setbox0=\copy\s@vedpage
          \s@tcolhts{\shortavail}\spl@tcols0
        \fi\fi
        % if either column height ends up negative, rollback and bail
        \ifrebalance\ifnum\ifdim\colhtA<\availA 1\else\ifdim\colhtB<\availB 1\else 0\fi\fi =1
          \advance\shortavail by \baselineskip
          \trace{o}{rebalance found column height < 0, back to \the\shortavail}\rebalancefalse
          \setbox0=\copy\s@vedpage
          \s@tcolhts{\shortavail}\spl@tcols0
        \fi\fi
		% if second column longer than first column by more than .3*line height, quit loop
        \colhtA=\ht\colA \colhtB=\ht\colB
        \dimen0=\colhtB \incr{\dimen0}{\toprightins} \incr{\dimen0}{\bottomrightins}%
        \advance\dimen0 by -\colhtA \decr{\dimen0}{\topleftins} \decr{\dimen0}{\bottomleftins}%
        \ifdim\dimen0<.95\baselineskip\ifdim\dimen0>-.95\baselineskip
          \shortavail=\colhtB\incr{\shortavail}{\toprightins}\incr{\shortavail}{\bottomrightins}%
          \ifdim\dimen0>0pt\else\advance\shortavail by -\dimen0\fi
          \advance\shortavail by \ifdim\dp\colA>\dp\colB \dp\colA\else\dp\colB\fi
          \trace{o}{Best height \the\shortavail}\rebalancefalse
        \fi\fi
		% give up if target size less than 2 lines (should not happen)
        \ifdim\shortavail<\baselineskip \MSG{Rebalancing bailed for short block}%
          \shortavail=\availht \rebalancefalse\fi
        \ifnum\loopcount>20 
          \advance\shortavail by \baselineskip
          \rebalancefalse\MSG{Rebalancing loop count bail}\fi
        \ifrebalance\repeat
    \fi
    \advance\shortavail by \ifdim\dp\colA>\dp\colB \dp\colA\else\dp\colB\fi
    \s@tcolhts{\shortavail}\spl@tcols\s@vedpage
    \trace{o}{A = \the\colhtA , B = \the\colhtB , remaining = \the\ht\s@vedpage}%
  \fi
}

\def\c@lcboxheights{
  \g@tcolhts
  \trace{b}{BALANCE pageout: cols=2: texta=\the\ht\colA , \the\dp\colA : textb=\the\ht\colB , \the\dp\colB : partial=\the\ht\partial, \the\dp\partial : topins=\the\ht\topins+\the\skip\topins, \the\ht\topleftins+\the\skip\topleftins, \the\ht\toprightins+\the\skip\toprightins : bottomins=\the\ht\bottomins+\the\skip\bottomins, \the\ht\bottomleftins+\the\skip\bottomleftins, \the\ht\bottomrightins+\the\skip\bottomrightins}
  \dimen0=\dp\colA \setbox\colA=\vbox to \availht{\unvbox\topleftins\box\colA\vfil\unvbox\bottomleftins}\dp\colA=\dimen0
  \ifdim\ht\colB<1pt\setbox\colB\vbox{\noindent\par}\else
    \ifdim\dimen0<\dp\colB\dimen0=\dp\colB\fi \setbox\colB=\vbox to \availht{\unvbox\toprightins\box\colB\vfil\unvbox\bottomrightins}\dp\colB=\dimen0
  \fi
  \ifdim\colhtA>\colhtB \dimen3=\colhtA\dimen0=\dp\colA\dp\colB=\dimen0\else
    \dimen3=\colhtB\dimen0=\dp\colB\dp\colA=\dimen0\fi
  \advance\dimen3 by \dimen0
}

\def\@mptyinserts{
  \global\setbox\topins=\box\voidb@x
  \global\setbox\bottomins=\box\voidb@x
  \global\setbox\topleftins=\box\voidb@x
  \global\setbox\toprightins=\box\voidb@x
  \global\setbox\bottomleftins=\box\voidb@x
  \global\setbox\bottomrightins=\box\voidb@x
  \global\setbox255=\box\voidb@x
  \global\holdinginserts=1
}

\xdef\p@gebotmark{}% No guarantee this will be universally set
\def\c@lcavailht{
  \edef\t@st{\p@gefirstmark}
  \ifx\t@st\empty\xdef\p@gefirstmark{\firstmark}\fi% remember first, if not already set
  \edef\t@st{\p@gebotmark}
  \ifx\t@st\empty\xdef\p@gebotmark{\botmark}\fi
  %NOT HERE! \global\trialheight=\textheight \global\advance\trialheight by -\ht\partial
  \availht=\trialheight % amount of space we think is available
  \trace{i}{C@lcavailht partial:\the\ht\partial \space av:\the\availht}
  \f@rstnotetrue
  \let\\=\reduceavailht\the\n@tecl@sses % reduce it by the space needed for each note class
  \trace{i}{after inserts: \the\availht}
  \trace{o}{availht = \the\availht, depth = \the\dp255}
  \advance\availht -\dp255
  % Round to lines accurate to 1/8pt in lineskip. 1/10pt causes overflow on longer pages. Support two column legal.
  \dimen0=8\baselineskip
  \multiply\availht by 8 \divide\availht by\dimen0 \multiply\availht by\dimen0 \divide\availht by 8
  \decr{\availht}{\topins} % and by the space needed for spanning pictures
  \decr{\availht}{\bottomins}
  \advance\availht by \dp255 % split includes depth so give it space for that
  \trace{o}{new availht=\the\availht, topins=\the\ht\topins, bottomins=\the\ht\bottomins, baselineskip=\the\baselineskip}
  \splittopskip=\topskip
}

\newif\ifrerunsavepartialpaged
\def\savepartialpaged{%
  \global\trialheight=\textheight \global\advance\trialheight by -\ht\partial
  \c@lcavailht
  \setbox\s@vedpage=\vbox{\unvbox255}
  \trace{o}{balancing from savepartialpaged, height=\the\ht\s@vedpage, hIns:\the\holdinginserts, vsz=\the\vsize}
  \ifdim\availht<\baselineskip 
    \rem@indertrue % No, I don't know what this does, but once If i  didn't do it then we get lost verses. 
    %\rem@inderfalse
    \balanced\fitonpagetrue % There's certainly no point shrinking past this
  \else
    \rem@inderfalse\balanced
  \fi
  \iffitonpage
    \c@lcboxheights
    \ht\colA=\dimen3 \ht\colB=\dimen3
    \dimen4=\ht\partial \dimen5=\dp\partial
    \global\setbox\partial=\vbox{
      % bodge the partial box which should really have been properly textually merged
      % the problem is that savepartialpaged can get called twice in succession because
      % not all the output text (after a \singlecolumn) was passed the first time
      \ifvoid\partial\else\trace{o}{nested partial}\unvbox\partial\vskip -2\dimen5\fi
      \ifvoid\topins\else \hbox{\hbox to \columnshift{}\box\topins} \vskip\skip\topins \fi % output top spanning pictures
      \dimen5=\textwidth\advance\dimen5 by -\ExtraRMargin\advance\dimen5 by -\columnshift %
      \vbox{\hbox to \dimen5{%\hbox to \columnshift{}
        \ifRTL \hbox to \ExtraRMargin{}\box\colB\makecolumngutter{\the\dimen3}{\the\dimen3}{\the\dimen0}\box\colA %
        \else \box\colA\makecolumngutter{\the\dimen3}{\the\dimen3}{\the\dimen0}\box\colB\hbox to \ExtraRMargin{}\fi}}}
    %\advance\dimen3 by \dimen4 \ht\partial=\dimen3
    \trace{o}{sppd: saved partial, ht:\the\ht\partial\space rem:\the\ht\s@vedpage}
    \global\holdinginserts=1
    \global\setbox255=\box\voidb@x
    \global\trialheight=\textheight \global\advance\trialheight by -\ht\partial
    \c@lcavailht
    \trace{o}{spdd: partial=\the\ht\partial, textheight=\the\textheight, availht=\the\availht}
    \ifdim\availht<2\baselineskip \twocolp@geout \fi
    \unvbox\s@vedpage\penalty\ifnum\outputpenalty=10000 0 \else \outputpenalty \fi
  \else
    \trace{o}{sppd: Not fit on page}
    \trace{i}{sppd: emptyinserts hIns=\the\holdinginserts, \the\ht\partial, vs:\the\vsize}
    \@mptyinserts
    \global\advance\vsize by -\baselineskip
    \global\let\whichtrial=\savepartialpaged
    %\global\rerunsavepartialpagedtrue
    \global\output={\backingup}%
    \tempfalse%\ifvoid\partial\else\temptrue\fi%
    % remember \holdingInserts=1 from \@emptyinserts
    \ifvoid\galley\temptrue\fi
    \ifm@rksonpage\else\gdef\p@gefirstmark{}\trace{H}{No marks found. Setting empty mark}\fi%No marks on the page, and it didn't fit, so add a blank mark
    \iftemp
     \trace{i}{Using savedpage.pen=\the\outputpenalty}%
     \unvbox\s@vedpage\penalty\ifnum\outputpenalty=10000 0 \else \outputpenalty \fi
    \else%
     \trace{i}{Using galley. pen=\the\galleypenalty}%
     \unvbox\s@vedpage\penalty\ifnum\galleypenalty=10000 0 \else \galleypenalty \fi
    \fi%
  \fi%
}

\def\m@kenotebox{
  \setbox2=\vbox{
  \let\\=\ins@rtn@tecl@ss \the\n@tecl@sses % output all note classes
  \iff@rstnote % no notes actually occurred!
    \trace{i}{No notes}
    \ifnoteseen\MSG{Page \the\pageno\space is being printed without any footnotes/xrefs, etc. But at least one was seen earlier. Maybe it's ended up on the previous page or moved to the next one, or maybe it's vanished. Human checking is needed.}\fi
    \kern-\dimen0 \vfil
  \else
    \setbox0=\lastbox
    \trace{i}{Inserted notes, ht \the\ht0\space dp \the\dimen0}
    \dimen0=\dp0 \box0 \kern-\dimen0
  \fi}}

\def\twocoltrial{% trial formatting to see if current contents will fit on the page
  \tracingparagraphs=0
  \global\trialheight=\textheight \global\advance\trialheight by -\ht\partial
  \c@lcavailht%
  \ifdim\availht<0pt
    \MSG{Page overfull with inserts. Perhaps a little more text and less pictures would help}
  \fi
  \setbox\s@vedpage=\copy255
  \trace{o}{balance from twocoltrial height \the\ht\s@vedpage}
  \rem@indertrue\balanced
  \dimen5=\baselineskip
  \iffitonpage\ifvoid\s@vedpage\else\fitonpagefalse\fi
  \else\ifnum\vsize<0 \fitonpagetrue\fi % no fit so dump the page we have now and try again with a new one.
  \fi% \dimen5=\ht\s@vedpage\fi\fi
  \iffitonpage
    \c@lcboxheights
    \ifdim\ht\colB<1pt\setbox\colB\vbox{\noindent\par}\fi
    \ht\colA=\the\availht \ht\colB=\the\availht
    \def\pagecontents{%
      \trace{o}{Topins=\the\ht\topins, Bottomins=\the\ht\bottomins, tlins=\the\ht\topleftins, blins=\the\ht\bottomleftins, trins=\the\ht\toprightins, brins=\the\ht\bottomrightins}%
      \dimen1=\textwidth\advance\dimen1 by \ExtraRMargin
      \ifvoid\partial\else %\msg{2colout partial: \the\ht\partial}\hskip\columnshift
        \vbox{\hbox to \dimen1{\hskip\columnshift\vbox{\unvbox\partial}}}\fi % output partial page
      \ifvoid\topins\else \hbox{\hbox to \columnshift{}\box\topins} \vskip\skip\topins \fi % output top spanning pictures
      \trace{p}{Text depth = \the\dimen3}%\showthe\dimen0
      \hbox to \dimen1{\hbox to \columnshift{}%
        \ifRTL \hbox to \ExtraRMargin{}\box\colB\makecolumngutter{\the\dimen3}{\the\availht}{\the\dimen0}\box\colA
        \else \box\colA\makecolumngutter{\the\dimen3}{\the\availht}{\the\dimen0}\box\colB\hbox to \ExtraRMargin{}\fi}
      \ifvoid\bottomins\else \kern-\dimen0 \dimen0=0pt \vskip\skip\bottomins \unvbox\bottomins \fi % ouput bottom spanning pictures
      \f@rstnotetrue
      \m@kenotebox
      \trace{b}{BALANCE pageouttxt: notes=\the\ht2 , \the\dp2}%
      \unvbox2
      \global\noteseenfalse%
    }
    \global\setbox255=\box\voidb@x
    \resetvsize % reset size of next page since it will not have any 1 column material
    \plainoutput 
    \xdef\p@gefirstmark{}%
    \xdef\p@gebotmark{}%
    \global\holdinginserts=1
    \ifrerunsavepartialpaged
      \global\output={\savepartialpaged}\unvbox\s@vedpage\global\rerunsavepartialpagedfalse
    \else\global\output={\twocols}\unvbox\s@vedpage\fi
  \else % the contents of the "galley" didn't fit into the columns,
        % so reduce \vsize and try again with an earlier break
    \trace{o}{Reducing vsize \the\vsize, by \the\dimen5}
    \global\advance\vsize by -\dimen5
	% clear insertions
    \trace{i}{2coltrial: emptyinserts hIns=\the\holdinginserts}
    \@mptyinserts
    \let\\=\cle@rn@tecl@ss \the\n@tecl@sses
    \global\let\whichtrial=\twocoltrial
    \global\output={\backingup}
    \ifm@rksonpage\else\gdef\p@gefirstmark{}\trace{H}{No marks found. Setting empty mark}\fi%No marks on the page, and it didn't fit, so add a blank mark
    \unvbox\galley \penalty\ifnum\galleypenalty=10000 0 \else \galleypenalty \fi
  \fi}
\newif\iffitonpage
\newif\ifrebalance

\newdimen\columnshift \columnshift=0pt
\newdimen\ColumnGutterRuleSkip \ColumnGutterRuleSkip=0pt

%makecolumngutter{rule height}{vbox height}{depth}
\def\makecolumngutter#1#2#3{\hfil
  \dimen4=#1\advance\dimen4-\ColumnGutterRuleSkip%\advance\dimen4 by #3
  \ifColumnGutterRule
    \setbox4=\vbox to #2{
      \vskip\ColumnGutterRuleSkip
      \hbox to 1pt{\vrule height \the\dimen4}%\dp5=#3 \box5
      \vfil}
    \dp4=#3 \box4
  \fi
  \hfil
  \hbox to \columnshift{}}
\newif\ifColumnGutterRule

\def\backingup{% this output routine is used when we reduce \vsize;
               % it will cause a new page break to be found, and then the \trial routine is called again
  \trace{o}{backingup hIns=\the\holdinginserts(==1)}
  \global\deadcycles=0
  \global\setbox\galley=\copy255
  \global\galleypenalty=\outputpenalty
  \global\output={\whichtrial}
  \global\holdinginserts=0
  \unvbox255% eject
  \penalty\ifnum\outputpenalty=10000 0 \else \outputpenalty \fi
}

\xdef\p@gefirstmark{}

\def\savepartialpagedbounce{
  % This output routine gets the partial galley (input text) before a 2->1 column transition
  % so that it can be re-run with holdinginserts=0, so footnotes etc can be seen.
  \global\trialheight=\textheight \global\advance\trialheight by -\ht\partial
  \c@lcavailht
  \vsize=2\availht
  \advance\vsize by \baselineskip
  \global\output={\savepartialpaged}
  \global\setbox\galley=\copy255 \global\setbox\galley=\vbox{\unvbox\galley}
  \bgroup\setbox0=\copy255 \setbox1=\vsplit0 to \maxdimen\egroup
   \edef\t@mp{\splitbotmark}
   \ifx\t@mp\empty\else\global\m@rksonpagetrue\trace{H}{Found mark \splitbotmark}\fi
  \trace{o}{sppb hIns=\the\holdinginserts, pt:\the\ht\partial, g:\the\ht255, \the\ht\galley}
  \global\holdinginserts=0
  \unvbox255\eject
}

\def\enddigl@t{
  \ifsk@pping \egroup \fi% if we were skipping nonpublishable text, end that mode
  \sk@ppingfalse
  \ifhe@dings\endhe@dings\fi%
  \Lneedsemptyingtrue%
%FIXME! If \ifendbooknoeject  happens to be true, then what? 
%Set the 2 columns, put the footnotes up in the air and carry on??
  \eject
}

% to switch back to diglot columns, first set the page  so far 
% and then change the rest.
\def\diglotcolumns{\ifdiglot\enddigl@t\fi\diglottrue\diglots@tup}

\def\diglots@tup{%
  % reset parameters for diglot-column formatting
  \global\hsize=\columnLwidth % always start with left column
  \global\vsize=\textheight
  \global\BodyColumns=1
  \global\advance\vsize by \baselineskip % don't get caught short by 1/2 line or so
  \global\advance\vsize by -\ht\partial % Don't mis-inform tex
  % make a macro reset vsize for remaining pages which do not have 1 column material
  \gdef\resetvsize{\global\vsize=\textheight \global\advance\vsize by -\ht\partial \global\advance\vsize by 0.5\baselineskip} 
  \global\availht=\vsize
  \global\c@rrentcols=2
  \global\output={\diglotLeft}
  % (\count of an insertion class is a scaling factor)
  \count255=1000
  \global\count\topins=\count255
  \global\count\bottomins=\count255
  %\let\\=\s@tn@tec@unt \the\n@tecl@sses % NECESSARY? reset \count for each note class
  \global\holdinginserts=1 % don't pull out inserts yet, we are still adjusting page
  %\msg{Starting diglot partial: \the\ht\partial, pagetotal: \the\pagetotal, vsize: \the\vsize}
} 

\def\twocolp@geout{%This gets used a few times
  \trace{o}{twocolp@geout cols=1: partial=\the\ht\partial , \the\dp\partial : topins=\the\ht\topins : bottomins=\the\ht\bottomins}
  \def\pagecontents{
    \trace{b}{BALANCE pageout: cols=1: partial=\the\ht\partial , \the\dp\partial : topins=\the\ht\topins : bottomins=\the\ht\bottomins}
    \vbox{\dimen1=\textwidth \advance\dimen1 \ExtraRMargin
    \ifvoid\partial\else \vbox{\hbox to \dimen1{\hskip\columnshift\box\partial}} \fi
	\dimen0=\dp\partial
    \ifvoid\bottomins\else \kern-\dimen0 \dimen0=0pt \vskip\skip\bottomins \unvbox\bottomins \fi
    \f@rstnotetrue%
    \m@kenotebox
    \trace{b}{BALANCE pageouttxt: notes=\the\ht2 , \the\dp2}%
    \unvbox2
    }}\plainoutput\noteseenfalse
    \xdef\p@gefirstmark{}\xdef\p@gebotmark{}%
}%

% to switch back to single column, we reset the page size parameters
\def\singlecolumn{%
  \trace{o}{singlecolumn hIns=\the\holdinginserts}%\tracingmacros=1
  \ifnum\c@rrentcols>1
    \ifhe@dings\endhe@dings\fi
    %\temptrue
    \global\output={\savepartialpagedbounce}\eject\eject % save any partially-full page into a box.
    %LOTS happens before this line is reached!
    \global\hsize=\textwidth \advance\hsize by -\columnshift
    \global\vsize=\textheight
    %\tempfalse
    \ifdim\ht\partial>\PageFullFactor\textheight
      \twocolp@geout \fi%Set up pagecontents and call \plainoutput
    \global\c@rrentcols=1
    \trace{o}{resetting resetvsize for singlecolumn}\def\resetvsize{\trace{o}{resetvsize}\global\vsize=\textheight} 
    \global\output={\onecol}
    \global\holdinginserts=1
    \count255=1000
    \global\count\topins=\count255
    \global\count\bottomins=\count255
    \let\\=\s@tn@tec@unt \the\n@tecl@sses % reset \count for each note class
    \ifdim\ht\partial>\baselineskip \vbox{\unvbox\partial}
      \ifendbooknoeject\ifx\p@gefirstmark\t@tle \else\vskip\baselineskip\hskip-\columnshift\hrule\vskip2\baselineskip\fi\fi
    \fi
  \fi}%

\count\topins=1000 \dimen\topins=\maxdimen \skip\topins=0pt
\newinsert\bottomins \count\bottomins=1000 \dimen\bottomins=\maxdimen
\newinsert\topleftins \count\topleftins=1000 \dimen\topleftins=\maxdimen
\newinsert\toprightins \count\toprightins=1000 \dimen\toprightins=\maxdimen
\newinsert\bottomleftins \count\bottomleftins=1000 \dimen\bottomleftins=\maxdimen
\newinsert\bottomrightins\count\bottomrightins=1000 \dimen\bottomrightins=\maxdimen

\def\p@rsesize#1*#2*#3\end{%
  \def\size@ption{#1}%
  \def\sizem@ltiple{#2}% empty if not provided
}

%%%%%%%%%%%%%% FIGURES %%%%%%%%%%%%%% 
% called from \fig or from figure list

% Place everything up to first | into \param-N.
% Redefine \p@rams to be rest of list
\def\stripspace#1 \end/#2\relax#3{\if\relax#2\relax\gdef\p@ram{#3}\else\stripspace#1\end/ \end/\relax{#1}\fi}
\def\FigP@ramAlt{alt} \def\FigP@ramSrc{src} \def\FigP@ramSize{size} \def\FigP@ramLoc{loc}
\def\FigP@ramCopy{copy} \def\FigP@ramRef{ref} \def\FigP@ramXtra{xetex}
\newif\ifusfmthree \usfmthreefalse
\def\getonepairp@ram#1="#2" {\stripspace#1\end/ \end/\relax{#1}\edef\p@ramid{\p@ram}%
  \stripspace#2\end/ \end/\relax{#2}\edef\p@ramval{\p@ram}%
  \trace{g}{getonepairparam: [\p@ramid] = [\p@ramval]}%
  \ifx\p@ramid\empty\let\next=\relax\else
    \ifx\p@ramid\FigP@ramSrc\global\p@ramnumber=2\else
    \ifx\p@ramid\FigP@ramSize\global\p@ramnumber=3 \trace{g}{Got Size}\else
    \ifx\p@ramid\FigP@ramLoc\global\p@ramnumber=4\else
    \ifx\p@ramid\FigP@ramCopy\global\p@ramnumber=5\else
    \ifx\p@ramid\FigP@ramAlt\global\p@ramnumber=6\else
    \ifx\p@ramid\FigP@ramRef\global\p@ramnumber=7\else
    \ifx\p@ramid\FigP@ramXtra\global\p@ramnumber=8\else %experimental extra parameter, post filename e.g. "rotated 90"
    \fi\fi\fi\fi\fi\fi\fi
    \tempfalse
    \ifx\p@ramval\empty\ifusfmthree\else
      \trace{g}{param-\the\p@ramnumber = "\p@ramid" but not "\p@ramval"}%
      \x@\global\x@\edef\csname param-\the\p@ramnumber\endcsname{\p@ramid}%
      \let\next=\relax
      \temptrue
    \fi\fi
    \iftemp\else
      \trace{g}{param-\the\p@ramnumber [\p@ramid]  = "#2" from "\p@ramval"}%
      \x@\gdef\csname param-\the\p@ramnumber\endcsname{#2}%
      \let\next=\getonepairp@ram \usfmthreetrue
    \fi
  \fi\next}
 
\def\getonep@ram#1|#2\end{\gdef\p@rams{#2}%
  \trace{g}{getoneparam: #1 | #2}%
  \getonepairp@ram#1 ="" \relax%
  \global\advance\p@ramnumber by 1 }

\newif\iffiglocleft \figloclefttrue
\newif\ifpicUsesIns \picUsesInstrue
\newif\ifpicNarrow \picNarrowfalse

\def\p@rsePicXtra#1#2\end{% Parse optional new picture options, and remember them
  \xdef\t@mpb{#1}%
  \xdef\t@mpc{#2}%
}
\def\p@rsePicUse#1#2\end{% Parse the new picture options, and remember them
  \picUsesInstrue
  \picNarrowfalse
  \gdef\pic@lign{c}%How does the picture align within the box?
  \gdef\picl@c{}% location code for in-line pictures
  \edef\t@mp{#1}
  \edef\t@mpb{#2}
  \ifx\t@mpb\empty\let\t@mpc=\t@mpb\else\x@\p@rsePicXtra\t@mpb\end\fi
  \trace{g}{doespicUseIns? #1 \t@mpb \t@mpc}
  \ifx\t@mp\loc@Inl
    \trace{g}{Experimental inline graphic}%
    \global\let\picl@c\loc@Inl
    \xdef\pic@lign{\t@mpb}
    \global\picUsesInsfalse
    \ifx\t@mpb\@lignLeft\picNarrowtrue\fi
    \ifx\t@mpb\@lignRight\picNarrowtrue\fi
  \fi
  \ifx\t@mp\loc@Cut
    \global\let\picl@c\loc@Cut
    \global\picUsesInsfalse
    \global\picNarrowtrue
    \xdef\l@cspec{\t@mpb}
    \ifx\t@mpc\empty\gdef\c@tskip{0}\else\xdef\c@tskip{\t@mpc}\fi%
    \trace{g}{Experimental graphic in cutout, after \c@tskip lines}%
  \fi
  \ifx\t@mp\loc@Par
    \global\let\picl@c=\loc@Par
    \trace{g}{Experimental after-paragraph graphic}%
    \xdef\pic@lign{\t@mpb}
    \global\picUsesInsfalse
    \ifx\t@mpb\@lignLeft\picNarrowtrue\fi
    \ifx\t@mpb\@lignRight\picNarrowtrue\fi
  \fi
  }
\def\d@figure#1{%
 \ifIncludeFigures
  \trace{g}{In d@figure #1}\usfmthreefalse
  \gdef\p@rams{#1|}% ensure there is a trailing | separator
  % place parts of text into \param-1, \param-2, ...
  \global\p@ramnumber=1
  \loop
    \x@\gdef\csname param-\the\p@ramnumber\endcsname{}%
    \global\advance\p@ramnumber by 1
    \ifnum\p@ramnumber<9\repeat
  \global\p@ramnumber=1
  \loop
    \x@\getonep@ram\p@rams\end
    \ifx\p@rams\empty \morep@ramsfalse \else \morep@ramstrue \fi
    \ifmorep@rams\repeat
  % get size
  \lowercase{\edef\size@ption{\csname param-3\endcsname}}%
  \expandafter\p@rsesize\size@ption**\end % extract possible multiplier
  % get location
  \lowercase{\edef\loc@ption{\csname param-4\endcsname}}%
  % 
  \ifx\loc@ption\empty % if location empty
    \ifx\size@ption\size@SPAN%
	   \def\loc@ption{t}% if size is SPAN default to top
	\else\def\loc@ption{tl}\fi % else default to top left
  \fi
  % set width to column width or span width
  \p@cwidth=\ifx\size@ption\size@COL \hsize
  \else\ifx\size@ption\size@SPAN \textwidth \advance\p@cwidth by -\columnshift
  \else \textwidth
    \errmessage{Unknown picture size "\size@ption", expected "col" or "span"}\fi\fi
  \let\p@cins=\relax
  % Parse the location option 
  \x@\p@rsePicUse\loc@ption\end
  % make \p@cins point to insertion class for this location
  \ifx\loc@ption\loc@T \let\p@cins=\topins
  \else\ifx\loc@ption\loc@B \let\p@cins=\bottomins
  \else
      \ifnum\c@rrentcols>1
        \ifx\loc@ption\loc@TL \let\p@cins=\topleftins
        \else\ifx\loc@ption\loc@TR \let\p@cins=\toprightins
        \else\ifx\loc@ption\loc@BL \let\p@cins=\bottomleftins
        \else\ifx\loc@ption\loc@BR \let\p@cins=\bottomrightins
        \fi\fi\fi\fi
      \else
	    \ifdiglot\MSG{HOW DID THAT HAPPEN? currentcols=\the\c@rrentcols}\fi
        \ifx\loc@ption\loc@TL \picw@rning{tl}{t}\let\p@cins=\topins
        \else\ifx\loc@ption\loc@TR \picw@rning{tr}{t}\let\p@cins=\topins
        \else\ifx\loc@ption\loc@BL \picw@rning{bl}{b}\let\p@cins=\bottomins
        \else\ifx\loc@ption\loc@BR \picw@rning{br}{b}\let\p@cins=\bottomins
        \fi\fi\fi\fi
      \fi
  \fi\fi
  \ifx\p@cins\relax
    %\errmessage{Unknown picture location "\loc@ption",
    %  expected one of t,b,tl,tr,bl,br}
    \ifnum\c@rrentcols>1
      \iffiglocleft\let\p@cins=\topleftins\figlocleftfalse
      \else\let\p@cins=\toprightins\figloclefttrue\fi
    \else \let\p@cins=\topins\fi
  \fi
  % create box to contain picture
  \trace{g}{Figure size=\size@ption, loc=\loc@ption, \ifpicUsesIns picins=\the\p@cins\fi}%
  \setbox0=\vbox{
    \ifpicNarrow
      \hsize=\sizem@ltiple\p@cwidth
    \else
      \hsize=\p@cwidth
    \fi
	% insert picture based on PicPath and file name, scaled to \p@cwidth
    \let\picfilecomm@nd=\XeTeXpicfile
    \edef\f@lename{\csname param-2\endcsname}%
    \expandafter\ch@ckpdf\f@lename..\endf@lename
    \setbox0=\hbox to \sizem@ltiple\p@cwidth{\hss
      \picfilecomm@nd "\the\PicPath\csname param-2\endcsname"
                     width \sizem@ltiple\p@cwidth \csname param-8\endcsname\hss}
    \ifFigurePlaceholders % replace graphic with a frame and the file name
      \setbox0=\vbox to \ht0{\offinterlineskip
        \hrule height .2pt \kern-.2pt
        \hbox to \wd0{\vrule height \ht0 width .2pt \hfil \vrule width .2pt}
        \vbox to 0pt{\kern-\ht0 \vss \hbox to \wd0{\hss\idf@nt
          \csname param-2\endcsname\hss}\vss}
        \kern-.2pt \hrule height .2pt}
    \fi
    \trace{g}{Picture height=\the\ht0, vs:\the\vsize, \the\partialLpenalty,  pg:\the\pagegoal, pt:\the\pagetotal, av:\the\availht}%
    \ifpicNarrow
      \ifx\pic@line\@lignRight
        \line{\hfil\box0}%
      \else\ifx\pic@lign\@lignLeft
          \line{\box0\hfil}%
        \else
          \line{\hfil\box0\hfil}%
        \fi
      \fi
    \else
      \line{\hfil\box0\hfil}%
    \fi
    \edef\r@f{\csname param-7\endcsname} % get reference
    \stripspace\r@f\end/ \end/\relax\r@f \edef\r@f{\p@ram} % strip space
    \edef\c@ption{\ifusfmthree\csname param-1\endcsname\else\csname param-6\endcsname\fi} % get caption
    \trace{g}{Fig caption: [\c@ption] \ifx\r@f\relax\else(\r@f)\fi}
    \ifx{\c@ption}\empty\else
      \everypar={}\let\par\endgraf
      \leftskip=0pt plus \hsize \rightskip=\leftskip \parfillskip=0pt
      \linepenalty=1000
      \s@tbaseline{fig}
      % insert caption box
      \noindent\leavevmode
        \s@tfont{fig}\c@ption\unskip
        \ifx\r@f\empty\else\nobreak\ (\r@f\unskip)\fi
      \par
    \fi
    \vskip.5\baselineskip
  }%
  \trace{b}{BALANCE fig: height=\the\ht0: baselineskip=\the\baselineskip}%
  % insert into proper insertion class for this position
  \ifpicUsesIns
    \trace{g}{Picture goes to insert}%
    \insert\p@cins{\penalty10000 % make sure this picture does not float away to another page
      \splittopskip\z@skip
      \splitmaxdepth\maxdimen \floatingpenalty20000 % was zero
      \gridp@ctrue\gridb@x0\gridp@cfalse% make sure we align to the page grid
    }%
  \else
    % Calculate lines used for the picture
    \dimen0=\ht0
    \advance\dimen0 by \dp0
    \advance\dimen0 by 0.5\baselineskip
    \divide\dimen0 by \baselineskip
    \count0=\dimen0
    \trace{g}{Inline picture takes \the\count0 lines}%
    \ifx\picl@c\loc@Inl
      \trace{g}{Picture is inline}%
      \advance\count0 by \prevgraf % tell cutouts what's happened
      \prevgraf=\count0
      \penalty10 % Allow the page to break
      \splittopskip\z@skip
      \splitmaxdepth\maxdimen \floatingpenalty20000 % was zero
      \ifpicNarrow\ifx\pic@lign\@lignLeft
        \setbox0\hbox to \colwidth{\box0\hss}\else\setbox0\hbox to \colwidth{\hss\box0}\fi
      \fi
      \gridp@ctrue\gridb@x0\gridp@cfalse% make sure we align to the page grid
      \penalty10
    \fi
    \ifx\picl@c\loc@Par
      \trace{g}{Picture is post-paragraph}%
      \x@\global\x@\setbox\x@\picb@x\box0
      \gdef\at@ndofthispar{
        \penalty10 % Allow the page to break
        \splittopskip\z@skip
        \splitmaxdepth\maxdimen \floatingpenalty20000 % was zero
        \ifpicNarrow\ifx\pic@lign\@lignLeft
          \setbox\picb@x\hbox to \colwidth{\box\picb@x\hss}\trace{g}{Pic aligned left}%
          \else\setbox\picb@x\hbox to \colwidth{\hss\box\picb@x}\trace{g}{Pic aligned right}\fi
        \fi
        \gridp@ctrue\gridb@x\picb@x\gridp@cfalse% make sure we align to the page grid
        \penalty10
       }%
    \fi
    \ifx\picl@c\loc@Cut
        \ifhmode
           \ifx\prev@rsemode\empty
             \trace{g}{Cutout in horizontal mode. forcing \m@rker}%
             \csname\m@rker\endcsname
           \else
             \trace{g}{Cutout in horizontal mode, but verse was in vertical. forcing \prev@rsemode and negative lineskip}%
             \csname\prev@rsemode\endcsname
             \vskip -\baselineskip
           \fi
        \fi 
      \trace{g}{Picture is cutout}%
      \dimen1=\baselineskip
      \advance\count0 by 1
      \multiply\dimen1 by \count0
      \advance\dimen1 by \c@tskip\baselineskip
      \dimen2=\wd0
      \advance\dimen2 by 10pt
      \ifx\l@cspec\@lignLeft
        \setbox\picb@x\hbox to 0pt {\lower\dimen1\box0\hss}%
        \ht\picb@x=0pt 
         \dp\picb@x=0pt
        \vskip-\baselineskip  %Yuck
        \box\picb@x %FIXME? Box here Adds vertical space
        \trace{g}{leftcutout \the\dimen2, \c@tskip, \the\count0}%
        \leftcutout{\the\dimen2}{\c@tskip}{\the\count0}%
      \else
        \ifx\l@cspec\@lignRight
          \setbox\picb@x\hbox to \hsize{\hss\lower\dimen1\box0}%
          \ht\picb@x=0pt 
          \dp\picb@x=0pt
          \vskip-\baselineskip %Yuck
          \box\picb@x% FIXME? Box here Adds vertical space
          \trace{g}{rightcutout \the\dimen2, \c@tskip, \the\count0}%\
          %\edef\t@mpd{\the\dimen2}
          \rightcutout{\the\dimen2}{\c@tskip}{\the\count0}%\emspace
        \else
          \message{*** Unknown position \l@cspec, picture misplaced}%
          \box0
        \fi
        \unskip
      \fi
    \fi
   \fi
 \fi% end \ifIncludeFigures
}
\newtoks\PicPath % directory containing picture files
\newif\ifIncludeFigures \IncludeFigurestrue
\newif\ifFigurePlaceholders

\def\ch@ckpdf#1.#2.#3\endf@lename{\lowercase{\edef\@xt{#2}}%
  \ifx\@xt\@pdf \let\picfilecomm@nd\XeTeXpdffile \fi}
\def\@pdf{pdf}

\def\picw@rning#1#2{\msg{converted picture placement "#1" to "#2" in single-column layout}}

\def\size@COL{col}
\def\size@SPAN{span}
\def\loc@T{t}
\def\loc@B{b}
\def\loc@TL{tl}
\def\loc@TR{tr}
\def\loc@BL{bl}
\def\loc@BR{br}
\def\loc@Inl{i}% In-line graphic, not a float.
\def\loc@Cut{c}% cutout
\def\loc@Par{p}% post-paragrpah
\def\@lignLeft{l}
\def\@lignRight{r}
\newbox\picb@x
\newcount\p@ramnumber
\newdimen\p@cwidth
\newif\ifmorep@rams

\def\includepdf{\@netimesetup % in case \ptxfile hasn't been used yet
  \ifx\XeTeXpdfpagecount\undefined
    \MSG{*** sorry, \string\includepdf\space requires XeTeX 0.997 or later}%
    \let\n@xt\relax
  \else \let\n@xt\incl@depdf \fi
  \n@xt}
\def\incl@depdf{\begingroup
  \m@kedigitsother \catcode`\[=12 \catcode`\]=12
  \futurelet\n@xt\t@stincl@de@ptions}
\def\t@stincl@de@ptions{\ifx\n@xt[\let\n@xt\incl@de@ptions
  \else\let\n@xt\incl@deno@ptions\fi\n@xt}
\def\incl@deno@ptions#1{\incl@de@ptions[]{#1}}
\def\incl@de@ptions[#1]#2{%
  \totalp@ges=\XeTeXpdfpagecount "#2"\relax
  \whichp@ge=0
  \loop \ifnum\whichp@ge<\totalp@ges
    \advance\whichp@ge by 1
    \setbox0=\hbox{\XeTeXpdffile "#2" page \whichp@ge #1}%
    \ifdim\wd0>\pdfpagewidth
      \setbox0=\hbox{\XeTeXpdffile "#2" page \whichp@ge #1 width \pdfpagewidth}%
    \fi
    \ifdim\ht0>\pdfpageheight
      \setbox0=\hbox{\XeTeXpdffile "#2" page \whichp@ge #1 height \pdfpageheight}%
    \fi
    {\def\c@rrID{#2 #1 page \number\whichp@ge}% for lower crop-mark info, if requested
      \shipcompletep@gewithcr@pmarks{\vbox{\box0}}}%
    \advancepageno
    \repeat
  \endgroup % begun in \incl@depdf
}
\newcount\totalp@ges
\newcount\whichp@ge

\newif\ifendbooknoeject \endbooknoejectfalse
\def\pagebreak{
  \ifsk@pping \egroup \fi%
  \ifhe@dings\endhe@dings\fi%
  \ifendbooknoeject\else%
  \vfill\eject\fi%
}
\let\pb=\pagebreak

\def\columnbreak{\vfill\eject}

\newcount\badspacepenalty \badspacepenalty=100
\tolerance=9000
\hbadness=10000
\emergencystretch=11in
\vbadness=10000
\vfuzz=2pt
\frenchspacing

\XeTeXdashbreakstate=1 % allow line-break after en- and em-dash even if no space

%% various Unicode characters that we handle in TeX... and protect in TOC and PDF bookmarks

\def\SFTHYPHEN{\-}
\def\NBSP{\nobreak\ } % make Unicode NO-BREAK SPACE into a no-break space
\def\ZWSP{\hskip0pt\relax} % ZERO WIDTH SPACE is a possible break
\def\WJ{\leavevmode\nobreak} % WORD JOINER
\def\ZWNBSP{\WJ} % ZERO WIDTH NO-BREAK SPACE
\def\NBHYPH{\leavevmode\hbox{-}} % NON-BREAKING HYPHEN
\def\NQUAD{\penalty\badspacepenalty\ } % subvert Unicode En-Quad as BAD BREAKING SPACE
\def\MQUAD{\penalty\badspacepenalty\hskip 1em}
\def\MSPACE{\hskip 1em}
\def\NSPACE{\ }
\def\THREEPEREMSPACE{\hskip .333em}
\def\FOURPEREMSPACE{\hskip .25em}
\def\SIXPEREMSPACE{\hskip .1666em}
\def\THINSPACE{\hskip .2em}
\def\HAIRSPACE{\hskip 1sp}

\let\pr@tect=\relax
\def\pr@tectspecials{%
  \let\SFTHYPHEN=\relax
  \let\NBSP=\relax
  \let\ZWSP=\relax
  \let\ZWNBSP=\relax
  \let\WJ=\relax
  \let\NBHYPH=\relax
  \let\NQUAD=\relax
  \let\MQUAD=\relax
  \let\MSPACE=\relax
  \let\NSPACE=\relax
  \let\THREEPEREMSPACE=\relax
  \let\FOURPEREMSPACE=\relax
  \let\SIXPEREMSPACE=\relax
  \let\THINSPACE=\relax
  \let\HAIRSPACE=\relax
}

\catcode"A0=12
\catcode"AD=12
\catcode"200B=12
\catcode"2060=12
\catcode"FEFF=12
\catcode"2000=12
\catcode"2001=12
\catcode"2002=12
\catcode"2003=12
\catcode"2004=12
\catcode"2005=12
\catcode"2006=12
\catcode"2009=12
\catcode"200A=12
\def\liter@lspecials{%
  \def\pr@tect{}%
  \def\NBSP{^^a0}%
  \def\SFTHYPHEN{^^ad}%
  \def\ZWSP{^^^^200b}%
  \def\WJ{^^^^2060}%
  \def\ZWNBSP{^^^^feff}%
  \def\NBHYPH{^^^^2011}%
  \def\NQUAD{^^^^2000}%
  \def\MQUAD{^^^^2001}%
  \def\NSPACE{^^^^2002}%
  \def\MSPACE{^^^^2003}%
  \def\THREEPEREMSPACE{^^^^2004}%
  \def\FOURPEREMSPACE{^^^^2005}%
  \def\SIXPEREMSPACE{^^^^2006}%
  \def\THINSPACE{^^^^2009}%
  \def\HAIRSPACE{^^^^200a}%
}

\catcode"A0=\active   \def^^a0{\pr@tect\NBSP}
\catcode"AD=\active   \def^^ad{\pr@tect\SFTHYPHEN}
\catcode"200B=\active \def^^^^200b{\pr@tect\ZWSP}
\catcode"2060=\active \def^^^^2060{\pr@tect\WJ}
\catcode"FEFF=\active \def^^^^feff{\pr@tect\ZWNBSP}
\catcode"2011=\active \def^^^^2011{\pr@tect\NBHYPH}
\catcode"2000=\active \def^^^^2000{\pr@tect\NQUAD}
\catcode"2001=\active \def^^^^2001{\pr@tect\MQUAD}
\catcode"2002=\active \def^^^^2002{\pr@tect\NSPACE}
\catcode"2003=\active \def^^^^2003{\pr@tect\MSPACE}
\catcode"2004=\active \def^^^^2004{\pr@tect\THREEPEREMSPACE}
\catcode"2005=\active \def^^^^2005{\pr@tect\FOURPEREMSPACE}
\catcode"2006=\active \def^^^^2006{\pr@tect\SIXPEREMSPACE}
\catcode"2009=\active \def^^^^2009{\pr@tect\THINSPACE}
\catcode"200A=\active \def^^^^200a{\pr@tect\HAIRSPACE}

\catcode`\@=12

\parskip=0pt
\lineskip=0pt
% NOTE: The baselineskip is set negative here. Later on it will
% be set to the default of 14pt unless the user has specified
% another setting. If that is the case, that setting will be
% used and text can then move off the baseline which makes
% balancing easier but causes text to move off the grid in
% some instances.
\baselineskip=-14pt

\widowpenalty=10000
\clubpenalty=10000
\brokenpenalty=50

\endinput
