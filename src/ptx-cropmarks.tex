%%%%%%%%%%%%%%%%%%%%%%%%%%%%%%%%%%%%%%%%%%%%%%%%%%%%%%%%%%%%%%%%%%%%%%%
% Part of the ptx2pdf macro package for formatting USFM text
% copyright (c) 2007 by SIL International
% written by Jonathan Kew
%
% Permission is hereby granted, free of charge, to any person obtaining  
% a copy of this software and associated documentation files (the  
% "Software"), to deal in the Software without restriction, including  
% without limitation the rights to use, copy, modify, merge, publish,  
% distribute, sublicense, and/or sell copies of the Software, and to  
% permit persons to whom the Software is furnished to do so, subject to  
% the following conditions:
%
% The above copyright notice and this permission notice shall be  
% included in all copies or substantial portions of the Software.
%
% THE SOFTWARE IS PROVIDED "AS IS", WITHOUT WARRANTY OF ANY KIND,  
% EXPRESS OR IMPLIED, INCLUDING BUT NOT LIMITED TO THE WARRANTIES OF  
% MERCHANTABILITY, FITNESS FOR A PARTICULAR PURPOSE AND  
% NONINFRINGEMENT. IN NO EVENT SHALL SIL INTERNATIONAL BE LIABLE FOR  
% ANY CLAIM, DAMAGES OR OTHER LIABILITY, WHETHER IN AN ACTION OF  
% CONTRACT, TORT OR OTHERWISE, ARISING FROM, OUT OF OR IN CONNECTION  
% WITH THE SOFTWARE OR THE USE OR OTHER DEALINGS IN THE SOFTWARE.
%
% Except as contained in this notice, the name of SIL International  
% shall not be used in advertising or otherwise to promote the sale,  
% use or other dealings in this Software without prior written  
% authorization from SIL International.
%%%%%%%%%%%%%%%%%%%%%%%%%%%%%%%%%%%%%%%%%%%%%%%%%%%%%%%%%%%%%%%%%%%%%%%

% cropmark support for the ptx2pdf package

%+c_makecropmarks
\newif\ifBookOpenLeft\BookOpenLeftfalse % The friend of RTL. In monoglot these two typically go together, in diglot they may not. ifOpenLeft controls where binding gutter goes, etc.
\newif\ifCropMarks
%\font\idf@nt=cmtt10 scaled 700 % font for the marginal job information
\font\idf@nt="Source Code Pro" at 8pt
\def\id@@@{}% just in case.
%\font\idf@nt="Times New Roman" at 10pt % FIXME: Use something not bitmap?
\newbox\topcr@p \newbox\bottomcr@p
\def\makecr@ps{% construct the cropmark boxes for top and bottom of page
  \global\setbox\topcr@p=\vbox to 0pt{\vss
    \hbox to \PaperWidth{%
      \kern -30pt
      \vrule height .2pt depth .2pt width 25pt
      \kern 4.8pt
      \vrule height 30pt depth -5pt width .4pt
      \kern -30.4pt
      \dimen0=\topm@rgin \advance\dimen0 by .4pt
      \vrule height -\topm@rgin depth \dimen0 width 15pt
      \kern 15pt
      \hss
      \raise20pt\vtop{\hsize\PaperWidth \everypar={}
        \lineskiplimit=0pt \baselineskip=10pt
        \leftskip=0pt plus 1fil \rightskip=\leftskip \parfillskip=0pt
        \noindent \beginL\idf@nt \jobname\ :: \timestamp\endL}%
      \hss
      \vrule height 30pt depth -5pt width .4pt
      \kern 4.8pt
      \vrule height .2pt depth .2pt width 25pt
      \kern -15pt
      \vrule height -\topm@rgin depth \dimen0 width 15pt
      \kern -30pt
    }%
  }\dp\topcr@p=0pt % end \vbox for \topcr@p
  \global\setbox\bottomcr@p=\vbox to 0pt{%
    \setbox0=\hbox to \PaperWidth{%
      \dimen0=\bottomm@rgin \advance\dimen0 by 0.4pt
      \kern -30pt
      \vrule height .2pt depth .2pt width 25pt
      \kern -15pt
      \vrule height \dimen0 depth -\bottomm@rgin width 15pt
      \kern 4.8pt
      \vrule height -5pt depth 30pt width .4pt
      \hss
      \vrule height -5pt depth 30pt width .4pt
      \kern 4.8pt
      \vrule height \dimen0 depth -\bottomm@rgin width 15pt
      \kern -15pt
      \vrule height .2pt depth .2pt width 25pt
      \kern -30pt
    }%
    \dp0=0pt\ht0=0pt\box0
  }\ht\bottomcr@p=0pt\dp\bottomcr@p=0pt% end \vbox for \bottomcr@p
}
%-c_makecropmarks

%+c_shipwithcropmarks
\newif\ifrotate\rotatefalse
\newdimen\pdfcropwidth
\newdimen\pdfcropheight
\newdimen\tabheight % In page-orientation
\newdimen\tabwidth % In page-orientation
\newdimen\TabsStart %Vertical offset to margins for start of thumb tabs
\newdimen\TabsEnd %Vertical offset to margins for end of thumb tabs 
\newcount\NumTabs %How many tabs to be set?
\newif\ifTabTopToEdgeEven %Is the top of the text at the pageedge on even pages
\newif\ifTabTopToEdgeOdd %Is the top of the text at the pageedge on odd pages
\TabTopToEdgeEvenfalse
\TabTopToEdgeOddfalse

\newif\ifTabAutoRotate \TabAutoRotatetrue % Does autorotation of text occur?
\newif\ifTabRotationNormal \TabRotationNormaltrue % TabRotationNormalfalse==rotate text (if auto-rotate off). (If auto-rotate on: invert autorotate logic, i.e. assume the \toc3 are narrower than they are tall).



\NumTabs=2
\tabheight=50pt
\tabwidth=15pt
\TabsStart=10pt %First Tab starts 10pt from the vertical margin
\TabsEnd=10pt %LastTab ends 10 from the vertical margin
% Thumb-tabs codes assume  setup in the form \settumbtab{Gen}{1}, 
% where Gen exactly matches the \toc3 entry for the book.
\def\ThumbTabStyle{toc3}%Which style defines the font? Default is toc3
\def\tabBoxCol{0 0 0}%Bacground as r g b 
\def\tabFontCol{1 1 1}%Foregreound as r g b

\def\sett@bname#1{\xdef\t@bname{t@b-#1}}
\def\sett@bgrpnam#1{\xdef\t@bgrp{t@bg-#1}}
\def\setthumbtab#1#2{\sett@bname{#1}\edef\t@mp{#2}\ifx\t@mp\empty\x@\let\csname\t@bname -num\endcsname\relax\else\x@\xdef\csname\t@bname -num\endcsname{#2}\ifnum\NumTabs<#2 \global\NumTabs=#2 \fi\fi}
\def\setthumbtabBg#1#2{\sett@bname{#1}\x@\xdef\csname\t@bname -boxcol\endcsname{#2}}
\def\setthumbtabFg#1#2{\sett@bname{#1}\x@\xdef\csname\t@bname -col\endcsname{#2}}
% Tab groups
\def\settabgroupFg#1#2{\sett@bgrpnam{#1}\x@\xdef\csname\t@bgrp -col\endcsname{#2}}
\def\settabgroupBg#1#2{\sett@bgrpnam{#1}\x@\xdef\csname\t@bgrp -boxcol\endcsname{#2}}
\def\gett@bgroup{\x@\let\x@\t@bgrp\csname\t@bname-grp\endcsname}

\def\setthumbtabgroup#1#2{% #1 - Name e.g. Pent, #2 list of book IDs e.g GEN,EXO,NUM,DEU,JOS
 \sett@bgrpnam{#1}\let\n@xtcmd\s@tt@bgroup \n@xtcmd #2,,\endtab}
\def\l@stcmd#1,\endtab{}

\def\s@tt@bgroup#1,#2,\endtab{%Iterate through comma separated list, setting group name
  \edef\tmp{#2}%
  \ifx\tmp\empty\let\n@xtcmd\l@stcmd\fi
  \sett@bname{#1}\x@\xdef\csname \t@bname-grp\endcsname{\t@bgrp}%
  \n@xtcmd #2,\endtab}


\def\colourbox#1#2#3#4#5#6#7{%1 - box colour, 2 - text colour 
  % 3- boxwidth 
  % 4- box depth 
  % 5 - box height
  % 6 - =0, no movement >0pt  separation of text from bottom =1sp, centre text.  < 0pt separation from top of box to top of text 
  % 7 - text (should include glue to align / centre e.g. \hss)
  %\dimen2=\baselineskip\advance\dimen2 by -.2ex
  % Box height is set by \tabwidth, text is set 2pt below top 
  \trace{pt}{(#1) (#2) #3 #4 #5 #6 #7}%
  \dimen2=#6
  \setbox0\hbox to #3{#7}%
  \ifdim\dimen2<0pt
   \advance\dimen2 by #5\advance\dimen2 by -\ht0
   \ifdim\dimen2<0pt\dimen2=0pt\fi
  \else 
    \ifdim\dimen2=1sp%Magic number to indicate vertical centering.
     %To centre text, it goes UP by ((#5-ht0)-(#4-dp0))/2
     %can't use -#4 because #4 may be -ve already.
     \dimen2=#4\multiply\dimen2 by -1
     \advance\dimen2 by #5\advance\dimen2 by -\ht0
     \advance\dimen2 by \dp0
     \divide\dimen2 by 2
    \else
      \ifdim\dimen2=0pt\else
        %Distance from bottom  shift=#6+dp0 -#4
        %can't use -#4 because #4 may be -ve already.
        \dimen2=#4\multiply\dimen2 by -1
        \advance\dimen2 by #6
        \advance\dimen2 by \dp0
      \fi
    \fi
  \fi
  \def\cp@p{\special{color pop}}%
  \edef\bgc@l{#1}\ifx\bgc@l\empty\let\endbgc@l\empty\else\edef\bgc@l{\special{color push rgb #1}}\let\endbgc@l=\cp@p\fi%
  \edef\fgc@l{#2}\ifx\fgc@l\empty\let\endfgc@l\empty\else\edef\fgc@l{\special{color push rgb #2}}\let\endfgc@l=\cp@p\fi%
  \setbox0\hbox{\hbox to 0pt{\bgc@l\vrule depth #4 height #5 width #3 \hss}\endbgc@l\fgc@l\raise \dimen2\box0}\ht0=#5\dp0=#4\box0\endfgc@l}

%Rotate box0 about a point on it's baseline, half way along its width. Adjust size so that height/depth and width are correct for rotated box. box height/depth becomes left-most edge.
\def\r@acwSet{\edef\@ngle{90}\kern\ht0}% anticlockwise
\def\r@cwSet{\edef\@ngle{-90}\kern\dp0}%clockwise.
\def\rot@tebz{\dimen1=\wd0\dimen2=\ht0
  \advance\dimen2 by \dp0%
  \hbox to \dimen2{\kern -0.5\wd0%
    %kern so that the correct edge of the box is at the centrepoint location, and set rotation
    %\ifodd\pageno \kern\ht0\else\kern\dp0\fi %Old-style
    \r@tSet
    \trace{pt}{Rotating tab \@ngle (+ is anticlockwise)}%
    \vbox to 0pt{\vss\hbox to 0pt{\kern 0.5\wd0\special{x:gsave}\special{x:rotate \@ngle}\hss}\vss}%
    \ht0=0.5\dimen1\dp0=0.5\dimen1\box0\special{x:grestore}\hss}}
\def\rot@tebzoneeighty{\dimen1=\wd0\dimen2=\ht0\dimen3=\dp0
  \advance\dimen2 by \dp0%
  \hbox to \dimen1{\kern -0.5\wd0%
    %kern so that the correct edge of the box is at the centrepoint location, and set rotation
    %\ifodd\pageno \kern\ht0\else\kern\dp0\fi %Old-style
    \r@tSet
    \trace{pt}{Rotating tab \@ngle (+ is anticlockwise)}%
    \vbox to 0pt{\vss\hbox to 0pt{\kern 0.5\wd0\special{x:gsave}\special{x:rotate \@ngle}\hss}\vss}%
    \ht0=\dimen3\dp0=\dimen2\kern-0.5\dimen1\box0\special{x:grestore}\hss}}


\def\TabTxtFarFromEdge#1{\if\st@rtatedge F\kern 2pt\else\hss\fi #1\if\st@rtatedge F\hss\else\kern 2pt\fi}% Normal position for horizontal text - left-aligned on odd pages / bottom when rotated
\def\TabTxtCloseToEdge#1{\if\st@rtatedge T\kern 2pt\else\hss\fi #1\if\st@rtatedge T\hss\else\kern 2pt\fi}% Right-aligned on odd-pages / Top when rotated
\def\TabTxtCentred#1{\hss #1\hss}% 
\let\horizThumbtabContents=\TabTxtFarFromEdge% How should the thumbtab be positioned when horizontal?
\let\vertThumbtabContents=\TabTxtCentred% How should the thumbtab be positioned when vertical?
\def\vertThumbtabVadj{-2pt}%2pt below top.
\def\horizThumbtabVadj{1sp}%vertically centred.
\def\lastw@rning{}

\def\in@tThumbT@bs{%Thumb-tab setup code goes here.
  \let\initThumbT@bs\relax
}

\def\reinitThumbTabs{%Function for users to call if they've altered something that affects initialisation
  \let\initThumbT@bs\in@tThumbT@bs
}
     
\reinitThumbTabs

\def\t@bbox{%
 \s@tfont{\ThumbTabStyle}%
 \x@\let\x@\t@bBoxCol\csname\t@bname -boxcol\endcsname
 \x@\let\x@\t@bFontCol\csname\t@bname -col\endcsname
 \ifx\t@bgrp\relax
   \let\t@bgFontCol\relax
   \let\t@bgBoxCol\relax
 \else
   \x@\let\x@\t@bgBoxCol\csname\t@bgrp -boxcol\endcsname
   \x@\let\x@\t@bgFontCol\csname\t@bgrp -col\endcsname
 \fi
 \ifx\t@bBoxCol\relax\ifx\t@bgBoxCol\relax\let\t@bBoxCol\tabBoxCol\else\let\t@bBoxCol\t@bgBoxCol\fi\fi
 \ifx\t@bFontCol\relax\ifx\t@bgFontCol\relax\let\t@bFontCol\tabFontCol\else\let\t@bFontCol\t@bgFontCol\fi\fi
 \tempfalse% Don't rotate
 \ifTabAutoRotate 
   \ifdim\tabheight>\tabwidth% Normally rotate
     \ifTabRotationNormal
       \temptrue% DO  rotate
     \fi
   \else%widith>height, don't normally rotate
     \ifTabRotationNormal\else
       \temptrue %Do rotate
     \fi
   \fi
 \else
   \ifTabRotationNormal\else
     \temptrue %Do rotate
   \fi
 \fi
 \setbox0\hbox{\b@okShort}%
 \iftemp%Rotated tabs
   \ifdim\wd0>\tabheight\ifx\lastw@rning\b@okShort\else\xdef\lastw@rning{\b@okShort}\message{Thumb tab contents "\b@okShort" too wide for tab height}\fi\fi
   \tempfalse
   \ifodd\pageno
     \ifTabTopToEdgeOdd\temptrue\fi
   \else
     \ifTabTopToEdgeEven\temptrue\fi
   \fi
   \trace{pt}{pg \the\pageno, Top to \iftemp edge \else page\fi}%
   \iftemp %Rotated text has its top at the edge of the page (decenders to text)
     \edef\bxht{2pt}% Bleed distance
     \edef\bxdp{\the\tabwidth}%
     \bgroup
       \dimen1=\vertThumbtabVadj%
       \ifdim\dimen1=1sp\else %Don't mess with special value!
         \multiply\dimen1 by -1%
       \fi
       \xdef\V@dj{\the\dimen1}%
     \egroup
   \else %Rotated text has its bottom at the edge of the page (descenders to edge)
     \edef\bxht{\the\tabwidth}%
     \edef\bxdp{2pt}% Bleed distance
     \edef\V@dj{\vertThumbtabVadj}%
   \fi
   \setbox0\hbox{\colourbox{\t@bBoxCol}{\t@bFontCol}{\tabheight}{\bxdp}{\bxht}{\V@dj}{\vertThumbtabContents{\b@okShort}}}%
   \ht0=\tabwidth
   \let\r@tSet\r@cwSet 
   \ifodd\pageno
     \ifTabTopToEdgeOdd\else\let\r@tSet\r@acwSet\fi
   \else
     \ifTabTopToEdgeEven\let\r@tSet\r@acwSet\fi
   \fi
   \hbox to 0pt{%
    \if\st@rtatedge F\kern -\tabwidth\else\hss\fi
     %\setbox2\hbox{\vrule height 0.5\wd0 depth 0.5\wd0 width 1pt}%
     %\copy2 \rot@tebz \box2
     \rot@tebz 
     \if\st@rtatedge F\hss\else \kern-1\tabwidth\fi
   }%
 \else%Unrotated tabs
   \ifdim\wd0>\tabwidth\ifx\lastw@rning\b@okShort\else\xdef\lastw@rning{\b@okShort}\message{Thumb tab contents "\b@okShort" too wide for tab width}\fi\fi
   \hbox to 0pt{%
    \if\st@rtatedge F\hss\else\kern -1pt\fi% Bleed of 1pt
    \setbox0\hbox{\colourbox{\t@bBoxCol}{\t@bFontCol}{\tabwidth}{0.1\tabheight}{0.9\tabheight}{\horizThumbtabVadj}{\horizThumbtabContents{\b@okShort}}}%
   % \setbox2\hbox{\vrule height \ht0 depth \dp0 width 1pt}%
   % \copy2 
   %\ifodd\pageno\else\showbox0\fi
    \box0
   % \box2 
    \if\st@rtatedge F\kern -1pt \else\hss\fi%bleed of 1pt.
    }%
 \fi
}

\def\putthumbtab{%Using the current book ID, and the book's short name complete and position a thumbtab mark
  \initThumbT@bs%becomes \relax after 1st run, unless reset
  \let\st@rtatedge=T% Does a horizontal chunk of text in a box  start nearest to the page edge away from the binding (False for English on odd-numbered pages) (depends on page sequence, matching LTR/RTL)
  \ifodd\pageno
    \ifRTL
      \ifBookOpenLeft\let\st@rtatedge=F\fi
    \else
      \ifBookOpenLeft\else\let\st@rtatedge=F\fi
    \fi
  \else
    \ifRTL
      \ifBookOpenLeft\else\let\st@rtatedge=F\fi
    \else
      \ifBookOpenLeft\let\st@rtatedge=F\fi
    \fi
  \fi
 \trace{pt}{PTT nt:\the\NumTabs, bk:\b@okShort, pg:\the\pageno}%
 \edef\@seglyphmetrics{\the\XeTeXuseglyphmetrics}%
 \XeTeXuseglyphmetrics=3
 \ifnum\NumTabs=0 \else
   \sett@bname{\id@@@}\gett@bgroup\x@\let\x@\tabnum\csname\t@bname -num\endcsname
  \trace{pt}{\t@bname, Tabnum: \tabnum}%
   \ifx\tabnum\relax\else %NOT ifx..  This allows setting the number to \relax and it'll disable it.
      {%Protect temporary dimensions and box
        \dimen1=\textheight
        \advance\dimen1 by -\TabsStart
        \advance\dimen1 by -\TabsEnd
        \advance\dimen1 by -\tabheight
        \count255=\NumTabs
        \advance\count255 by -1
        \divide\dimen1 by\count255
        \count255=\tabnum
        \advance\count255 by -1
        \multiply\dimen1 by \count255
        \advance\dimen1 by \TabsStart
        \advance\dimen1 by \dimen0
        \advance\dimen1 by \topm@rgin
        \trace{pt}{d1=\the\dimen1}%
        \vbox to 0pt{%
          \kern\dimen1
          \setbox0\hbox to \PaperWidth{%
             \ifodd\pageno\hss\fi\t@bbox\ifodd\pageno\else\hss\fi}%
          \moveright\dimen0\box0
          %\setbox0\hbox to \PaperWidth{\kern -0.5pt\vrule width 1pt height 5pt\hss\vrule width 1pt height 5pt\kern -0.5pt}%
          %\moveright\dimen0\box0
          %\kern\textheight
          \vss
          %\hrule
        }%
      }%
   \fi
 \fi %ifnum
 \XeTeXuseglyphmetrics=\@seglyphmetrics% Restore to normal value.
}
\def\shipwithcr@pmarks#1{% \shipout box #1, adding cropmarks if required
  \dimen0=0\ifCropMarks .5\fi in
  \advance\pdfpagewidth by 2\dimen0 % increase PDF media size
  \advance\pdfpageheight by 2\dimen0
  \hoffset=-1in \voffset=-1in % shift the origin to (0,0)
  \let\pr@tect=\noexpand
  \shipout\vbox to 0pt{% ship the actual page content, with \BindingGutter added if wanted
    %%%%% The following was added to help produce proper x1a PDF output (djd - 20150504)
    %%%%% Note the \special line here seems to work but there may be a better location for it
    %%%%% Also, the numbers used are hard coded now but need to be calculated
    \ifCropMarks
      \pdfcropwidth=\pdfpagewidth\advance\pdfcropwidth by -\dimen0
      \pdfcropheight=\pdfpageheight\advance\pdfcropheight by -\dimen0
      \def\tmp{\dimen0}
      \special {pdf:put @thispage <</MediaBox [0 0 \strip@pt\pdfpagewidth \space
        \strip@pt\pdfpageheight ] /TrimBox [\strip@pt\tmp \space
        \strip@pt\tmp \space \strip@pt\pdfcropwidth \space
        \strip@pt\pdfcropheight ] >>}
    \else\trace{s}{pdfpagewidth=\the\pdfpagewidth}\special {pdf:put @thispage <</MediaBox [0 0 \strip@pt{\pdfpagewidth} \space
        \strip@pt{\pdfpageheight} ] /TrimBox [0 0 \strip@pt{\pdfpagewidth} \space
        \strip@pt{\pdfpageheight} ]>> }\fi
    % End inserted \special line
    \let\g@tterside=0 % figure out if we're adding a binding gutter, and which side
    \ifBindingGutter
      \ifDoubleSided
        \ifodd\pageno\ifBookOpenLeft\let\g@tterside=R \else\let\g@tterside=L\fi
        \else\ifBookOpenLeft\let\g@tterside=L \else\let\g@tterside=R\fi
        \fi
      \else\ifBookOpenLeft\let\g@tterside=R\else\let\g@tterside=L\fi
      \fi
    \fi
    \edef\oldup{\the\XeTeXupwardsmode}
    \XeTeXupwardsmode=0 %
    \ifrotate
      \vbox to 0pt{\kern.5\pdfpagewidth  %\PaperWidth % swapped because rotated
        \hbox to 0pt{\kern.5\pdfpagewidth \special{x:gsave}\special{x:rotate -90}\hss}
        \vss}
    \fi
    \m@rgepdf
    \pl@ceborder % add PageBorder (or watermark) graphic, if defined
    \offinterlineskip
    \vbox to \ifrotate\pdfpagewidth\else\pdfpageheight\fi{\vss
      \kern\topm@rgin
      %\trace{g}{\the\topm@rgin - \the\bottomm@rgin}%
      \vbox{\hbox to \ifrotate\pdfpageheight\else\pdfpagewidth\fi{\hss\hbox{%
        \if\g@tterside L\kern\BindingGutter\fi
        \XeTeXupwardsmode=\oldup
        #1%
        \if\g@tterside R\kern\BindingGutter\fi
      }\hss}}
      \kern\bottomm@rgin
      \vss}
    \vss
    \docr@pmarks
    \putthumbtab
    \ifrotate\special{x:grestore}\fi
  }%
  \let\pr@tect=\relax
}
%-c_shipwithcropmarks

%+c_shipcomplete
\def\shipcompletep@gewithcr@pmarks#1{% \shipout box #1, adding cropmarks if required
                                     % but without adding margins, borders, etc
                                     % (used for \includepdf)
  \dimen0=0\ifCropMarks .5\fi in
  \advance\pdfpagewidth by 2\dimen0 % increase PDF media size
  \advance\pdfpageheight by 2\dimen0
  \hoffset=-1in \voffset=-1in % shift the origin to (0,0)
  \shipout\vbox to 0pt{% ship the actual page
    %%%%% The following was added to help produce proper x1a PDF output (djd - 20150504)
    %%%%% Note the \special line here seems to work but there may be a better location for it
    %%%%% Also, the numbers used are hard coded now but need to be calculated
    \ifCropMarks
      \pdfcropwidth=\pdfpagewidth\advance\pdfcropwidth by -\dimen0
      \pdfcropheight=\pdfpageheight\advance\pdfcropheight by -\dimen0
      \def\tmp{\dimen0}
      \special {pdf: put @thispage <</MediaBox [0 0 \strip@pt\pdfpagewidth \space
        \strip@pt\pdfpageheight ] /TrimBox [\strip@pt\tmp \space
        \strip@pt\tmp \space \strip@pt\pdfcropwidth \space
        \strip@pt\pdfcropheight ] /ArtBox [\strip@pt\tmp \space
        \strip@pt\tmp \space \strip@pt\pdfcropwidth \space \strip@pt\pdfcropheight ]>>}
    \else
      \special {pdf: put @thispage <</MediaBox [0 0 \strip@pt\pdfpagewidth \space
        \strip@pt\pdfpageheight ] /TrimBox [0 0 \strip@pt\pdfpagewidth \space
        \strip@pt\pdfpageheight ] /ArtBox [0 0 \strip@pt\pdfpagewidth \space
        \strip@pt\pdfpageheight ]>>}
    \fi
    \trace{p}{Pagesize \the\pdfpagewidth, \the\pdfpageheight}
    % End inserted \special line
    \offinterlineskip                                                           %(1)
    \vbox to \pdfpageheight{\vss
      \hbox to \pdfpagewidth{\hss#1\hss}
      \vss}
    \vss
    \docr@pmarks
  }%
}
%-c_shipcomplete

%+c_docropmarks
\def\docr@pmarks{%
    \ifCropMarks % if crop marks are enabled
      \ifvoid\topcr@p \makecr@ps \fi % create them (first time)
      \vbox to 0pt{
        \kern\dimen0
        \moveright\dimen0\copy\topcr@p
        \kern\PaperHeight
        \moveright\dimen0\copy\bottomcr@p
        \moveright\dimen0\vbox to 0pt{\kern15pt\hsize\PaperWidth \everypar={}
          \lineskiplimit=0pt \baselineskip=10pt \linepenalty=200
          \leftskip=0pt plus 1fil \rightskip=\leftskip \parfillskip=0pt
          \noindent \beginL\idf@nt
            \csname c@rrID\endcsname\endL\endgraf % add the current \id line
          \vss}
        \vss
      }
    \fi}
%-c_docropmarks

%+c_pageborders
\def\PageBorder{}
\newbox\b@rder
\def\pl@ceborder{\ifx\PageBorder\empty\else % if \PageBorder is empty, this does nothing
  \ifvoid\b@rder % set up the \b@rder box the first time it's needed
    \global\setbox\b@rder=\hbox{\XeTeXpdffile \PageBorder \relax}%
    \global\setbox\b@rder=\vbox to \pdfpageheight{\vss
      \hbox to \pdfpagewidth{\hss\box\b@rder\hss}\vss}%
  \fi
  \vbox to 0pt{% respect binding gutter, just like main page content
    \hbox to \pdfpagewidth{\hss\hbox{%
      \if\g@tterside L\kern\BindingGutter\fi
      \copy\b@rder
      \if\g@tterside R\kern\BindingGutter\fi
    }\hss}
  \vss}% output a copy of \box\b@rder
\fi}

\def\MergePDF{}
\def\addqu@tes#1#2\E{\if #1"\relax \def\m@rgePDF{#1#2 }\else\def\m@rgePDF{"#1#2" }\fi}
\def\m@rgepdf{\ifx\MergePDF\empty\else
  \x@\addqu@tes \MergePDF\E  %Make \m@rgePDF a copy of \MergePDF, adding quotes if needed
  \vbox to 0pt{\vbox to \pdfpageheight{\vss\hbox to \pdfpagewidth{\hss\XeTeXpdffile \m@rgePDF media\relax\hss}\vss}\vss}
\fi}
%-c_pageborders

%+c_plainoutput
\m@rksonpagefalse
\newif\ifp@geone\p@geonetrue
% redefine plain TeX's output routine to add the cropmarks
\def\plainoutput{%
  \ifm@rksonpage\else\trace{m}{No marks on page \the\pageno}\fi
  \dimen9=\pdfpagewidth \dimen8=\pdfpageheight
  \shipwithcr@pmarks{\vbox{\makeheadline\pagebody\makefootline}}%
  \pdfpagewidth=\dimen9 \pdfpageheight=\dimen8
  \advancepageno
  \immediate\write\p@rlocs{\string\@pgstart\string{\the\pageno\string}}%
  \global\p@geonefalse
  \ifnum\pageno>10000\MSG{Error: Runaway page output}\s@ve@nd\fi
  \ifnum\outputpenalty>-\@MM \else\dosupereject\fi\global\m@rksonpagefalse\global\noteseenfalse
  }
%-c_plainoutput

\endinput
